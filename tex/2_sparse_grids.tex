\setdictum{%
  B-splines are not enough!%
}{%
  In a talk at the 2017 SIAM Conference on\\
  Computational Science and Engineering%
}
% SIAM CSE 2017, Minisymposium MS154 "Flooding the Cores--Computing Flooding
% Events on Modern Architecture--Part I of II",
% Craig Michoski, UT Austin,
% "Scaling at Exascale in Blended Isogeometric, Discontinuous
% Galerkin, and Particle-in-Cell Approaches"

\chapter{Hierarchical B-Splines and Sparse Grids}
\label{sec:sg}

\todo{write}

\section{Hierarchical Bases and Sparse Grids}

\todo{write}

\subsection{Nodal Spaces}

\todo{insert citations}

Let us first consider univariate functions
that are defined on the unit interval $[0, 1]$.
\newgsymbol{l}{$l$}{Level $\in \NN_0$}%
We discretize this domain by splitting it into $2^l$ equidistant segments,
where $l \in \NN_0$ is the \term{level}.
\newgsymbol{xli}{$x_{l,i}$}{Grid point $:= i \cdot h_l$}%
\newgsymbol{i}{$i$}{Index $= 0, \dotsc, 2^l$}%
\newgsymbol{hl}{$h_l$}{Mesh size $:= 2^{-l}$}%
The resulting $2^l + 1$ \term{grid points} are given by
\begin{equation}
  x_{l,i} := i \cdot h_l,\quad
  i = 0, \dotsc, 2^l,
\end{equation}
where $i$ is the \term{index} and $h_l := 2^{-l}$ is the \term{mesh size}.
\newgsymbol{phili}{$\varphi_{l,i}$}{%
  Hierarchical basis function of level $l$, index $i$%
}%
Every grid point is associated with a \term{basis function}
\begin{equation}
  \varphi_{l,i}\colon [0, 1] \to \RR.
\end{equation}
In this thesis, we assume $\varphi_{l,i}$ to be arbitrary,
satisfying required assumptions when needed.
However, it helps for both the theory and the intuition to have a
specific example of basis functions in mind.
\newgsymbol{phili1}{$\varphi_{l,i}^1$}{Hat function of level $l$, index $i$}%
The most common choice for $\varphi_{l,i}$ are the so-called \term{hat functions}
(linear B-splines), which are defined as
\begin{equation}
  \varphi_{l,i}^1(x)
  := \max(1 - |x/h_l - i|, 0).
\end{equation}
Here and in the following,
the superscript ``1'' is the degree of the linear B-spline and
is not to be read as an exponent.
We will generalize this notation to B-splines $\varphi_{l,i}^p$ of
arbitrary degrees $p$ (see \cref{sec:sgBspl}).

\newgsymbol{Vl}{$V_l$}{Nodal space of level $l$}%
The \emph{nodal space} $V_l$ of level $l$
is defined as the linear span of all basis functions
$\varphi_{l,i}$, $i = 0, \dotsc, 2^l$:
\begin{equation}
  V_l := \spn\{\varphi_{l,i} \mid i = 0, \dotsc, 2^l\}.
\end{equation}
We assume that the functions $\varphi_{l,i}$ form a basis of $V_l$, i.e.,
every linear combination of these functions is unique.
\newgsymbol{Vl1}{$V_l^1$}{Nodal piecewise linear space of level $l$}%
In the case of hat functions $\varphi_{l,i}^1$,
the nodal space $V_l^1$ is defined analogously as the span of the
$\varphi_{l,i}^1$ functions.
The space $V_l^1$ is the space of the linear splines,
that is, the space of all continuous functions on $[0, 1]$ that are
piecewise linear polynomials on $[x_{l,i}, x_{l,i+1}]$ for
$i = 0, \dotsc, 2^l - 1$.

\newgsymbol{d}{$d$}{Dimensionality $\in \NN$}%
For the multivariate case with $d \in \NN$ dimensions,
we proceed with the usual tensor product approach,
for which we replace all indices, points, and functions with
multi-indices, Cartesian products and tensor products, respectively.
\newgsymbol{0!}{$\ß0$}{$(0, \dotsc, 0) \in \NN_0^d$}%
\newgsymbol{1!}{$\ß1$}{$(1, \dotsc, 1) \in \NN^d$}%
\newgsymbol{01!}{$[\ß0, \ß1]$}{%
  Unit hypercube $:= [0, 1]^d$%
}
\newgsymbol{ab!}{$[\ßa, \ßb]$}{%
  Hypercube $:= [a_1, b_1] \times \dotsb \times [a_d, b_d]$%
}
\newgsymbol{l!}{$\ßl$}{Multivariate level $\in \NN_0^d$}%
Therefore, the domain is now $[\ß0, \ß1] := [0, 1]^d$,
which can be partitioned into
$\prod_{t=1}^d (2^{l_t} + 1)$ equally sized hyper-cubes,
where $\ßl = (l_1, \dotsc, l_d) \in \NN_0^d$ is the $d$-dimensional level.
\newgsymbol{xli!}{$\ßx_{\ßl,\ßi}$}{Multivariate grid point $:= \ßi \cdot \ßh_\ßl$}%
\newgsymbol{i!}{$\ßi$}{Multivariate index $= \ß0, \dotsc, \ß2^l$}%
\newgsymbol{hl!}{$\ßh_\ßl$}{Multivariate mesh size $:= \ß2^{-\ßl}$}%
The corners of the hyper-cubes are given by the grid points
\begin{equation}
  \label{eq:gridPointMultivariate}
  \ßx_{\ßl,\ßi} := \ßi \cdot \ßh_\ßl,\quad
  \ßi = \ß0, \dotsc, \ß2^{\ßl}.
\end{equation}
In this thesis, relations and operations with vectors (in bold face)
are to be read coordinate-wise, unless stated otherwise.
Bold-faced numbers like $\ß0$ are defined to be the vector $(0, \dotsc, 0)$
in which every entry is equal to that number.
This allows for a somewhat intuitive and suggestive notation.
For example, \eqref{eq:gridPointMultivariate} is equivalent to
\begin{equation}
  \ßx_{\ßl,\ßi}
  := (i_1 h_{l_1},\; \dotsc,\; i_d h_{l_d}),\quad
  i_t = 0, \dotsc, 2^{l_t},\quad
  t = 1, \dotsc, d,
\end{equation}
with the $d$-dimensional mesh size
$\ßh_\ßl := \ß2^{-\ßl} = (h_{l_1}, \dotsc, h_{l_d})$.
\newgsymbol{phili!}{$\varphi_{\ßl,\ßi}$}{%
  Multivariate hierarchical basis function of level $\ßl$, index $\ßi$%
}%
Again, every grid point is associated with a basis function that is defined
as the tensor product of the univariate functions:%
\footnote{%
  Note that one could employ basis functions of different types in
  each dimension, for example B-splines of different degrees.
  For simplicity, we first restrict ourselves to the case of a single type
  for all dimensions, but we will treat the more general case in
  \todo{insert reference}.%
}
\begin{equation}
  \varphi_{\ßl,\ßi}\colon [\ß0, \ß1] \to \RR,\quad
  \varphi_{\ßl,\ßi}(\ßx)
  := \prod_{t=1}^d \varphi_{l_t,i_t}(x_t).
\end{equation}

\newgsymbol{Vl!}{$V_\ßl$}{Multivariate nodal space of level $\ßl$}%
The multivariate nodal space $V_\ßl$ is defined analogously to
the univariate case:
\begin{equation}
  V_\ßl
  := \spn\{\varphi_{\ßl,\ßi} \mid \ßi = \ß0, \dotsc, \ß2^{\ßl}\}.
\end{equation}
\newgsymbol{phili1!}{$\varphi_{\ßl,\ßi}$}{%
  Multivariate hat function of level $\ßl$, index $\ßi$%
}%
\newgsymbol{Vl1!}{$V_\ßl^1$}{%
  Multivariate nodal piecewise linear space of level $\ßl$%
}%
In the case of hat functions $\varphi_{\ßl,\ßi}^1$,
the nodal space $V_\ßl^1$ is the $d$-linear spline space, i.e.,
the space of all continuous functions
on $[\ß0, \ß1]$ that are piecewise $d$-linear polynomial on
all hyper-cubes
\begin{equation}
  [\ßx_{\ßl,\ßi}, \ßx_{\ßl,\ßi+\ß1}]
  := [x_{l_1,i_1}, x_{l_1,i_1+1}] \times \dotsb \times
  [x_{l_d,i_d}, x_{l_d,i_d+1}],\quad
  \ßi = \ß0, \dotsc, \ß2^\ßl - \ß1.
\end{equation}
We can now interpolate objective functions $f\colon [\ß0, \ß1] \to \RR$
in the nodal space $V_\ßl$.
A function $\tilde{f}_\ßl\colon [\ß0, \ß1] \to \RR$ is called \term{interpolant}
in $V_\ßl$, if
\begin{equation}
  \tilde{f}_\ßl
  = \sum_{\ßi=\ß0}^{\ß2^\ßl} c_\ßi \varphi_{\ßl,\ßi},\quad
  \fa{\ßi = \ß0, \dotsc, \ß2^\ßl}{\tilde{f}_\ßl(\ßx_{\ßl,\ßi}) = f(\ßx_{\ßl,\ßi})},
\end{equation}
where $c_\ßi \in \RR$ and
the sum is over all $\ßi = \ß0, \dotsc, \ß2^\ßl$.
One can prove that the functions
$\varphi_{\ßl,\ßi}$ ($\ßi = \ß0, \dotsc, \ß2^\ßl$)
form a basis of $V_\ßl$, if the univariate functions
$\varphi_{l,i}$ ($i = 0, \dotsc, 2^l$)
form a basis of the univariate nodal space $V_l$.%
\footnote{%
  For example in 2D, assume that $\tilde{f}_{(l_1,l_2)} \equiv 0$.
  Then for all $i_1'$ and $i_2'$, due to the tensor product construction,
  it holds that
  $\sum_{i_1} \big[\sum_{i_2} c_{(i_1,i_2)}
  \varphi_{l_2,i_2}(x_{l_2,i_2'})\big] \varphi_{l_1,i_1}(x_{l_1,i_1'})  = 0$.
  We can now apply the linear independence in 1D twice to
  infer that the bracket vanishes and consequently $c_{(i_1,i_2)}$.
}
\newgsymbol{Omegal!}{$\Omega_\ßl$}{%
  Set of grid points of the full grid of level $\ßl$%
}%
This is equivalent to the statement that the coefficients $c_\ßi \in \RR$
exist for every objective function $f$ and are uniquely determined by
the values at the grid points
\begin{equation}
  \Omega_\ßl
  := \{\ßx_{\ßl,\ßi} \mid \ßi = \ß0, \dotsc, \ß2^{\ßl} - \ß1\}.
\end{equation}

\subsection{Hierarchical Subspaces}

The dimension of the nodal space $V_\ßl$ is given by
\begin{equation}
  \dim V_\ßl
  = |\Omega_\ßl|
  = \prod_{t=1}^d (2^{l_t} + 1).
\end{equation}
\newgsymbol{Omega}{$\Omega(f(x))$}{Big-$\Omega$ Landau notation}%
\newgsymbol{norm1}{$\norm{\cdot}_1$}{1-norm $\norm{\ßx}_1 := \sum_{t=1}^d x_t$}%
In asymptotic terms, this dimension is in $\Omega(2^{\norm{\ßl}_1})$,
where the 1-norm $\norm{\ßl}_1$ of $\ßl$ is defined as
$\norm{\ßl}_1 := \sum_{t=1}^d l_t$.
\newgsymbol{n}{$n$}{Level $\in \natural_0$ of full or sparse grid}
If we choose the same level $n \in \NN_0$ in all dimensions, i.e.,
$\ßl = n \cdot \ß1$, this means that the dimension and the
number of grid points grow at least as fast as $2^{nd}$.
This exponential dependency between $\dim V_n$ and $d$ is known as the
\term{curse of dimensionality}.
The curse makes interpolation on $V_\ßl$ computationally infeasible
for dimensionalities $d > 4$,
as we have would have to calculate and store $\dim V_\ßl$ coefficients $c_\ßi$.%
\footnote{%
  The number of necessary basis evaluations to evaluate the interpolant once
  would not be as large, as most types of basis functions
  (like hat functions $\varphi_{l,i}^1$ and higher-order B-splines)
  are locally supported.%
}

\todo{write}

\subsection{Regular Sparse Grids}

\todo{write}

\subsection{Boundary Treatment}

\todo{write}

\section{Defining Hierarchical B-Splines on Sparse Grids}
\label{sec:sgBspl}

\todo{write}

\subsection{Uniform Hierarchical B-Splines}

\todo{write}

\subsection{Linear Independence and Linear Span}

\todo{write}

\subsection{Modification}

\todo{write}

\subsection{Non-Uniform Hierarchical B-Splines}

\todo{write}

\section{Boundary Behavior of Hierarchical B-Splines}

\todo{write}

\subsection{Problems with Uniform Hierarchical B-Splines}

\todo{write}

\subsection{Hierarchical Not-A-Knot B-Splines}

\todo{write}

\subsection{Modified and Non-Uniform Hierarchical Not-A-Knot B-Splines}

\todo{write}

\cite{zenger91}

\blindmathpaper
