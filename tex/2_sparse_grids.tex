\setdictum{%
  B-splines are not enough!%
}{%
  In a talk at the 2017 SIAM Conference on\\
  Computational Science and Engineering%
}
% SIAM CSE 2017, Minisymposium MS154 "Flooding the Cores--Computing Flooding
% Events on Modern Architecture--Part I of II",
% Craig Michoski, UT Austin,
% "Scaling at Exascale in Blended Isogeometric, Discontinuous
% Galerkin, and Particle-in-Cell Approaches"

\todo{write}

\chapter{Hierarchical B-Splines and Sparse Grids}
\label{sec:sg}

In the following, we consider functions that are defined on the unit interval
$[0, 1]$.
We discretize this domain by splitting it into $2^l$ equidistant segments,
where $l \in \NN$ is the \term{level}.
The resulting $2^l + 1$ \term{grid points} are given by
\begin{equation}
  x_{l,i} := i \cdot h_l,\quad
  i = 0, \dotsc, 2^l,
\end{equation}
where $h_l := 2^{-l}$ is the \term{mesh size}.
Every grid point is associated with a \term{basis function}
\begin{equation}
  \varphi_{l,i}\colon [0, 1] \to \RR.
\end{equation}
In this thesis, we assume $\varphi_{l,i}$ to be arbitrary,
satisfying required assumptions when needed.
However, it helps for both the theory and the intuition to have a
specific example of basis functions in mind.
The most common choice for $\varphi_{l,i}$ are the so-called \term{hat functions}
(linear B-splines), which are defined as
\begin{equation}
  \varphi_{l,i}^1(x)
  := \max(1 - |x/h_l - i|, 0).
\end{equation}
Here and in the following,
the superscript ``1'' is the degree of the linear B-spline and
is not to be read as an exponent.
We will generalize this notation to B-splines $\varphi_{l,i}^p$ of
arbitrary degrees $p$ (see \Cref{sec:sg_bspl}).

The \emph{nodal space} $V_l$ of level $l$
is defined as the linear span of all basis functions
$\varphi_{l,i}$, $i = 0, \dotsc, 2^l$:
\begin{equation}
  V_l := \spn\{\varphi_{l,i} \mid i = 0, \dotsc, 2^l\}.
\end{equation}

\todo{write}

\section{Hierarchical Bases and Sparse Grids}

\todo{write}

\subsection{Hierarchical Subspaces}

\todo{write}

\subsection{Regular Sparse Grids}

\todo{write}

\subsection{Boundary Treatment}

\todo{write}

\section{Defining Hierarchical B-Splines on Sparse Grids}
\label{sec:sg_bspl}

\todo{write}

\subsection{Uniform Hierarchical B-Splines}

\todo{write}

\subsection{Linear Independence and Linear Span}

\todo{write}

\subsection{Modification}

\todo{write}

\subsection{Non-Uniform Hierarchical B-Splines}

\todo{write}

\section{Boundary Behavior of Hierarchical B-Splines}

\todo{write}

\subsection{Problems with Uniform Hierarchical B-Splines}

\todo{write}

\subsection{Hierarchical Not-A-Knot B-Splines}

\todo{write}

\subsection{Modified and Non-Uniform Hierarchical Not-A-Knot B-Splines}

\todo{write}

\cite{zenger91}

\blindmathpaper
