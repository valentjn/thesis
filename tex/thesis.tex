%Warns about outdated packages and missing caption delcarations
%See https://www.ctan.org/pkg/nag
\RequirePackage[l2tabu, orthodox]{nag}
\documentclass[paper=a4,           %
               twoside,            %
               % fontsize=10pt,      %
               % DIV=calc,           %
               headings=small,     %
               bibliography=totoc, %
               listof=totoc,        %
               headsepline,        %
               footsepline,         %
               plainfootsepline
]{scrbook}
\usepackage[utf8]{inputenc}

% Sprachsupport - !!bitte einstellen!!
% last language: Default
\usepackage[ngerman,american]{babel}
% Deutsche Ausarbeitung
% \usepackage[american,ngerman]{babel}

% Selectlanguage is reported not to work, therefore use package options
% \selectlanguage{american}
% \selectlanguage{ngerman}

\makeatletter
\@ifpackageloaded{tex4ht}{\def\iftex4ht{\iftrue}}
{\def\iftex4ht{\iffalse}}
\makeatother


% IfThenElse
\usepackage{ifthen}
%%% Doc: http://mirror.ctan.org/tex-archive/macros/latex/contrib/oberdiek/ifpdf.sty
% command for testing for pdf-creation
\usepackage{ifpdf} %\ifpdf \else \fi

%%% Internal Commands: ----------------------------------------------
\makeatletter
%
\providecommand{\IfPackageLoaded}[2]{\@ifpackageloaded{#1}{#2}{}}
\providecommand{\IfPackageNotLoaded}[2]{\@ifpackageloaded{#1}{}{#2}}
\providecommand{\IfElsePackageLoaded}[3]{\@ifpackageloaded{#1}{#2}{#3}}
%
\newboolean{chapteravailable}%
\setboolean{chapteravailable}{false}%

\ifcsname chapter\endcsname
\setboolean{chapteravailable}{true}%
\else
\setboolean{chapteravailable}{false}%
\fi


\providecommand{\IfChapterDefined}[1]{\ifthenelse{\boolean{chapteravailable}}{#1}{}}%
\providecommand{\IfElseChapterDefined}[2]{\ifthenelse{\boolean{chapteravailable}}{#1}{#2}}%

\providecommand{\IfDefined}[2]{%
  \ifcsname #1\endcsname
  #2 %
  \else
  % do nothing
  \fi
}

\providecommand{\IfElseDefined}[3]{%
  \ifcsname #1\endcsname
  #2 %
  \else
  #3 %
  \fi
}

\providecommand{\IfElseUnDefined}[3]{%
  \ifcsname #1\endcsname
  #3 %
  \else
  #2 %
  \fi
}

%\usepackage{fontspec}
%\setmainfont{univers.ttf}
%\usepackage[sfdefault,type1]{universalis}
%\setmainfont{frutiger}
\usepackage{sansmathfonts}
\usepackage{helvet}
\renewcommand{\familydefault}{\sfdefault}

%
% Check for 'draft' mode - commands.
\newcommand{\IfNotDraft}[1]{\ifx\@draft\@undefined #1 \fi}
\newcommand{\IfNotDraftElse}[2]{\ifx\@draft\@undefined #1 \else #2 \fi}
\newcommand{\IfDraft}[1]{\ifx\@draft\@undefined \else #1 \fi}
%

% Definde frontmatter, mainmatter and backmatter if not defined
\@ifundefined{frontmatter}{%
  \newcommand{\frontmatter}{%
    % In Roemischen Buchstaben nummerieren (i, ii, iii)
    \pagenumbering{roman}
  }
}{}
\@ifundefined{mainmatter}{%
  % scrpage2 benoetigt den folgenden switch
  % wenn \mainmatter definiert ist.
  \newif\if@mainmatter\@mainmattertrue
  \newcommand{\mainmatter}{%
    % -- Seitennummerierung auf Arabische Zahlen zuruecksetzen (1,2,3)
    \pagenumbering{arabic}%
    \setcounter{page}{1}%
  }
}{}
\@ifundefined{backmatter}{%
  \newcommand{\backmatter}{
    % In Roemischen Buchstaben nummerieren (i, ii, iii)
    \pagenumbering{roman}
  }
}{}

% Pakete speichern die spaeter geladen werden sollen
\newcommand{\LoadPackagesNow}{}
\newcommand{\LoadPackageLater}[1]{%
  \g@addto@macro{\LoadPackagesNow}{%
    \usepackage{#1}%
  }%
}

\usepackage{fncychap}
% Code for fancy chapters.
\makeatletter
\ChNameVar{\normalfont\Large\scshape\bfseries} % sets the style for name
\ChNumVar{\normalfont\Large\scshape\bfseries} % sets the style for number
\ChTitleVar{\normalfont\Huge\scshape\color{black}\bfseries}


% Code for fancy chapters.
% \makeatletter
% \ChNameVar{\normalfont\Large\scshape} % sets the style for name
% \ChNumVar{\fontsize{85}{100}\selectfont\sffamily\color{white}}
% \ChTitleVar{\normalfont\Huge\scshape\color{black}\raggedleft}
% \renewcommand{\DOCH}{%
% \setlength{\fboxrule}{\RW}
% \makebox[12.34cm][r]{
% \rotatebox{90}{
% \makebox[2cm]{
% \resizebox{2.5cm}{!}{ %\fboxsep=6pt
% \CNV\FmN{\space\@chapapp}
% }
% }
% }
%   \resizebox{!}{2.5cm}{ \fboxsep=6pt
%   \CNoV\colorbox{black!40}{\thechapter}
% }
% }\par\nobreak
%   \vskip 0\p@}
%   \renewcommand{\DOTI}[1]{%
%   \CTV\FmTi{#1}\par\nobreak
%   \vskip 40\p@}
%   \renewcommand{\DOTIS}[1]{%
%   \CTV\FmTi{#1}\par\nobreak
%   \vskip 40\p@}
% \makeatother



\makeatother
%%% ----------------------------------------------------------------

%% colors
\usepackage[dvipsnames, table]{xcolor}
% Für die finale Fassung empfohlen:
% \usepackage[dvipsnames, table, gray]{xcolor}


% Farben fuer die Links im PDF
\definecolor{pdfurlcolor}{rgb}{0,0,0.6}
\definecolor{pdffilecolor}{rgb}{0.7,0,0}
\definecolor{pdflinkcolor}{rgb}{0,0,0.6}
\definecolor{pdfcitecolor}{rgb}{0,0,0.6}


\usepackage{calc}

% see http://tex.stackexchange.com/a/48549/9075
\usepackage[normalem]{ulem} % use normalem to protect \emph

\usepackage{graphicx}

% Base folder, so there is no need to repeat this over and over again.
\graphicspath{ {figures/} }

% use TODO with \todo{}
\usepackage[color=gray!17]{todonotes}
\let\xtodo\todo
\renewcommand{\todo}[1]{\xtodo[inline,color=black!5]{#1}}
\newcommand{\todofix}[1]{\xtodo[color=red!40]{#1}}

%% fonts
% \usepackage[charter, expert]{mathdesign}
% \usepackage{XCharter}
% \usepackage[scaled=.95]{helvet}
\usepackage[T1]{fontenc} % T1 Schrift Encoding
\usepackage{textcomp}	 % Zusatzliche Symbole (Text Companion font extension)

% math stuff
\usepackage[
centertags, % (default) center tags vertically
% tbtags,    % 'Top-or-bottom tags': For a split equation, place equation numbers level
% with the last (resp. first) line, if numbers are on the right (resp. left).
sumlimits,  %(default) Place the subscripts and superscripts of summation
% symbols above and below
% nosumlimits, % Always place the subscripts and superscripts of summation-type
% symbols to the side, even in displayed equations.
intlimits,  % Like sumlimits, but for integral symbols.
% nointlimits, % (default) Opposite of intlimits.
namelimits, % (default) Like sumlimits, but for certain 'operator names' such as
% det, inf, lim, max, min, that traditionally have subscripts placed underneath
% when they occur in a displayed equation.
% nonamelimits, % Opposite of namelimits.
% leqno,     % Place equation numbers on the left.
reqno,     % Place equation numbers on the right.
% fleqn,     % Position equations at a fixed indent from the left margin
% rather than centered in the text column.
]{amsmath} %
% \SetMathAlphabet{\mathcal}{normal}{OMS}{amsa}{m}{n} %% AMS font for mathcal
\usepackage{amssymb}

%%% Doc: http://mirror.ctan.org/tex-archive/macros/latex/contrib/mh/doc/mathtools.pdf
% Erweitert amsmath und behebt einige Bugs
\usepackage[fixamsmath,disallowspaces]{mathtools}

%%% Doc: http://www.ctan.org/info?id=fixmath
% LaTeX's default style of typesetting mathematics does not comply
% with the International Standards ISO31-0:1992 to ISO31-13:1992
% which indicate that uppercase Greek letters always be typset
% upright, as opposed to italic (even though they usually
% represent variables) and allow for typsetting of variables in a
% boldface italic style (even though the required fonts are
% available). This package ensures that uppercase Greek be typeset
% in italic style, that upright $\Delta$ and $\Omega$ symbols are
% available through the commands \upDelta and \upOmega; and
% provides a new math alphabet \mathbold for boldface
% italic letters, including Greek.
\usepackage{fixmath}

% for theorems, replacement for amsthm
\usepackage[amsmath,hyperref]{ntheorem}
\theorempreskipamount 2ex plus1ex minus0.5ex
\theorempostskipamount 2ex plus1ex minus0.5ex
\theoremstyle{break}
\newtheorem{definition}{Definition}[chapter]
%


%%% Doc: http://mirror.ctan.org/tex-archive/macros/latex/contrib/onlyamsmath/onlyamsmath.dvi
% Warnt bei Benutzung von Befehlen die mit amsmath inkompatibel sind.

% TODO: DM --> den shit brauchmer net
% \usepackage[
% all,
% warning
% ]{onlyamsmath}

%% Tables
% ~~~~~~~~~~~~~~~~~~~~~~~~~~~~~~~~~~~~~~~~~~~~~~~~~~~~~~~~~~~~~~~~~~~~~~~~
% Tables (Tabular)
% ~~~~~~~~~~~~~~~~~~~~~~~~~~~~~~~~~~~~~~~~~~~~~~~~~~~~~~~~~~~~~~~~~~~~~~~~

% Basispaket fuer alle Tabellenfunktionen
% -> wird automatisch durch andere Pakete geladen
% \usepackage{array}
%
% bessere Abstaende innerhalb der Tabelle (Layout))
% -------------------------------------------------
%%% Doc: http://mirror.ctan.org/tex-archive/macros/latex/contrib/booktabs/booktabs.pdf
\usepackage{booktabs}
%
% Farbige Tabellen
% ----------------
% Das Paket colortbl wird inzwischen automatisch durch xcolor geladen
%
% Erweiterte Funktionen innerhalb von Tabellen
% --------------------------------------------
%%% Doc: http://mirror.ctan.org/tex-archive/macros/latex/contrib/multirow/multirow.sty
\usepackage{multirow} % Mehrfachspalten
%
%%% Doc: Documentation inside dtx Package
\usepackage{dcolumn}  % Ausrichtung an Komma oder Punkt


%%% Doc: http://mirror.ctan.org/tex-archive/macros/latex/contrib/supertabular/supertabular.pdf
% \usepackage{supertabular}


%% page setup
\usepackage{setspace}
\setstretch{1.2}
\setkomafont{sectioning}{\normalfont\normalcolor\rmfamily\mdseries}
\setkomafont{descriptionlabel}{}
\typearea[current]{last}


% section titles
\usepackage{titlesec}
\titleformat{\section}{\normalfont\Large\bfseries}{\thesection}{1em}{}
\titleformat{\subsection}{\normalfont\large\bfseries}{\thesubsection}{1em}{}
\titleformat{\subsubsection}{\normalfont\large\bfseries}{\thesubsubsection}{1em}{}


\usepackage{scrpage2}

\IfElseChapterDefined{%
  \pagestyle{scrheadings} % Seite mit Headern
}{
  \pagestyle{scrplain} % Seiten ohne Header
}
% \pagestyle{empty} % Seiten ohne Header
%
% loescht voreingestellte Stile
\clearscrheadings
\clearscrplain
\clearscrheadfoot

\newcommand{\setchapterfooter}[1]{%
  % \def\mydummy{#1}
  % \renewcommand{\chaptermark}[1]{%
  %   \markright{\thechapter~|~\mydummy}{}
  % }
}

\newcommand{\resetchapterfooter}{%
  % \renewcommand{\chaptermark}[1]{%
  %   \markright{\thechapter~|~##1}{}}
}

% \renewcommand{\chaptermark}[1]{%
%   \markright{\thechapter~|~#1}{}}

% \renewcommand{\sectionmark}[1]{%
%   \markleft{\thesection~|~#1}{}}

%
% Was steht wo...
\IfElseChapterDefined{
  \ofoot[\pagemark]{\pagemark}
  \rehead{\rightmark}
  % \refoot{\chaptermark -- \thechapter}
  \lohead{\leftmark}
}{
  \cfoot[\pagemark]{\pagemark} % Mitte unten: Seitenzahlen bei plain
}


% daniels stil für abstract (zusammenfassung / abstract)
\deftripstyle{justpagenums}{}{}{}{}{}{\textbf{\thepage}}


% TODO: modified by DM
\setkomafont{pageheadfoot}{} % \small\sffamily}
\setkomafont{pagenumber}{} % \small\sffamily\bfseries}

% Define commands that don't eat spaces.
\usepackage{xspace}

\usepackage{url} % Setzen von URLs. In Verbindung mit hyperref sind diese auch aktive Links.

%%% Fussnoten/Endnoten ===================================================
%
%%% Doc: http://mirror.ctan.org/tex-archive/macros/latex/contrib/footmisc/footmisc.pdf
%
\usepackage[
bottom,      % Footnotes appear always on bottom. This is necessary
% especially when floats are used
stable,      % Make footnotes stable in section titles
perpage,     % Reset on each page
% para,       % Place footnotes side by side of in one paragraph.
% side,       % Place footnotes in the margin
ragged,      % Use RaggedRight
% norule,     % suppress rule above footnotes
multiple,    % rearrange multiple footnotes intelligent in the text.
% symbol,     % use symbols instead of numbers
]{footmisc}

\interfootnotelinepenalty=10000 % Verhindert das Fortsetzen von
% Fussnoten auf der gegenüberligenden Seite

% ~~~~~~~~~~~~~~~~~~~~~~~~~~~~~~~~~~~~~~~~~~~~~~~~~~~~~~~~~~~~~~~~~~~~~~~~
% Pakete zum Zitieren
% ~~~~~~~~~~~~~~~~~~~~~~~~~~~~~~~~~~~~~~~~~~~~~~~~~~~~~~~~~~~~~~~~~~~~~~~~

% Quotes =================================================================
%% Doc: http://mirror.ctan.org/tex-archive/macros/latex/contrib/csquotes/csquotes.pdf
% Advanced features for clever quotations
\usepackage[%
babel,            % the style of all quotation marks will be adapted
% to the document language as chosen by 'babel'
german=quotes,		% Styles of quotes in each language
english=american,
french=guillemets
]{csquotes}

% ~~~~~~~~~~~~~~~~~~~~~~~~~~~~~~~~~~~~~~~~~~~~~~~~~~~~~~~~~~~~~~~~~~~~~~~~
% figures and placement
% ~~~~~~~~~~~~~~~~~~~~~~~~~~~~~~~~~~~~~~~~~~~~~~~~~~~~~~~~~~~~~~~~~~~~~~~~

%% Bilder und Graphiken ==================================================

%%% Doc: float.div
% floatrow is the successor of float
% \usepackage{floatrow}             % Offers option [H] for float positioning
% !! DOES NOT WORK WITH tex4ht !!
% !! in seldom cases, floatrow causes trouble with biblatex's \citeauthor !!
% --> use predecessor float
\usepackage{float}

%%% Doc: No Documentation
% \usepackage{flafter}          % Floats immer erst nach der Referenz setzen

% Defines a \FloatBarrier command, beyond which floats may not
% pass; useful, for example, to ensure all floats for a section
% appear before the next \section command.
\usepackage[
section		% "\section" command will be redefined with "\FloatBarrier"
]{placeins}
%
%%% Doc: http://mirror.ctan.org/tex-archive/macros/latex/contrib/subfig/subfig.pdf
% Incompatible: loads package capt-of. Loading of 'capt-of' afterwards will fail therefor
% \usepackage{subfig} % Layout wird weiter unten festgelegt !

%%% Bilder von Text Umfliessen lassen : (empfehle wrapfig)
%
%%% Doc: http://mirror.ctan.org/tex-archive/macros/latex/contrib/wrapfig/wrapfig.sty
\usepackage{wrapfig}	        % defines wrapfigure and wrapfloat
% \setlength{\wrapoverhang}{\marginparwidth} % aeerlapp des Bildes ...
% \addtolength{\wrapoverhang}{\marginparsep} % ... in den margin
\setlength{\intextsep}{0.5\baselineskip} % Platz ober- und unterhalb des Bildes
% \intextsep ignoiert bei draft ???
% \setlength{\columnsep}{1em} % Abstand zum Text

%%% Doc: Documentation inside dtx Package
% \usepackage{floatflt}   	  % LaTeX2e Paket von 1996
% [rflt] - Standard float auf der rechten Seite

%%% Doc: http://mirror.ctan.org/tex-archive/macros/latex209/contrib/picins/picins.doc
% \usepackage{picins}          % LaTeX 2.09 Paket von 1992. aber Layout kombatibel
% Olly prefers picins over wrapfig

% Make float placement easier
% See also http://dcwww.camd.dtu.dk/~schiotz/comp/LatexTips/LatexTips.html#figplacement
\renewcommand{\floatpagefraction}{.75} % vorher: .5
\renewcommand{\textfraction}{.1}       % vorher: .2
\renewcommand{\topfraction}{.8}        % vorher: .7
\renewcommand{\bottomfraction}{.5}     % vorher: .3
\setcounter{topnumber}{3}              % vorher: 2
\setcounter{bottomnumber}{2}           % vorher: 1
\setcounter{totalnumber}{5}            % vorher: 3


%%% Doc: http://mirror.ctan.org/tex-archive/macros/latex/contrib/psfrag/pfgguide.pdf
% \usepackage{psfrag}	% Ersetzen von Zeichen in eps Bildern


%%% Doc: http://www.ctan.org/tex-archive/macros/latex/contrib/sidecap/sidecap.pdf
\usepackage[%
%	outercaption,%	(default) caption is placed always on the outside side
%	innercaption,% caption placed on the inner side
%	leftcaption,%  caption placed on the left side
rightcaption,% caption placed on the right side
%	wide,%			caption of float my extend into the margin if necessary
%	margincaption,% caption set into margin
ragged,% caption is set ragged
]{sidecap}

\renewcommand\sidecaptionsep{2em}
% \renewcommand\sidecaptionrelwidth{20}
\sidecaptionvpos{table}{c}
\sidecaptionvpos{figure}{c}

%% Index
\usepackage{makeidx}		% Index
\IfDraft{
  \usepackage{showidx}    % Indizierte Begriffe am Rand (Korrekturlesen)
}

% Listings Paket ------------------------------------------------------
%% Doc: http://mirror.ctan.org/tex-archive/macros/latex/contrib/listings/listings-1.3.pdf
\usepackage{listings}
\lstset{language=XML,
  captionpos=b,
  basicstyle=\small\ttfamily, % Standardschrift
  numbers=left,               % Ort der Zeilennummern
  numberstyle=\tiny,          % Stil der Zeilennummern
  stepnumber=2,               % Abstand zwischen den Zeilennummern
  numbersep=5pt,              % Abstand der Nummern zum Text
  tabsize=2,                  % Groesse von Tabs
  extendedchars=true,         %
  breaklines=true,            % Zeilen werden umbrochen
  % keywordstyle=[1]\textbf,    % Stil der Keywords
  % keywordstyle=[2]\textbf,    %
  % keywordstyle=[3]\textbf,    %
  % keywordstyle=[4]\textbf,    %
  % stringstyle=\color{stringcolor}, % Farbe der Strings, stringcolor derzeit undefiniert
  showspaces=false,           % Leerzeichen anzeigen ?
  showtabs=false,             % Tabs anzeigen ?
  showstringspaces=false      % Leerzeichen in Strings anzeigen ?
}
\lstloadlanguages{% Check Dokumentation for further languages ...
  % [Visual]Basic
  % Pascal
  % C
  % C++
  % XML
  % HTML
}

% Moderner als Listings: minted / pygmentize library
% See https://github.com/gpoore/minted
%
% Derzeit deaktiviert, da
% 1. -shell-esacpe auf jeden Fall benötigt wird
% 2. \setminted Version 2.0 erfordert, die noch nicht in MiKTeX enthalten ist
\iffalse
% \usepackage[cache]{minted}
\usepackage{minted}

% \usemintedstyle{autumn}
% \usemintedstyle{rrt}
% \usemintedstyle{borland}
\usemintedstyle{friendly}
% http://www.jevon.org/wiki/Eclipse_Pygments_Style
% \usemintedstyle{eclipse}

% Zeilennummern innerhalb vom Rand:
\setminted{numbersep=5pt, xleftmargin=12pt}
\fi

%% Inhaltsverzeichnis (Schrift, Aussehen) sowie weitere Verzeichnisse ====
\setcounter{secnumdepth}{3}    % Abbildungsnummerierung mit groesserer Tiefe
\setcounter{tocdepth}{2}       % Inhaltsverzeichnis mit groesserer Tiefe
%

%%% Doc: http://mirror.ctan.org/tex-archive/macros/latex/contrib/microtype/microtype.pdf
% Optischer Randausgleich mit pdfTeX
\ifpdf
\usepackage[%
expansion=true, % better typography, but with much larger PDF file.
protrusion=true,
babel=true % use current language when doing microtyping
]{microtype}
\fi

\newcommand\bhrule{\typeout{------------------------------------------------------------------------------}}
\newcommand\btypeout[1]{\bhrule\typeout{\space #1}\bhrule}

% See http://tex.stackexchange.com/a/83051/9075
% Normally, doesn't work with hyperref, but cleveref fixes that
% Deutsch
% \usepackage[ngerman]{varioref}
% Englisch
\usepackage{varioref}

%%% Doc: http://mirror.ctan.org/tex-archive/macros/latex/contrib/hyperref/doc/manual.pdf
\usepackage[
% Farben fuer die Links
% colorlinks=true,         % Links erhalten Farben statt Kaeten
urlcolor=pdfurlcolor,    % \href{...}{...} external (URL)
filecolor=pdffilecolor,  % \href{...} local file
linkcolor=pdflinkcolor,  %\ref{...} and \pageref{...}
citecolor=pdfcitecolor,  %
% Links
raiselinks=true,			 % calculate real height of the link
breaklinks,              % Links ueberstehen Zeilenumbruch
% backref=none,            % Backlinks im Literaturverzeichnis (section, slide, page, none)
% pagebackref=true,        % Backlinks im Literaturverzeichnis mit Seitenangabe. Jetzt mit biblatex
hypertexnames=false,     %http://tex.stackexchange.com/a/156404/9075
verbose,
hyperindex=true,         % backlinkex index
linktocpage=false,       % Inhaltsverzeichnis verlinkt komplette Ueberschrift
hyperfootnotes=true,     % Links auf Fussnoten
% Bookmarks
bookmarks=true,          % Erzeugung von Bookmarks fuer PDF-Viewer
bookmarksopenlevel=1,    % Gliederungstiefe der Bookmarks
bookmarksopen=true,      % Expandierte Untermenues in Bookmarks
bookmarksnumbered=true,  % Nummerierung der Bookmarks
bookmarkstype=toc,       % Art der Verzeichnisses
% Anchors
plainpages=false,        % Anchors even on plain pages ?
pageanchor=true,         % Pages are linkable
% PDF Informationen
pdftitle={},             % Titel
pdfauthor={},            % Autor
pdfcreator={LaTeX, hyperref, KOMA-Script}, % Ersteller
% pdfproducer={pdfeTeX 1.10b-2.1} %Produzent
pdfdisplaydoctitle=true, % Dokumententitel statt Dateiname im Fenstertitel
pdfstartview=FitH,       % Dokument wird Fit Width geaefnet
pdfpagemode=UseOutlines, % Bookmarks im Viewer anzeigen
pdfpagelabels=true,           % set PDF page labels
pdfpagelayout=TwoPageRight, % zweiseitige Darstellung: ungerade Seiten
% rechts im PDF-Viewer
% pdfpagelayout=SinglePage, % einseitige Darstellung
]{hyperref}

\usepackage{pdfcomment}

% bib stuff
% Biblatex has to be loaded after hyperref
\usepackage[
backend       = biber, %minalphanames only works with biber backend
bibstyle      = numeric,
citestyle     = numeric,
firstinits    = true,
useprefix     = true, %print "von, van, etc.", too.
minnames      = 1,
minalphanames = 3,
maxalphanames = 4,
maxbibnames   = 99,
maxcitenames  = 3,
doi           = false, %source: http://tex.stackexchange.com/a/23118/9075
isbn          = false, %source: http://tex.stackexchange.com/a/23118/9075
backref       = true]{biblatex}

% \DefineBibliographyStrings{ngerman}{
% backrefpage  = {Zitiert auf S\adddot},
% backrefpages = {Zitiert auf S\adddot},
% andothers    = {et\ \addabbrvspace al\adddot},
% % Tipp von http://www.mrunix.de/forums/showthread.php?64665-biblatex-Kann-%DCberschrift-vom-Inhaltsverzeichnis-nicht-%E4ndern&p=293656&viewfull=1#post293656
% bibliography = {Literaturverzeichnis}
% }


%   Reduce size of bib font: http://tex.stackexchange.com/questions/329/how-to-change-font-size-for-bibliography
\renewcommand{\bibfont}{\normalfont\small}

% enable hyperlinked author names when using \citeauthor
% source: http://tex.stackexchange.com/a/75916/9075
\DeclareCiteCommand{\citeauthor}
{\boolfalse{citetracker}%
  \boolfalse{pagetracker}%
  \usebibmacro{prenote}}
{\ifciteindex
  {\indexnames{labelname}}
  {}%
  \printtext[bibhyperref]{\printnames{labelname}}}
{\multicitedelim}
{\usebibmacro{postnote}}

% natbib compatibility
\newcommand{\citep}[1]{\cite{#1}}
\newcommand{\citet}[1]{\citeauthor{#1} \cite{#1}}
% Beginning of sentence - analogous to cleveref - important for names such as "zur Muehlen"
\newcommand{\Citep}[1]{\cite{#1}}
\newcommand{\Citet}[1]{\Citeauthor{#1} \cite{#1}}

% Cleveref für clevere Referenzierungen mittels \cref bzw. \Cref am Satzanfang.
% Deutsch
% \usepackage[ngerman,capitalise,nameinlink,noabbrev]{cleveref}
% \crefname{figure}{Abbildung}{Abbildungen}
% \Crefname{figure}{Abbildung}{Abbildungen}
% Englisch
\usepackage[capitalise,nameinlink,noabbrev]{cleveref}
\crefname{figure}{Fig.}{Figs.}
\Crefname{figure}{Fig.}{Figs.}

\crefname{chapter}{chapter}{chapters}
\Crefname{chapter}{Chapter}{Chapters}
\crefname{section}{section}{sections}
\Crefname{section}{Section}{Sections}
\crefname{subsection}{subsection}{subsections}
\Crefname{subsection}{Subsection}{Subsections}


%%%
% Ermoeglicht es, Abbildungen um 90 Grad zu drehen
% Alternatives Paket: rotating Allerdings wird hier nur das Bild gedreht, während bei lscape auch die PDF-Seite gedreht wird.
% Das Paket lscape dreht die Seite auch nicht
\usepackage{pdflscape}
%
%%%

% For making subfigures
% Part of the caption package. See http://www.ctan.org/pkg/caption
%
% subfigure is outdated. subfig is maintained, but subcaption is better.
% See: http://tex.stackexchange.com/questions/13625/subcaption-vs-subfig-best-package-for-referencing-a-subfigure
% \usepackage{subcaption}

\usepackage[chapter]{algorithm}
\usepackage{algpseudocode}
\floatname{algorithm}{Algorithm}
\renewcommand{\listalgorithmname}{List of Algorithms}
\usepackage{algorithmicx}

\usepackage{minted}

\floatstyle{ruled} %%???

\usepackage{layout} % use with \layout, print page dimensions
\usepackage{blindtext}

%% PDF stuff

\setlength{\pdfpageheight}{\paperheight}
\setlength{\pdfpagewidth}{\paperwidth}

%%% Doc: http://mirror.ctan.org/tex-archive/macros/latex/contrib/oberdiek/hypcap.pdf
% Links auf Gleitumgebungen springen nicht zur Beschriftung,
% sondern zum Anfang der Gleitumgebung
\IfPackageLoaded{hyperref}{%
  \usepackage[figure,table]{hypcap}
}



%%% Doc: http://mirror.ctan.org/tex-archive/macros/latex/contrib/pdfpages/pdfpages.pdf
\usepackage{pdfpages} % Include pages from external PDF documents in LaTeX documents

%%% Neue Tabellen-Umgebungen:
% ---------------------------
% Spalten automatischer Breite:
%%% Doc: Documentation inside dtx Package
\usepackage{tabularx}
% -> nach hyperref Laden
% -> wird von ltxtable geladen


% Tabellen ueber mehere Seiten
% ----------------------------
%%% Doc: http://mirror.ctan.org/tex-archive/macros/latex/contrib/carlisle/ltxtable.pdf
\usepackage{ltxtable} % Longtable + tabularx
% (multi-page tables) + (auto-sized columns in a fixed width table)
% -> nach hyperref laden

\raggedbottom     % Variable Seitenhoehen zulassen


% DM: line-breaking-description env vom daniel w.

% credit goes to daniel w. :-)
%% --- Descriptions with line breaks in labels ---------------------------------
\usepackage{calc}

\newcommand*\Descriptionlabel[1]
{\raisebox{0pt}[1ex][0pt]
  {\makebox[\labelwidth][1]
    {\parbox[t]{\labelwidth}
      {\hspace{0pt}\textbf{#1:}}}}}

\newcommand*\Descriptionlabelx[1]
{\parbox[t]{\textwidth}
  {\textbf{#1}\\\mbox{}}
}

\newenvironment{Description}
{\begin{list}{}{\let\makelabel\Descriptionlabelx
      \setlength\labelwidth{1em}
      \setlength\leftmargin{\labelwidth+\labelsep}}}
  {\end{list}}


% globally change line spacing of lists
% paralist has suspended development since 10 years.
% enumitem has been updated 2011-09-28
\usepackage[inline]{enumitem}
\setlist{partopsep=0pt,itemsep=1pt}

% ------------------------------------------------------------------------
% Fixes wrong spacing in the TOC
\usepackage[tocgraduated]{tocstyle}
\usetocstyle{standard}

% Enables named references, where the name is typeset
% Available on ctan at http://www.ctan.org/pkg/refenums and in latest MiKTeX and TeXLive
\usepackage{refenums}

% introduce \powerset - hint by http://matheplanet.com/matheplanet/nuke/html/viewtopic.php?topic=136492&post_id=997377
\DeclareFontFamily{U}{MnSymbolC}{}
\DeclareSymbolFont{MnSyC}{U}{MnSymbolC}{m}{n}
\DeclareFontShape{U}{MnSymbolC}{m}{n}{
  <-6>  MnSymbolC5
  <6-7>  MnSymbolC6
  <7-8>  MnSymbolC7
  <8-9>  MnSymbolC8
  <9-10> MnSymbolC9
  <10-12> MnSymbolC10
  <12->   MnSymbolC12%
}{}
\DeclareMathSymbol{\powerset}{\mathord}{MnSyC}{180}

% nomenclature
\usepackage{nomencl}

% ------------------------------------------------------------------------
% set size of the footer
\usepackage[bottom=12em]{geometry}

% ------------------------------------------------------------------------
% environment to add "inspirational" citation under each chapter
\makeatletter
\newenvironment{chapquote}[2][2em]
  {\setlength{\@tempdima}{#1}%
   \def\chapquote@author{#2}%
   \parshape 1 \@tempdima \dimexpr\textwidth-2\@tempdima\relax%
   \itshape}
  {\par\normalfont\hfill--\ \chapquote@author\hspace*{\@tempdima}\par\bigskip}
\makeatother

\usepackage{preambel/titlepage}

% \newcommand with variable command names
% (https://tex.stackexchange.com/a/317094)
\makeatletter
\newcommand*{\newnamecommand}{\@star@or@long\new@name@command}
\newcommand*{\new@name@command}[1]{\expandafter\new@command\csname #1\endcsname}
\makeatother

% black-board letters
% define \bbN and \NN (N = arbitrary letter)
\newcommand*{\defbb}[1]{%
  \newnamecommand{bb#1}{\mathbb{#1}}%
  \newnamecommand{#1#1}{\mathbb{#1}}%
}
\defbb{N}
\defbb{R}
\defbb{Z}

% calligraphic letters
% define \calC (C = arbitrary letter)
\newcommand*{\defcal}[1]{\newnamecommand{cal#1}{\mathcal{#1}}}
\defcal{C}
\defcal{O}

% defined terms
\newcommand*{\term}[1]{\emph{#1}}

% mathematic operators
\DeclareMathOperator{\spn}{span}
\DeclareMathOperator{\xor}{xor}

% subsets
\renewcommand*{\subset}{\subseteq}

% differential for integral/derivatives
\newcommand*{\diff}{\mathop{}\!\mathrm{d}}
\newcommand*{\dx}{\diff{}x}

% vectors
\renewcommand*{\vec}[1]{{\boldsymbol{#1}}}
\newcommand*{\ß}[1]{\vec{#1}}
\newcommand*{\vecmax}{\mathop{\vec{\max}}}

% quantors
\newcommand*{\fa}[2]{\forall_{#1}\;\;#2}
\newcommand*{\ex}[2]{\exists_{#1}\;\;#2}

% norm
\newcommand*{\norm}[1]{\lVert{#1}\rVert}
\newcommand*{\bignorm}[1]{\left\lVert{#1}\right\rVert}

% superscripts
\newcommand*{\modified}{\mathrm{mod}}
\newcommand*{\sparse}{\mathrm{s}}

% disjoint union
\newcommand*{\dotcup}{\mathbin{\dot{\cup}}}
\DeclareMathOperator*{\bigdotcup}{\dot{\bigcup}}

% interior and boundary
\newcommand*{\interior}[1]{\mathring{#1}}
\newcommand*{\bndry}[1]{\mathop{}\!\partial#1}

% open interval
\newcommand*{\openinterval}[1]{\mathopen]#1\mathclose[}

\bibliography{bibliography}

% nomenclature
\renewcommand{\nomgroup}[1]{
  \ifthenelse{\equal{#1}{S}}{\item[\bfseries\sffamily\large{List of Symbols}]}{
    \ifthenelse{\equal{#1}{A}}{\item[\bfseries\sffamily\large{List of Abbreviations}]}{}}
}
\makenomenclature

\begin{document}

%Tipp von http://goemonx.blogspot.de/2012/01/pdflatex-ligaturen-und-copynpaste.html
%siehe auch http://tex.stackexchange.com/questions/4397/make-ligatures-in-linux-libertine-copyable-and-searchable
%
%ONLY WORKS ON MiKTeX
%On other systems, download glyphtounicode.tex from http://pdftex.sarovar.org/misc/
%
% this file was converted from the following files:
%   - glyphlist.txt       (Adobe Glyph List v2.0)
%   - zapfdingbats.txt    (ITC Zapf Dingbats Glyph List)
%   - texglyphlist.txt    (lcdf-typetools texglyphlist.txt, v2.33)
%   - additional.tex      (additional entries)
%
% Notes:
% - entries containing duplicates in glyphlist.txt like
%   'dalethatafpatah;05D3 05B2' are ignored (commented out)
%
% - entries containing duplicates in texglyphlist.txt like
%   'angbracketleft;27E8,2329' are changed so that only the first
%   choice remains, ie 'angbracketleft;27E8'
%
% - a few entries in texglyphlist.txt like Delta, Omega, etc. are
%   commented out (they are already in glyphlist.txt)

% entries from glyphlist.txt:
\pdfglyphtounicode{A}{0041}
\pdfglyphtounicode{AE}{00C6}
\pdfglyphtounicode{AEacute}{01FC}
\pdfglyphtounicode{AEmacron}{01E2}
\pdfglyphtounicode{AEsmall}{F7E6}
\pdfglyphtounicode{Aacute}{00C1}
\pdfglyphtounicode{Aacutesmall}{F7E1}
\pdfglyphtounicode{Abreve}{0102}
\pdfglyphtounicode{Abreveacute}{1EAE}
\pdfglyphtounicode{Abrevecyrillic}{04D0}
\pdfglyphtounicode{Abrevedotbelow}{1EB6}
\pdfglyphtounicode{Abrevegrave}{1EB0}
\pdfglyphtounicode{Abrevehookabove}{1EB2}
\pdfglyphtounicode{Abrevetilde}{1EB4}
\pdfglyphtounicode{Acaron}{01CD}
\pdfglyphtounicode{Acircle}{24B6}
\pdfglyphtounicode{Acircumflex}{00C2}
\pdfglyphtounicode{Acircumflexacute}{1EA4}
\pdfglyphtounicode{Acircumflexdotbelow}{1EAC}
\pdfglyphtounicode{Acircumflexgrave}{1EA6}
\pdfglyphtounicode{Acircumflexhookabove}{1EA8}
\pdfglyphtounicode{Acircumflexsmall}{F7E2}
\pdfglyphtounicode{Acircumflextilde}{1EAA}
\pdfglyphtounicode{Acute}{F6C9}
\pdfglyphtounicode{Acutesmall}{F7B4}
\pdfglyphtounicode{Acyrillic}{0410}
\pdfglyphtounicode{Adblgrave}{0200}
\pdfglyphtounicode{Adieresis}{00C4}
\pdfglyphtounicode{Adieresiscyrillic}{04D2}
\pdfglyphtounicode{Adieresismacron}{01DE}
\pdfglyphtounicode{Adieresissmall}{F7E4}
\pdfglyphtounicode{Adotbelow}{1EA0}
\pdfglyphtounicode{Adotmacron}{01E0}
\pdfglyphtounicode{Agrave}{00C0}
\pdfglyphtounicode{Agravesmall}{F7E0}
\pdfglyphtounicode{Ahookabove}{1EA2}
\pdfglyphtounicode{Aiecyrillic}{04D4}
\pdfglyphtounicode{Ainvertedbreve}{0202}
\pdfglyphtounicode{Alpha}{0391}
\pdfglyphtounicode{Alphatonos}{0386}
\pdfglyphtounicode{Amacron}{0100}
\pdfglyphtounicode{Amonospace}{FF21}
\pdfglyphtounicode{Aogonek}{0104}
\pdfglyphtounicode{Aring}{00C5}
\pdfglyphtounicode{Aringacute}{01FA}
\pdfglyphtounicode{Aringbelow}{1E00}
\pdfglyphtounicode{Aringsmall}{F7E5}
\pdfglyphtounicode{Asmall}{F761}
\pdfglyphtounicode{Atilde}{00C3}
\pdfglyphtounicode{Atildesmall}{F7E3}
\pdfglyphtounicode{Aybarmenian}{0531}
\pdfglyphtounicode{B}{0042}
\pdfglyphtounicode{Bcircle}{24B7}
\pdfglyphtounicode{Bdotaccent}{1E02}
\pdfglyphtounicode{Bdotbelow}{1E04}
\pdfglyphtounicode{Becyrillic}{0411}
\pdfglyphtounicode{Benarmenian}{0532}
\pdfglyphtounicode{Beta}{0392}
\pdfglyphtounicode{Bhook}{0181}
\pdfglyphtounicode{Blinebelow}{1E06}
\pdfglyphtounicode{Bmonospace}{FF22}
\pdfglyphtounicode{Brevesmall}{F6F4}
\pdfglyphtounicode{Bsmall}{F762}
\pdfglyphtounicode{Btopbar}{0182}
\pdfglyphtounicode{C}{0043}
\pdfglyphtounicode{Caarmenian}{053E}
\pdfglyphtounicode{Cacute}{0106}
\pdfglyphtounicode{Caron}{F6CA}
\pdfglyphtounicode{Caronsmall}{F6F5}
\pdfglyphtounicode{Ccaron}{010C}
\pdfglyphtounicode{Ccedilla}{00C7}
\pdfglyphtounicode{Ccedillaacute}{1E08}
\pdfglyphtounicode{Ccedillasmall}{F7E7}
\pdfglyphtounicode{Ccircle}{24B8}
\pdfglyphtounicode{Ccircumflex}{0108}
\pdfglyphtounicode{Cdot}{010A}
\pdfglyphtounicode{Cdotaccent}{010A}
\pdfglyphtounicode{Cedillasmall}{F7B8}
\pdfglyphtounicode{Chaarmenian}{0549}
\pdfglyphtounicode{Cheabkhasiancyrillic}{04BC}
\pdfglyphtounicode{Checyrillic}{0427}
\pdfglyphtounicode{Chedescenderabkhasiancyrillic}{04BE}
\pdfglyphtounicode{Chedescendercyrillic}{04B6}
\pdfglyphtounicode{Chedieresiscyrillic}{04F4}
\pdfglyphtounicode{Cheharmenian}{0543}
\pdfglyphtounicode{Chekhakassiancyrillic}{04CB}
\pdfglyphtounicode{Cheverticalstrokecyrillic}{04B8}
\pdfglyphtounicode{Chi}{03A7}
\pdfglyphtounicode{Chook}{0187}
\pdfglyphtounicode{Circumflexsmall}{F6F6}
\pdfglyphtounicode{Cmonospace}{FF23}
\pdfglyphtounicode{Coarmenian}{0551}
\pdfglyphtounicode{Csmall}{F763}
\pdfglyphtounicode{D}{0044}
\pdfglyphtounicode{DZ}{01F1}
\pdfglyphtounicode{DZcaron}{01C4}
\pdfglyphtounicode{Daarmenian}{0534}
\pdfglyphtounicode{Dafrican}{0189}
\pdfglyphtounicode{Dcaron}{010E}
\pdfglyphtounicode{Dcedilla}{1E10}
\pdfglyphtounicode{Dcircle}{24B9}
\pdfglyphtounicode{Dcircumflexbelow}{1E12}
\pdfglyphtounicode{Dcroat}{0110}
\pdfglyphtounicode{Ddotaccent}{1E0A}
\pdfglyphtounicode{Ddotbelow}{1E0C}
\pdfglyphtounicode{Decyrillic}{0414}
\pdfglyphtounicode{Deicoptic}{03EE}
\pdfglyphtounicode{Delta}{2206}
\pdfglyphtounicode{Deltagreek}{0394}
\pdfglyphtounicode{Dhook}{018A}
\pdfglyphtounicode{Dieresis}{F6CB}
\pdfglyphtounicode{DieresisAcute}{F6CC}
\pdfglyphtounicode{DieresisGrave}{F6CD}
\pdfglyphtounicode{Dieresissmall}{F7A8}
\pdfglyphtounicode{Digammagreek}{03DC}
\pdfglyphtounicode{Djecyrillic}{0402}
\pdfglyphtounicode{Dlinebelow}{1E0E}
\pdfglyphtounicode{Dmonospace}{FF24}
\pdfglyphtounicode{Dotaccentsmall}{F6F7}
\pdfglyphtounicode{Dslash}{0110}
\pdfglyphtounicode{Dsmall}{F764}
\pdfglyphtounicode{Dtopbar}{018B}
\pdfglyphtounicode{Dz}{01F2}
\pdfglyphtounicode{Dzcaron}{01C5}
\pdfglyphtounicode{Dzeabkhasiancyrillic}{04E0}
\pdfglyphtounicode{Dzecyrillic}{0405}
\pdfglyphtounicode{Dzhecyrillic}{040F}
\pdfglyphtounicode{E}{0045}
\pdfglyphtounicode{Eacute}{00C9}
\pdfglyphtounicode{Eacutesmall}{F7E9}
\pdfglyphtounicode{Ebreve}{0114}
\pdfglyphtounicode{Ecaron}{011A}
\pdfglyphtounicode{Ecedillabreve}{1E1C}
\pdfglyphtounicode{Echarmenian}{0535}
\pdfglyphtounicode{Ecircle}{24BA}
\pdfglyphtounicode{Ecircumflex}{00CA}
\pdfglyphtounicode{Ecircumflexacute}{1EBE}
\pdfglyphtounicode{Ecircumflexbelow}{1E18}
\pdfglyphtounicode{Ecircumflexdotbelow}{1EC6}
\pdfglyphtounicode{Ecircumflexgrave}{1EC0}
\pdfglyphtounicode{Ecircumflexhookabove}{1EC2}
\pdfglyphtounicode{Ecircumflexsmall}{F7EA}
\pdfglyphtounicode{Ecircumflextilde}{1EC4}
\pdfglyphtounicode{Ecyrillic}{0404}
\pdfglyphtounicode{Edblgrave}{0204}
\pdfglyphtounicode{Edieresis}{00CB}
\pdfglyphtounicode{Edieresissmall}{F7EB}
\pdfglyphtounicode{Edot}{0116}
\pdfglyphtounicode{Edotaccent}{0116}
\pdfglyphtounicode{Edotbelow}{1EB8}
\pdfglyphtounicode{Efcyrillic}{0424}
\pdfglyphtounicode{Egrave}{00C8}
\pdfglyphtounicode{Egravesmall}{F7E8}
\pdfglyphtounicode{Eharmenian}{0537}
\pdfglyphtounicode{Ehookabove}{1EBA}
\pdfglyphtounicode{Eightroman}{2167}
\pdfglyphtounicode{Einvertedbreve}{0206}
\pdfglyphtounicode{Eiotifiedcyrillic}{0464}
\pdfglyphtounicode{Elcyrillic}{041B}
\pdfglyphtounicode{Elevenroman}{216A}
\pdfglyphtounicode{Emacron}{0112}
\pdfglyphtounicode{Emacronacute}{1E16}
\pdfglyphtounicode{Emacrongrave}{1E14}
\pdfglyphtounicode{Emcyrillic}{041C}
\pdfglyphtounicode{Emonospace}{FF25}
\pdfglyphtounicode{Encyrillic}{041D}
\pdfglyphtounicode{Endescendercyrillic}{04A2}
\pdfglyphtounicode{Eng}{014A}
\pdfglyphtounicode{Enghecyrillic}{04A4}
\pdfglyphtounicode{Enhookcyrillic}{04C7}
\pdfglyphtounicode{Eogonek}{0118}
\pdfglyphtounicode{Eopen}{0190}
\pdfglyphtounicode{Epsilon}{0395}
\pdfglyphtounicode{Epsilontonos}{0388}
\pdfglyphtounicode{Ercyrillic}{0420}
\pdfglyphtounicode{Ereversed}{018E}
\pdfglyphtounicode{Ereversedcyrillic}{042D}
\pdfglyphtounicode{Escyrillic}{0421}
\pdfglyphtounicode{Esdescendercyrillic}{04AA}
\pdfglyphtounicode{Esh}{01A9}
\pdfglyphtounicode{Esmall}{F765}
\pdfglyphtounicode{Eta}{0397}
\pdfglyphtounicode{Etarmenian}{0538}
\pdfglyphtounicode{Etatonos}{0389}
\pdfglyphtounicode{Eth}{00D0}
\pdfglyphtounicode{Ethsmall}{F7F0}
\pdfglyphtounicode{Etilde}{1EBC}
\pdfglyphtounicode{Etildebelow}{1E1A}
\pdfglyphtounicode{Euro}{20AC}
\pdfglyphtounicode{Ezh}{01B7}
\pdfglyphtounicode{Ezhcaron}{01EE}
\pdfglyphtounicode{Ezhreversed}{01B8}
\pdfglyphtounicode{F}{0046}
\pdfglyphtounicode{Fcircle}{24BB}
\pdfglyphtounicode{Fdotaccent}{1E1E}
\pdfglyphtounicode{Feharmenian}{0556}
\pdfglyphtounicode{Feicoptic}{03E4}
\pdfglyphtounicode{Fhook}{0191}
\pdfglyphtounicode{Fitacyrillic}{0472}
\pdfglyphtounicode{Fiveroman}{2164}
\pdfglyphtounicode{Fmonospace}{FF26}
\pdfglyphtounicode{Fourroman}{2163}
\pdfglyphtounicode{Fsmall}{F766}
\pdfglyphtounicode{G}{0047}
\pdfglyphtounicode{GBsquare}{3387}
\pdfglyphtounicode{Gacute}{01F4}
\pdfglyphtounicode{Gamma}{0393}
\pdfglyphtounicode{Gammaafrican}{0194}
\pdfglyphtounicode{Gangiacoptic}{03EA}
\pdfglyphtounicode{Gbreve}{011E}
\pdfglyphtounicode{Gcaron}{01E6}
\pdfglyphtounicode{Gcedilla}{0122}
\pdfglyphtounicode{Gcircle}{24BC}
\pdfglyphtounicode{Gcircumflex}{011C}
\pdfglyphtounicode{Gcommaaccent}{0122}
\pdfglyphtounicode{Gdot}{0120}
\pdfglyphtounicode{Gdotaccent}{0120}
\pdfglyphtounicode{Gecyrillic}{0413}
\pdfglyphtounicode{Ghadarmenian}{0542}
\pdfglyphtounicode{Ghemiddlehookcyrillic}{0494}
\pdfglyphtounicode{Ghestrokecyrillic}{0492}
\pdfglyphtounicode{Gheupturncyrillic}{0490}
\pdfglyphtounicode{Ghook}{0193}
\pdfglyphtounicode{Gimarmenian}{0533}
\pdfglyphtounicode{Gjecyrillic}{0403}
\pdfglyphtounicode{Gmacron}{1E20}
\pdfglyphtounicode{Gmonospace}{FF27}
\pdfglyphtounicode{Grave}{F6CE}
\pdfglyphtounicode{Gravesmall}{F760}
\pdfglyphtounicode{Gsmall}{F767}
\pdfglyphtounicode{Gsmallhook}{029B}
\pdfglyphtounicode{Gstroke}{01E4}
\pdfglyphtounicode{H}{0048}
\pdfglyphtounicode{H18533}{25CF}
\pdfglyphtounicode{H18543}{25AA}
\pdfglyphtounicode{H18551}{25AB}
\pdfglyphtounicode{H22073}{25A1}
\pdfglyphtounicode{HPsquare}{33CB}
\pdfglyphtounicode{Haabkhasiancyrillic}{04A8}
\pdfglyphtounicode{Hadescendercyrillic}{04B2}
\pdfglyphtounicode{Hardsigncyrillic}{042A}
\pdfglyphtounicode{Hbar}{0126}
\pdfglyphtounicode{Hbrevebelow}{1E2A}
\pdfglyphtounicode{Hcedilla}{1E28}
\pdfglyphtounicode{Hcircle}{24BD}
\pdfglyphtounicode{Hcircumflex}{0124}
\pdfglyphtounicode{Hdieresis}{1E26}
\pdfglyphtounicode{Hdotaccent}{1E22}
\pdfglyphtounicode{Hdotbelow}{1E24}
\pdfglyphtounicode{Hmonospace}{FF28}
\pdfglyphtounicode{Hoarmenian}{0540}
\pdfglyphtounicode{Horicoptic}{03E8}
\pdfglyphtounicode{Hsmall}{F768}
\pdfglyphtounicode{Hungarumlaut}{F6CF}
\pdfglyphtounicode{Hungarumlautsmall}{F6F8}
\pdfglyphtounicode{Hzsquare}{3390}
\pdfglyphtounicode{I}{0049}
\pdfglyphtounicode{IAcyrillic}{042F}
\pdfglyphtounicode{IJ}{0132}
\pdfglyphtounicode{IUcyrillic}{042E}
\pdfglyphtounicode{Iacute}{00CD}
\pdfglyphtounicode{Iacutesmall}{F7ED}
\pdfglyphtounicode{Ibreve}{012C}
\pdfglyphtounicode{Icaron}{01CF}
\pdfglyphtounicode{Icircle}{24BE}
\pdfglyphtounicode{Icircumflex}{00CE}
\pdfglyphtounicode{Icircumflexsmall}{F7EE}
\pdfglyphtounicode{Icyrillic}{0406}
\pdfglyphtounicode{Idblgrave}{0208}
\pdfglyphtounicode{Idieresis}{00CF}
\pdfglyphtounicode{Idieresisacute}{1E2E}
\pdfglyphtounicode{Idieresiscyrillic}{04E4}
\pdfglyphtounicode{Idieresissmall}{F7EF}
\pdfglyphtounicode{Idot}{0130}
\pdfglyphtounicode{Idotaccent}{0130}
\pdfglyphtounicode{Idotbelow}{1ECA}
\pdfglyphtounicode{Iebrevecyrillic}{04D6}
\pdfglyphtounicode{Iecyrillic}{0415}
\pdfglyphtounicode{Ifraktur}{2111}
\pdfglyphtounicode{Igrave}{00CC}
\pdfglyphtounicode{Igravesmall}{F7EC}
\pdfglyphtounicode{Ihookabove}{1EC8}
\pdfglyphtounicode{Iicyrillic}{0418}
\pdfglyphtounicode{Iinvertedbreve}{020A}
\pdfglyphtounicode{Iishortcyrillic}{0419}
\pdfglyphtounicode{Imacron}{012A}
\pdfglyphtounicode{Imacroncyrillic}{04E2}
\pdfglyphtounicode{Imonospace}{FF29}
\pdfglyphtounicode{Iniarmenian}{053B}
\pdfglyphtounicode{Iocyrillic}{0401}
\pdfglyphtounicode{Iogonek}{012E}
\pdfglyphtounicode{Iota}{0399}
\pdfglyphtounicode{Iotaafrican}{0196}
\pdfglyphtounicode{Iotadieresis}{03AA}
\pdfglyphtounicode{Iotatonos}{038A}
\pdfglyphtounicode{Ismall}{F769}
\pdfglyphtounicode{Istroke}{0197}
\pdfglyphtounicode{Itilde}{0128}
\pdfglyphtounicode{Itildebelow}{1E2C}
\pdfglyphtounicode{Izhitsacyrillic}{0474}
\pdfglyphtounicode{Izhitsadblgravecyrillic}{0476}
\pdfglyphtounicode{J}{004A}
\pdfglyphtounicode{Jaarmenian}{0541}
\pdfglyphtounicode{Jcircle}{24BF}
\pdfglyphtounicode{Jcircumflex}{0134}
\pdfglyphtounicode{Jecyrillic}{0408}
\pdfglyphtounicode{Jheharmenian}{054B}
\pdfglyphtounicode{Jmonospace}{FF2A}
\pdfglyphtounicode{Jsmall}{F76A}
\pdfglyphtounicode{K}{004B}
\pdfglyphtounicode{KBsquare}{3385}
\pdfglyphtounicode{KKsquare}{33CD}
\pdfglyphtounicode{Kabashkircyrillic}{04A0}
\pdfglyphtounicode{Kacute}{1E30}
\pdfglyphtounicode{Kacyrillic}{041A}
\pdfglyphtounicode{Kadescendercyrillic}{049A}
\pdfglyphtounicode{Kahookcyrillic}{04C3}
\pdfglyphtounicode{Kappa}{039A}
\pdfglyphtounicode{Kastrokecyrillic}{049E}
\pdfglyphtounicode{Kaverticalstrokecyrillic}{049C}
\pdfglyphtounicode{Kcaron}{01E8}
\pdfglyphtounicode{Kcedilla}{0136}
\pdfglyphtounicode{Kcircle}{24C0}
\pdfglyphtounicode{Kcommaaccent}{0136}
\pdfglyphtounicode{Kdotbelow}{1E32}
\pdfglyphtounicode{Keharmenian}{0554}
\pdfglyphtounicode{Kenarmenian}{053F}
\pdfglyphtounicode{Khacyrillic}{0425}
\pdfglyphtounicode{Kheicoptic}{03E6}
\pdfglyphtounicode{Khook}{0198}
\pdfglyphtounicode{Kjecyrillic}{040C}
\pdfglyphtounicode{Klinebelow}{1E34}
\pdfglyphtounicode{Kmonospace}{FF2B}
\pdfglyphtounicode{Koppacyrillic}{0480}
\pdfglyphtounicode{Koppagreek}{03DE}
\pdfglyphtounicode{Ksicyrillic}{046E}
\pdfglyphtounicode{Ksmall}{F76B}
\pdfglyphtounicode{L}{004C}
\pdfglyphtounicode{LJ}{01C7}
\pdfglyphtounicode{LL}{F6BF}
\pdfglyphtounicode{Lacute}{0139}
\pdfglyphtounicode{Lambda}{039B}
\pdfglyphtounicode{Lcaron}{013D}
\pdfglyphtounicode{Lcedilla}{013B}
\pdfglyphtounicode{Lcircle}{24C1}
\pdfglyphtounicode{Lcircumflexbelow}{1E3C}
\pdfglyphtounicode{Lcommaaccent}{013B}
\pdfglyphtounicode{Ldot}{013F}
\pdfglyphtounicode{Ldotaccent}{013F}
\pdfglyphtounicode{Ldotbelow}{1E36}
\pdfglyphtounicode{Ldotbelowmacron}{1E38}
\pdfglyphtounicode{Liwnarmenian}{053C}
\pdfglyphtounicode{Lj}{01C8}
\pdfglyphtounicode{Ljecyrillic}{0409}
\pdfglyphtounicode{Llinebelow}{1E3A}
\pdfglyphtounicode{Lmonospace}{FF2C}
\pdfglyphtounicode{Lslash}{0141}
\pdfglyphtounicode{Lslashsmall}{F6F9}
\pdfglyphtounicode{Lsmall}{F76C}
\pdfglyphtounicode{M}{004D}
\pdfglyphtounicode{MBsquare}{3386}
\pdfglyphtounicode{Macron}{F6D0}
\pdfglyphtounicode{Macronsmall}{F7AF}
\pdfglyphtounicode{Macute}{1E3E}
\pdfglyphtounicode{Mcircle}{24C2}
\pdfglyphtounicode{Mdotaccent}{1E40}
\pdfglyphtounicode{Mdotbelow}{1E42}
\pdfglyphtounicode{Menarmenian}{0544}
\pdfglyphtounicode{Mmonospace}{FF2D}
\pdfglyphtounicode{Msmall}{F76D}
\pdfglyphtounicode{Mturned}{019C}
\pdfglyphtounicode{Mu}{039C}
\pdfglyphtounicode{N}{004E}
\pdfglyphtounicode{NJ}{01CA}
\pdfglyphtounicode{Nacute}{0143}
\pdfglyphtounicode{Ncaron}{0147}
\pdfglyphtounicode{Ncedilla}{0145}
\pdfglyphtounicode{Ncircle}{24C3}
\pdfglyphtounicode{Ncircumflexbelow}{1E4A}
\pdfglyphtounicode{Ncommaaccent}{0145}
\pdfglyphtounicode{Ndotaccent}{1E44}
\pdfglyphtounicode{Ndotbelow}{1E46}
\pdfglyphtounicode{Nhookleft}{019D}
\pdfglyphtounicode{Nineroman}{2168}
\pdfglyphtounicode{Nj}{01CB}
\pdfglyphtounicode{Njecyrillic}{040A}
\pdfglyphtounicode{Nlinebelow}{1E48}
\pdfglyphtounicode{Nmonospace}{FF2E}
\pdfglyphtounicode{Nowarmenian}{0546}
\pdfglyphtounicode{Nsmall}{F76E}
\pdfglyphtounicode{Ntilde}{00D1}
\pdfglyphtounicode{Ntildesmall}{F7F1}
\pdfglyphtounicode{Nu}{039D}
\pdfglyphtounicode{O}{004F}
\pdfglyphtounicode{OE}{0152}
\pdfglyphtounicode{OEsmall}{F6FA}
\pdfglyphtounicode{Oacute}{00D3}
\pdfglyphtounicode{Oacutesmall}{F7F3}
\pdfglyphtounicode{Obarredcyrillic}{04E8}
\pdfglyphtounicode{Obarreddieresiscyrillic}{04EA}
\pdfglyphtounicode{Obreve}{014E}
\pdfglyphtounicode{Ocaron}{01D1}
\pdfglyphtounicode{Ocenteredtilde}{019F}
\pdfglyphtounicode{Ocircle}{24C4}
\pdfglyphtounicode{Ocircumflex}{00D4}
\pdfglyphtounicode{Ocircumflexacute}{1ED0}
\pdfglyphtounicode{Ocircumflexdotbelow}{1ED8}
\pdfglyphtounicode{Ocircumflexgrave}{1ED2}
\pdfglyphtounicode{Ocircumflexhookabove}{1ED4}
\pdfglyphtounicode{Ocircumflexsmall}{F7F4}
\pdfglyphtounicode{Ocircumflextilde}{1ED6}
\pdfglyphtounicode{Ocyrillic}{041E}
\pdfglyphtounicode{Odblacute}{0150}
\pdfglyphtounicode{Odblgrave}{020C}
\pdfglyphtounicode{Odieresis}{00D6}
\pdfglyphtounicode{Odieresiscyrillic}{04E6}
\pdfglyphtounicode{Odieresissmall}{F7F6}
\pdfglyphtounicode{Odotbelow}{1ECC}
\pdfglyphtounicode{Ogoneksmall}{F6FB}
\pdfglyphtounicode{Ograve}{00D2}
\pdfglyphtounicode{Ogravesmall}{F7F2}
\pdfglyphtounicode{Oharmenian}{0555}
\pdfglyphtounicode{Ohm}{2126}
\pdfglyphtounicode{Ohookabove}{1ECE}
\pdfglyphtounicode{Ohorn}{01A0}
\pdfglyphtounicode{Ohornacute}{1EDA}
\pdfglyphtounicode{Ohorndotbelow}{1EE2}
\pdfglyphtounicode{Ohorngrave}{1EDC}
\pdfglyphtounicode{Ohornhookabove}{1EDE}
\pdfglyphtounicode{Ohorntilde}{1EE0}
\pdfglyphtounicode{Ohungarumlaut}{0150}
\pdfglyphtounicode{Oi}{01A2}
\pdfglyphtounicode{Oinvertedbreve}{020E}
\pdfglyphtounicode{Omacron}{014C}
\pdfglyphtounicode{Omacronacute}{1E52}
\pdfglyphtounicode{Omacrongrave}{1E50}
\pdfglyphtounicode{Omega}{2126}
\pdfglyphtounicode{Omegacyrillic}{0460}
\pdfglyphtounicode{Omegagreek}{03A9}
\pdfglyphtounicode{Omegaroundcyrillic}{047A}
\pdfglyphtounicode{Omegatitlocyrillic}{047C}
\pdfglyphtounicode{Omegatonos}{038F}
\pdfglyphtounicode{Omicron}{039F}
\pdfglyphtounicode{Omicrontonos}{038C}
\pdfglyphtounicode{Omonospace}{FF2F}
\pdfglyphtounicode{Oneroman}{2160}
\pdfglyphtounicode{Oogonek}{01EA}
\pdfglyphtounicode{Oogonekmacron}{01EC}
\pdfglyphtounicode{Oopen}{0186}
\pdfglyphtounicode{Oslash}{00D8}
\pdfglyphtounicode{Oslashacute}{01FE}
\pdfglyphtounicode{Oslashsmall}{F7F8}
\pdfglyphtounicode{Osmall}{F76F}
\pdfglyphtounicode{Ostrokeacute}{01FE}
\pdfglyphtounicode{Otcyrillic}{047E}
\pdfglyphtounicode{Otilde}{00D5}
\pdfglyphtounicode{Otildeacute}{1E4C}
\pdfglyphtounicode{Otildedieresis}{1E4E}
\pdfglyphtounicode{Otildesmall}{F7F5}
\pdfglyphtounicode{P}{0050}
\pdfglyphtounicode{Pacute}{1E54}
\pdfglyphtounicode{Pcircle}{24C5}
\pdfglyphtounicode{Pdotaccent}{1E56}
\pdfglyphtounicode{Pecyrillic}{041F}
\pdfglyphtounicode{Peharmenian}{054A}
\pdfglyphtounicode{Pemiddlehookcyrillic}{04A6}
\pdfglyphtounicode{Phi}{03A6}
\pdfglyphtounicode{Phook}{01A4}
\pdfglyphtounicode{Pi}{03A0}
\pdfglyphtounicode{Piwrarmenian}{0553}
\pdfglyphtounicode{Pmonospace}{FF30}
\pdfglyphtounicode{Psi}{03A8}
\pdfglyphtounicode{Psicyrillic}{0470}
\pdfglyphtounicode{Psmall}{F770}
\pdfglyphtounicode{Q}{0051}
\pdfglyphtounicode{Qcircle}{24C6}
\pdfglyphtounicode{Qmonospace}{FF31}
\pdfglyphtounicode{Qsmall}{F771}
\pdfglyphtounicode{R}{0052}
\pdfglyphtounicode{Raarmenian}{054C}
\pdfglyphtounicode{Racute}{0154}
\pdfglyphtounicode{Rcaron}{0158}
\pdfglyphtounicode{Rcedilla}{0156}
\pdfglyphtounicode{Rcircle}{24C7}
\pdfglyphtounicode{Rcommaaccent}{0156}
\pdfglyphtounicode{Rdblgrave}{0210}
\pdfglyphtounicode{Rdotaccent}{1E58}
\pdfglyphtounicode{Rdotbelow}{1E5A}
\pdfglyphtounicode{Rdotbelowmacron}{1E5C}
\pdfglyphtounicode{Reharmenian}{0550}
\pdfglyphtounicode{Rfraktur}{211C}
\pdfglyphtounicode{Rho}{03A1}
\pdfglyphtounicode{Ringsmall}{F6FC}
\pdfglyphtounicode{Rinvertedbreve}{0212}
\pdfglyphtounicode{Rlinebelow}{1E5E}
\pdfglyphtounicode{Rmonospace}{FF32}
\pdfglyphtounicode{Rsmall}{F772}
\pdfglyphtounicode{Rsmallinverted}{0281}
\pdfglyphtounicode{Rsmallinvertedsuperior}{02B6}
\pdfglyphtounicode{S}{0053}
\pdfglyphtounicode{SF010000}{250C}
\pdfglyphtounicode{SF020000}{2514}
\pdfglyphtounicode{SF030000}{2510}
\pdfglyphtounicode{SF040000}{2518}
\pdfglyphtounicode{SF050000}{253C}
\pdfglyphtounicode{SF060000}{252C}
\pdfglyphtounicode{SF070000}{2534}
\pdfglyphtounicode{SF080000}{251C}
\pdfglyphtounicode{SF090000}{2524}
\pdfglyphtounicode{SF100000}{2500}
\pdfglyphtounicode{SF110000}{2502}
\pdfglyphtounicode{SF190000}{2561}
\pdfglyphtounicode{SF200000}{2562}
\pdfglyphtounicode{SF210000}{2556}
\pdfglyphtounicode{SF220000}{2555}
\pdfglyphtounicode{SF230000}{2563}
\pdfglyphtounicode{SF240000}{2551}
\pdfglyphtounicode{SF250000}{2557}
\pdfglyphtounicode{SF260000}{255D}
\pdfglyphtounicode{SF270000}{255C}
\pdfglyphtounicode{SF280000}{255B}
\pdfglyphtounicode{SF360000}{255E}
\pdfglyphtounicode{SF370000}{255F}
\pdfglyphtounicode{SF380000}{255A}
\pdfglyphtounicode{SF390000}{2554}
\pdfglyphtounicode{SF400000}{2569}
\pdfglyphtounicode{SF410000}{2566}
\pdfglyphtounicode{SF420000}{2560}
\pdfglyphtounicode{SF430000}{2550}
\pdfglyphtounicode{SF440000}{256C}
\pdfglyphtounicode{SF450000}{2567}
\pdfglyphtounicode{SF460000}{2568}
\pdfglyphtounicode{SF470000}{2564}
\pdfglyphtounicode{SF480000}{2565}
\pdfglyphtounicode{SF490000}{2559}
\pdfglyphtounicode{SF500000}{2558}
\pdfglyphtounicode{SF510000}{2552}
\pdfglyphtounicode{SF520000}{2553}
\pdfglyphtounicode{SF530000}{256B}
\pdfglyphtounicode{SF540000}{256A}
\pdfglyphtounicode{Sacute}{015A}
\pdfglyphtounicode{Sacutedotaccent}{1E64}
\pdfglyphtounicode{Sampigreek}{03E0}
\pdfglyphtounicode{Scaron}{0160}
\pdfglyphtounicode{Scarondotaccent}{1E66}
\pdfglyphtounicode{Scaronsmall}{F6FD}
\pdfglyphtounicode{Scedilla}{015E}
\pdfglyphtounicode{Schwa}{018F}
\pdfglyphtounicode{Schwacyrillic}{04D8}
\pdfglyphtounicode{Schwadieresiscyrillic}{04DA}
\pdfglyphtounicode{Scircle}{24C8}
\pdfglyphtounicode{Scircumflex}{015C}
\pdfglyphtounicode{Scommaaccent}{0218}
\pdfglyphtounicode{Sdotaccent}{1E60}
\pdfglyphtounicode{Sdotbelow}{1E62}
\pdfglyphtounicode{Sdotbelowdotaccent}{1E68}
\pdfglyphtounicode{Seharmenian}{054D}
\pdfglyphtounicode{Sevenroman}{2166}
\pdfglyphtounicode{Shaarmenian}{0547}
\pdfglyphtounicode{Shacyrillic}{0428}
\pdfglyphtounicode{Shchacyrillic}{0429}
\pdfglyphtounicode{Sheicoptic}{03E2}
\pdfglyphtounicode{Shhacyrillic}{04BA}
\pdfglyphtounicode{Shimacoptic}{03EC}
\pdfglyphtounicode{Sigma}{03A3}
\pdfglyphtounicode{Sixroman}{2165}
\pdfglyphtounicode{Smonospace}{FF33}
\pdfglyphtounicode{Softsigncyrillic}{042C}
\pdfglyphtounicode{Ssmall}{F773}
\pdfglyphtounicode{Stigmagreek}{03DA}
\pdfglyphtounicode{T}{0054}
\pdfglyphtounicode{Tau}{03A4}
\pdfglyphtounicode{Tbar}{0166}
\pdfglyphtounicode{Tcaron}{0164}
\pdfglyphtounicode{Tcedilla}{0162}
\pdfglyphtounicode{Tcircle}{24C9}
\pdfglyphtounicode{Tcircumflexbelow}{1E70}
\pdfglyphtounicode{Tcommaaccent}{0162}
\pdfglyphtounicode{Tdotaccent}{1E6A}
\pdfglyphtounicode{Tdotbelow}{1E6C}
\pdfglyphtounicode{Tecyrillic}{0422}
\pdfglyphtounicode{Tedescendercyrillic}{04AC}
\pdfglyphtounicode{Tenroman}{2169}
\pdfglyphtounicode{Tetsecyrillic}{04B4}
\pdfglyphtounicode{Theta}{0398}
\pdfglyphtounicode{Thook}{01AC}
\pdfglyphtounicode{Thorn}{00DE}
\pdfglyphtounicode{Thornsmall}{F7FE}
\pdfglyphtounicode{Threeroman}{2162}
\pdfglyphtounicode{Tildesmall}{F6FE}
\pdfglyphtounicode{Tiwnarmenian}{054F}
\pdfglyphtounicode{Tlinebelow}{1E6E}
\pdfglyphtounicode{Tmonospace}{FF34}
\pdfglyphtounicode{Toarmenian}{0539}
\pdfglyphtounicode{Tonefive}{01BC}
\pdfglyphtounicode{Tonesix}{0184}
\pdfglyphtounicode{Tonetwo}{01A7}
\pdfglyphtounicode{Tretroflexhook}{01AE}
\pdfglyphtounicode{Tsecyrillic}{0426}
\pdfglyphtounicode{Tshecyrillic}{040B}
\pdfglyphtounicode{Tsmall}{F774}
\pdfglyphtounicode{Twelveroman}{216B}
\pdfglyphtounicode{Tworoman}{2161}
\pdfglyphtounicode{U}{0055}
\pdfglyphtounicode{Uacute}{00DA}
\pdfglyphtounicode{Uacutesmall}{F7FA}
\pdfglyphtounicode{Ubreve}{016C}
\pdfglyphtounicode{Ucaron}{01D3}
\pdfglyphtounicode{Ucircle}{24CA}
\pdfglyphtounicode{Ucircumflex}{00DB}
\pdfglyphtounicode{Ucircumflexbelow}{1E76}
\pdfglyphtounicode{Ucircumflexsmall}{F7FB}
\pdfglyphtounicode{Ucyrillic}{0423}
\pdfglyphtounicode{Udblacute}{0170}
\pdfglyphtounicode{Udblgrave}{0214}
\pdfglyphtounicode{Udieresis}{00DC}
\pdfglyphtounicode{Udieresisacute}{01D7}
\pdfglyphtounicode{Udieresisbelow}{1E72}
\pdfglyphtounicode{Udieresiscaron}{01D9}
\pdfglyphtounicode{Udieresiscyrillic}{04F0}
\pdfglyphtounicode{Udieresisgrave}{01DB}
\pdfglyphtounicode{Udieresismacron}{01D5}
\pdfglyphtounicode{Udieresissmall}{F7FC}
\pdfglyphtounicode{Udotbelow}{1EE4}
\pdfglyphtounicode{Ugrave}{00D9}
\pdfglyphtounicode{Ugravesmall}{F7F9}
\pdfglyphtounicode{Uhookabove}{1EE6}
\pdfglyphtounicode{Uhorn}{01AF}
\pdfglyphtounicode{Uhornacute}{1EE8}
\pdfglyphtounicode{Uhorndotbelow}{1EF0}
\pdfglyphtounicode{Uhorngrave}{1EEA}
\pdfglyphtounicode{Uhornhookabove}{1EEC}
\pdfglyphtounicode{Uhorntilde}{1EEE}
\pdfglyphtounicode{Uhungarumlaut}{0170}
\pdfglyphtounicode{Uhungarumlautcyrillic}{04F2}
\pdfglyphtounicode{Uinvertedbreve}{0216}
\pdfglyphtounicode{Ukcyrillic}{0478}
\pdfglyphtounicode{Umacron}{016A}
\pdfglyphtounicode{Umacroncyrillic}{04EE}
\pdfglyphtounicode{Umacrondieresis}{1E7A}
\pdfglyphtounicode{Umonospace}{FF35}
\pdfglyphtounicode{Uogonek}{0172}
\pdfglyphtounicode{Upsilon}{03A5}
\pdfglyphtounicode{Upsilon1}{03D2}
\pdfglyphtounicode{Upsilonacutehooksymbolgreek}{03D3}
\pdfglyphtounicode{Upsilonafrican}{01B1}
\pdfglyphtounicode{Upsilondieresis}{03AB}
\pdfglyphtounicode{Upsilondieresishooksymbolgreek}{03D4}
\pdfglyphtounicode{Upsilonhooksymbol}{03D2}
\pdfglyphtounicode{Upsilontonos}{038E}
\pdfglyphtounicode{Uring}{016E}
\pdfglyphtounicode{Ushortcyrillic}{040E}
\pdfglyphtounicode{Usmall}{F775}
\pdfglyphtounicode{Ustraightcyrillic}{04AE}
\pdfglyphtounicode{Ustraightstrokecyrillic}{04B0}
\pdfglyphtounicode{Utilde}{0168}
\pdfglyphtounicode{Utildeacute}{1E78}
\pdfglyphtounicode{Utildebelow}{1E74}
\pdfglyphtounicode{V}{0056}
\pdfglyphtounicode{Vcircle}{24CB}
\pdfglyphtounicode{Vdotbelow}{1E7E}
\pdfglyphtounicode{Vecyrillic}{0412}
\pdfglyphtounicode{Vewarmenian}{054E}
\pdfglyphtounicode{Vhook}{01B2}
\pdfglyphtounicode{Vmonospace}{FF36}
\pdfglyphtounicode{Voarmenian}{0548}
\pdfglyphtounicode{Vsmall}{F776}
\pdfglyphtounicode{Vtilde}{1E7C}
\pdfglyphtounicode{W}{0057}
\pdfglyphtounicode{Wacute}{1E82}
\pdfglyphtounicode{Wcircle}{24CC}
\pdfglyphtounicode{Wcircumflex}{0174}
\pdfglyphtounicode{Wdieresis}{1E84}
\pdfglyphtounicode{Wdotaccent}{1E86}
\pdfglyphtounicode{Wdotbelow}{1E88}
\pdfglyphtounicode{Wgrave}{1E80}
\pdfglyphtounicode{Wmonospace}{FF37}
\pdfglyphtounicode{Wsmall}{F777}
\pdfglyphtounicode{X}{0058}
\pdfglyphtounicode{Xcircle}{24CD}
\pdfglyphtounicode{Xdieresis}{1E8C}
\pdfglyphtounicode{Xdotaccent}{1E8A}
\pdfglyphtounicode{Xeharmenian}{053D}
\pdfglyphtounicode{Xi}{039E}
\pdfglyphtounicode{Xmonospace}{FF38}
\pdfglyphtounicode{Xsmall}{F778}
\pdfglyphtounicode{Y}{0059}
\pdfglyphtounicode{Yacute}{00DD}
\pdfglyphtounicode{Yacutesmall}{F7FD}
\pdfglyphtounicode{Yatcyrillic}{0462}
\pdfglyphtounicode{Ycircle}{24CE}
\pdfglyphtounicode{Ycircumflex}{0176}
\pdfglyphtounicode{Ydieresis}{0178}
\pdfglyphtounicode{Ydieresissmall}{F7FF}
\pdfglyphtounicode{Ydotaccent}{1E8E}
\pdfglyphtounicode{Ydotbelow}{1EF4}
\pdfglyphtounicode{Yericyrillic}{042B}
\pdfglyphtounicode{Yerudieresiscyrillic}{04F8}
\pdfglyphtounicode{Ygrave}{1EF2}
\pdfglyphtounicode{Yhook}{01B3}
\pdfglyphtounicode{Yhookabove}{1EF6}
\pdfglyphtounicode{Yiarmenian}{0545}
\pdfglyphtounicode{Yicyrillic}{0407}
\pdfglyphtounicode{Yiwnarmenian}{0552}
\pdfglyphtounicode{Ymonospace}{FF39}
\pdfglyphtounicode{Ysmall}{F779}
\pdfglyphtounicode{Ytilde}{1EF8}
\pdfglyphtounicode{Yusbigcyrillic}{046A}
\pdfglyphtounicode{Yusbigiotifiedcyrillic}{046C}
\pdfglyphtounicode{Yuslittlecyrillic}{0466}
\pdfglyphtounicode{Yuslittleiotifiedcyrillic}{0468}
\pdfglyphtounicode{Z}{005A}
\pdfglyphtounicode{Zaarmenian}{0536}
\pdfglyphtounicode{Zacute}{0179}
\pdfglyphtounicode{Zcaron}{017D}
\pdfglyphtounicode{Zcaronsmall}{F6FF}
\pdfglyphtounicode{Zcircle}{24CF}
\pdfglyphtounicode{Zcircumflex}{1E90}
\pdfglyphtounicode{Zdot}{017B}
\pdfglyphtounicode{Zdotaccent}{017B}
\pdfglyphtounicode{Zdotbelow}{1E92}
\pdfglyphtounicode{Zecyrillic}{0417}
\pdfglyphtounicode{Zedescendercyrillic}{0498}
\pdfglyphtounicode{Zedieresiscyrillic}{04DE}
\pdfglyphtounicode{Zeta}{0396}
\pdfglyphtounicode{Zhearmenian}{053A}
\pdfglyphtounicode{Zhebrevecyrillic}{04C1}
\pdfglyphtounicode{Zhecyrillic}{0416}
\pdfglyphtounicode{Zhedescendercyrillic}{0496}
\pdfglyphtounicode{Zhedieresiscyrillic}{04DC}
\pdfglyphtounicode{Zlinebelow}{1E94}
\pdfglyphtounicode{Zmonospace}{FF3A}
\pdfglyphtounicode{Zsmall}{F77A}
\pdfglyphtounicode{Zstroke}{01B5}
\pdfglyphtounicode{a}{0061}
\pdfglyphtounicode{aabengali}{0986}
\pdfglyphtounicode{aacute}{00E1}
\pdfglyphtounicode{aadeva}{0906}
\pdfglyphtounicode{aagujarati}{0A86}
\pdfglyphtounicode{aagurmukhi}{0A06}
\pdfglyphtounicode{aamatragurmukhi}{0A3E}
\pdfglyphtounicode{aarusquare}{3303}
\pdfglyphtounicode{aavowelsignbengali}{09BE}
\pdfglyphtounicode{aavowelsigndeva}{093E}
\pdfglyphtounicode{aavowelsigngujarati}{0ABE}
\pdfglyphtounicode{abbreviationmarkarmenian}{055F}
\pdfglyphtounicode{abbreviationsigndeva}{0970}
\pdfglyphtounicode{abengali}{0985}
\pdfglyphtounicode{abopomofo}{311A}
\pdfglyphtounicode{abreve}{0103}
\pdfglyphtounicode{abreveacute}{1EAF}
\pdfglyphtounicode{abrevecyrillic}{04D1}
\pdfglyphtounicode{abrevedotbelow}{1EB7}
\pdfglyphtounicode{abrevegrave}{1EB1}
\pdfglyphtounicode{abrevehookabove}{1EB3}
\pdfglyphtounicode{abrevetilde}{1EB5}
\pdfglyphtounicode{acaron}{01CE}
\pdfglyphtounicode{acircle}{24D0}
\pdfglyphtounicode{acircumflex}{00E2}
\pdfglyphtounicode{acircumflexacute}{1EA5}
\pdfglyphtounicode{acircumflexdotbelow}{1EAD}
\pdfglyphtounicode{acircumflexgrave}{1EA7}
\pdfglyphtounicode{acircumflexhookabove}{1EA9}
\pdfglyphtounicode{acircumflextilde}{1EAB}
\pdfglyphtounicode{acute}{00B4}
\pdfglyphtounicode{acutebelowcmb}{0317}
\pdfglyphtounicode{acutecmb}{0301}
\pdfglyphtounicode{acutecomb}{0301}
\pdfglyphtounicode{acutedeva}{0954}
\pdfglyphtounicode{acutelowmod}{02CF}
\pdfglyphtounicode{acutetonecmb}{0341}
\pdfglyphtounicode{acyrillic}{0430}
\pdfglyphtounicode{adblgrave}{0201}
\pdfglyphtounicode{addakgurmukhi}{0A71}
\pdfglyphtounicode{adeva}{0905}
\pdfglyphtounicode{adieresis}{00E4}
\pdfglyphtounicode{adieresiscyrillic}{04D3}
\pdfglyphtounicode{adieresismacron}{01DF}
\pdfglyphtounicode{adotbelow}{1EA1}
\pdfglyphtounicode{adotmacron}{01E1}
\pdfglyphtounicode{ae}{00E6}
\pdfglyphtounicode{aeacute}{01FD}
\pdfglyphtounicode{aekorean}{3150}
\pdfglyphtounicode{aemacron}{01E3}
\pdfglyphtounicode{afii00208}{2015}
\pdfglyphtounicode{afii08941}{20A4}
\pdfglyphtounicode{afii10017}{0410}
\pdfglyphtounicode{afii10018}{0411}
\pdfglyphtounicode{afii10019}{0412}
\pdfglyphtounicode{afii10020}{0413}
\pdfglyphtounicode{afii10021}{0414}
\pdfglyphtounicode{afii10022}{0415}
\pdfglyphtounicode{afii10023}{0401}
\pdfglyphtounicode{afii10024}{0416}
\pdfglyphtounicode{afii10025}{0417}
\pdfglyphtounicode{afii10026}{0418}
\pdfglyphtounicode{afii10027}{0419}
\pdfglyphtounicode{afii10028}{041A}
\pdfglyphtounicode{afii10029}{041B}
\pdfglyphtounicode{afii10030}{041C}
\pdfglyphtounicode{afii10031}{041D}
\pdfglyphtounicode{afii10032}{041E}
\pdfglyphtounicode{afii10033}{041F}
\pdfglyphtounicode{afii10034}{0420}
\pdfglyphtounicode{afii10035}{0421}
\pdfglyphtounicode{afii10036}{0422}
\pdfglyphtounicode{afii10037}{0423}
\pdfglyphtounicode{afii10038}{0424}
\pdfglyphtounicode{afii10039}{0425}
\pdfglyphtounicode{afii10040}{0426}
\pdfglyphtounicode{afii10041}{0427}
\pdfglyphtounicode{afii10042}{0428}
\pdfglyphtounicode{afii10043}{0429}
\pdfglyphtounicode{afii10044}{042A}
\pdfglyphtounicode{afii10045}{042B}
\pdfglyphtounicode{afii10046}{042C}
\pdfglyphtounicode{afii10047}{042D}
\pdfglyphtounicode{afii10048}{042E}
\pdfglyphtounicode{afii10049}{042F}
\pdfglyphtounicode{afii10050}{0490}
\pdfglyphtounicode{afii10051}{0402}
\pdfglyphtounicode{afii10052}{0403}
\pdfglyphtounicode{afii10053}{0404}
\pdfglyphtounicode{afii10054}{0405}
\pdfglyphtounicode{afii10055}{0406}
\pdfglyphtounicode{afii10056}{0407}
\pdfglyphtounicode{afii10057}{0408}
\pdfglyphtounicode{afii10058}{0409}
\pdfglyphtounicode{afii10059}{040A}
\pdfglyphtounicode{afii10060}{040B}
\pdfglyphtounicode{afii10061}{040C}
\pdfglyphtounicode{afii10062}{040E}
\pdfglyphtounicode{afii10063}{F6C4}
\pdfglyphtounicode{afii10064}{F6C5}
\pdfglyphtounicode{afii10065}{0430}
\pdfglyphtounicode{afii10066}{0431}
\pdfglyphtounicode{afii10067}{0432}
\pdfglyphtounicode{afii10068}{0433}
\pdfglyphtounicode{afii10069}{0434}
\pdfglyphtounicode{afii10070}{0435}
\pdfglyphtounicode{afii10071}{0451}
\pdfglyphtounicode{afii10072}{0436}
\pdfglyphtounicode{afii10073}{0437}
\pdfglyphtounicode{afii10074}{0438}
\pdfglyphtounicode{afii10075}{0439}
\pdfglyphtounicode{afii10076}{043A}
\pdfglyphtounicode{afii10077}{043B}
\pdfglyphtounicode{afii10078}{043C}
\pdfglyphtounicode{afii10079}{043D}
\pdfglyphtounicode{afii10080}{043E}
\pdfglyphtounicode{afii10081}{043F}
\pdfglyphtounicode{afii10082}{0440}
\pdfglyphtounicode{afii10083}{0441}
\pdfglyphtounicode{afii10084}{0442}
\pdfglyphtounicode{afii10085}{0443}
\pdfglyphtounicode{afii10086}{0444}
\pdfglyphtounicode{afii10087}{0445}
\pdfglyphtounicode{afii10088}{0446}
\pdfglyphtounicode{afii10089}{0447}
\pdfglyphtounicode{afii10090}{0448}
\pdfglyphtounicode{afii10091}{0449}
\pdfglyphtounicode{afii10092}{044A}
\pdfglyphtounicode{afii10093}{044B}
\pdfglyphtounicode{afii10094}{044C}
\pdfglyphtounicode{afii10095}{044D}
\pdfglyphtounicode{afii10096}{044E}
\pdfglyphtounicode{afii10097}{044F}
\pdfglyphtounicode{afii10098}{0491}
\pdfglyphtounicode{afii10099}{0452}
\pdfglyphtounicode{afii10100}{0453}
\pdfglyphtounicode{afii10101}{0454}
\pdfglyphtounicode{afii10102}{0455}
\pdfglyphtounicode{afii10103}{0456}
\pdfglyphtounicode{afii10104}{0457}
\pdfglyphtounicode{afii10105}{0458}
\pdfglyphtounicode{afii10106}{0459}
\pdfglyphtounicode{afii10107}{045A}
\pdfglyphtounicode{afii10108}{045B}
\pdfglyphtounicode{afii10109}{045C}
\pdfglyphtounicode{afii10110}{045E}
\pdfglyphtounicode{afii10145}{040F}
\pdfglyphtounicode{afii10146}{0462}
\pdfglyphtounicode{afii10147}{0472}
\pdfglyphtounicode{afii10148}{0474}
\pdfglyphtounicode{afii10192}{F6C6}
\pdfglyphtounicode{afii10193}{045F}
\pdfglyphtounicode{afii10194}{0463}
\pdfglyphtounicode{afii10195}{0473}
\pdfglyphtounicode{afii10196}{0475}
\pdfglyphtounicode{afii10831}{F6C7}
\pdfglyphtounicode{afii10832}{F6C8}
\pdfglyphtounicode{afii10846}{04D9}
\pdfglyphtounicode{afii299}{200E}
\pdfglyphtounicode{afii300}{200F}
\pdfglyphtounicode{afii301}{200D}
\pdfglyphtounicode{afii57381}{066A}
\pdfglyphtounicode{afii57388}{060C}
\pdfglyphtounicode{afii57392}{0660}
\pdfglyphtounicode{afii57393}{0661}
\pdfglyphtounicode{afii57394}{0662}
\pdfglyphtounicode{afii57395}{0663}
\pdfglyphtounicode{afii57396}{0664}
\pdfglyphtounicode{afii57397}{0665}
\pdfglyphtounicode{afii57398}{0666}
\pdfglyphtounicode{afii57399}{0667}
\pdfglyphtounicode{afii57400}{0668}
\pdfglyphtounicode{afii57401}{0669}
\pdfglyphtounicode{afii57403}{061B}
\pdfglyphtounicode{afii57407}{061F}
\pdfglyphtounicode{afii57409}{0621}
\pdfglyphtounicode{afii57410}{0622}
\pdfglyphtounicode{afii57411}{0623}
\pdfglyphtounicode{afii57412}{0624}
\pdfglyphtounicode{afii57413}{0625}
\pdfglyphtounicode{afii57414}{0626}
\pdfglyphtounicode{afii57415}{0627}
\pdfglyphtounicode{afii57416}{0628}
\pdfglyphtounicode{afii57417}{0629}
\pdfglyphtounicode{afii57418}{062A}
\pdfglyphtounicode{afii57419}{062B}
\pdfglyphtounicode{afii57420}{062C}
\pdfglyphtounicode{afii57421}{062D}
\pdfglyphtounicode{afii57422}{062E}
\pdfglyphtounicode{afii57423}{062F}
\pdfglyphtounicode{afii57424}{0630}
\pdfglyphtounicode{afii57425}{0631}
\pdfglyphtounicode{afii57426}{0632}
\pdfglyphtounicode{afii57427}{0633}
\pdfglyphtounicode{afii57428}{0634}
\pdfglyphtounicode{afii57429}{0635}
\pdfglyphtounicode{afii57430}{0636}
\pdfglyphtounicode{afii57431}{0637}
\pdfglyphtounicode{afii57432}{0638}
\pdfglyphtounicode{afii57433}{0639}
\pdfglyphtounicode{afii57434}{063A}
\pdfglyphtounicode{afii57440}{0640}
\pdfglyphtounicode{afii57441}{0641}
\pdfglyphtounicode{afii57442}{0642}
\pdfglyphtounicode{afii57443}{0643}
\pdfglyphtounicode{afii57444}{0644}
\pdfglyphtounicode{afii57445}{0645}
\pdfglyphtounicode{afii57446}{0646}
\pdfglyphtounicode{afii57448}{0648}
\pdfglyphtounicode{afii57449}{0649}
\pdfglyphtounicode{afii57450}{064A}
\pdfglyphtounicode{afii57451}{064B}
\pdfglyphtounicode{afii57452}{064C}
\pdfglyphtounicode{afii57453}{064D}
\pdfglyphtounicode{afii57454}{064E}
\pdfglyphtounicode{afii57455}{064F}
\pdfglyphtounicode{afii57456}{0650}
\pdfglyphtounicode{afii57457}{0651}
\pdfglyphtounicode{afii57458}{0652}
\pdfglyphtounicode{afii57470}{0647}
\pdfglyphtounicode{afii57505}{06A4}
\pdfglyphtounicode{afii57506}{067E}
\pdfglyphtounicode{afii57507}{0686}
\pdfglyphtounicode{afii57508}{0698}
\pdfglyphtounicode{afii57509}{06AF}
\pdfglyphtounicode{afii57511}{0679}
\pdfglyphtounicode{afii57512}{0688}
\pdfglyphtounicode{afii57513}{0691}
\pdfglyphtounicode{afii57514}{06BA}
\pdfglyphtounicode{afii57519}{06D2}
\pdfglyphtounicode{afii57534}{06D5}
\pdfglyphtounicode{afii57636}{20AA}
\pdfglyphtounicode{afii57645}{05BE}
\pdfglyphtounicode{afii57658}{05C3}
\pdfglyphtounicode{afii57664}{05D0}
\pdfglyphtounicode{afii57665}{05D1}
\pdfglyphtounicode{afii57666}{05D2}
\pdfglyphtounicode{afii57667}{05D3}
\pdfglyphtounicode{afii57668}{05D4}
\pdfglyphtounicode{afii57669}{05D5}
\pdfglyphtounicode{afii57670}{05D6}
\pdfglyphtounicode{afii57671}{05D7}
\pdfglyphtounicode{afii57672}{05D8}
\pdfglyphtounicode{afii57673}{05D9}
\pdfglyphtounicode{afii57674}{05DA}
\pdfglyphtounicode{afii57675}{05DB}
\pdfglyphtounicode{afii57676}{05DC}
\pdfglyphtounicode{afii57677}{05DD}
\pdfglyphtounicode{afii57678}{05DE}
\pdfglyphtounicode{afii57679}{05DF}
\pdfglyphtounicode{afii57680}{05E0}
\pdfglyphtounicode{afii57681}{05E1}
\pdfglyphtounicode{afii57682}{05E2}
\pdfglyphtounicode{afii57683}{05E3}
\pdfglyphtounicode{afii57684}{05E4}
\pdfglyphtounicode{afii57685}{05E5}
\pdfglyphtounicode{afii57686}{05E6}
\pdfglyphtounicode{afii57687}{05E7}
\pdfglyphtounicode{afii57688}{05E8}
\pdfglyphtounicode{afii57689}{05E9}
\pdfglyphtounicode{afii57690}{05EA}
\pdfglyphtounicode{afii57694}{FB2A}
\pdfglyphtounicode{afii57695}{FB2B}
\pdfglyphtounicode{afii57700}{FB4B}
\pdfglyphtounicode{afii57705}{FB1F}
\pdfglyphtounicode{afii57716}{05F0}
\pdfglyphtounicode{afii57717}{05F1}
\pdfglyphtounicode{afii57718}{05F2}
\pdfglyphtounicode{afii57723}{FB35}
\pdfglyphtounicode{afii57793}{05B4}
\pdfglyphtounicode{afii57794}{05B5}
\pdfglyphtounicode{afii57795}{05B6}
\pdfglyphtounicode{afii57796}{05BB}
\pdfglyphtounicode{afii57797}{05B8}
\pdfglyphtounicode{afii57798}{05B7}
\pdfglyphtounicode{afii57799}{05B0}
\pdfglyphtounicode{afii57800}{05B2}
\pdfglyphtounicode{afii57801}{05B1}
\pdfglyphtounicode{afii57802}{05B3}
\pdfglyphtounicode{afii57803}{05C2}
\pdfglyphtounicode{afii57804}{05C1}
\pdfglyphtounicode{afii57806}{05B9}
\pdfglyphtounicode{afii57807}{05BC}
\pdfglyphtounicode{afii57839}{05BD}
\pdfglyphtounicode{afii57841}{05BF}
\pdfglyphtounicode{afii57842}{05C0}
\pdfglyphtounicode{afii57929}{02BC}
\pdfglyphtounicode{afii61248}{2105}
\pdfglyphtounicode{afii61289}{2113}
\pdfglyphtounicode{afii61352}{2116}
\pdfglyphtounicode{afii61573}{202C}
\pdfglyphtounicode{afii61574}{202D}
\pdfglyphtounicode{afii61575}{202E}
\pdfglyphtounicode{afii61664}{200C}
\pdfglyphtounicode{afii63167}{066D}
\pdfglyphtounicode{afii64937}{02BD}
\pdfglyphtounicode{agrave}{00E0}
\pdfglyphtounicode{agujarati}{0A85}
\pdfglyphtounicode{agurmukhi}{0A05}
\pdfglyphtounicode{ahiragana}{3042}
\pdfglyphtounicode{ahookabove}{1EA3}
\pdfglyphtounicode{aibengali}{0990}
\pdfglyphtounicode{aibopomofo}{311E}
\pdfglyphtounicode{aideva}{0910}
\pdfglyphtounicode{aiecyrillic}{04D5}
\pdfglyphtounicode{aigujarati}{0A90}
\pdfglyphtounicode{aigurmukhi}{0A10}
\pdfglyphtounicode{aimatragurmukhi}{0A48}
\pdfglyphtounicode{ainarabic}{0639}
\pdfglyphtounicode{ainfinalarabic}{FECA}
\pdfglyphtounicode{aininitialarabic}{FECB}
\pdfglyphtounicode{ainmedialarabic}{FECC}
\pdfglyphtounicode{ainvertedbreve}{0203}
\pdfglyphtounicode{aivowelsignbengali}{09C8}
\pdfglyphtounicode{aivowelsigndeva}{0948}
\pdfglyphtounicode{aivowelsigngujarati}{0AC8}
\pdfglyphtounicode{akatakana}{30A2}
\pdfglyphtounicode{akatakanahalfwidth}{FF71}
\pdfglyphtounicode{akorean}{314F}
\pdfglyphtounicode{alef}{05D0}
\pdfglyphtounicode{alefarabic}{0627}
\pdfglyphtounicode{alefdageshhebrew}{FB30}
\pdfglyphtounicode{aleffinalarabic}{FE8E}
\pdfglyphtounicode{alefhamzaabovearabic}{0623}
\pdfglyphtounicode{alefhamzaabovefinalarabic}{FE84}
\pdfglyphtounicode{alefhamzabelowarabic}{0625}
\pdfglyphtounicode{alefhamzabelowfinalarabic}{FE88}
\pdfglyphtounicode{alefhebrew}{05D0}
\pdfglyphtounicode{aleflamedhebrew}{FB4F}
\pdfglyphtounicode{alefmaddaabovearabic}{0622}
\pdfglyphtounicode{alefmaddaabovefinalarabic}{FE82}
\pdfglyphtounicode{alefmaksuraarabic}{0649}
\pdfglyphtounicode{alefmaksurafinalarabic}{FEF0}
\pdfglyphtounicode{alefmaksurainitialarabic}{FEF3}
\pdfglyphtounicode{alefmaksuramedialarabic}{FEF4}
\pdfglyphtounicode{alefpatahhebrew}{FB2E}
\pdfglyphtounicode{alefqamatshebrew}{FB2F}
\pdfglyphtounicode{aleph}{2135}
\pdfglyphtounicode{allequal}{224C}
\pdfglyphtounicode{alpha}{03B1}
\pdfglyphtounicode{alphatonos}{03AC}
\pdfglyphtounicode{amacron}{0101}
\pdfglyphtounicode{amonospace}{FF41}
\pdfglyphtounicode{ampersand}{0026}
\pdfglyphtounicode{ampersandmonospace}{FF06}
\pdfglyphtounicode{ampersandsmall}{F726}
\pdfglyphtounicode{amsquare}{33C2}
\pdfglyphtounicode{anbopomofo}{3122}
\pdfglyphtounicode{angbopomofo}{3124}
\pdfglyphtounicode{angkhankhuthai}{0E5A}
\pdfglyphtounicode{angle}{2220}
\pdfglyphtounicode{anglebracketleft}{3008}
\pdfglyphtounicode{anglebracketleftvertical}{FE3F}
\pdfglyphtounicode{anglebracketright}{3009}
\pdfglyphtounicode{anglebracketrightvertical}{FE40}
\pdfglyphtounicode{angleleft}{2329}
\pdfglyphtounicode{angleright}{232A}
\pdfglyphtounicode{angstrom}{212B}
\pdfglyphtounicode{anoteleia}{0387}
\pdfglyphtounicode{anudattadeva}{0952}
\pdfglyphtounicode{anusvarabengali}{0982}
\pdfglyphtounicode{anusvaradeva}{0902}
\pdfglyphtounicode{anusvaragujarati}{0A82}
\pdfglyphtounicode{aogonek}{0105}
\pdfglyphtounicode{apaatosquare}{3300}
\pdfglyphtounicode{aparen}{249C}
\pdfglyphtounicode{apostrophearmenian}{055A}
\pdfglyphtounicode{apostrophemod}{02BC}
\pdfglyphtounicode{apple}{F8FF}
\pdfglyphtounicode{approaches}{2250}
\pdfglyphtounicode{approxequal}{2248}
\pdfglyphtounicode{approxequalorimage}{2252}
\pdfglyphtounicode{approximatelyequal}{2245}
\pdfglyphtounicode{araeaekorean}{318E}
\pdfglyphtounicode{araeakorean}{318D}
\pdfglyphtounicode{arc}{2312}
\pdfglyphtounicode{arighthalfring}{1E9A}
\pdfglyphtounicode{aring}{00E5}
\pdfglyphtounicode{aringacute}{01FB}
\pdfglyphtounicode{aringbelow}{1E01}
\pdfglyphtounicode{arrowboth}{2194}
\pdfglyphtounicode{arrowdashdown}{21E3}
\pdfglyphtounicode{arrowdashleft}{21E0}
\pdfglyphtounicode{arrowdashright}{21E2}
\pdfglyphtounicode{arrowdashup}{21E1}
\pdfglyphtounicode{arrowdblboth}{21D4}
\pdfglyphtounicode{arrowdbldown}{21D3}
\pdfglyphtounicode{arrowdblleft}{21D0}
\pdfglyphtounicode{arrowdblright}{21D2}
\pdfglyphtounicode{arrowdblup}{21D1}
\pdfglyphtounicode{arrowdown}{2193}
\pdfglyphtounicode{arrowdownleft}{2199}
\pdfglyphtounicode{arrowdownright}{2198}
\pdfglyphtounicode{arrowdownwhite}{21E9}
\pdfglyphtounicode{arrowheaddownmod}{02C5}
\pdfglyphtounicode{arrowheadleftmod}{02C2}
\pdfglyphtounicode{arrowheadrightmod}{02C3}
\pdfglyphtounicode{arrowheadupmod}{02C4}
\pdfglyphtounicode{arrowhorizex}{F8E7}
\pdfglyphtounicode{arrowleft}{2190}
\pdfglyphtounicode{arrowleftdbl}{21D0}
\pdfglyphtounicode{arrowleftdblstroke}{21CD}
\pdfglyphtounicode{arrowleftoverright}{21C6}
\pdfglyphtounicode{arrowleftwhite}{21E6}
\pdfglyphtounicode{arrowright}{2192}
\pdfglyphtounicode{arrowrightdblstroke}{21CF}
\pdfglyphtounicode{arrowrightheavy}{279E}
\pdfglyphtounicode{arrowrightoverleft}{21C4}
\pdfglyphtounicode{arrowrightwhite}{21E8}
\pdfglyphtounicode{arrowtableft}{21E4}
\pdfglyphtounicode{arrowtabright}{21E5}
\pdfglyphtounicode{arrowup}{2191}
\pdfglyphtounicode{arrowupdn}{2195}
\pdfglyphtounicode{arrowupdnbse}{21A8}
\pdfglyphtounicode{arrowupdownbase}{21A8}
\pdfglyphtounicode{arrowupleft}{2196}
\pdfglyphtounicode{arrowupleftofdown}{21C5}
\pdfglyphtounicode{arrowupright}{2197}
\pdfglyphtounicode{arrowupwhite}{21E7}
\pdfglyphtounicode{arrowvertex}{F8E6}
\pdfglyphtounicode{asciicircum}{005E}
\pdfglyphtounicode{asciicircummonospace}{FF3E}
\pdfglyphtounicode{asciitilde}{007E}
\pdfglyphtounicode{asciitildemonospace}{FF5E}
\pdfglyphtounicode{ascript}{0251}
\pdfglyphtounicode{ascriptturned}{0252}
\pdfglyphtounicode{asmallhiragana}{3041}
\pdfglyphtounicode{asmallkatakana}{30A1}
\pdfglyphtounicode{asmallkatakanahalfwidth}{FF67}
\pdfglyphtounicode{asterisk}{002A}
\pdfglyphtounicode{asteriskaltonearabic}{066D}
\pdfglyphtounicode{asteriskarabic}{066D}
\pdfglyphtounicode{asteriskmath}{2217}
\pdfglyphtounicode{asteriskmonospace}{FF0A}
\pdfglyphtounicode{asterisksmall}{FE61}
\pdfglyphtounicode{asterism}{2042}
\pdfglyphtounicode{asuperior}{F6E9}
\pdfglyphtounicode{asymptoticallyequal}{2243}
\pdfglyphtounicode{at}{0040}
\pdfglyphtounicode{atilde}{00E3}
\pdfglyphtounicode{atmonospace}{FF20}
\pdfglyphtounicode{atsmall}{FE6B}
\pdfglyphtounicode{aturned}{0250}
\pdfglyphtounicode{aubengali}{0994}
\pdfglyphtounicode{aubopomofo}{3120}
\pdfglyphtounicode{audeva}{0914}
\pdfglyphtounicode{augujarati}{0A94}
\pdfglyphtounicode{augurmukhi}{0A14}
\pdfglyphtounicode{aulengthmarkbengali}{09D7}
\pdfglyphtounicode{aumatragurmukhi}{0A4C}
\pdfglyphtounicode{auvowelsignbengali}{09CC}
\pdfglyphtounicode{auvowelsigndeva}{094C}
\pdfglyphtounicode{auvowelsigngujarati}{0ACC}
\pdfglyphtounicode{avagrahadeva}{093D}
\pdfglyphtounicode{aybarmenian}{0561}
\pdfglyphtounicode{ayin}{05E2}
\pdfglyphtounicode{ayinaltonehebrew}{FB20}
\pdfglyphtounicode{ayinhebrew}{05E2}
\pdfglyphtounicode{b}{0062}
\pdfglyphtounicode{babengali}{09AC}
\pdfglyphtounicode{backslash}{005C}
\pdfglyphtounicode{backslashmonospace}{FF3C}
\pdfglyphtounicode{badeva}{092C}
\pdfglyphtounicode{bagujarati}{0AAC}
\pdfglyphtounicode{bagurmukhi}{0A2C}
\pdfglyphtounicode{bahiragana}{3070}
\pdfglyphtounicode{bahtthai}{0E3F}
\pdfglyphtounicode{bakatakana}{30D0}
\pdfglyphtounicode{bar}{007C}
\pdfglyphtounicode{barmonospace}{FF5C}
\pdfglyphtounicode{bbopomofo}{3105}
\pdfglyphtounicode{bcircle}{24D1}
\pdfglyphtounicode{bdotaccent}{1E03}
\pdfglyphtounicode{bdotbelow}{1E05}
\pdfglyphtounicode{beamedsixteenthnotes}{266C}
\pdfglyphtounicode{because}{2235}
\pdfglyphtounicode{becyrillic}{0431}
\pdfglyphtounicode{beharabic}{0628}
\pdfglyphtounicode{behfinalarabic}{FE90}
\pdfglyphtounicode{behinitialarabic}{FE91}
\pdfglyphtounicode{behiragana}{3079}
\pdfglyphtounicode{behmedialarabic}{FE92}
\pdfglyphtounicode{behmeeminitialarabic}{FC9F}
\pdfglyphtounicode{behmeemisolatedarabic}{FC08}
\pdfglyphtounicode{behnoonfinalarabic}{FC6D}
\pdfglyphtounicode{bekatakana}{30D9}
\pdfglyphtounicode{benarmenian}{0562}
\pdfglyphtounicode{bet}{05D1}
\pdfglyphtounicode{beta}{03B2}
\pdfglyphtounicode{betasymbolgreek}{03D0}
\pdfglyphtounicode{betdagesh}{FB31}
\pdfglyphtounicode{betdageshhebrew}{FB31}
\pdfglyphtounicode{bethebrew}{05D1}
\pdfglyphtounicode{betrafehebrew}{FB4C}
\pdfglyphtounicode{bhabengali}{09AD}
\pdfglyphtounicode{bhadeva}{092D}
\pdfglyphtounicode{bhagujarati}{0AAD}
\pdfglyphtounicode{bhagurmukhi}{0A2D}
\pdfglyphtounicode{bhook}{0253}
\pdfglyphtounicode{bihiragana}{3073}
\pdfglyphtounicode{bikatakana}{30D3}
\pdfglyphtounicode{bilabialclick}{0298}
\pdfglyphtounicode{bindigurmukhi}{0A02}
\pdfglyphtounicode{birusquare}{3331}
\pdfglyphtounicode{blackcircle}{25CF}
\pdfglyphtounicode{blackdiamond}{25C6}
\pdfglyphtounicode{blackdownpointingtriangle}{25BC}
\pdfglyphtounicode{blackleftpointingpointer}{25C4}
\pdfglyphtounicode{blackleftpointingtriangle}{25C0}
\pdfglyphtounicode{blacklenticularbracketleft}{3010}
\pdfglyphtounicode{blacklenticularbracketleftvertical}{FE3B}
\pdfglyphtounicode{blacklenticularbracketright}{3011}
\pdfglyphtounicode{blacklenticularbracketrightvertical}{FE3C}
\pdfglyphtounicode{blacklowerlefttriangle}{25E3}
\pdfglyphtounicode{blacklowerrighttriangle}{25E2}
\pdfglyphtounicode{blackrectangle}{25AC}
\pdfglyphtounicode{blackrightpointingpointer}{25BA}
\pdfglyphtounicode{blackrightpointingtriangle}{25B6}
\pdfglyphtounicode{blacksmallsquare}{25AA}
\pdfglyphtounicode{blacksmilingface}{263B}
\pdfglyphtounicode{blacksquare}{25A0}
\pdfglyphtounicode{blackstar}{2605}
\pdfglyphtounicode{blackupperlefttriangle}{25E4}
\pdfglyphtounicode{blackupperrighttriangle}{25E5}
\pdfglyphtounicode{blackuppointingsmalltriangle}{25B4}
\pdfglyphtounicode{blackuppointingtriangle}{25B2}
\pdfglyphtounicode{blank}{2423}
\pdfglyphtounicode{blinebelow}{1E07}
\pdfglyphtounicode{block}{2588}
\pdfglyphtounicode{bmonospace}{FF42}
\pdfglyphtounicode{bobaimaithai}{0E1A}
\pdfglyphtounicode{bohiragana}{307C}
\pdfglyphtounicode{bokatakana}{30DC}
\pdfglyphtounicode{bparen}{249D}
\pdfglyphtounicode{bqsquare}{33C3}
\pdfglyphtounicode{braceex}{F8F4}
\pdfglyphtounicode{braceleft}{007B}
\pdfglyphtounicode{braceleftbt}{F8F3}
\pdfglyphtounicode{braceleftmid}{F8F2}
\pdfglyphtounicode{braceleftmonospace}{FF5B}
\pdfglyphtounicode{braceleftsmall}{FE5B}
\pdfglyphtounicode{bracelefttp}{F8F1}
\pdfglyphtounicode{braceleftvertical}{FE37}
\pdfglyphtounicode{braceright}{007D}
\pdfglyphtounicode{bracerightbt}{F8FE}
\pdfglyphtounicode{bracerightmid}{F8FD}
\pdfglyphtounicode{bracerightmonospace}{FF5D}
\pdfglyphtounicode{bracerightsmall}{FE5C}
\pdfglyphtounicode{bracerighttp}{F8FC}
\pdfglyphtounicode{bracerightvertical}{FE38}
\pdfglyphtounicode{bracketleft}{005B}
\pdfglyphtounicode{bracketleftbt}{F8F0}
\pdfglyphtounicode{bracketleftex}{F8EF}
\pdfglyphtounicode{bracketleftmonospace}{FF3B}
\pdfglyphtounicode{bracketlefttp}{F8EE}
\pdfglyphtounicode{bracketright}{005D}
\pdfglyphtounicode{bracketrightbt}{F8FB}
\pdfglyphtounicode{bracketrightex}{F8FA}
\pdfglyphtounicode{bracketrightmonospace}{FF3D}
\pdfglyphtounicode{bracketrighttp}{F8F9}
\pdfglyphtounicode{breve}{02D8}
\pdfglyphtounicode{brevebelowcmb}{032E}
\pdfglyphtounicode{brevecmb}{0306}
\pdfglyphtounicode{breveinvertedbelowcmb}{032F}
\pdfglyphtounicode{breveinvertedcmb}{0311}
\pdfglyphtounicode{breveinverteddoublecmb}{0361}
\pdfglyphtounicode{bridgebelowcmb}{032A}
\pdfglyphtounicode{bridgeinvertedbelowcmb}{033A}
\pdfglyphtounicode{brokenbar}{00A6}
\pdfglyphtounicode{bstroke}{0180}
\pdfglyphtounicode{bsuperior}{F6EA}
\pdfglyphtounicode{btopbar}{0183}
\pdfglyphtounicode{buhiragana}{3076}
\pdfglyphtounicode{bukatakana}{30D6}
\pdfglyphtounicode{bullet}{2022}
\pdfglyphtounicode{bulletinverse}{25D8}
\pdfglyphtounicode{bulletoperator}{2219}
\pdfglyphtounicode{bullseye}{25CE}
\pdfglyphtounicode{c}{0063}
\pdfglyphtounicode{caarmenian}{056E}
\pdfglyphtounicode{cabengali}{099A}
\pdfglyphtounicode{cacute}{0107}
\pdfglyphtounicode{cadeva}{091A}
\pdfglyphtounicode{cagujarati}{0A9A}
\pdfglyphtounicode{cagurmukhi}{0A1A}
\pdfglyphtounicode{calsquare}{3388}
\pdfglyphtounicode{candrabindubengali}{0981}
\pdfglyphtounicode{candrabinducmb}{0310}
\pdfglyphtounicode{candrabindudeva}{0901}
\pdfglyphtounicode{candrabindugujarati}{0A81}
\pdfglyphtounicode{capslock}{21EA}
\pdfglyphtounicode{careof}{2105}
\pdfglyphtounicode{caron}{02C7}
\pdfglyphtounicode{caronbelowcmb}{032C}
\pdfglyphtounicode{caroncmb}{030C}
\pdfglyphtounicode{carriagereturn}{21B5}
\pdfglyphtounicode{cbopomofo}{3118}
\pdfglyphtounicode{ccaron}{010D}
\pdfglyphtounicode{ccedilla}{00E7}
\pdfglyphtounicode{ccedillaacute}{1E09}
\pdfglyphtounicode{ccircle}{24D2}
\pdfglyphtounicode{ccircumflex}{0109}
\pdfglyphtounicode{ccurl}{0255}
\pdfglyphtounicode{cdot}{010B}
\pdfglyphtounicode{cdotaccent}{010B}
\pdfglyphtounicode{cdsquare}{33C5}
\pdfglyphtounicode{cedilla}{00B8}
\pdfglyphtounicode{cedillacmb}{0327}
\pdfglyphtounicode{cent}{00A2}
\pdfglyphtounicode{centigrade}{2103}
\pdfglyphtounicode{centinferior}{F6DF}
\pdfglyphtounicode{centmonospace}{FFE0}
\pdfglyphtounicode{centoldstyle}{F7A2}
\pdfglyphtounicode{centsuperior}{F6E0}
\pdfglyphtounicode{chaarmenian}{0579}
\pdfglyphtounicode{chabengali}{099B}
\pdfglyphtounicode{chadeva}{091B}
\pdfglyphtounicode{chagujarati}{0A9B}
\pdfglyphtounicode{chagurmukhi}{0A1B}
\pdfglyphtounicode{chbopomofo}{3114}
\pdfglyphtounicode{cheabkhasiancyrillic}{04BD}
\pdfglyphtounicode{checkmark}{2713}
\pdfglyphtounicode{checyrillic}{0447}
\pdfglyphtounicode{chedescenderabkhasiancyrillic}{04BF}
\pdfglyphtounicode{chedescendercyrillic}{04B7}
\pdfglyphtounicode{chedieresiscyrillic}{04F5}
\pdfglyphtounicode{cheharmenian}{0573}
\pdfglyphtounicode{chekhakassiancyrillic}{04CC}
\pdfglyphtounicode{cheverticalstrokecyrillic}{04B9}
\pdfglyphtounicode{chi}{03C7}
\pdfglyphtounicode{chieuchacirclekorean}{3277}
\pdfglyphtounicode{chieuchaparenkorean}{3217}
\pdfglyphtounicode{chieuchcirclekorean}{3269}
\pdfglyphtounicode{chieuchkorean}{314A}
\pdfglyphtounicode{chieuchparenkorean}{3209}
\pdfglyphtounicode{chochangthai}{0E0A}
\pdfglyphtounicode{chochanthai}{0E08}
\pdfglyphtounicode{chochingthai}{0E09}
\pdfglyphtounicode{chochoethai}{0E0C}
\pdfglyphtounicode{chook}{0188}
\pdfglyphtounicode{cieucacirclekorean}{3276}
\pdfglyphtounicode{cieucaparenkorean}{3216}
\pdfglyphtounicode{cieuccirclekorean}{3268}
\pdfglyphtounicode{cieuckorean}{3148}
\pdfglyphtounicode{cieucparenkorean}{3208}
\pdfglyphtounicode{cieucuparenkorean}{321C}
\pdfglyphtounicode{circle}{25CB}
\pdfglyphtounicode{circlemultiply}{2297}
\pdfglyphtounicode{circleot}{2299}
\pdfglyphtounicode{circleplus}{2295}
\pdfglyphtounicode{circlepostalmark}{3036}
\pdfglyphtounicode{circlewithlefthalfblack}{25D0}
\pdfglyphtounicode{circlewithrighthalfblack}{25D1}
\pdfglyphtounicode{circumflex}{02C6}
\pdfglyphtounicode{circumflexbelowcmb}{032D}
\pdfglyphtounicode{circumflexcmb}{0302}
\pdfglyphtounicode{clear}{2327}
\pdfglyphtounicode{clickalveolar}{01C2}
\pdfglyphtounicode{clickdental}{01C0}
\pdfglyphtounicode{clicklateral}{01C1}
\pdfglyphtounicode{clickretroflex}{01C3}
\pdfglyphtounicode{club}{2663}
\pdfglyphtounicode{clubsuitblack}{2663}
\pdfglyphtounicode{clubsuitwhite}{2667}
\pdfglyphtounicode{cmcubedsquare}{33A4}
\pdfglyphtounicode{cmonospace}{FF43}
\pdfglyphtounicode{cmsquaredsquare}{33A0}
\pdfglyphtounicode{coarmenian}{0581}
\pdfglyphtounicode{colon}{003A}
\pdfglyphtounicode{colonmonetary}{20A1}
\pdfglyphtounicode{colonmonospace}{FF1A}
\pdfglyphtounicode{colonsign}{20A1}
\pdfglyphtounicode{colonsmall}{FE55}
\pdfglyphtounicode{colontriangularhalfmod}{02D1}
\pdfglyphtounicode{colontriangularmod}{02D0}
\pdfglyphtounicode{comma}{002C}
\pdfglyphtounicode{commaabovecmb}{0313}
\pdfglyphtounicode{commaaboverightcmb}{0315}
\pdfglyphtounicode{commaaccent}{F6C3}
\pdfglyphtounicode{commaarabic}{060C}
\pdfglyphtounicode{commaarmenian}{055D}
\pdfglyphtounicode{commainferior}{F6E1}
\pdfglyphtounicode{commamonospace}{FF0C}
\pdfglyphtounicode{commareversedabovecmb}{0314}
\pdfglyphtounicode{commareversedmod}{02BD}
\pdfglyphtounicode{commasmall}{FE50}
\pdfglyphtounicode{commasuperior}{F6E2}
\pdfglyphtounicode{commaturnedabovecmb}{0312}
\pdfglyphtounicode{commaturnedmod}{02BB}
\pdfglyphtounicode{compass}{263C}
\pdfglyphtounicode{congruent}{2245}
\pdfglyphtounicode{contourintegral}{222E}
\pdfglyphtounicode{control}{2303}
\pdfglyphtounicode{controlACK}{0006}
\pdfglyphtounicode{controlBEL}{0007}
\pdfglyphtounicode{controlBS}{0008}
\pdfglyphtounicode{controlCAN}{0018}
\pdfglyphtounicode{controlCR}{000D}
\pdfglyphtounicode{controlDC1}{0011}
\pdfglyphtounicode{controlDC2}{0012}
\pdfglyphtounicode{controlDC3}{0013}
\pdfglyphtounicode{controlDC4}{0014}
\pdfglyphtounicode{controlDEL}{007F}
\pdfglyphtounicode{controlDLE}{0010}
\pdfglyphtounicode{controlEM}{0019}
\pdfglyphtounicode{controlENQ}{0005}
\pdfglyphtounicode{controlEOT}{0004}
\pdfglyphtounicode{controlESC}{001B}
\pdfglyphtounicode{controlETB}{0017}
\pdfglyphtounicode{controlETX}{0003}
\pdfglyphtounicode{controlFF}{000C}
\pdfglyphtounicode{controlFS}{001C}
\pdfglyphtounicode{controlGS}{001D}
\pdfglyphtounicode{controlHT}{0009}
\pdfglyphtounicode{controlLF}{000A}
\pdfglyphtounicode{controlNAK}{0015}
\pdfglyphtounicode{controlRS}{001E}
\pdfglyphtounicode{controlSI}{000F}
\pdfglyphtounicode{controlSO}{000E}
\pdfglyphtounicode{controlSOT}{0002}
\pdfglyphtounicode{controlSTX}{0001}
\pdfglyphtounicode{controlSUB}{001A}
\pdfglyphtounicode{controlSYN}{0016}
\pdfglyphtounicode{controlUS}{001F}
\pdfglyphtounicode{controlVT}{000B}
\pdfglyphtounicode{copyright}{00A9}
\pdfglyphtounicode{copyrightsans}{F8E9}
\pdfglyphtounicode{copyrightserif}{F6D9}
\pdfglyphtounicode{cornerbracketleft}{300C}
\pdfglyphtounicode{cornerbracketlefthalfwidth}{FF62}
\pdfglyphtounicode{cornerbracketleftvertical}{FE41}
\pdfglyphtounicode{cornerbracketright}{300D}
\pdfglyphtounicode{cornerbracketrighthalfwidth}{FF63}
\pdfglyphtounicode{cornerbracketrightvertical}{FE42}
\pdfglyphtounicode{corporationsquare}{337F}
\pdfglyphtounicode{cosquare}{33C7}
\pdfglyphtounicode{coverkgsquare}{33C6}
\pdfglyphtounicode{cparen}{249E}
\pdfglyphtounicode{cruzeiro}{20A2}
\pdfglyphtounicode{cstretched}{0297}
\pdfglyphtounicode{curlyand}{22CF}
\pdfglyphtounicode{curlyor}{22CE}
\pdfglyphtounicode{currency}{00A4}
\pdfglyphtounicode{cyrBreve}{F6D1}
\pdfglyphtounicode{cyrFlex}{F6D2}
\pdfglyphtounicode{cyrbreve}{F6D4}
\pdfglyphtounicode{cyrflex}{F6D5}
\pdfglyphtounicode{d}{0064}
\pdfglyphtounicode{daarmenian}{0564}
\pdfglyphtounicode{dabengali}{09A6}
\pdfglyphtounicode{dadarabic}{0636}
\pdfglyphtounicode{dadeva}{0926}
\pdfglyphtounicode{dadfinalarabic}{FEBE}
\pdfglyphtounicode{dadinitialarabic}{FEBF}
\pdfglyphtounicode{dadmedialarabic}{FEC0}
\pdfglyphtounicode{dagesh}{05BC}
\pdfglyphtounicode{dageshhebrew}{05BC}
\pdfglyphtounicode{dagger}{2020}
\pdfglyphtounicode{daggerdbl}{2021}
\pdfglyphtounicode{dagujarati}{0AA6}
\pdfglyphtounicode{dagurmukhi}{0A26}
\pdfglyphtounicode{dahiragana}{3060}
\pdfglyphtounicode{dakatakana}{30C0}
\pdfglyphtounicode{dalarabic}{062F}
\pdfglyphtounicode{dalet}{05D3}
\pdfglyphtounicode{daletdagesh}{FB33}
\pdfglyphtounicode{daletdageshhebrew}{FB33}
% dalethatafpatah;05D3 05B2
% dalethatafpatahhebrew;05D3 05B2
% dalethatafsegol;05D3 05B1
% dalethatafsegolhebrew;05D3 05B1
\pdfglyphtounicode{dalethebrew}{05D3}
% dalethiriq;05D3 05B4
% dalethiriqhebrew;05D3 05B4
% daletholam;05D3 05B9
% daletholamhebrew;05D3 05B9
% daletpatah;05D3 05B7
% daletpatahhebrew;05D3 05B7
% daletqamats;05D3 05B8
% daletqamatshebrew;05D3 05B8
% daletqubuts;05D3 05BB
% daletqubutshebrew;05D3 05BB
% daletsegol;05D3 05B6
% daletsegolhebrew;05D3 05B6
% daletsheva;05D3 05B0
% daletshevahebrew;05D3 05B0
% dalettsere;05D3 05B5
% dalettserehebrew;05D3 05B5
\pdfglyphtounicode{dalfinalarabic}{FEAA}
\pdfglyphtounicode{dammaarabic}{064F}
\pdfglyphtounicode{dammalowarabic}{064F}
\pdfglyphtounicode{dammatanaltonearabic}{064C}
\pdfglyphtounicode{dammatanarabic}{064C}
\pdfglyphtounicode{danda}{0964}
\pdfglyphtounicode{dargahebrew}{05A7}
\pdfglyphtounicode{dargalefthebrew}{05A7}
\pdfglyphtounicode{dasiapneumatacyrilliccmb}{0485}
\pdfglyphtounicode{dblGrave}{F6D3}
\pdfglyphtounicode{dblanglebracketleft}{300A}
\pdfglyphtounicode{dblanglebracketleftvertical}{FE3D}
\pdfglyphtounicode{dblanglebracketright}{300B}
\pdfglyphtounicode{dblanglebracketrightvertical}{FE3E}
\pdfglyphtounicode{dblarchinvertedbelowcmb}{032B}
\pdfglyphtounicode{dblarrowleft}{21D4}
\pdfglyphtounicode{dblarrowright}{21D2}
\pdfglyphtounicode{dbldanda}{0965}
\pdfglyphtounicode{dblgrave}{F6D6}
\pdfglyphtounicode{dblgravecmb}{030F}
\pdfglyphtounicode{dblintegral}{222C}
\pdfglyphtounicode{dbllowline}{2017}
\pdfglyphtounicode{dbllowlinecmb}{0333}
\pdfglyphtounicode{dbloverlinecmb}{033F}
\pdfglyphtounicode{dblprimemod}{02BA}
\pdfglyphtounicode{dblverticalbar}{2016}
\pdfglyphtounicode{dblverticallineabovecmb}{030E}
\pdfglyphtounicode{dbopomofo}{3109}
\pdfglyphtounicode{dbsquare}{33C8}
\pdfglyphtounicode{dcaron}{010F}
\pdfglyphtounicode{dcedilla}{1E11}
\pdfglyphtounicode{dcircle}{24D3}
\pdfglyphtounicode{dcircumflexbelow}{1E13}
\pdfglyphtounicode{dcroat}{0111}
\pdfglyphtounicode{ddabengali}{09A1}
\pdfglyphtounicode{ddadeva}{0921}
\pdfglyphtounicode{ddagujarati}{0AA1}
\pdfglyphtounicode{ddagurmukhi}{0A21}
\pdfglyphtounicode{ddalarabic}{0688}
\pdfglyphtounicode{ddalfinalarabic}{FB89}
\pdfglyphtounicode{dddhadeva}{095C}
\pdfglyphtounicode{ddhabengali}{09A2}
\pdfglyphtounicode{ddhadeva}{0922}
\pdfglyphtounicode{ddhagujarati}{0AA2}
\pdfglyphtounicode{ddhagurmukhi}{0A22}
\pdfglyphtounicode{ddotaccent}{1E0B}
\pdfglyphtounicode{ddotbelow}{1E0D}
\pdfglyphtounicode{decimalseparatorarabic}{066B}
\pdfglyphtounicode{decimalseparatorpersian}{066B}
\pdfglyphtounicode{decyrillic}{0434}
\pdfglyphtounicode{degree}{00B0}
\pdfglyphtounicode{dehihebrew}{05AD}
\pdfglyphtounicode{dehiragana}{3067}
\pdfglyphtounicode{deicoptic}{03EF}
\pdfglyphtounicode{dekatakana}{30C7}
\pdfglyphtounicode{deleteleft}{232B}
\pdfglyphtounicode{deleteright}{2326}
\pdfglyphtounicode{delta}{03B4}
\pdfglyphtounicode{deltaturned}{018D}
\pdfglyphtounicode{denominatorminusonenumeratorbengali}{09F8}
\pdfglyphtounicode{dezh}{02A4}
\pdfglyphtounicode{dhabengali}{09A7}
\pdfglyphtounicode{dhadeva}{0927}
\pdfglyphtounicode{dhagujarati}{0AA7}
\pdfglyphtounicode{dhagurmukhi}{0A27}
\pdfglyphtounicode{dhook}{0257}
\pdfglyphtounicode{dialytikatonos}{0385}
\pdfglyphtounicode{dialytikatonoscmb}{0344}
\pdfglyphtounicode{diamond}{2666}
\pdfglyphtounicode{diamondsuitwhite}{2662}
\pdfglyphtounicode{dieresis}{00A8}
\pdfglyphtounicode{dieresisacute}{F6D7}
\pdfglyphtounicode{dieresisbelowcmb}{0324}
\pdfglyphtounicode{dieresiscmb}{0308}
\pdfglyphtounicode{dieresisgrave}{F6D8}
\pdfglyphtounicode{dieresistonos}{0385}
\pdfglyphtounicode{dihiragana}{3062}
\pdfglyphtounicode{dikatakana}{30C2}
\pdfglyphtounicode{dittomark}{3003}
\pdfglyphtounicode{divide}{00F7}
\pdfglyphtounicode{divides}{2223}
\pdfglyphtounicode{divisionslash}{2215}
\pdfglyphtounicode{djecyrillic}{0452}
\pdfglyphtounicode{dkshade}{2593}
\pdfglyphtounicode{dlinebelow}{1E0F}
\pdfglyphtounicode{dlsquare}{3397}
\pdfglyphtounicode{dmacron}{0111}
\pdfglyphtounicode{dmonospace}{FF44}
\pdfglyphtounicode{dnblock}{2584}
\pdfglyphtounicode{dochadathai}{0E0E}
\pdfglyphtounicode{dodekthai}{0E14}
\pdfglyphtounicode{dohiragana}{3069}
\pdfglyphtounicode{dokatakana}{30C9}
\pdfglyphtounicode{dollar}{0024}
\pdfglyphtounicode{dollarinferior}{F6E3}
\pdfglyphtounicode{dollarmonospace}{FF04}
\pdfglyphtounicode{dollaroldstyle}{F724}
\pdfglyphtounicode{dollarsmall}{FE69}
\pdfglyphtounicode{dollarsuperior}{F6E4}
\pdfglyphtounicode{dong}{20AB}
\pdfglyphtounicode{dorusquare}{3326}
\pdfglyphtounicode{dotaccent}{02D9}
\pdfglyphtounicode{dotaccentcmb}{0307}
\pdfglyphtounicode{dotbelowcmb}{0323}
\pdfglyphtounicode{dotbelowcomb}{0323}
\pdfglyphtounicode{dotkatakana}{30FB}
\pdfglyphtounicode{dotlessi}{0131}
\pdfglyphtounicode{dotlessj}{F6BE}
\pdfglyphtounicode{dotlessjstrokehook}{0284}
\pdfglyphtounicode{dotmath}{22C5}
\pdfglyphtounicode{dottedcircle}{25CC}
\pdfglyphtounicode{doubleyodpatah}{FB1F}
\pdfglyphtounicode{doubleyodpatahhebrew}{FB1F}
\pdfglyphtounicode{downtackbelowcmb}{031E}
\pdfglyphtounicode{downtackmod}{02D5}
\pdfglyphtounicode{dparen}{249F}
\pdfglyphtounicode{dsuperior}{F6EB}
\pdfglyphtounicode{dtail}{0256}
\pdfglyphtounicode{dtopbar}{018C}
\pdfglyphtounicode{duhiragana}{3065}
\pdfglyphtounicode{dukatakana}{30C5}
\pdfglyphtounicode{dz}{01F3}
\pdfglyphtounicode{dzaltone}{02A3}
\pdfglyphtounicode{dzcaron}{01C6}
\pdfglyphtounicode{dzcurl}{02A5}
\pdfglyphtounicode{dzeabkhasiancyrillic}{04E1}
\pdfglyphtounicode{dzecyrillic}{0455}
\pdfglyphtounicode{dzhecyrillic}{045F}
\pdfglyphtounicode{e}{0065}
\pdfglyphtounicode{eacute}{00E9}
\pdfglyphtounicode{earth}{2641}
\pdfglyphtounicode{ebengali}{098F}
\pdfglyphtounicode{ebopomofo}{311C}
\pdfglyphtounicode{ebreve}{0115}
\pdfglyphtounicode{ecandradeva}{090D}
\pdfglyphtounicode{ecandragujarati}{0A8D}
\pdfglyphtounicode{ecandravowelsigndeva}{0945}
\pdfglyphtounicode{ecandravowelsigngujarati}{0AC5}
\pdfglyphtounicode{ecaron}{011B}
\pdfglyphtounicode{ecedillabreve}{1E1D}
\pdfglyphtounicode{echarmenian}{0565}
\pdfglyphtounicode{echyiwnarmenian}{0587}
\pdfglyphtounicode{ecircle}{24D4}
\pdfglyphtounicode{ecircumflex}{00EA}
\pdfglyphtounicode{ecircumflexacute}{1EBF}
\pdfglyphtounicode{ecircumflexbelow}{1E19}
\pdfglyphtounicode{ecircumflexdotbelow}{1EC7}
\pdfglyphtounicode{ecircumflexgrave}{1EC1}
\pdfglyphtounicode{ecircumflexhookabove}{1EC3}
\pdfglyphtounicode{ecircumflextilde}{1EC5}
\pdfglyphtounicode{ecyrillic}{0454}
\pdfglyphtounicode{edblgrave}{0205}
\pdfglyphtounicode{edeva}{090F}
\pdfglyphtounicode{edieresis}{00EB}
\pdfglyphtounicode{edot}{0117}
\pdfglyphtounicode{edotaccent}{0117}
\pdfglyphtounicode{edotbelow}{1EB9}
\pdfglyphtounicode{eegurmukhi}{0A0F}
\pdfglyphtounicode{eematragurmukhi}{0A47}
\pdfglyphtounicode{efcyrillic}{0444}
\pdfglyphtounicode{egrave}{00E8}
\pdfglyphtounicode{egujarati}{0A8F}
\pdfglyphtounicode{eharmenian}{0567}
\pdfglyphtounicode{ehbopomofo}{311D}
\pdfglyphtounicode{ehiragana}{3048}
\pdfglyphtounicode{ehookabove}{1EBB}
\pdfglyphtounicode{eibopomofo}{311F}
\pdfglyphtounicode{eight}{0038}
\pdfglyphtounicode{eightarabic}{0668}
\pdfglyphtounicode{eightbengali}{09EE}
\pdfglyphtounicode{eightcircle}{2467}
\pdfglyphtounicode{eightcircleinversesansserif}{2791}
\pdfglyphtounicode{eightdeva}{096E}
\pdfglyphtounicode{eighteencircle}{2471}
\pdfglyphtounicode{eighteenparen}{2485}
\pdfglyphtounicode{eighteenperiod}{2499}
\pdfglyphtounicode{eightgujarati}{0AEE}
\pdfglyphtounicode{eightgurmukhi}{0A6E}
\pdfglyphtounicode{eighthackarabic}{0668}
\pdfglyphtounicode{eighthangzhou}{3028}
\pdfglyphtounicode{eighthnotebeamed}{266B}
\pdfglyphtounicode{eightideographicparen}{3227}
\pdfglyphtounicode{eightinferior}{2088}
\pdfglyphtounicode{eightmonospace}{FF18}
\pdfglyphtounicode{eightoldstyle}{F738}
\pdfglyphtounicode{eightparen}{247B}
\pdfglyphtounicode{eightperiod}{248F}
\pdfglyphtounicode{eightpersian}{06F8}
\pdfglyphtounicode{eightroman}{2177}
\pdfglyphtounicode{eightsuperior}{2078}
\pdfglyphtounicode{eightthai}{0E58}
\pdfglyphtounicode{einvertedbreve}{0207}
\pdfglyphtounicode{eiotifiedcyrillic}{0465}
\pdfglyphtounicode{ekatakana}{30A8}
\pdfglyphtounicode{ekatakanahalfwidth}{FF74}
\pdfglyphtounicode{ekonkargurmukhi}{0A74}
\pdfglyphtounicode{ekorean}{3154}
\pdfglyphtounicode{elcyrillic}{043B}
\pdfglyphtounicode{element}{2208}
\pdfglyphtounicode{elevencircle}{246A}
\pdfglyphtounicode{elevenparen}{247E}
\pdfglyphtounicode{elevenperiod}{2492}
\pdfglyphtounicode{elevenroman}{217A}
\pdfglyphtounicode{ellipsis}{2026}
\pdfglyphtounicode{ellipsisvertical}{22EE}
\pdfglyphtounicode{emacron}{0113}
\pdfglyphtounicode{emacronacute}{1E17}
\pdfglyphtounicode{emacrongrave}{1E15}
\pdfglyphtounicode{emcyrillic}{043C}
\pdfglyphtounicode{emdash}{2014}
\pdfglyphtounicode{emdashvertical}{FE31}
\pdfglyphtounicode{emonospace}{FF45}
\pdfglyphtounicode{emphasismarkarmenian}{055B}
\pdfglyphtounicode{emptyset}{2205}
\pdfglyphtounicode{enbopomofo}{3123}
\pdfglyphtounicode{encyrillic}{043D}
\pdfglyphtounicode{endash}{2013}
\pdfglyphtounicode{endashvertical}{FE32}
\pdfglyphtounicode{endescendercyrillic}{04A3}
\pdfglyphtounicode{eng}{014B}
\pdfglyphtounicode{engbopomofo}{3125}
\pdfglyphtounicode{enghecyrillic}{04A5}
\pdfglyphtounicode{enhookcyrillic}{04C8}
\pdfglyphtounicode{enspace}{2002}
\pdfglyphtounicode{eogonek}{0119}
\pdfglyphtounicode{eokorean}{3153}
\pdfglyphtounicode{eopen}{025B}
\pdfglyphtounicode{eopenclosed}{029A}
\pdfglyphtounicode{eopenreversed}{025C}
\pdfglyphtounicode{eopenreversedclosed}{025E}
\pdfglyphtounicode{eopenreversedhook}{025D}
\pdfglyphtounicode{eparen}{24A0}
\pdfglyphtounicode{epsilon}{03B5}
\pdfglyphtounicode{epsilontonos}{03AD}
\pdfglyphtounicode{equal}{003D}
\pdfglyphtounicode{equalmonospace}{FF1D}
\pdfglyphtounicode{equalsmall}{FE66}
\pdfglyphtounicode{equalsuperior}{207C}
\pdfglyphtounicode{equivalence}{2261}
\pdfglyphtounicode{erbopomofo}{3126}
\pdfglyphtounicode{ercyrillic}{0440}
\pdfglyphtounicode{ereversed}{0258}
\pdfglyphtounicode{ereversedcyrillic}{044D}
\pdfglyphtounicode{escyrillic}{0441}
\pdfglyphtounicode{esdescendercyrillic}{04AB}
\pdfglyphtounicode{esh}{0283}
\pdfglyphtounicode{eshcurl}{0286}
\pdfglyphtounicode{eshortdeva}{090E}
\pdfglyphtounicode{eshortvowelsigndeva}{0946}
\pdfglyphtounicode{eshreversedloop}{01AA}
\pdfglyphtounicode{eshsquatreversed}{0285}
\pdfglyphtounicode{esmallhiragana}{3047}
\pdfglyphtounicode{esmallkatakana}{30A7}
\pdfglyphtounicode{esmallkatakanahalfwidth}{FF6A}
\pdfglyphtounicode{estimated}{212E}
\pdfglyphtounicode{esuperior}{F6EC}
\pdfglyphtounicode{eta}{03B7}
\pdfglyphtounicode{etarmenian}{0568}
\pdfglyphtounicode{etatonos}{03AE}
\pdfglyphtounicode{eth}{00F0}
\pdfglyphtounicode{etilde}{1EBD}
\pdfglyphtounicode{etildebelow}{1E1B}
\pdfglyphtounicode{etnahtafoukhhebrew}{0591}
\pdfglyphtounicode{etnahtafoukhlefthebrew}{0591}
\pdfglyphtounicode{etnahtahebrew}{0591}
\pdfglyphtounicode{etnahtalefthebrew}{0591}
\pdfglyphtounicode{eturned}{01DD}
\pdfglyphtounicode{eukorean}{3161}
\pdfglyphtounicode{euro}{20AC}
\pdfglyphtounicode{evowelsignbengali}{09C7}
\pdfglyphtounicode{evowelsigndeva}{0947}
\pdfglyphtounicode{evowelsigngujarati}{0AC7}
\pdfglyphtounicode{exclam}{0021}
\pdfglyphtounicode{exclamarmenian}{055C}
\pdfglyphtounicode{exclamdbl}{203C}
\pdfglyphtounicode{exclamdown}{00A1}
\pdfglyphtounicode{exclamdownsmall}{F7A1}
\pdfglyphtounicode{exclammonospace}{FF01}
\pdfglyphtounicode{exclamsmall}{F721}
\pdfglyphtounicode{existential}{2203}
\pdfglyphtounicode{ezh}{0292}
\pdfglyphtounicode{ezhcaron}{01EF}
\pdfglyphtounicode{ezhcurl}{0293}
\pdfglyphtounicode{ezhreversed}{01B9}
\pdfglyphtounicode{ezhtail}{01BA}
\pdfglyphtounicode{f}{0066}
\pdfglyphtounicode{fadeva}{095E}
\pdfglyphtounicode{fagurmukhi}{0A5E}
\pdfglyphtounicode{fahrenheit}{2109}
\pdfglyphtounicode{fathaarabic}{064E}
\pdfglyphtounicode{fathalowarabic}{064E}
\pdfglyphtounicode{fathatanarabic}{064B}
\pdfglyphtounicode{fbopomofo}{3108}
\pdfglyphtounicode{fcircle}{24D5}
\pdfglyphtounicode{fdotaccent}{1E1F}
\pdfglyphtounicode{feharabic}{0641}
\pdfglyphtounicode{feharmenian}{0586}
\pdfglyphtounicode{fehfinalarabic}{FED2}
\pdfglyphtounicode{fehinitialarabic}{FED3}
\pdfglyphtounicode{fehmedialarabic}{FED4}
\pdfglyphtounicode{feicoptic}{03E5}
\pdfglyphtounicode{female}{2640}
\pdfglyphtounicode{ff}{FB00}
\pdfglyphtounicode{ffi}{FB03}
\pdfglyphtounicode{ffl}{FB04}
\pdfglyphtounicode{fi}{FB01}
\pdfglyphtounicode{fifteencircle}{246E}
\pdfglyphtounicode{fifteenparen}{2482}
\pdfglyphtounicode{fifteenperiod}{2496}
\pdfglyphtounicode{figuredash}{2012}
\pdfglyphtounicode{filledbox}{25A0}
\pdfglyphtounicode{filledrect}{25AC}
\pdfglyphtounicode{finalkaf}{05DA}
\pdfglyphtounicode{finalkafdagesh}{FB3A}
\pdfglyphtounicode{finalkafdageshhebrew}{FB3A}
\pdfglyphtounicode{finalkafhebrew}{05DA}
% finalkafqamats;05DA 05B8
% finalkafqamatshebrew;05DA 05B8
% finalkafsheva;05DA 05B0
% finalkafshevahebrew;05DA 05B0
\pdfglyphtounicode{finalmem}{05DD}
\pdfglyphtounicode{finalmemhebrew}{05DD}
\pdfglyphtounicode{finalnun}{05DF}
\pdfglyphtounicode{finalnunhebrew}{05DF}
\pdfglyphtounicode{finalpe}{05E3}
\pdfglyphtounicode{finalpehebrew}{05E3}
\pdfglyphtounicode{finaltsadi}{05E5}
\pdfglyphtounicode{finaltsadihebrew}{05E5}
\pdfglyphtounicode{firsttonechinese}{02C9}
\pdfglyphtounicode{fisheye}{25C9}
\pdfglyphtounicode{fitacyrillic}{0473}
\pdfglyphtounicode{five}{0035}
\pdfglyphtounicode{fivearabic}{0665}
\pdfglyphtounicode{fivebengali}{09EB}
\pdfglyphtounicode{fivecircle}{2464}
\pdfglyphtounicode{fivecircleinversesansserif}{278E}
\pdfglyphtounicode{fivedeva}{096B}
\pdfglyphtounicode{fiveeighths}{215D}
\pdfglyphtounicode{fivegujarati}{0AEB}
\pdfglyphtounicode{fivegurmukhi}{0A6B}
\pdfglyphtounicode{fivehackarabic}{0665}
\pdfglyphtounicode{fivehangzhou}{3025}
\pdfglyphtounicode{fiveideographicparen}{3224}
\pdfglyphtounicode{fiveinferior}{2085}
\pdfglyphtounicode{fivemonospace}{FF15}
\pdfglyphtounicode{fiveoldstyle}{F735}
\pdfglyphtounicode{fiveparen}{2478}
\pdfglyphtounicode{fiveperiod}{248C}
\pdfglyphtounicode{fivepersian}{06F5}
\pdfglyphtounicode{fiveroman}{2174}
\pdfglyphtounicode{fivesuperior}{2075}
\pdfglyphtounicode{fivethai}{0E55}
\pdfglyphtounicode{fl}{FB02}
\pdfglyphtounicode{florin}{0192}
\pdfglyphtounicode{fmonospace}{FF46}
\pdfglyphtounicode{fmsquare}{3399}
\pdfglyphtounicode{fofanthai}{0E1F}
\pdfglyphtounicode{fofathai}{0E1D}
\pdfglyphtounicode{fongmanthai}{0E4F}
\pdfglyphtounicode{forall}{2200}
\pdfglyphtounicode{four}{0034}
\pdfglyphtounicode{fourarabic}{0664}
\pdfglyphtounicode{fourbengali}{09EA}
\pdfglyphtounicode{fourcircle}{2463}
\pdfglyphtounicode{fourcircleinversesansserif}{278D}
\pdfglyphtounicode{fourdeva}{096A}
\pdfglyphtounicode{fourgujarati}{0AEA}
\pdfglyphtounicode{fourgurmukhi}{0A6A}
\pdfglyphtounicode{fourhackarabic}{0664}
\pdfglyphtounicode{fourhangzhou}{3024}
\pdfglyphtounicode{fourideographicparen}{3223}
\pdfglyphtounicode{fourinferior}{2084}
\pdfglyphtounicode{fourmonospace}{FF14}
\pdfglyphtounicode{fournumeratorbengali}{09F7}
\pdfglyphtounicode{fouroldstyle}{F734}
\pdfglyphtounicode{fourparen}{2477}
\pdfglyphtounicode{fourperiod}{248B}
\pdfglyphtounicode{fourpersian}{06F4}
\pdfglyphtounicode{fourroman}{2173}
\pdfglyphtounicode{foursuperior}{2074}
\pdfglyphtounicode{fourteencircle}{246D}
\pdfglyphtounicode{fourteenparen}{2481}
\pdfglyphtounicode{fourteenperiod}{2495}
\pdfglyphtounicode{fourthai}{0E54}
\pdfglyphtounicode{fourthtonechinese}{02CB}
\pdfglyphtounicode{fparen}{24A1}
\pdfglyphtounicode{fraction}{2044}
\pdfglyphtounicode{franc}{20A3}
\pdfglyphtounicode{g}{0067}
\pdfglyphtounicode{gabengali}{0997}
\pdfglyphtounicode{gacute}{01F5}
\pdfglyphtounicode{gadeva}{0917}
\pdfglyphtounicode{gafarabic}{06AF}
\pdfglyphtounicode{gaffinalarabic}{FB93}
\pdfglyphtounicode{gafinitialarabic}{FB94}
\pdfglyphtounicode{gafmedialarabic}{FB95}
\pdfglyphtounicode{gagujarati}{0A97}
\pdfglyphtounicode{gagurmukhi}{0A17}
\pdfglyphtounicode{gahiragana}{304C}
\pdfglyphtounicode{gakatakana}{30AC}
\pdfglyphtounicode{gamma}{03B3}
\pdfglyphtounicode{gammalatinsmall}{0263}
\pdfglyphtounicode{gammasuperior}{02E0}
\pdfglyphtounicode{gangiacoptic}{03EB}
\pdfglyphtounicode{gbopomofo}{310D}
\pdfglyphtounicode{gbreve}{011F}
\pdfglyphtounicode{gcaron}{01E7}
\pdfglyphtounicode{gcedilla}{0123}
\pdfglyphtounicode{gcircle}{24D6}
\pdfglyphtounicode{gcircumflex}{011D}
\pdfglyphtounicode{gcommaaccent}{0123}
\pdfglyphtounicode{gdot}{0121}
\pdfglyphtounicode{gdotaccent}{0121}
\pdfglyphtounicode{gecyrillic}{0433}
\pdfglyphtounicode{gehiragana}{3052}
\pdfglyphtounicode{gekatakana}{30B2}
\pdfglyphtounicode{geometricallyequal}{2251}
\pdfglyphtounicode{gereshaccenthebrew}{059C}
\pdfglyphtounicode{gereshhebrew}{05F3}
\pdfglyphtounicode{gereshmuqdamhebrew}{059D}
\pdfglyphtounicode{germandbls}{00DF}
\pdfglyphtounicode{gershayimaccenthebrew}{059E}
\pdfglyphtounicode{gershayimhebrew}{05F4}
\pdfglyphtounicode{getamark}{3013}
\pdfglyphtounicode{ghabengali}{0998}
\pdfglyphtounicode{ghadarmenian}{0572}
\pdfglyphtounicode{ghadeva}{0918}
\pdfglyphtounicode{ghagujarati}{0A98}
\pdfglyphtounicode{ghagurmukhi}{0A18}
\pdfglyphtounicode{ghainarabic}{063A}
\pdfglyphtounicode{ghainfinalarabic}{FECE}
\pdfglyphtounicode{ghaininitialarabic}{FECF}
\pdfglyphtounicode{ghainmedialarabic}{FED0}
\pdfglyphtounicode{ghemiddlehookcyrillic}{0495}
\pdfglyphtounicode{ghestrokecyrillic}{0493}
\pdfglyphtounicode{gheupturncyrillic}{0491}
\pdfglyphtounicode{ghhadeva}{095A}
\pdfglyphtounicode{ghhagurmukhi}{0A5A}
\pdfglyphtounicode{ghook}{0260}
\pdfglyphtounicode{ghzsquare}{3393}
\pdfglyphtounicode{gihiragana}{304E}
\pdfglyphtounicode{gikatakana}{30AE}
\pdfglyphtounicode{gimarmenian}{0563}
\pdfglyphtounicode{gimel}{05D2}
\pdfglyphtounicode{gimeldagesh}{FB32}
\pdfglyphtounicode{gimeldageshhebrew}{FB32}
\pdfglyphtounicode{gimelhebrew}{05D2}
\pdfglyphtounicode{gjecyrillic}{0453}
\pdfglyphtounicode{glottalinvertedstroke}{01BE}
\pdfglyphtounicode{glottalstop}{0294}
\pdfglyphtounicode{glottalstopinverted}{0296}
\pdfglyphtounicode{glottalstopmod}{02C0}
\pdfglyphtounicode{glottalstopreversed}{0295}
\pdfglyphtounicode{glottalstopreversedmod}{02C1}
\pdfglyphtounicode{glottalstopreversedsuperior}{02E4}
\pdfglyphtounicode{glottalstopstroke}{02A1}
\pdfglyphtounicode{glottalstopstrokereversed}{02A2}
\pdfglyphtounicode{gmacron}{1E21}
\pdfglyphtounicode{gmonospace}{FF47}
\pdfglyphtounicode{gohiragana}{3054}
\pdfglyphtounicode{gokatakana}{30B4}
\pdfglyphtounicode{gparen}{24A2}
\pdfglyphtounicode{gpasquare}{33AC}
\pdfglyphtounicode{gradient}{2207}
\pdfglyphtounicode{grave}{0060}
\pdfglyphtounicode{gravebelowcmb}{0316}
\pdfglyphtounicode{gravecmb}{0300}
\pdfglyphtounicode{gravecomb}{0300}
\pdfglyphtounicode{gravedeva}{0953}
\pdfglyphtounicode{gravelowmod}{02CE}
\pdfglyphtounicode{gravemonospace}{FF40}
\pdfglyphtounicode{gravetonecmb}{0340}
\pdfglyphtounicode{greater}{003E}
\pdfglyphtounicode{greaterequal}{2265}
\pdfglyphtounicode{greaterequalorless}{22DB}
\pdfglyphtounicode{greatermonospace}{FF1E}
\pdfglyphtounicode{greaterorequivalent}{2273}
\pdfglyphtounicode{greaterorless}{2277}
\pdfglyphtounicode{greateroverequal}{2267}
\pdfglyphtounicode{greatersmall}{FE65}
\pdfglyphtounicode{gscript}{0261}
\pdfglyphtounicode{gstroke}{01E5}
\pdfglyphtounicode{guhiragana}{3050}
\pdfglyphtounicode{guillemotleft}{00AB}
\pdfglyphtounicode{guillemotright}{00BB}
\pdfglyphtounicode{guilsinglleft}{2039}
\pdfglyphtounicode{guilsinglright}{203A}
\pdfglyphtounicode{gukatakana}{30B0}
\pdfglyphtounicode{guramusquare}{3318}
\pdfglyphtounicode{gysquare}{33C9}
\pdfglyphtounicode{h}{0068}
\pdfglyphtounicode{haabkhasiancyrillic}{04A9}
\pdfglyphtounicode{haaltonearabic}{06C1}
\pdfglyphtounicode{habengali}{09B9}
\pdfglyphtounicode{hadescendercyrillic}{04B3}
\pdfglyphtounicode{hadeva}{0939}
\pdfglyphtounicode{hagujarati}{0AB9}
\pdfglyphtounicode{hagurmukhi}{0A39}
\pdfglyphtounicode{haharabic}{062D}
\pdfglyphtounicode{hahfinalarabic}{FEA2}
\pdfglyphtounicode{hahinitialarabic}{FEA3}
\pdfglyphtounicode{hahiragana}{306F}
\pdfglyphtounicode{hahmedialarabic}{FEA4}
\pdfglyphtounicode{haitusquare}{332A}
\pdfglyphtounicode{hakatakana}{30CF}
\pdfglyphtounicode{hakatakanahalfwidth}{FF8A}
\pdfglyphtounicode{halantgurmukhi}{0A4D}
\pdfglyphtounicode{hamzaarabic}{0621}
% hamzadammaarabic;0621 064F
% hamzadammatanarabic;0621 064C
% hamzafathaarabic;0621 064E
% hamzafathatanarabic;0621 064B
\pdfglyphtounicode{hamzalowarabic}{0621}
% hamzalowkasraarabic;0621 0650
% hamzalowkasratanarabic;0621 064D
% hamzasukunarabic;0621 0652
\pdfglyphtounicode{hangulfiller}{3164}
\pdfglyphtounicode{hardsigncyrillic}{044A}
\pdfglyphtounicode{harpoonleftbarbup}{21BC}
\pdfglyphtounicode{harpoonrightbarbup}{21C0}
\pdfglyphtounicode{hasquare}{33CA}
\pdfglyphtounicode{hatafpatah}{05B2}
\pdfglyphtounicode{hatafpatah16}{05B2}
\pdfglyphtounicode{hatafpatah23}{05B2}
\pdfglyphtounicode{hatafpatah2f}{05B2}
\pdfglyphtounicode{hatafpatahhebrew}{05B2}
\pdfglyphtounicode{hatafpatahnarrowhebrew}{05B2}
\pdfglyphtounicode{hatafpatahquarterhebrew}{05B2}
\pdfglyphtounicode{hatafpatahwidehebrew}{05B2}
\pdfglyphtounicode{hatafqamats}{05B3}
\pdfglyphtounicode{hatafqamats1b}{05B3}
\pdfglyphtounicode{hatafqamats28}{05B3}
\pdfglyphtounicode{hatafqamats34}{05B3}
\pdfglyphtounicode{hatafqamatshebrew}{05B3}
\pdfglyphtounicode{hatafqamatsnarrowhebrew}{05B3}
\pdfglyphtounicode{hatafqamatsquarterhebrew}{05B3}
\pdfglyphtounicode{hatafqamatswidehebrew}{05B3}
\pdfglyphtounicode{hatafsegol}{05B1}
\pdfglyphtounicode{hatafsegol17}{05B1}
\pdfglyphtounicode{hatafsegol24}{05B1}
\pdfglyphtounicode{hatafsegol30}{05B1}
\pdfglyphtounicode{hatafsegolhebrew}{05B1}
\pdfglyphtounicode{hatafsegolnarrowhebrew}{05B1}
\pdfglyphtounicode{hatafsegolquarterhebrew}{05B1}
\pdfglyphtounicode{hatafsegolwidehebrew}{05B1}
\pdfglyphtounicode{hbar}{0127}
\pdfglyphtounicode{hbopomofo}{310F}
\pdfglyphtounicode{hbrevebelow}{1E2B}
\pdfglyphtounicode{hcedilla}{1E29}
\pdfglyphtounicode{hcircle}{24D7}
\pdfglyphtounicode{hcircumflex}{0125}
\pdfglyphtounicode{hdieresis}{1E27}
\pdfglyphtounicode{hdotaccent}{1E23}
\pdfglyphtounicode{hdotbelow}{1E25}
\pdfglyphtounicode{he}{05D4}
\pdfglyphtounicode{heart}{2665}
\pdfglyphtounicode{heartsuitblack}{2665}
\pdfglyphtounicode{heartsuitwhite}{2661}
\pdfglyphtounicode{hedagesh}{FB34}
\pdfglyphtounicode{hedageshhebrew}{FB34}
\pdfglyphtounicode{hehaltonearabic}{06C1}
\pdfglyphtounicode{heharabic}{0647}
\pdfglyphtounicode{hehebrew}{05D4}
\pdfglyphtounicode{hehfinalaltonearabic}{FBA7}
\pdfglyphtounicode{hehfinalalttwoarabic}{FEEA}
\pdfglyphtounicode{hehfinalarabic}{FEEA}
\pdfglyphtounicode{hehhamzaabovefinalarabic}{FBA5}
\pdfglyphtounicode{hehhamzaaboveisolatedarabic}{FBA4}
\pdfglyphtounicode{hehinitialaltonearabic}{FBA8}
\pdfglyphtounicode{hehinitialarabic}{FEEB}
\pdfglyphtounicode{hehiragana}{3078}
\pdfglyphtounicode{hehmedialaltonearabic}{FBA9}
\pdfglyphtounicode{hehmedialarabic}{FEEC}
\pdfglyphtounicode{heiseierasquare}{337B}
\pdfglyphtounicode{hekatakana}{30D8}
\pdfglyphtounicode{hekatakanahalfwidth}{FF8D}
\pdfglyphtounicode{hekutaarusquare}{3336}
\pdfglyphtounicode{henghook}{0267}
\pdfglyphtounicode{herutusquare}{3339}
\pdfglyphtounicode{het}{05D7}
\pdfglyphtounicode{hethebrew}{05D7}
\pdfglyphtounicode{hhook}{0266}
\pdfglyphtounicode{hhooksuperior}{02B1}
\pdfglyphtounicode{hieuhacirclekorean}{327B}
\pdfglyphtounicode{hieuhaparenkorean}{321B}
\pdfglyphtounicode{hieuhcirclekorean}{326D}
\pdfglyphtounicode{hieuhkorean}{314E}
\pdfglyphtounicode{hieuhparenkorean}{320D}
\pdfglyphtounicode{hihiragana}{3072}
\pdfglyphtounicode{hikatakana}{30D2}
\pdfglyphtounicode{hikatakanahalfwidth}{FF8B}
\pdfglyphtounicode{hiriq}{05B4}
\pdfglyphtounicode{hiriq14}{05B4}
\pdfglyphtounicode{hiriq21}{05B4}
\pdfglyphtounicode{hiriq2d}{05B4}
\pdfglyphtounicode{hiriqhebrew}{05B4}
\pdfglyphtounicode{hiriqnarrowhebrew}{05B4}
\pdfglyphtounicode{hiriqquarterhebrew}{05B4}
\pdfglyphtounicode{hiriqwidehebrew}{05B4}
\pdfglyphtounicode{hlinebelow}{1E96}
\pdfglyphtounicode{hmonospace}{FF48}
\pdfglyphtounicode{hoarmenian}{0570}
\pdfglyphtounicode{hohipthai}{0E2B}
\pdfglyphtounicode{hohiragana}{307B}
\pdfglyphtounicode{hokatakana}{30DB}
\pdfglyphtounicode{hokatakanahalfwidth}{FF8E}
\pdfglyphtounicode{holam}{05B9}
\pdfglyphtounicode{holam19}{05B9}
\pdfglyphtounicode{holam26}{05B9}
\pdfglyphtounicode{holam32}{05B9}
\pdfglyphtounicode{holamhebrew}{05B9}
\pdfglyphtounicode{holamnarrowhebrew}{05B9}
\pdfglyphtounicode{holamquarterhebrew}{05B9}
\pdfglyphtounicode{holamwidehebrew}{05B9}
\pdfglyphtounicode{honokhukthai}{0E2E}
\pdfglyphtounicode{hookabovecomb}{0309}
\pdfglyphtounicode{hookcmb}{0309}
\pdfglyphtounicode{hookpalatalizedbelowcmb}{0321}
\pdfglyphtounicode{hookretroflexbelowcmb}{0322}
\pdfglyphtounicode{hoonsquare}{3342}
\pdfglyphtounicode{horicoptic}{03E9}
\pdfglyphtounicode{horizontalbar}{2015}
\pdfglyphtounicode{horncmb}{031B}
\pdfglyphtounicode{hotsprings}{2668}
\pdfglyphtounicode{house}{2302}
\pdfglyphtounicode{hparen}{24A3}
\pdfglyphtounicode{hsuperior}{02B0}
\pdfglyphtounicode{hturned}{0265}
\pdfglyphtounicode{huhiragana}{3075}
\pdfglyphtounicode{huiitosquare}{3333}
\pdfglyphtounicode{hukatakana}{30D5}
\pdfglyphtounicode{hukatakanahalfwidth}{FF8C}
\pdfglyphtounicode{hungarumlaut}{02DD}
\pdfglyphtounicode{hungarumlautcmb}{030B}
\pdfglyphtounicode{hv}{0195}
\pdfglyphtounicode{hyphen}{002D}
\pdfglyphtounicode{hypheninferior}{F6E5}
\pdfglyphtounicode{hyphenmonospace}{FF0D}
\pdfglyphtounicode{hyphensmall}{FE63}
\pdfglyphtounicode{hyphensuperior}{F6E6}
\pdfglyphtounicode{hyphentwo}{2010}
\pdfglyphtounicode{i}{0069}
\pdfglyphtounicode{iacute}{00ED}
\pdfglyphtounicode{iacyrillic}{044F}
\pdfglyphtounicode{ibengali}{0987}
\pdfglyphtounicode{ibopomofo}{3127}
\pdfglyphtounicode{ibreve}{012D}
\pdfglyphtounicode{icaron}{01D0}
\pdfglyphtounicode{icircle}{24D8}
\pdfglyphtounicode{icircumflex}{00EE}
\pdfglyphtounicode{icyrillic}{0456}
\pdfglyphtounicode{idblgrave}{0209}
\pdfglyphtounicode{ideographearthcircle}{328F}
\pdfglyphtounicode{ideographfirecircle}{328B}
\pdfglyphtounicode{ideographicallianceparen}{323F}
\pdfglyphtounicode{ideographiccallparen}{323A}
\pdfglyphtounicode{ideographiccentrecircle}{32A5}
\pdfglyphtounicode{ideographicclose}{3006}
\pdfglyphtounicode{ideographiccomma}{3001}
\pdfglyphtounicode{ideographiccommaleft}{FF64}
\pdfglyphtounicode{ideographiccongratulationparen}{3237}
\pdfglyphtounicode{ideographiccorrectcircle}{32A3}
\pdfglyphtounicode{ideographicearthparen}{322F}
\pdfglyphtounicode{ideographicenterpriseparen}{323D}
\pdfglyphtounicode{ideographicexcellentcircle}{329D}
\pdfglyphtounicode{ideographicfestivalparen}{3240}
\pdfglyphtounicode{ideographicfinancialcircle}{3296}
\pdfglyphtounicode{ideographicfinancialparen}{3236}
\pdfglyphtounicode{ideographicfireparen}{322B}
\pdfglyphtounicode{ideographichaveparen}{3232}
\pdfglyphtounicode{ideographichighcircle}{32A4}
\pdfglyphtounicode{ideographiciterationmark}{3005}
\pdfglyphtounicode{ideographiclaborcircle}{3298}
\pdfglyphtounicode{ideographiclaborparen}{3238}
\pdfglyphtounicode{ideographicleftcircle}{32A7}
\pdfglyphtounicode{ideographiclowcircle}{32A6}
\pdfglyphtounicode{ideographicmedicinecircle}{32A9}
\pdfglyphtounicode{ideographicmetalparen}{322E}
\pdfglyphtounicode{ideographicmoonparen}{322A}
\pdfglyphtounicode{ideographicnameparen}{3234}
\pdfglyphtounicode{ideographicperiod}{3002}
\pdfglyphtounicode{ideographicprintcircle}{329E}
\pdfglyphtounicode{ideographicreachparen}{3243}
\pdfglyphtounicode{ideographicrepresentparen}{3239}
\pdfglyphtounicode{ideographicresourceparen}{323E}
\pdfglyphtounicode{ideographicrightcircle}{32A8}
\pdfglyphtounicode{ideographicsecretcircle}{3299}
\pdfglyphtounicode{ideographicselfparen}{3242}
\pdfglyphtounicode{ideographicsocietyparen}{3233}
\pdfglyphtounicode{ideographicspace}{3000}
\pdfglyphtounicode{ideographicspecialparen}{3235}
\pdfglyphtounicode{ideographicstockparen}{3231}
\pdfglyphtounicode{ideographicstudyparen}{323B}
\pdfglyphtounicode{ideographicsunparen}{3230}
\pdfglyphtounicode{ideographicsuperviseparen}{323C}
\pdfglyphtounicode{ideographicwaterparen}{322C}
\pdfglyphtounicode{ideographicwoodparen}{322D}
\pdfglyphtounicode{ideographiczero}{3007}
\pdfglyphtounicode{ideographmetalcircle}{328E}
\pdfglyphtounicode{ideographmooncircle}{328A}
\pdfglyphtounicode{ideographnamecircle}{3294}
\pdfglyphtounicode{ideographsuncircle}{3290}
\pdfglyphtounicode{ideographwatercircle}{328C}
\pdfglyphtounicode{ideographwoodcircle}{328D}
\pdfglyphtounicode{ideva}{0907}
\pdfglyphtounicode{idieresis}{00EF}
\pdfglyphtounicode{idieresisacute}{1E2F}
\pdfglyphtounicode{idieresiscyrillic}{04E5}
\pdfglyphtounicode{idotbelow}{1ECB}
\pdfglyphtounicode{iebrevecyrillic}{04D7}
\pdfglyphtounicode{iecyrillic}{0435}
\pdfglyphtounicode{ieungacirclekorean}{3275}
\pdfglyphtounicode{ieungaparenkorean}{3215}
\pdfglyphtounicode{ieungcirclekorean}{3267}
\pdfglyphtounicode{ieungkorean}{3147}
\pdfglyphtounicode{ieungparenkorean}{3207}
\pdfglyphtounicode{igrave}{00EC}
\pdfglyphtounicode{igujarati}{0A87}
\pdfglyphtounicode{igurmukhi}{0A07}
\pdfglyphtounicode{ihiragana}{3044}
\pdfglyphtounicode{ihookabove}{1EC9}
\pdfglyphtounicode{iibengali}{0988}
\pdfglyphtounicode{iicyrillic}{0438}
\pdfglyphtounicode{iideva}{0908}
\pdfglyphtounicode{iigujarati}{0A88}
\pdfglyphtounicode{iigurmukhi}{0A08}
\pdfglyphtounicode{iimatragurmukhi}{0A40}
\pdfglyphtounicode{iinvertedbreve}{020B}
\pdfglyphtounicode{iishortcyrillic}{0439}
\pdfglyphtounicode{iivowelsignbengali}{09C0}
\pdfglyphtounicode{iivowelsigndeva}{0940}
\pdfglyphtounicode{iivowelsigngujarati}{0AC0}
\pdfglyphtounicode{ij}{0133}
\pdfglyphtounicode{ikatakana}{30A4}
\pdfglyphtounicode{ikatakanahalfwidth}{FF72}
\pdfglyphtounicode{ikorean}{3163}
\pdfglyphtounicode{ilde}{02DC}
\pdfglyphtounicode{iluyhebrew}{05AC}
\pdfglyphtounicode{imacron}{012B}
\pdfglyphtounicode{imacroncyrillic}{04E3}
\pdfglyphtounicode{imageorapproximatelyequal}{2253}
\pdfglyphtounicode{imatragurmukhi}{0A3F}
\pdfglyphtounicode{imonospace}{FF49}
\pdfglyphtounicode{increment}{2206}
\pdfglyphtounicode{infinity}{221E}
\pdfglyphtounicode{iniarmenian}{056B}
\pdfglyphtounicode{integral}{222B}
\pdfglyphtounicode{integralbottom}{2321}
\pdfglyphtounicode{integralbt}{2321}
\pdfglyphtounicode{integralex}{F8F5}
\pdfglyphtounicode{integraltop}{2320}
\pdfglyphtounicode{integraltp}{2320}
\pdfglyphtounicode{intersection}{2229}
\pdfglyphtounicode{intisquare}{3305}
\pdfglyphtounicode{invbullet}{25D8}
\pdfglyphtounicode{invcircle}{25D9}
\pdfglyphtounicode{invsmileface}{263B}
\pdfglyphtounicode{iocyrillic}{0451}
\pdfglyphtounicode{iogonek}{012F}
\pdfglyphtounicode{iota}{03B9}
\pdfglyphtounicode{iotadieresis}{03CA}
\pdfglyphtounicode{iotadieresistonos}{0390}
\pdfglyphtounicode{iotalatin}{0269}
\pdfglyphtounicode{iotatonos}{03AF}
\pdfglyphtounicode{iparen}{24A4}
\pdfglyphtounicode{irigurmukhi}{0A72}
\pdfglyphtounicode{ismallhiragana}{3043}
\pdfglyphtounicode{ismallkatakana}{30A3}
\pdfglyphtounicode{ismallkatakanahalfwidth}{FF68}
\pdfglyphtounicode{issharbengali}{09FA}
\pdfglyphtounicode{istroke}{0268}
\pdfglyphtounicode{isuperior}{F6ED}
\pdfglyphtounicode{iterationhiragana}{309D}
\pdfglyphtounicode{iterationkatakana}{30FD}
\pdfglyphtounicode{itilde}{0129}
\pdfglyphtounicode{itildebelow}{1E2D}
\pdfglyphtounicode{iubopomofo}{3129}
\pdfglyphtounicode{iucyrillic}{044E}
\pdfglyphtounicode{ivowelsignbengali}{09BF}
\pdfglyphtounicode{ivowelsigndeva}{093F}
\pdfglyphtounicode{ivowelsigngujarati}{0ABF}
\pdfglyphtounicode{izhitsacyrillic}{0475}
\pdfglyphtounicode{izhitsadblgravecyrillic}{0477}
\pdfglyphtounicode{j}{006A}
\pdfglyphtounicode{jaarmenian}{0571}
\pdfglyphtounicode{jabengali}{099C}
\pdfglyphtounicode{jadeva}{091C}
\pdfglyphtounicode{jagujarati}{0A9C}
\pdfglyphtounicode{jagurmukhi}{0A1C}
\pdfglyphtounicode{jbopomofo}{3110}
\pdfglyphtounicode{jcaron}{01F0}
\pdfglyphtounicode{jcircle}{24D9}
\pdfglyphtounicode{jcircumflex}{0135}
\pdfglyphtounicode{jcrossedtail}{029D}
\pdfglyphtounicode{jdotlessstroke}{025F}
\pdfglyphtounicode{jecyrillic}{0458}
\pdfglyphtounicode{jeemarabic}{062C}
\pdfglyphtounicode{jeemfinalarabic}{FE9E}
\pdfglyphtounicode{jeeminitialarabic}{FE9F}
\pdfglyphtounicode{jeemmedialarabic}{FEA0}
\pdfglyphtounicode{jeharabic}{0698}
\pdfglyphtounicode{jehfinalarabic}{FB8B}
\pdfglyphtounicode{jhabengali}{099D}
\pdfglyphtounicode{jhadeva}{091D}
\pdfglyphtounicode{jhagujarati}{0A9D}
\pdfglyphtounicode{jhagurmukhi}{0A1D}
\pdfglyphtounicode{jheharmenian}{057B}
\pdfglyphtounicode{jis}{3004}
\pdfglyphtounicode{jmonospace}{FF4A}
\pdfglyphtounicode{jparen}{24A5}
\pdfglyphtounicode{jsuperior}{02B2}
\pdfglyphtounicode{k}{006B}
\pdfglyphtounicode{kabashkircyrillic}{04A1}
\pdfglyphtounicode{kabengali}{0995}
\pdfglyphtounicode{kacute}{1E31}
\pdfglyphtounicode{kacyrillic}{043A}
\pdfglyphtounicode{kadescendercyrillic}{049B}
\pdfglyphtounicode{kadeva}{0915}
\pdfglyphtounicode{kaf}{05DB}
\pdfglyphtounicode{kafarabic}{0643}
\pdfglyphtounicode{kafdagesh}{FB3B}
\pdfglyphtounicode{kafdageshhebrew}{FB3B}
\pdfglyphtounicode{kaffinalarabic}{FEDA}
\pdfglyphtounicode{kafhebrew}{05DB}
\pdfglyphtounicode{kafinitialarabic}{FEDB}
\pdfglyphtounicode{kafmedialarabic}{FEDC}
\pdfglyphtounicode{kafrafehebrew}{FB4D}
\pdfglyphtounicode{kagujarati}{0A95}
\pdfglyphtounicode{kagurmukhi}{0A15}
\pdfglyphtounicode{kahiragana}{304B}
\pdfglyphtounicode{kahookcyrillic}{04C4}
\pdfglyphtounicode{kakatakana}{30AB}
\pdfglyphtounicode{kakatakanahalfwidth}{FF76}
\pdfglyphtounicode{kappa}{03BA}
\pdfglyphtounicode{kappasymbolgreek}{03F0}
\pdfglyphtounicode{kapyeounmieumkorean}{3171}
\pdfglyphtounicode{kapyeounphieuphkorean}{3184}
\pdfglyphtounicode{kapyeounpieupkorean}{3178}
\pdfglyphtounicode{kapyeounssangpieupkorean}{3179}
\pdfglyphtounicode{karoriisquare}{330D}
\pdfglyphtounicode{kashidaautoarabic}{0640}
\pdfglyphtounicode{kashidaautonosidebearingarabic}{0640}
\pdfglyphtounicode{kasmallkatakana}{30F5}
\pdfglyphtounicode{kasquare}{3384}
\pdfglyphtounicode{kasraarabic}{0650}
\pdfglyphtounicode{kasratanarabic}{064D}
\pdfglyphtounicode{kastrokecyrillic}{049F}
\pdfglyphtounicode{katahiraprolongmarkhalfwidth}{FF70}
\pdfglyphtounicode{kaverticalstrokecyrillic}{049D}
\pdfglyphtounicode{kbopomofo}{310E}
\pdfglyphtounicode{kcalsquare}{3389}
\pdfglyphtounicode{kcaron}{01E9}
\pdfglyphtounicode{kcedilla}{0137}
\pdfglyphtounicode{kcircle}{24DA}
\pdfglyphtounicode{kcommaaccent}{0137}
\pdfglyphtounicode{kdotbelow}{1E33}
\pdfglyphtounicode{keharmenian}{0584}
\pdfglyphtounicode{kehiragana}{3051}
\pdfglyphtounicode{kekatakana}{30B1}
\pdfglyphtounicode{kekatakanahalfwidth}{FF79}
\pdfglyphtounicode{kenarmenian}{056F}
\pdfglyphtounicode{kesmallkatakana}{30F6}
\pdfglyphtounicode{kgreenlandic}{0138}
\pdfglyphtounicode{khabengali}{0996}
\pdfglyphtounicode{khacyrillic}{0445}
\pdfglyphtounicode{khadeva}{0916}
\pdfglyphtounicode{khagujarati}{0A96}
\pdfglyphtounicode{khagurmukhi}{0A16}
\pdfglyphtounicode{khaharabic}{062E}
\pdfglyphtounicode{khahfinalarabic}{FEA6}
\pdfglyphtounicode{khahinitialarabic}{FEA7}
\pdfglyphtounicode{khahmedialarabic}{FEA8}
\pdfglyphtounicode{kheicoptic}{03E7}
\pdfglyphtounicode{khhadeva}{0959}
\pdfglyphtounicode{khhagurmukhi}{0A59}
\pdfglyphtounicode{khieukhacirclekorean}{3278}
\pdfglyphtounicode{khieukhaparenkorean}{3218}
\pdfglyphtounicode{khieukhcirclekorean}{326A}
\pdfglyphtounicode{khieukhkorean}{314B}
\pdfglyphtounicode{khieukhparenkorean}{320A}
\pdfglyphtounicode{khokhaithai}{0E02}
\pdfglyphtounicode{khokhonthai}{0E05}
\pdfglyphtounicode{khokhuatthai}{0E03}
\pdfglyphtounicode{khokhwaithai}{0E04}
\pdfglyphtounicode{khomutthai}{0E5B}
\pdfglyphtounicode{khook}{0199}
\pdfglyphtounicode{khorakhangthai}{0E06}
\pdfglyphtounicode{khzsquare}{3391}
\pdfglyphtounicode{kihiragana}{304D}
\pdfglyphtounicode{kikatakana}{30AD}
\pdfglyphtounicode{kikatakanahalfwidth}{FF77}
\pdfglyphtounicode{kiroguramusquare}{3315}
\pdfglyphtounicode{kiromeetorusquare}{3316}
\pdfglyphtounicode{kirosquare}{3314}
\pdfglyphtounicode{kiyeokacirclekorean}{326E}
\pdfglyphtounicode{kiyeokaparenkorean}{320E}
\pdfglyphtounicode{kiyeokcirclekorean}{3260}
\pdfglyphtounicode{kiyeokkorean}{3131}
\pdfglyphtounicode{kiyeokparenkorean}{3200}
\pdfglyphtounicode{kiyeoksioskorean}{3133}
\pdfglyphtounicode{kjecyrillic}{045C}
\pdfglyphtounicode{klinebelow}{1E35}
\pdfglyphtounicode{klsquare}{3398}
\pdfglyphtounicode{kmcubedsquare}{33A6}
\pdfglyphtounicode{kmonospace}{FF4B}
\pdfglyphtounicode{kmsquaredsquare}{33A2}
\pdfglyphtounicode{kohiragana}{3053}
\pdfglyphtounicode{kohmsquare}{33C0}
\pdfglyphtounicode{kokaithai}{0E01}
\pdfglyphtounicode{kokatakana}{30B3}
\pdfglyphtounicode{kokatakanahalfwidth}{FF7A}
\pdfglyphtounicode{kooposquare}{331E}
\pdfglyphtounicode{koppacyrillic}{0481}
\pdfglyphtounicode{koreanstandardsymbol}{327F}
\pdfglyphtounicode{koroniscmb}{0343}
\pdfglyphtounicode{kparen}{24A6}
\pdfglyphtounicode{kpasquare}{33AA}
\pdfglyphtounicode{ksicyrillic}{046F}
\pdfglyphtounicode{ktsquare}{33CF}
\pdfglyphtounicode{kturned}{029E}
\pdfglyphtounicode{kuhiragana}{304F}
\pdfglyphtounicode{kukatakana}{30AF}
\pdfglyphtounicode{kukatakanahalfwidth}{FF78}
\pdfglyphtounicode{kvsquare}{33B8}
\pdfglyphtounicode{kwsquare}{33BE}
\pdfglyphtounicode{l}{006C}
\pdfglyphtounicode{labengali}{09B2}
\pdfglyphtounicode{lacute}{013A}
\pdfglyphtounicode{ladeva}{0932}
\pdfglyphtounicode{lagujarati}{0AB2}
\pdfglyphtounicode{lagurmukhi}{0A32}
\pdfglyphtounicode{lakkhangyaothai}{0E45}
\pdfglyphtounicode{lamaleffinalarabic}{FEFC}
\pdfglyphtounicode{lamalefhamzaabovefinalarabic}{FEF8}
\pdfglyphtounicode{lamalefhamzaaboveisolatedarabic}{FEF7}
\pdfglyphtounicode{lamalefhamzabelowfinalarabic}{FEFA}
\pdfglyphtounicode{lamalefhamzabelowisolatedarabic}{FEF9}
\pdfglyphtounicode{lamalefisolatedarabic}{FEFB}
\pdfglyphtounicode{lamalefmaddaabovefinalarabic}{FEF6}
\pdfglyphtounicode{lamalefmaddaaboveisolatedarabic}{FEF5}
\pdfglyphtounicode{lamarabic}{0644}
\pdfglyphtounicode{lambda}{03BB}
\pdfglyphtounicode{lambdastroke}{019B}
\pdfglyphtounicode{lamed}{05DC}
\pdfglyphtounicode{lameddagesh}{FB3C}
\pdfglyphtounicode{lameddageshhebrew}{FB3C}
\pdfglyphtounicode{lamedhebrew}{05DC}
% lamedholam;05DC 05B9
% lamedholamdagesh;05DC 05B9 05BC
% lamedholamdageshhebrew;05DC 05B9 05BC
% lamedholamhebrew;05DC 05B9
\pdfglyphtounicode{lamfinalarabic}{FEDE}
\pdfglyphtounicode{lamhahinitialarabic}{FCCA}
\pdfglyphtounicode{laminitialarabic}{FEDF}
\pdfglyphtounicode{lamjeeminitialarabic}{FCC9}
\pdfglyphtounicode{lamkhahinitialarabic}{FCCB}
\pdfgentounicode=1

% hyphenations
%
% der Befehl \hypenation versteht keine Sonderzeichen, also weder �
% noch "a noch \"a. W�rter die derartige Zeichen enthalten m�ssen
% direkt im Text getrennt werden, z.B. W�r\-ter
%
\hyphenation{Ma-nage-ment}
\hyphenation{Ma-nage-ment-agent}
\hyphenation{Ma-nage-ment-agent-en}
\hyphenation{Ma-nage-ment-ar-chi-tek-tur}
\hyphenation{Ma-nage-ment-ar-chi-tek-tu-ren}
\hyphenation{Ma-nage-ment-an-wen-dung}
\hyphenation{Ma-nage-ment-an-wen-dung-en}
\hyphenation{Ma-nage-ment-an-for-der-ung}
\hyphenation{Ma-nage-ment-funk-ti-on}
\hyphenation{Ma-nage-ment-funk-ti-onen}
\hyphenation{Ma-nage-ment-kon-zep-te}
\hyphenation{Ma-nage-ment-res-source}
\hyphenation{Ma-nage-ment-in-for-ma-ti-on}
\hyphenation{Ma-nage-ment-res-sour-cen}
\hyphenation{ma-nage-ment-re-le-vante}
\hyphenation{ma-nage-ment-sy-stem}
\hyphenation{ma-nage-ment-sy-steme}
\hyphenation{Ma-nage-ment-in-stru-men-tie-rung}
\hyphenation{Ma-nage-ment-platt-form}
\hyphenation{Sys-te-men}
\hyphenation{Sys-tem-um-ge-bun-gen}
\hyphenation{Sys-tem-ma-nage-ment}
\hyphenation{DHCP}
\hyphenation{Ma-nage-ment-diszi-plinen}
\hyphenation{System-management-architekturen}
\hyphenation{Verwendungs-nachweise}
\hyphenation{Video-einricht-ungen}
\hyphenation{Res-source}
\hyphenation{Res-sourcen}
\hyphenation{Grund-anwendung}
\hyphenation{Grund-anwendungen}
\hyphenation{Basis-anwendung}
\hyphenation{Core}
\hyphenation{Kom-mu-ni-ka-ti-on}
\hyphenation{De-sign-ent-schei-dung}
\hyphenation{Sprung-ad-res-sen}
\hyphenation{Klas-si-fi-ka-ti-on}
\hyphenation{Schreib-recht}
\hyphenation{Be-nut-zer-zer-ti-fi-kat}
\hyphenation{Bau-stein-ent-wi-ckler}
\hyphenation{ad-mi-ni-stra-ti-ve}
\hyphenation{Ver-bin-dungs-weg}
\hyphenation{che-mi-schen}
\hyphenation{ent-we-der}
\hyphenation{zeich-ne-te}
\hyphenation{Au-to-ka-ros-se-ri-en}
\hyphenation{sol-cher}
\hyphenation{de-mo-kra-ti-schen}
\hyphenation{Aus-hand-lungs-pro-zes-ses}

% Die Seitennummerierung erfolgt durchlaufend ab der Titelseite. Also keine
% Spielereien mit römischen Ziffern usw. - Die ISO 7144 schreibt das sogar für
% wissenschaftliche Werke vor.
% Von Promitionsordnung verlangt!
% Deshalb ist \frontmatter DEAKTIVIERT
%\frontmatter
\title{Sparse Grids and Uncertainty Quantification}
%\author{\texorpdfstring{\href{http://www.example.org/}{Author Name}}{Author Name}}
\author{Fabian Franzelin}
\date{\today}
\keywords{TODO}
\firstexaminer{Prof.~Dr.~Dirk Pflüger}
\secondexaminer{?? noch offen ??}
\dateofexamination{01. Februar 2017}
\placeofbirth{Bozen}
\faculty{Fakultät für Informatik, Elektrotechnik und Informationstechnik}
\department{Institut für Parallele und Verteilte Systeme}

\maketitle

% This is necessary if you have more than 9 sections/subsections/subsubsections
\makeatletter
\renewcommand\l@section{\@dottedtocline{1}{1.5em}{3em}}
\renewcommand\l@subsection{\@dottedtocline{2}{1.5em}{4.3em}}
\renewcommand\l@subsubsection{\@dottedtocline{3}{1.5em}{5.6em}}
\makeatother

\tableofcontents
\clearpage

\pagestyle{justpagenums}
% \addchap{Abstract/\foreignlanguage{ngerman}{Kurzzusammenfassung}}

\printornamentsfalse

\section*{Abstract}

In simulation technology, computationally expensive objective functions
are often replaced by cheap surrogates,
which can be obtained by interpolation.
Full grid interpolation methods suffer from the
so-called curse of dimensionality,
rendering them infeasible if the parameter domain of the function
is higher-dimensional (four or more parameters).
Sparse grids constitute a discretization method that does not suffer from the
curse, while the approximation quality deteriorates only insignificantly.
However, conventional basis functions such as piecewise linear functions
are not smooth (continuously differentiable).
Hence, these basis functions are unsuitable for applications
in which gradients are required.
One example for such an application is gradient-based optimization,
in which the availability of gradients greatly improves the speed of
convergence and the accuracy of results.

This thesis demonstrates that hierarchical B-splines on sparse grids are
well-suited for obtaining smooth interpolants for higher dimensionalities.
The thesis is organized in two main parts:
In the first part, we derive new B-spline bases on sparse grids and study
their implications on theory and algorithms.
In the second part, we consider three real-world applications in optimization:
topology optimization, biomechanical continuum-mechanics, and
dynamic portfolio choice models in finance.
The results reveal that the optimization problems of these applications
can be solved accurately and efficiently with hierarchical B-splines on
sparse grids.
% 209 words

\newpage

\begin{otherlanguage}{ngerman}
  \section*{Kurzzusammenfassung}
  
  In der Simulationstechnologie werden zeitaufwendige Zielfunktionen
  oft durch einfache Surrogate ersetzt, die durch Interpolation
  gewonnen werden können.
  Vollgitter-Interpolationsmethoden leiden unter dem
  sogenannten Fluch der Dimensionalität,
  der sie unbrauchbar macht, falls der Parameterbereich der Funktion
  höherdimensional ist (vier oder mehr Parameter).
  Dünne Gitter sind eine Diskretisierungsmethode, die nicht unter
  dem Fluch leidet, aber die Approximationsqualität nur leicht verschlechtert.
  Leider sind konventionelle Basisfunktionen wie die stückweise
  lineare Funktionen nicht glatt (stetig differenzierbar).
  Daher sind sie für Anwendungen ungeeignet, in denen Gradienten
  benötigt werden.
  Ein Beispiel für eine solche Anwendung ist gradientenbasierte Optimierung,
  in der die Verfügbarkeit von Gradienten die Konvergenzgeschwindigkeit und
  die Ergebnisgenauigkeit deutlich verbessert.
  
  Diese Dissertation demonstriert, dass hierarchische B-Splines auf
  dünnen Gittern gut geeignet sind,
  um glatte Interpolierende für höhere Dimensionalitäten zu erhalten.
  Die Dissertation ist in zwei Hauptbereiche gegliedert:
  Der erste Teil leitet neue B-Spline-Basen auf dünnen Gittern her und
  untersucht ihre Implikationen bezüglich Theorie und Algorithmen.
  Der zweite Teil behandelt drei Realwelt-Anwendungen aus der Optimierung:
  Topologieoptimierung, biomechanische Kontinuumsmechanik und
  Modelle der dynamischen Portfolio-Wahl in der Finanzmathematik.
  Die Ergebnisse zeigen, dass die Optimierungsprobleme dieser
  Anwendungen durch hierarchische B-Splines auf dünnen Gittern
  genau und effizient gelöst werden können.
  % 188 Wörter
\end{otherlanguage}

\printornamentstrue
\cleardoublepage


%\mainmatter
\pagestyle{scrheadings}

%% ---- abstracts ----------------------------------------------

%% --- Zusammenfassung ---------------------------------------------
\chapter*{Zusammenfassung}
%Silbentrennung Deutsch
\begin{otherlanguage}{ngerman}
  Kurzfassung der Arbeit.
\end{otherlanguage}
\clearpage

%% --- Abstract Page---------------------------------------------
\chapter*{Abstract}
%Silbentrennung Englisch
\begin{otherlanguage}{american}
  The Thesis Abstract is written here (and usually kept to just this
  page). The page is kept centered vertically so it can expand into
  the blank space above the title too\ldots Some people think monkeys
  are weird, but I disagree.
\end{otherlanguage}
\clearpage

%% --- Acknowledgements page------------------------------------
\chapter*{Danksagungen}
I would like to thank the little green men from Mars. Without their
anal probing equipment, the medical experiments on the monkeys
described in this thesis would have been considerably more unpleasant
for all parties involved.
\clearpage


%%% --------------------------------------------------------------------
%%% Local Variables:
%%% mode: latex
%%% TeX-master: "../thesis"
%%% End:

%% -------------------------------------------------------------
\pagestyle{scrheadings}

%% ---- Content ------------------------------------------------
% 
\chapter{Test}
\label{chap:test}

\Citet{WSPA} beschreiben die Umsetzung einer serviceorientierten Architektur mittels Web-Services, während \citet{zMR2005} BPMN um den Aspekt des Risikomanagements erweitern.



Das Potenzmengensymbol $\powerset$ ist auch korrekt und kein kleines Weierstraß-p ($\wp$).

\textmarker{Markierter Text.}
\modified{Manuelle Markierung für Text, der seit der letzten Version geändert wurde.}

%provided indirectly by pdfcomment.sty (soulpos).
\hl{In Gelb hervorgehoben.}

Checkmark: \dingcheck. Crossmark: \dingcross.

Referencetest: \Cref{ssec:example}, \cref{fig:Abbildung} und \cref{alg:example}.

\begin{figure}
\missingfigure{}
\caption{Abbildung}
\label{fig:Abbildung}
\end{figure}

\begin{landscape}
\begin{figure}
\missingfigure{}
\caption{Gedrehte Abbildung}
\label{fig:AbbildungGedreht}
\end{figure}
\end{landscape}

\begin{algorithm}
\caption{$algo$}
\label{alg:example}
\begin{algorithmic}[1]
\State $a \gets 0$
\State State 2\label{alg1:state2}
\end{algorithmic}
\end{algorithm}

\section{Definitionen}
\begin{definition}[Title]
\label{def:def1}
Definition Text
\end{definition}

\begin{definition}[Title]
\label{def:def1}
Definition Text
\end{definition}

\begin{definition}[Title]
\label{def:def1}
Definition Text
\end{definition}

\begin{definition}[Title]
\label{def:def1}
Definition Text
\end{definition}

\begin{definition}[Title]
\label{def:def1}
Definition Text
\end{definition}

\begin{definition}[Title]
\label{def:def1}
Definition Text
\end{definition}

\begin{definition}[Title]
\label{def:def1}
Definition Text
\end{definition}

\begin{definition}[Title]
\label{def:def1}
Definition Text
\end{definition}

\begin{definition}[Title]
\label{def:def1}
Definition Text
\end{definition}

\begin{definition}[Title]
\label{def:def1}
Definition Text
\end{definition}

\begin{definition}[Title]
\label{def:def1}
Definition Text
\end{definition}

\begin{definition}[Title]
\label{def:def1}
Definition Text
\end{definition}

\begin{definition}[Title]
\label{def:def1}
Definition Text
\end{definition}

\begin{definition}[Title]
\label{def:def1}
Definition Text
\end{definition}

\begin{definition}[Title]
\label{def:def1}
Definition Text
\end{definition}

\begin{definition}[Title]
\label{def:def1}
Definition Text
\end{definition}

\begin{definition}[Title]
\label{def:def1}
Definition Text
\end{definition}

\begin{definition}[Title]
\label{def:def1}
Definition Text
\end{definition}

\begin{definition}[Title]
\label{def:def1}
Definition Text
\end{definition}

\begin{definition}[Title]
\label{def:def1}
Definition Text
\end{definition}

\begin{definition}[Title]
\label{def:def1}
Definition Text
\end{definition}

\begin{definition}[Title]
\label{def:def1}
Definition Text
\end{definition}

\begin{definition}[Title]
\label{def:def1}
Definition Text
\end{definition}

\begin{definition}[Title]
\label{def:def1}
Definition Text
\end{definition}

\begin{definition}[Title]
\label{def:def1}
Definition Text
\end{definition}

\begin{definition}[Title]
\label{def:def1}
Definition Text
\end{definition}

\begin{definition}[Title]
\label{def:def1}
Definition Text
\end{definition}

\begin{definition}[Title]
\label{def:def1}
Definition Text
\end{definition}

\begin{definition}[Title]
\label{def:def1}
Definition Text
\end{definition}

\begin{definition}[Title]
\label{def:def1}
Definition Text
\end{definition}

\begin{definition}[Title]
\label{def:def1}
Definition Text
\end{definition}

\begin{definition}[Title]
\label{def:def1}
Definition Text
\end{definition}

\begin{definition}[Title]
\label{def:def1}
Definition Text
\end{definition}

\begin{definition}[Title]
\label{def:def1}
Definition Text
\end{definition}

\begin{definition}[Title]
\label{def:def1}
Definition Text
\end{definition}

\begin{definition}[Title]
\label{def:def1}
Definition Text
\end{definition}

\begin{definition}[Title]
\label{def:def1}
Definition Text
\end{definition}

\begin{definition}[Title]
\label{def:def1}
Definition Text
\end{definition}

\begin{definition}[Title]
\label{def:def1}
Definition Text
\end{definition}

\begin{definition}[Title]
\label{def:def1}
Definition Text
\end{definition}

\begin{definition}[Title]
\label{def:def1}
Definition Text
\end{definition}

\begin{definition}[Title]
\label{def:def1}
Definition Text
\end{definition}

\begin{definition}[Title]
\label{def:def1}
Definition Text
\end{definition}

\begin{definition}[Title]
\label{def:def1}
Definition Text
\end{definition}

\begin{definition}[Title]
\label{def:def1}
Definition Text
\end{definition}

\begin{definition}[Title]
\label{def:def1}
Definition Text
\end{definition}

\begin{definition}[Title]
\label{def:def1}
Definition Text
\end{definition}

\begin{definition}[Title]
\label{def:def1}
Definition Text
\end{definition}

\begin{definition}[Title]
\label{def:def1}
Definition Text
\end{definition}

\begin{definition}[Title]
\label{def:def1}
Definition Text
\end{definition}

\begin{definition}[Title]
\label{def:def1}
Definition Text
\end{definition}

\begin{definition}[Title]
\label{def:def1}
Definition Text
\end{definition}

\begin{definition}[Title]
\label{def:def1}
Definition Text
\end{definition}

\begin{definition}[Title]
\label{def:def1}
Definition Text
\end{definition}

\begin{definition}[Title]
\label{def:def1}
Definition Text
\end{definition}

\begin{definition}[Title]
\label{def:def1}
Definition Text
\end{definition}

\begin{definition}[Title]
\label{def:def1}
Definition Text
\end{definition}

\begin{definition}[Title]
\label{def:def1}
Definition Text
\end{definition}

\begin{definition}[Title]
\label{def:def1}
Definition Text
\end{definition}

\begin{definition}[Title]
\label{def:def1}
Definition Text
\end{definition}

\begin{definition}[Title]
\label{def:def1}
Definition Text
\end{definition}

\begin{definition}[Title]
\label{def:def1}
Definition Text
\end{definition}

\begin{definition}[Title]
\label{def:def1}
Definition Text
\end{definition}

\begin{definition}[Title]
\label{def:def1}
Definition Text
\end{definition}

\begin{definition}[Title]
\label{def:def1}
Definition Text
\end{definition}

\begin{definition}[Title]
\label{def:def1}
Definition Text
\end{definition}

\begin{definition}[Title]
\label{def:def1}
Definition Text
\end{definition}

\begin{definition}[Title]
\label{def:def1}
Definition Text
\end{definition}

\begin{definition}[Title]
\label{def:def1}
Definition Text
\end{definition}

\begin{definition}[Title]
\label{def:def1}
Definition Text
\end{definition}

\begin{definition}[Title]
\label{def:def1}
Definition Text
\end{definition}

\begin{definition}[Title]
\label{def:def1}
Definition Text
\end{definition}

\begin{definition}[Title]
\label{def:def1}
Definition Text
\end{definition}

\begin{definition}[Title]
\label{def:def1}
Definition Text
\end{definition}

\begin{definition}[Title]
\label{def:def1}
Definition Text
\end{definition}

\begin{definition}[Title]
\label{def:def1}
Definition Text
\end{definition}

\begin{definition}[Title]
\label{def:def1}
Definition Text
\end{definition}

\begin{definition}[Title]
\label{def:def1}
Definition Text
\end{definition}

\begin{definition}[Title]
\label{def:def1}
Definition Text
\end{definition}

\begin{definition}[Title]
\label{def:def1}
Definition Text
\end{definition}

\begin{definition}[Title]
\label{def:def1}
Definition Text
\end{definition}

\begin{definition}[Title]
\label{def:def1}
Definition Text
\end{definition}

\begin{definition}[Title]
\label{def:def1}
Definition Text
\end{definition}

\begin{definition}[Title]
\label{def:def1}
Definition Text
\end{definition}

\Cref{def:def1} zeigt \ldots


\section{Aufzählungen}
\begin{enumerate}[label=\alph*)]
\item a
\item b
\item c
\item d
\end{enumerate}

Equivalent to paralist's inparaenum:
\begin{enumerate*}[label=\alph*)]
\item a
\item b
\item c
\item d
\end{enumerate*}

\begin{description}
\item[first] Erstens
\item[second] Zweitens
\item[third] Drittens
\end{description}

\begin{description}
\item[\texttt{first}] Erstens
\item[\texttt{second}] Zweitens
\item[\texttt{third}] Drittens
\end{description}

%works only if package enumitem is loaded
\begin{description}[font=\ttfamily]
\item[first] Erstens
\item[second] Zweitens
\item[third] Drittens
\end{description}

\begin{description}[style=unboxed]
\item[first label with a long description text breaking over one line. Enabled by enumitem package] Erstens
\item[second] Zweitens
\item[third] Drittens
\end{description}

\begin{Description}
\item[first label with a long description text breaking over one line. Defined in template.tex] Erstens
\item[second] Zweitens
\item[third] Drittens
\end{Description}

\begin{itemize}
\item Erstens
\item Zweitens
\item Drittens
\end{itemize}

Optionaler Parameter ändert den Marker, der vorangestellt ist. Siehe \url{http://www.weinelt.de/latex/item.html}.
\begin{itemize}
\item[A] Erstens
\item[B] Zweitens
\item[C] Drittens
\end{itemize}

Falsche Benutzung des optionalen Parameters wie folgt:
\begin{itemize}
\item[first] Erstens
\item[second] Zweitens
\item[third] Drittens
\end{itemize}
Dabei ist zu beachten, dass es sich bei Einbindung von \texttt{enumitem} anders verhält als bei \texttt{paralist}.

\todo{Hier muss noch kräftig Text produziert werden}

\section{Varioref Demonstration}
Siehe \vref{chap:test}.

\section{Algorithmen}
\begin{algorithm}

\caption{Algorithm 2}
\label{alg:example2}
\begin{algorithmic}[1]
\State $a \gets 0$
\State State 2\label{alg2:state2}
\end{algorithmic}
\end{algorithm}

\Cref{alg:example} hat bereits einen Algorithmus gezeigt.
Test der Zeilenreferenzierung: Zeile~\ref{alg1:state2} (\cref{alg:example}) und Zeile~\ref{alg2:state2} (\cref{alg:example2}).

\section{Listing Demonstration}
Minted ist das beste, aber funktioniert nur, wenn \href{http://pygments.org/download/}{pygments} installiert ist und pdflatex mit \texttt{-shell-escape} ausgeführt wird.

\iffalse
\begin{listing}
\begin{minted}[linenos=true]{xml}
<process name="reisebuero">
</process>
\end{minted}
\caption{Beispielprozess}
\label{lst:beispielprozess}
\end{listing}
\fi

Alter Stil mittels des Listings-Pakets ist in \cref{lst:beispielprozess} gezeigt.
%Listing-Umgebung wurde durch \newfloat{Listing} definiert
\begin{lstlisting}[float,caption={Beispielprozess},label={lst:beispielprozess}]
<process name="reisebuero">
</process>
\end{lstlisting}

\section{Demonstration von refenums}
\label{sec:method}
%See http://mirror.ctan.org/macros/latex/contrib/refenums/demo.tex for a full usage guide
%Setup ``Steps'' not having a print form
\setupRefEnums[~]{step}{ONLYSHORT}

\defRefEnum[subsection]{step}{Anforderungsanalyse}{rqa}
\label{sec:rqa}
\blindtext

\defRefEnum[subsection]{step}{Spezifikation}{spec}
\blindtext

\subsection{Zusammenfassung}
In \cref{sec:method} wurde zuerst \refEnumFull{step}{rqa} vorgestellt.
Anschließend wurde \refEnumFull{step}{spec} beschrieben.

\section{Demonstration für Kommentare}
Das ist ein Text.
\change{FL1: Text angepasst}{Geänderter Text}.
Hier nur ein Kommentar\comment{Kommentar}.

Alternativ: \modified{Nur geändert, ohne Rückverweis auf Korrekturkommentar}.


\section{Section}

\section{Section}

\section{Section}

\section{Section}

\section{Section}

\section{Section}

\section{Section}

\section{Section}

\section{Section}

\section{Section}
\blindtext

\section{Section}

\section{Section}

\section{Section}

\section{Section}

\section{Section}

\section{Section}

\section{Section}

\section{Section}

\section{Section}

\section{Section}

\section{Section}

\subsection{Subsection}
\label{ssec:example}
\blindtext

\subsection{Subsection}

\chapter{Transformation of WS-BPEL 2.0 into Executable Workflow Networks}
\label{chap:bpel-ewfn-transformation}

%Um die Fußzeile zu demonstrieren
\blindtext[3]

% \resetchapterfooter

%%% --------------------------------------------------------------------
%%% Local Variables:
%%% mode: latex
%%% TeX-master: "../thesis"
%%% End:


\chapter{Introduction}
\label{cha:introduction}

% -------------------------------------------------------------------------------------

\chapter{Uncertainty quantification}
\label{sec:uncert-quant}

\begin{chapquote}{Lewis Carroll, \textit{Alice in Wonderland}}
  ``Begin at the beginning,'' the King said, gravely, ``and go on till you
  come to an end; then stop.''
\end{chapquote}

\begin{itemize}
\item quantities of interest
\item sources of uncertainty
\end{itemize}

\blindtext
\blindtext
\blindtext
\blindtext
\blindtext
\blindtext
\blindtext
\blindtext
\blindtext
\blindtext
\blindtext
\blindtext
\blindtext
\blindtext
\blindtext
\blindtext
% -------------------------------------------------------------------------------------

\chapter{Sparse Grids}
\label{cha:sparse-grids}

\begin{figure}
\centering
\begin{minipage}{0.48\textwidth}
\includegraphics[width=0.90\columnwidth]{figures/subspace.pdf}
\end{minipage}
\begin{minipage}{0.48\textwidth}
\includegraphics[width=0.90\columnwidth]{figures/sparseGrid.pdf}
\end{minipage}
\caption{Subspace tableau and resulting sparse grid in 2D on level 3.}
\end{figure}

\cite{zenger91sparse}
\cite{zenger91sparse, Adams12Rigorous}

% -------------------------------------------------------------------------------------

\chapter{Polynomial chaos expansion}
\label{cha:polyn-chaos-expans}


% -------------------------------------------------------------------------------------

\chapter{Input modeling}
\label{cha:input-modeling}

\begin{itemize}
\item random variables
\end{itemize}

\blindtext
\blindtext
\blindtext
\blindtext
\blindtext
\blindtext
\blindtext
\blindtext
\blindtext
\blindtext
\blindtext
\blindtext
\blindtext
\blindtext
\blindtext
\blindtext

\section{Density Estimation}
\label{sec:}

\blindtext
\blindtext
\blindtext
\blindtext
\blindtext
\blindtext
\blindtext
\blindtext
\blindtext
\blindtext
\blindtext
\blindtext
\blindtext
\blindtext
\blindtext
\blindtext

\subsection{Kernel density estimators}
\label{sec:kern-dens-estim}

\blindtext
\blindtext
\blindtext
\blindtext
\blindtext
\blindtext
\blindtext
\blindtext
\blindtext
\blindtext
\blindtext
\blindtext
\blindtext
\blindtext
\blindtext
\blindtext


\begin{itemize}
\item general approach, motivation via smoothing histograms
\item special case: Gaussian kernels
\end{itemize}

\begin{equation}
  p(x) = \frac{1}{n} \sum_{i = 0}^n K_i(x)
\end{equation}

\subsection{Sparse Grid Density Estimation}
\label{sec:sparse-grid-density}

\begin{itemize}
\item general approach
\item \todo{positivity}
\end{itemize}

\section{On probabilistic transformations}
\label{sec:on-prob-transformations}

\begin{itemize}
\item Nataf transformation
  \begin{itemize}
  \item numerical issues with cdf/ppf
  \item computation of corrected correlation matrix via
    Gauss-quadrature and bisection
  \item equal to Rosenblatt for correlated RVs
  \item density estimation needed just for the marginals
  \end{itemize}
\item Rosenblatt transformation
  \begin{itemize}
  \item general approach if cdf and conditional cdf is defined
  \item show Rosenblatt for KDE (refer to Sandia) and SGDE (refer to
    Benjamins work in the context of sampling)
  \end{itemize}
\end{itemize}

\section{Numerical Results}
\label{sec:im-numerical-results}

% -------------------------------------------------------------------------------------

\chapter{Uncertainty propagation}
\label{cha:uncert-prop}

\section{Problem formulation}
\label{sec:problem-formulation}

\section{Adaptive sparse grid collocation}
\label{sec:adaptive-sparse-grid}

\subsection{$hp$-refinement}
\label{sec:hp-refinement}

\begin{itemize}
\item piecewise polynomial basis
\item refinement criteria
\item \todo{adding grid points instead of refining}
\item \todo{sublinearity}
\end{itemize}

\subsection{Probabilistic analysis}
\label{sec:probabi-analysis}

\begin{itemize}
\item moment estimation: quadrature (analytic, monte carlo)
\item sobol indices: based on moment estimation
\item risk analysis: based on Monte Carlo quadrature
\end{itemize}

\section{Non-intrusive polynomial chaos expansion}
\label{sec:non-intr-polyn}

\begin{itemize}
\item definition
\item higher-
\end{itemize}

\subsection{Askey scheme}
\label{sec:askey-scheme}

\begin{itemize}
\item present case for independent marginals from a family of
  distributions
\end{itemize}

\subsection{Arbitrary expansion}
\label{sec:arbitrary-expansion}

\begin{itemize}
\item present case for arbitrary distributions
\item marginals: three-term-recurrence
\item orthogonalization: transformation matrix based on moments of
  distributions; implementation: Cholesky factor of Gramian matrix or,
  better, QR of Vandermonde like marix
\end{itemize}

\subsection{Sampling rules}
\label{sec:sampling-rules}

\begin{itemize}
\item probabilistic transformation of optimal samplings in independent
  case
\item Leja sequences
\item root search of orthogonal polynomial (out of scope)
\end{itemize}

\section{Numerical Results}
\label{sec:up-numerical-results}

% -------------------------------------------------------------------------------------

\chapter{Data assimilation}
\label{cha:data-assimilation}

\section{Markov-Chain Monte Carlo}
\label{sec:markov-chain-monte}


\section{Numerical Results}
\label{sec:da-numerical-results}


\begin{itemize}
\item Bayesian inverse problem: definition
\item application of leja + apce and sgde + asgc
\end{itemize}

% -------------------------------------------------------------------------------------

\chapter{Summary and Outlook}
\label{cha:summary-outlook}



%%% --------------------------------------------------------------------
%%% Local Variables:
%%% mode: latex
%%% TeX-master: "../thesis"
%%% End:

\chapter{Some Random Ideas That May or May Not Be Helpful}
\section{Univers-Font}
To disable setting this document using University of Stuttgart's very own
``Univers'' font, comment out the following lines in preamble/template.tex:
\begin{minted}{latex}
\usepackage{fontspec}
\setmainfont{Univers for UniS 55 Roman Rg}
\end{minted}
Note, that the font needs to be installed for building the document.

\section{Sync Bibfile}
In case you want to use the official SGS .bib - file from this repository, it
may be helpful to check if the file in your diss repo is in sync with the SGS
repo.

The following pre- or post-commit hook might be helpful.
\begin{minted}{python}
#!/usr/bin/env python

import difflib
import sys

def syncFile():
    print ("synching lib file")
    file_in_repo = open("library.bib")
    c_repo = file_in_repo.readlines()
    file_in_sgs  = open("/home/lahnerml/SGitS/bibliothek/jabref/library.bib")
    c_sgs = file_in_sgs.readlines()
    diff = difflib.unified_diff(c_repo, c_sgs)
    if len(list(diff)):
        print ("library.bib differs from SGS repo.")
        sys.exit(-1)

def main(args=None):
    syncFile()

if __name__ == "__main__":
    main()
\end{minted}

\section{Build with latexmk}
If you prefer latexmk over SCons, you may consider the following Makefile
\begin{minted}{Makefile}
NAME=diss
FIGURES=figures

all: build reload

build:
	$(MAKE) -C $(FIGURES)
	@latexmk ${NAME}.tex

reload:
	@pkill -HUP mupdf || true

clean:
	$(MAKE) -C $(FIGURES) clean
	@latexmk -c

clean-all:
	$(MAKE) -C $(FIGURES) clean-all
	@latexmk -C
\end{minted}
The Makefile for the figures folder may look like this:
\begin{minted}{Makefile}
all: build

build:
	@latexmk

clean:
	@latexmk -c

clean-all:
	@latexmk -C
\end{minted}

The template requires some customizations for latexmk.
latexmk can be customized by creating a file called ``latexmkrc'' in the
respective folder.
The customizations for the main directory may look like this:
\begin{minted}{perl}
 # Custom dependency and function for nomencl package
 add_cus_dep( 'nlo', 'nls', 0, 'makenlo2nls' );
 sub makenlo2nls {
 system( "makeindex -s nomencl.ist -o \"$_[0].nls\" \"$_[0].nlo\"" );
 }

# $makeindex = 'texindy'; #$
add_cus_dep('idx', 'ind', 0, 'texindy');
sub texindy{
        system("texindy \"$_[0].idx\""); #$
}

# build a pdf
$pdf_mode = 1; #$

# Custom cleaning rule
$clean_ext = 'synctex.gz synctex.gz(busy) run.xml tex.bak thm upa upb nls nlo \
              bbl bcf fdb_latexmk run tdo %R-blx.bib'
\end{minted}

%% -------------------------------------------------------------

\printbibliography
\noindent
Alle URLs wurden zuletzt am \today~gepr\"uft.

\clearpage
\listoffigures
\listoftables

\markleft{List of Theorems}{}
\markright{List of Theorems}{}
\chapter*{List of Theorems}
\addcontentsline{toc}{chapter}{List of Theorems}
\listtheorems{definition}

\appendix

% 'Anhang' ins Inhaltsverzeichnis
\phantomsection
\addcontentsline{toc}{part}{Appendix}

\chapter{Various}

%\input{content/Z-Anhang}

%% --------------------------------------------------------------
% nomenclature


%% -------------------------------------------------------
%% nomenclature
%% -------------------------------------------------------
% % introduction
\nomenclature[s1000]{\textbf{1}}{\textbf{introduction}}%
\nomenclature[a1001]{UQ}{uncertainty quantification}%

%% -------------------------------------------------------
% % problem formulation
\nomenclature[s2000]{\textbf{2}}{\textbf{problem formulation}}%
\nomenclature[s2001]{$(\Omega, \Sigma, P)$}{sample space $\Omega$,
  sigma-algebra $\Sigma$ and measure $P$}%
\nomenclature[s2002]{$\vxi = (\xi_1, \ldots, \xi_d) \in
  \Omega$}{random event of the samples space $\Omega$}%
\nomenclature[s2003]{$f$}{probability density function of $\vxi$}%
\nomenclature[s2004]{$\vx \in D \subset \bbR^{d_s}$}{physical domain
  $D$ of dimensionality $1 \leq d_s \leq 3$}%
\nomenclature[s2005]{$t \in T$}{temporal domain $T \subset
  \mathbb{R}$}%
\nomenclature[s2006]{$\model$}{physical model}%
\nomenclature[s2007]{$u(\vx, t, \vxi) = \model(\vx, t, \vxi) \in
  \bbR^{d_r}$}{model function with corresponding parameters and
  dimensionality $d_r$ of the response vector}%
\nomenclature[s2008]{$Q[u(\vx, t, \vxi)]$}{quantity of interest from
  model function $u$, this is going to be approximated by $g_\Ind$}%
\nomenclature[s2009]{$\mathcal{D} := \{\vxik\}_{i = 1}^n$}{sample set;
  one realization of $f$}%
\nomenclature[s2010]{$\vxik$}{one sample of dimensionality $d$}%
\nomenclature[s2011]{$n$}{number of samples available from $f$}%
\nomenclature[s2012]{$d$}{the number of stochastic dimensions,
  $|\vxik| = d$}%
\nomenclature[s2013]{$\mean_f(u)$}{expectation value of $u$}%
\nomenclature[s2014]{$\var_f(u)$}{variance of $u$}%

% --------------------------------------------------------
% % stochastic collocation
\nomenclature[s3100]{\textbf{3.1}}{\textbf{stochastic collocation}}%
\nomenclature[s3101]{$g$}{surrogate model for model function $u$}%
\nomenclature[s3102]{$N$}{the number of collocation nodes}%
\nomenclature[s3103]{$\Xi_N$}{set of collocation nodes, which needs to
  be evaluated}%
\nomenclature[s3104]{$\gi$}{surrogate model for model function $u$ at
  selected point in time $t_i$ and space $x_i$}%
\nomenclature[s3105]{$x_i$}{selected point in space}%
\nomenclature[s3106]{$t_i$}{selected point in time}%
% % --------------------------------------------------------
% % sparse grids
\nomenclature[s3200]{\textbf{3.2}}{\textbf{sparse grids}}%
\nomenclature[s3201]{$\level$}{maximum discretization level}
\nomenclature[s3202]{$V_\level$}{space of piece wise $d$-linear
  functions}%
\nomenclature[s3203]{$V_\level^{(1)}$}{sparse grid space of piece wise
  $d$-linear functions}%
\nomenclature[s3204]{$\vli$}{multi-dimensional level index pair with
  $(\vli) \in \Ind$}%
\nomenclature[s3205]{$\Ind$}{set of collocation nodes as pair of level
  and index}%
\nomenclature[s3206]{$\psi_\vli$}{piecewise linear basis}%
\nomenclature[s3207]{$W_\vl$}{hierarchical subspaces}%
\nomenclature[s3208]{$|\vl|_\infty$}{infinity norm of level $\vl$}%
\nomenclature[s3209]{$|\vl|_1$}{one norm of level $\vl$}%
\nomenclature[s3210]{$g_\Il$}{regular sparse grid function}%
\nomenclature[s3210]{$g_\Ind$}{adaptively refined sparse grid function}%
\nomenclature[s3211]{$v_\vli$}{hierarchical coefficients, for SG
  surrogate model}%
\nomenclature[s3212]{$p$}{degree of polynomial basis}%
\nomenclature[s3213]{$\psip_\vli$}{piecewise $d$-polynomial basis}%
% \nomenclature[s]{$w_\vli$}{hierarchical coefficients, for $\hat{p}$}%
% \nomenclature[s]{$\varphi_\vkj$}{piecewise linear basis $p = 1$,
% polynomial basis $p > 1$ functions}%
% % --------------------------------------------------------
% % density estimation
\nomenclature[s3300]{\textbf{3.3}}{\textbf{Sparse Grid Density
    Estimation method}}%
\nomenclature[a3301]{SGDE}{Sparse Grid Density Estimation}%
\nomenclature[s3302]{$R$}{functional to be minimized}%
\nomenclature[s3303]{$\fhk$}{sparse grid function that minimizes $R$
  defined on level, index set $\Knd$}%
\nomenclature[s3304]{$\lambda$}{regularization parameter}%
\nomenclature[s3305]{$S$}{regularization method}%

% % --------------------------------------------------------
% % Sparse Grid Collocation
\nomenclature[s3400]{\textbf{3.4}}{\textbf{Sparse Grid Collocation}}%
\nomenclature[a3401]{SGC}{Sparse Grid Collocation}%

% % --------------------------------------------------------
% % Moment estimation
\nomenclature[s3500]{\textbf{3.5}}{\textbf{Moment Estimation}}%
\nomenclature[s3501]{$\mu$}{abbreviation for true expectation value,
  i.e. $\mean_f(u)$}%
\nomenclature[s3502]{$\sigma^2$}{abbreviation for true variance,
  i.e. $\var_f(u)$}%
\nomenclature[s3503]{$\fh$}{estimated density function}%
\nomenclature[s3504]{$\Dhcal$}{sample set drawn from $\fh$}%
\nomenclature[s3505]{$\nh$}{number of samples of $\Dhcal$}%
\nomenclature[s3506]{$\fhd$}{dirac delta density}%
\nomenclature[s3507]{$\delta$}{dirac delty function}%
\nomenclature[s3508]{$\vkj$}{alternative hierarchical coefficients
  to $\vli$}%
\nomenclature[s3509]{$\vlit$}{alternative hierarchical coefficients
  to $\vli$}%
\nomenclature[s3510]{$\varphiq$}{alternative basis to $\psip$}%
\nomenclature[s3511]{$q$}{polynomial degree of $\varphiq$ basis}%

% % --------------------------------------------------------
% % The Sparse Grid data-driven UQ forward Pipeline
\nomenclature[s3600]{\textbf{3.6}}{\textbf{The Sparse Grid data-driven
    UQ forward Pipeline}}%
\nomenclature[s3601]{$\epsilon$}{machine epsilon}%

% % --------------------------------------------------------
% % Analytic Example
\nomenclature[s4100]{\textbf{4.1}}{\textbf{Analytic Example}}%
\nomenclature[s4101]{$\Bcal$}{Beta distribution}%
\nomenclature[s4102]{$\alpha, \beta$}{shape parameters for beta
  distribution}%
\nomenclature[s4103]{$c_k$}{$1/B(\alpha, \beta)$ coefficient of pdf of
beta distribution, where $B$ is the beta function}%
\nomenclature[s4104]{$\Gamma$}{gamma function}%
\nomenclature[s4105]{$\Omega_1$}{domain for $\xi_1$}%
\nomenclature[s4106]{$\Omega_2$}{domain for $\xi_2$}%
\nomenclature[s4107]{$\Omega$}{complete domain $\Omega = \Omega_1
  \times \Omega_2$}%
\nomenclature[s4108]{$\partial \Omega$}{border of domain $\Omega$}%
\nomenclature[s4109]{$k$}{number of realizations of $\Dcal$}%
\nomenclature[s4110]{$\Dcal_k$}{$k^{\text{th}}$ realization of $f$}%
\nomenclature[a4111]{SG$^{\text{++}}$}{SGpp}%
\nomenclature[s4112]{$\Tcal$}{test set of size $m$}%
\nomenclature[s4113]{$m$}{size of test set}%
\nomenclature[s4114]{$L$}{cross entropy (negative log-likelihood)}%
\nomenclature[a4115]{L}{cross entropy (negative log-likelihood)}%
\nomenclature[a4116]{KL}{Kullback-Leibler divergence}%

% % --------------------------------------------------------
% % Multivariate Stochastic Application
\nomenclature[s4200]{\textbf{4.2}}{\textbf{Multivariate Stochastic
    Application}}%
\nomenclature[s4201]{$\phi$}{reservoir porosity}%
\nomenclature[s4202]{$\Ka$}{reservoir permeability}%
\nomenclature[s4203]{$\KL$}{leaky well permeability}%
\nomenclature[s4204]{$f_1, f_2, f_3$}{density functions for different
  parameters ($\phi, \Ka, \KL$)}%
\nomenclature[s4205]{$\nu$}{variation}%
\nomenclature[s4206]{$h$}{transformation function from stochastic
  space to parameter space}%
\nomenclature[s4207]{$\Omega_\phi$}{domain of $\phi$}%
\nomenclature[s4208]{$\Omega_\nu$}{domain of $\nu$}%
\nomenclature[s4209]{$\Omega_\KL$}{domain of $\KL$}%
\nomenclature[s4210]{$\Tcal$}{test set}%
\nomenclature[s4211]{$m$}{number of samples in the test set}%

%%% --------------------------------------------------
%%% Local Variables:
%%% mode: latex
%%% TeX-master: "thesis"
%%% End:
%%% --------------------------------------------------

\IfDefined{printindex}{\printindex}
\IfDefined{printnomenclature}{
  \markleft{Nomenclature}{}
  \markright{Nomenclature}{}
  \printnomenclature
}

\end{document}
