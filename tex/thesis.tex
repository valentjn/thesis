% needed for custom switches
\RequirePackage{etoolbox}

% checkmode switch:
% show all overfull/underfull boxes, show unknown hyphenations
\newtoggle{checkmode}
\toggletrue{checkmode}

% debugmode switch:
% show all boxes, glues, and kerning info
\newtoggle{debugmode}
\togglefalse{debugmode}

% warns about outdated packages and missing caption declarations
\RequirePackage[
  l2tabu,
  orthodox,
]{nag}

% set main document class and options
\documentclass[
  % choose font size (PO: should be between 12pt and 14pt)
  fontsize=12pt,
  % let typearea calculate appropriate page margins
  DIV=calc,
  % don't use KOMA-Script's default "big" headings
  headings=normal,
]{scrbook}

% check mode
\iftoggle{checkmode}{
  % show overfull boxes
  \overfullrule=1mm
  % show underfull vboxes
  \directlua{dofile("detect_underfull.lua")}
  % show non-whitelisted hyphenations
  \usepackage[mark]{lua-check-hyphen}
  % whitelist is in hyphenation_whitelist.txt:
  % one word per line with dashes (-) indicating the places where
  % hyphenation is whitelisted
  \LuaCheckHyphen{whitelist=hyphenation_whitelist.txt}
}{}

% debug mode
\iftoggle{debugmode}{
  % show boxes, glues, and kerning
  \usepackage{lua-visual-debug}
}{}

% language-specific things (hyphenation, ...)
\usepackage[ngerman,american]{babel}

% biblatex package wants to have csquotes (otherwise: warning)
\usepackage{csquotes}

% dummy texts
\usepackage[math]{blindtext}

% basic AMS math commands
\usepackage{amsmath}

% micro-typographic adjustments
% (final: don't deactivate extensions due to document class draft option)
\usepackage[final]{microtype}

% Unicode umlauts
\usepackage[utf8]{luainputenc}

% need T1 font encoding for Charter,
% otherwise there will be "undefined font shape" warnings
\usepackage[T1]{fontenc}

% use Bitstream Charter as main font
\usepackage[bitstream-charter]{mathdesign}

% don't use another, sans-serif font for headings
\setkomafont{disposition}{\normalcolor\bfseries}

% graphics
\usepackage{graphicx}

% X column type for tables (filling rest of line)
\usepackage{tabularx}

% line spacing (one and a half lines)
\usepackage[onehalfspacing]{setspace}

% location of graphics files
\graphicspath{{../gfx/}}

% chapter heading: make number bigger
\renewcommand*{\chapterformat}{%
  \mbox{%
    \chapappifchapterprefix{\nobreakspace}%
    \scalebox{3.5}{\thechapter\autodot}%
    \IfUsePrefixLine{}{\enskip}%
  }%
}

% chapter heading: align number and text at the baseline of the last line
% of the text, add some more space between number and text
\renewcommand*{\chapterlinesformat}[3]{%
  \begin{tabularx}{\textwidth}{@{}l@{}X@{}}%
    % chapter number
    #2%
    % add hspace between number and text, but only if \chapter (not \chapter*)
    % has been used (i.e., if the chapter number #2 is not empty)
    \if\relax\detokenize{#2}\relax\else\hspace*{5mm}\fi&%
    % text of heading, [b] is for alignment, \linewidth is width of X cell,
    % \raggedchapter for not justifying headings
    \parbox[b]{\linewidth}{\raggedchapter#3}%
  \end{tabularx}%
}

% PDF links
\usepackage{hyperref}

% BibLaTeX
\usepackage{biblatex}

% location of *.bib file
\addbibresource{../bib/bibliography.bib}

% named references
\usepackage{cleveref}

% meta-data variables
\newcommand*{\thetitle}{%
  B-Splines for Sparse Grids:\texorpdfstring{\\}{}
  Algorithms and Application to
  Higher-Dimensional Optimization%
}
\newcommand*{\theauthor}{Julian Valentin}
\newcommand*{\thedate}{TODO(date)}

% set up hyperref
\hypersetup{
  pdftitle={\thetitle},
  pdfauthor={\theauthor},
  pdfcreator={LaTeX, KOMA-Script, hyperref},
}

% change bibliography heading (default: "Bibliography")
\DefineBibliographyStrings{english}{bibliography={References}}

\begin{document}
  % title page
  % set title page one-sided
\KOMAoptions{twoside=false}

\begin{titlepage}
  \begin{spacing}{1}
    \begin{center}
      \begin{otherlanguage}{ngerman}
        % no indentation of paragraphs
        \setlength{\parindent}{0pt}
        
        \includegraphics[width=60mm]{logo_us}
        
        \vfill
        
        {\bfseries\huge\thetitle\par}
        
        \vfill
        
        Vom Stuttgarter Zentrum für Simulationswissenschaften der\\
        Universität~Stuttgart zur Erlangung der Würde eines Doktors der\\
        Naturwissenschaften (Dr.~rer.~nat.) genehmigte Abhandlung
        
        \vfill
        
        Vorgelegt von
        
        {\bfseries\Large\theauthor\par}
        
        aus Stuttgart
        
        \vfill
        
        \begin{tabular}{ll}
          Hauptberichter:&
          Prof.\ Dr.\ Dirk Pflüger\\[0.5em]
          Mitberichter:&
          \TODO{insert examiner}\\
          &\TODO{insert examiner}\\
          &\TODO{insert examiner}\\[1em]
          \multicolumn{2}{l}{%
            Tag der mündlichen Prüfung:\quad%
            \TODO{insert defense date}%
          }
        \end{tabular}
        
        \vfill
        
        Institut für Parallele und Verteilte Systeme
        
        \vspace{1em}
        
        \TODO{insert year}
      \end{otherlanguage}
    \end{center}
  \end{spacing}
\end{titlepage}

% set other pages two-sided
\KOMAoptions{twoside}

  
  % dedication page
  \cleardoublepage
  % don't display page number
\thispagestyle{empty}

\vspace*{\fill}

\begin{center}
  \includegraphics[width=0.6\textwidth]{dedicationQuote}
  
  \begin{minipage}{0.6\textwidth}%
    \begin{flushright}
      \small--- Carl de~Boor \cite{Boor16Comment}
    \end{flushright}
  \end{minipage}
\end{center}

\vspace*{\fill}

\begin{center}
  \includegraphics{logoBSplines}%
\end{center}

\vspace*{\fill}

\cleardoublepage

  
  % table of contents
  \tableofcontents
  
  % abstract
  \addchap{Abstract/\foreignlanguage{ngerman}{Kurzzusammenfassung}}

\section*{Abstract}

In simulation technology, computationally expensive objective
functions are often replaced
by cheap surrogates, which can be obtained by interpolation.
Full-grid interpolation methods are affected by the curse of dimensionality,
rendering them infeasible if the function has four or more parameters.
Sparse grids are an interpolation method that does not suffer from the
curse while not much deteriorating the approximation quality.
However, conventional basis functions such as piecewise linear functions
are not smooth (continuously differentiable).
Thus, they are unsuitable for applications in which gradients are required,
for example, gradient-based optimization.
This thesis demonstrates that hierarchical B-splines are well-fitted for
obtaining smooth interpolants in higher dimensionalities.
The thesis considers theoretical, algorithmic, and numerical aspects of the
new basis as well as three different real-world applications in optimization:
topology optimization, biomechanical continuum-mechanics, and
dynamic portfolio choice models in finance.
The results show that the optimization problems in these applications
can be solved accurately and efficiently with hierarchical B-splines on
sparse grids.
% 158 words

\begin{otherlanguage}{ngerman}
  \section*{Kurzzusammenfassung}
  
  In der Simulationstechnologie werden zeitaufwendige Zielfunktionen
  oft durch einfache Surrogate ersetzt, die durch Interpolation
  gewonnen werden können.
  Vollgitter-Interpolationsmethoden sind vom Fluch der
  Dimensionalität betroffen, der sie für Funktionen mit vier oder mehr
  Parametern unbrauchbar macht.
  Dünne Gitter sind eine Interpolationsmethode, die nicht unter
  dem Fluch leidet, aber den Approximationsfehler nicht stark verschlechtert.
  Leider sind konventionelle Basisfunktionen wie die stückweise
  lineare Funktionen nicht glatt (stetig differenzierbar).
  Daher sind sie für Anwendungen unbrauchbar, in denen Gradienten
  benötigt werden, zum Beispiel gradientenbasierte Optimierung.
  Diese Dissertation demonstriert, dass hierarchische B-Splines gut
  geeignet sind, um glatte Interpolierende für höhere
  Dimensionalitäten zu erhalten.
  Die Dissertation behandelt sowohl theoretische, algorithmische und numerische
  Aspekte der neuen Basis als auch drei verschiedene
  Realwelt-Anwendungen:
  Topologieoptimierung, biomechanische Kontinuumsmechanik und
  Modelle der dynamischen Portfolio-Wahl in der Finanzmathematik.
  Die Ergebnisse zeigen, dass die Optimierungsprobleme in diesen
  Anwendungen durch hierarchische B-Splines auf dünnen Gittern
  gut und effizient gelöst werden können.
  % 146 Wörter
\end{otherlanguage}

  
  % preface
  \addchap{Preface}

\blindtext

TODO(preface)

\vspace{5mm}

\noindent
Stuttgart, \thedate

\noindent
Julian Valentin

  
  % TODO: dummy chapters
  \chapter{A Long Heading Without Any Meaning and Using Two Lines}
  Hello World!
  
  Umlaute: äöüß
    
  \begin{gather}
    X \times Y\\
    A \cdot \vec{x} = \vec{b}\\
    \min_{\vec{x} \in [0, 1]} \int_\Omega f(\vec{x}, \vec{y}) d\vec{y}
  \end{gather}
  
  \section{Abschnitt}
  
  \blindtext
  
  \subsection{Unterabschnitt}
  
  \blindtext
  
  \chapter{Kapitel 1}
  \blindmathpaper
  
  \chapter{Kapitel 2}
  \blindmathpaper
  
  % start appendix
  \appendix
  
  % TODO: cite all for debug purposes
  \nocite{*}
  % references (add to TOC)
  \printbibliography[heading=bibintoc]
\end{document}

\endinput









documentclass[
  paper=a4,
  twoside,
  fontsize=12pt,
  DIV=calc,
  headings=small,
  bibliography=totoc,
  listof=totoc,
  headsepline,
  footsepline,
  plainfootsepline,
]{scrbook}

%input{template}
usepackage{titlepage}

input{commands}
bibliography{../bib/bibliography}

\graphicspath{{../gfx/}}

% nomenclature
\renewcommand{\nomgroup}[1]{
  \ifthenelse{\equal{#1}{S}}{\item[\bfseries\sffamily\large{List of Symbols}]}{
    \ifthenelse{\equal{#1}{A}}{\item[\bfseries\sffamily\large{List of Abbreviations}]}{}}
}
\makenomenclature

\begin{document}
% hyphenations
input{hyphenation}

% Die Seitennummerierung erfolgt durchlaufend ab der Titelseite. Also keine
% Spielereien mit römischen Ziffern usw. - Die ISO 7144 schreibt das sogar für
% wissenschaftliche Werke vor.
% Von Promitionsordnung verlangt!
% Deshalb ist \frontmatter DEAKTIVIERT
%\frontmatter
\title{%
  B-Splines for Sparse Grids: Algorithms and
  Application to Higher-Dimensional Optimization%
}
\author{Julian Valentin}
\date{TODO}
\keywords{TODO}
\firstexaminer{Prof.~Dr.~Dirk Pflüger}
\secondexaminer{TODO\\TODO\\TODO}
\dateofexamination{TODO}
\placeofbirth{Stuttgart}
\faculty[Vom]{Stuttgart Research Centre for Simulation Technology}
\institute{Institut für Parallele und Verteilte Systeme}

\maketitle

% This is necessary if you have more than 9 sections/subsections/subsubsections
\makeatletter
\renewcommand\l@section{\@dottedtocline{1}{1.5em}{3em}}
\renewcommand\l@subsection{\@dottedtocline{2}{1.5em}{4.3em}}
\renewcommand\l@subsubsection{\@dottedtocline{3}{1.5em}{5.6em}}
\makeatother

\tableofcontents
\clearpage

\pagestyle{justpagenums}
% input{abstract}

%\mainmatter
\pagestyle{scrheadings}

%% ---- abstracts ----------------------------------------------
input{abstracts-danksagung}
%% -------------------------------------------------------------
\pagestyle{scrheadings}

%% ---- Content ------------------------------------------------
% input{examples}
input{introduction}
input{this_and_that}
%% -------------------------------------------------------------

\printbibliography
\noindent
Alle URLs wurden zuletzt am \today~gepr\"uft.

\clearpage
\listoffigures
\listoftables

\markleft{List of Theorems}{}
\markright{List of Theorems}{}
\chapter*{List of Theorems}
\addcontentsline{toc}{chapter}{List of Theorems}
\listtheorems{definition}

\appendix

% 'Anhang' ins Inhaltsverzeichnis
\phantomsection
\addcontentsline{toc}{part}{Appendix}

\chapter{Various}

%input{Z-Anhang}

%% --------------------------------------------------------------
% nomenclature
input{nomenclature}

\IfDefined{printindex}{\printindex}
\IfDefined{printnomenclature}{
  \markleft{Nomenclature}{}
  \markright{Nomenclature}{}
  \printnomenclature
}

\end{document}
