\section{Hierarchization on Full Grids (Unidirectional Principle)}

If $\Omega^\sparse$ is a full grid $\Omega_\ßl$
(see \cref{sec:21nodalSpaces}),
the well-known \emph{\up}
can be used to apply $L$ to input data $\ßu$.
Its idea is to apply the corresponding one-dimensional operators on the
one-dimensional subgrids (the \emph{poles}) of $\Omega^\sparse$,
which is repeated for all dimensions.
In this section, we first formulate the \up for
general linear operators $L$ and then prove its correctness for
the case $L = A^{-1}$ of hierarchization.
The correctness for the general case of arbitrary tensor product operators
will follow from the considerations in \cref{sec:44spatAdaptive}.

The \up is given in \cref{alg:unidirectionalPrinciple}.
The algorithm is given a permutation $(t_1, \dotsc, t_d)$ of $(1, \dotsc, d)$
that specifies the order of dimensions in which the \up should be applied.
We denote with $L_{t_j}$ the one-dimensional version of $L$
applied in dimension $t_j$ ($j = 1, \dotsc, d$).

\begin{algorithm}
  \begin{algorithmic}[1]
    \Function{$\ßy =$ unidirectionalPrinciple}{%
      $K$, $(t_1, \dotsc, t_d)$, $\ßu$%
    }
      \State{$\ßy^{(0)} \gets \ßu$}
      \For{$j = 1, \dotsc, d$}
        \State{%
          Define equivalence relation $\sim$ on $K$
          as $\ßk \sim \ßk' \iff \ßk_{-t_j} = \ßk'_{-t_j}$%
        }
        \For{$K_\mathrm{pole} \in K/{\sim}$}
          \State{%
            $(y_\ßk^{(j)})_{\ßk \in K_\mathrm{pole}} \gets
            L_{t_j}\Big[(y_\ßk^{(j-1)})_{\ßk \in K_\mathrm{pole}}\Big]$%
          }
          \label{line:algUnidirectionalPrinciple1}
        \EndFor{}
      \EndFor{}
      \State{$\ßy \gets \ßy^{(d)}$}
    \EndFunction{}
  \end{algorithmic}
  \caption{%
    Application of a tensor product operator $L$ with
    the unidirectional principle.
    Inputs are the set $K$ of grid indices,
    the permutation $(t_1, \dotsc, t_d)$ specifying the order in which
    the one-dimensional operators $L_{t_j}$ should be applied, and
    the vector $\ßu = (u_\ßk)_{\ßk \in K}$ of input data.
    The output is the vector $\ßy = (y_\ßk)_{\ßk \in K}$
    of output data.%
  }%
  \label{alg:unidirectionalPrinciple}%
\end{algorithm}

\begin{proposition}[%
  invariant of \cref{alg:unidirectionalPrinciple} for hierarchization%
]
  \label{prop:invariantUnidirectionalPrinciple}
  Let $T$ be the hierarchization operator on a full grid,
  i.e.,
  $T = A^{-1}$,
  $\ßu = (f_\ßk)_{\ßk \in K}$,
  $\ßy = (\alpha_\ßk)_{\ßk \in K}$,
  $T_{t_j}$ is the 1D interpolation operator $(A_{t_j})^{-1}$, and
  $K = \{\ß0, \dotsc, \ß2^\ßl\}$
  corresponds to a full grid $\Omega_\ßl$ of level $\ßl$.
  After iteration $j$ of \cref{alg:unidirectionalPrinciple}
  ($j = 1, \dotsc, d$), it holds for $T := (t_1, \dotsc, t_j)$
  \begin{equation}
    \sum_{\ßk_T=\ß0}^{\ß2^{\ßl_T}}
    \alpha^{(j)}_{(\ßk_T,\ßk'_{-T})} \varphi_{\ßk_T}(\ßx_{\ßk'_T})
    = f_{\ßk'},\quad
    \ßk' = \ß0, \dotsc, \ß2^\ßl,
  \end{equation}
  where $(\ßk_T,\ßk'_{-T})$ is defined to be the index $\ßk''$
  with $\ßk''_T := \ßk_T$ and $\ßk''_{-T} := \ßk'_{-T}$.
\end{proposition}

\begin{proof}
  We prove the assertion by induction over $j = 1, \dotsc, d$.
  We set $T' := (t_1, \dotsc, t_{j-1})$,
  $T := (t_1, \dotsc, t_{j-1}, t_j)$,
  and we exploit the tensor-product structure of the basis
  to write the left-hand side of the assertion for $j$ as
  \begin{equation}
    \underbrace{
      \sum_{\ßk_T=\ß0}^{\ß2^{\ßl_T}}
      \alpha^{(j)}_{(\ßk_T,\ßk'_{-T})} \varphi_{\ßk_T}(\ßx_{\ßk'_T})
    }_{{} =: (\ast)}
    = \sum_{\ßk_{T'}=\ß0}^{\ß2^{\ßl_{T'}}}
    \varphi_{\ßk_{T'}}(\ßx_{\ßk'_{T'}}) \cdot
    \underbrace{
      \sum_{k_{t_j}=0}^{2^{l_{t_j}}}
      \alpha^{(j)}_{(\ßk_T,\ßk'_{-T})} \varphi_{k_{t_j}}(x_{k'_{t_j}})
    }_{{} =: (\ast\ast)},
  \end{equation}
  where $\ßk' = \ß0, \dotsc, \ß2^\ßl$ is arbitrary.
  For the equivalence class $K_\mathrm{pole} := [(\ßk_T,\ßk'_{-T})]_\sim$
  ($\ßk_T$ arbitrary, $\sim$ as defined in iteration $j$), we have
  \begin{equation}
    \label{eq:proofInvariantUnidirectionalPrinciple1}
    (\ast\ast)
    = \sum_{\ßk \in K_\mathrm{pole}}
    \alpha^{(j)}_\ßk \varphi_{k_{t_j}}(x_{k'_{t_j}})
    %= ((L_{t_j})^{-1}
    %\Big[(\alpha_\ßk^{(j)})_{\ßk \in K_\mathrm{pole}}\Big])_{k'_{t_j}}
    %= ((\alpha_\ßk^{(j-1)})_{\ßk \in K_\mathrm{pole}})_{k'_{t_j}}
    = \alpha^{(j-1)}_{(\ßk_{T'},\ßk'_{-T'})}
  \end{equation}
  by the 1D interpolation operator $T_{t_j}$
  (\cref{line:algUnidirectionalPrinciple1} of
  \cref{alg:unidirectionalPrinciple}).
  Consequently,
  \begin{equation}
    (\ast)
    = \sum_{\ßk_{T'}=\ß0}^{\ß2^{\ßl_{T'}}}
    \alpha^{(j-1)}_{(\ßk_{T'},\ßk'_{-T'})}
    \varphi_{\ßk_{T'}}(\ßx_{\ßk'_{T'}}),
  \end{equation}
  which, by the induction hypothesis, equals $f_{\ßk'}$ as desired.
  \Cref{eq:proofInvariantUnidirectionalPrinciple1} can be used
  for $j = 0$ to establish the base case.
\end{proof}

\begin{corollary}
  \Cref{alg:unidirectionalPrinciple}
  is correct for the case of hierarchization.
\end{corollary}

\begin{proof}
  We apply \cref{prop:invariantUnidirectionalPrinciple} to obtain
  $\sum_{\ßk=\ß0}^{\ß2^\ßl}
  \alpha^{(j)}_\ßk \varphi_{\ßk}(\ßx_{\ßk'})
  = f_{\ßk'}$
  for all $\ßk' = \ß0, \dotsc, \ß2^\ßl$, i.e.,
  the $\alpha^{(j)}_\ßk$ are the correct interpolation coefficients
  according to \eqref{eq:hierarchizationProblem}.
\end{proof}

\blindtext{}
