\section{Hierarchization on Full Grids (Unidirectional Principle)}

\todo{get acronyms to work}
If $\Omega^\sparse$ is a full grid $\Omega_\ßl$
(see \cref{sec:21nodalSpaces}),
the well-known \emph{unidirectional principle}
can be used to apply $L$ to input data $\ßu$.
\todo{get abbreviations to work}
Its idea is to apply the corresponding one-dimensional operators on the
one-dimensional subgrids (the \emph{poles}) of $\Omega^\sparse$,
which is repeated for all dimensions.
In this section, we first formulate the unidirectional principle for
general linear operators $L$ and then prove its correctness for
the case $L = A^{-1}$ of hierarchization.
The correctness for the general case of arbitrary tensor product operators
will follow from the considerations in \cref{sec:44spatAdaptive}.

The unidirectional principle is given in \cref{alg:unidirectionalPrinciple}.
The algorithm is given a permutation $(t_1, \dotsc, t_d)$ of $(1, \dotsc, d)$
that specifies the order of dimensions in which the
unidirectional principle should be applied.
We denote with $L_{t_j}$ the one-dimensional version of $L$
applied in dimension $t_j$ ($j = 1, \dotsc, d$).

\begin{algorithm}
  \begin{algorithmic}[1]
    \Function{$\ßy =$ unidirectionalPrinciple}{%
      $K$, $(t_1, \dotsc, t_d)$, $\ßu$%
    }
      \State{$\ßy^{(0)} \gets \ßu$}
      \For{$j = 1, \dotsc, d$}
        \State{%
          Define equivalence relation $\sim$ on $K$
          as $\ßk \sim \ßk' \iff \ßk_{-t_j} = \ßk'_{-t_j}$%
        }
        \For{$K_\mathrm{pole} \in K/{\sim}$}
          \State{%
            $(y_\ßk^{(j)})_{\ßk \in K_\mathrm{pole}} \gets
            L_{t_j}\Big[(y_\ßk^{(j-1)})_{\ßk \in K_\mathrm{pole}}\Big]$%
          }
        \EndFor{}
      \EndFor{}
      \State{$\ßy \gets \ßy^{(d)}$}
    \EndFunction{}
  \end{algorithmic}
  \caption{%
    Application of a tensor product operator $L$ with
    the unidirectional principle.
    Inputs are the set $K$ of grid indices,
    the permutation $(t_1, \dotsc, t_d)$ specifying the order in which
    the one-dimensional operators $L_{t_j}$ should be applied, and
    the vector $\ßu = (u_\ßk)_{\ßk \in K}$ of input data.
    The output is the vector $\ßy = (y_\ßk)_{\ßk \in K}$
    of output data.%
  }%
  \label{alg:unidirectionalPrinciple}%
\end{algorithm}

\begin{proposition}[%
  invariant of \cref{alg:unidirectionalPrinciple} for hierarchization%
]
  Let $T$ be the hierarchization operator and $\Omega^\sparse$ be the
  full grid $\Omega_\ßl$ of level $\ßl \in \NN_0^d$.
  After iteration $j$ of \cref{alg:unidirectionalPrinciple}
  ($j = 1, \dotsc, d$), it holds for $T := (t_1, \dotsc, t_j)$
  \begin{equation}
    \sum_{\ßk_T=\ß0}^{\ß2^{\ßl_T}}
    \alpha^{(j)}_{\ßk''} \varphi_{\ßk_T}(\ßx_{\ßk'_T})
    = f_{\ßk'},\quad
    \ßk' = \ß0, \dotsc, \ß2^\ßl,\quad
    \ßk''_T := \ßk_T,\quad
    \ßk''_{-T} := \ßk'_{-T}.
  \end{equation}
\end{proposition}

\begin{corollary}
  \Cref{alg:unidirectionalPrinciple} is correct for the case of hierarchization.
\end{corollary}

\blindtext{}
