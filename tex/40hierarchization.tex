\setdictum{%
  Who are you? How did you get in my house?%
}{%
  Donald E. Knuth about one-based array indices in algorithms
  (according to \texttt{xkcd}\footnotemark)%
}

\chapter{Algorithms for B-Splines on Sparse Grids}
\footnotetext{\url{https://xkcd.com/163/}}

%\lettrine{A}{s} we have seen in the last chapter,
%hierarchical B-splines constitute an interesting basis on sparse grids,
%enabling smooth approximations of objective functions.
%When correctly incorporating boundary conditions into the hierarchical basis,
%the basis can exactly represent polynomials,
%which is a prerequisite for higher approximation orders.

\lettrine{L}{et}
us study the algorithmic challenges that the new hierarchical B-spline
bases pose.
As we are usually not able to apply the sparse grid algorithms
that were designed for hat functions, we have to generalize the algorithms
to higher-order B-splines or even to arbitrary tensor product basis functions.
We learn in this chapter that the main problem is the larger support
of higher-order B-splines (degree $p > 1$) when compared to degree $p = 1$.
Unfortunately, it is not possible to simply ``transform'' the basis
functions $\bspl{l,i}{p}{}$ to shrink the size of their support
to $\supp \bspl{l,i}{1}{}$.

In this chapter, we give an overview of algorithms for B-splines
on sparse grids.
Some algorithms are already known from the literature,
while others are new.
In any case, we give correctness results for every algorithm.
We take hierarchization as the example problem for our algorithms,
but the algorithms can be generalized to any linear operator.
Furthermore, most algorithms are not tailored for B-splines $\bspl{l,i}{p}{}$,
but applicable for general tensor product basis functions $\basis{l,i}$.
The types of feasible approaches for algorithms
depend on the ``regularity'' of the sparse grid at hand
(full grid/dimensionally adaptive sparse grid/spatially adaptive sparse grid).
Naturally, the more assumptions the grid satisfies, the faster and
easier will the corresponding algorithms be.

\Cref{sec:41problem} explains hierarchization as our example problem
and defines the notation used in this chapter.
The remaining three sections treat the different cases of grids:
First, \cref{sec:42fullGrids} deals with full grids to formalize and repeat
the well-known unidirectional principle.
Second, \cref{sec:43dimAdaptive} studies algorithms for
sparse grids that have been generated dimensionally adaptively.
Third, \cref{sec:44spatAdaptive} treats the for us most important case
of arbitrary (spatially adaptive) sparse grids.

This original chapter is the main theoretical contribution of this thesis.
Although the unidirectional principle in \cref{sec:42fullGrids} and
the combination technique in \cref{sec:43dimAdaptive} are well-known,
the presentation with formal proofs of correctness is new.
Parts of the chapter have already been published in scientific papers
\cite{Valentin18Fundamental}.



\section{The Hierarchization Problem}
\label{sec:41problem}

Let $\sgset \subset \clint{0, 1}^d$ be a general (sparse) grid that
may be spatially adaptive, i.e.,
of the form $\setsize{\sgset} = \{\vgp{\ßl,\ßi} \mid (\ßl, \ßi) \in \liset\}$,
where $\liset$ is a set of level-index pairs $(\ßl, \ßi)$ with $\ßl \in \natz$
and $\ßi \in \hiset{\ßl}$ such that
$\ngp := \setsize{\sgset} = \setsize{\liset} < \infty$
(see \cref{sec:233spatiallyAdaptiveSG}).
The \term{hierarchization problem} is finding
\term{hierarchical surpluses}
$(\surplus{\ßl',\ßi'})_{(\ßl',\ßi') \in \liset} \in \real^\ngp$ such that
\begin{equation}
  \label{eq:hierarchizationProblem}
  \sum_{\mathclap{(\ßl', \ßi') \in \liset}} \surplus{\ßl',\ßi'}
  \basis{\ßl',\ßi'}(\vgp{\ßl,\ßi}) = \fcnval{\ßl,\ßi}
  \quad\text{for all}\quad
  (\ßl, \ßi) \in \liset,
\end{equation}
where $(\fcnval{\ßl,\ßi})_{(\ßl,\ßi) \in \liset} \in \real^\ngp$ is a given set of
function values $\objfun(\vgp{\ßl,\ßi})$ at the grid points $\vgp{\ßl,\ßi}$.
The hierarchical surpluses then define the interpolant $\sgintp$ as
\begin{equation}
  \sgintp\colon \clint{\ß0, \ß1} \to \real,\quad
  \sgintp :=
  \sum_{\mathclap{(\ßl', \ßi') \in \liset}} \surplus{\ßl',\ßi'}
  \basis{\ßl',\ßi'},
\end{equation}
which interpolates $\objfun$ at the grid points $\vgp{\ßl,\ßi}$ of $\sgset$.

The basis functions $\basis{\ßl',\ßi'}$ are
arbitrary tensor product functions.
We explicitly allow $\basis{\ßl',\ßi'}$ to be a non-hierarchical
nodal basis, in which case $\ßl'$ is constant and
$\sgset$ is usually a full grid.
Strictly speaking, the problem is then a \term{interpolation problem}
and the $\surplus{\ßl',\ßi'}$ are \term{interpolation coefficients}.
However, we still apply the terms
``hierarchization'' and ``hierarchical surpluses'' in this case
to keep the terminology consistent.

\paragraph{Hierarchization as a linear operator}

The example of hierarchization can be generalized
to arbitrary linear operators
\begin{equation}
  \linop\colon \real^\ngp \to \real^\ngp,\quad
  \vlinin \mapsto \vlinout = \linop[\vlinin],
\end{equation}
where $\linop$ depends on the grid $\sgset$ at hand.
Input $\vlinin$ and output $\vlinout$ are scalar-valued data%
%\footnote{%
%  The setting could be generalized even more,
%  for example to vector-valued data.%
%  We restrict ourselves to the scalar-valued case,
%  keeping hierarchization as our main application in mind.%
%}
\begin{equation}
  \vlinin = (\linin{\ßl,\ßi})_{(\ßl,\ßi) \in \liset} \in \real^\ngp,\quad
  \vlinout = (\linout{\ßl,\ßi})_{(\ßl,\ßi) \in \liset} \in \real^\ngp,
\end{equation}
which give one scalar per grid point $\vgp{\ßl,\ßi} \in \sgset$.
For the case of hierarchization,
$\linop$ is the inverse of the \term{interpolation matrix} $A \in \real^{\ngp \times \ngp}$:
\begin{subequations}
  \label{eq:hierarchizationSLE}
  \begin{equation}
    \linop = A^{-1},\quad
    A = (\basis{\ßl',\ßi'}(\vgp{\ßl,\ßi}))_{(\ßl,\ßi),(\ßl',\ßi') \in \liset},\quad
    \vlinin = (\fcnval{\ßl,\ßi})_{(\ßl,\ßi) \in \liset},\quad
    \vlinout = (\surplus{\ßl',\ßi'})_{(\ßl',\ßi') \in \liset}.
  \end{equation}
  This means that we can determine the $\surplus{\ßl',\ßi'}$ by solving
  the $\ngp \times \ngp$ system of linear equations
  \begin{equation}
    \vlinout = \linop[\vlinin]
    \quad\iff\quad
    A \cdot (\surplus{\ßl',\ßi'})_{(\ßl',\ßi') \in \liset}
    = (\fcnval{\ßl,\ßi})_{(\ßl,\ßi) \in \liset}.
  \end{equation}
\end{subequations}

\paragraph{Complexity of B-spline hierarchization}

As noted in \cite{Valentin18Fundamental},
hierarchization on sparse grids with hierarchical B-splines
$\basis{\ßl,\ßi}^\ßp$ of degree $\ßp$
as basis functions $\basis{\ßl,\ßi}$ is a tedious task.
The corresponding linear system \eqref{eq:hierarchizationSLE} is in general
non-symmetric
(i.e., $\basis{\ßl',\ßi'}^\ßp(\vgp{\ßl,\ßi}) \not=
\basis{\ßl,\ßi}^\ßp(\vgp{\ßl',\ßi'})$) and densely populated.
This is because the matrix entry in the $(\ßl,\ßi)$-th row and
$(\ßl',\ßi')$-th column vanishes if and only if
\begin{equation}
  \vgp{\ßl,\ßi} \notin \interiorsupp \basis{\ßl',\ßi'}^\ßp
  \iff
  \ex{t = 1, \dotsc, d}{
    \gp{l_t,i_t} \notin
    \opintscaled{
      \gp{l'_t,i'_t} - \tfrac{p_t+1}{2} \ms{l'_t},\,
      \gp{l'_t,i'_t} + \tfrac{p_t+1}{2} \ms{l'_t}
    }
  },
\end{equation}
where $\interiorsupp$ is the interior of the support
\cite{Valentin18Fundamental}.
For coarse levels $\ßl'$, the mesh size $\ms{l'_t}$ is large in
every dimension $t$, which implies that $\interiorsupp \basis{\ßl',\ßi'}^\ßp$
contains most of the grid points.
In contrast to the hat function case ($\ßp = \ß1$),
the value of $\surplus{\ßl',\ßi'}$ depends not only on
$\fcnval{\ßl,\ßi}$ and the data of its $3^d - 1$ neighboring grid points
on the boundary of $\supp \basis{\ßl',\ßi'}^\ß1$,
but potentially on the data of the whole grid.

This prohibits the use of the unidirectional principle,
which will be discussed in \cref{sec:42fullGrids}
and which takes $\landauO{Nd}$ computing time%
\footnote{in the number of univariate basis evaluations},
on sparse grids with hierarchical B-splines.
Consequently, we have to solve the linear system
\eqref{eq:hierarchizationSLE}, which is significantly more time-consuming,
as it takes between $\Omega(\ngp^2 d)$ and $\landauO{\ngp^3 d}$ time
(via Gaussian elimination).
In addition, if we use an explicit solver for the linear system,
we additionally have to store an $\ngp \times \ngp$ matrix in memory.
However, a grid of size $\ngp = \num{50000}$ would already exceed the memory
of a \SI{16}{\gibi\byte} workstation.

\paragraph{Notation}

We do not need the hierarchical level-index information $(\ßl, \ßi)$ in
$\sgset$, $\liset$, $\vlinin$, and $\vlinout$
for most of the considerations in this chapter.
Therefore, we assume that in each dimension $t$, the level-index pairs
$(l_t, i_t)$ ($l_t \in \natz$, $i_t \in \hiset{l_t}$)
are continuously enumerated by a single index $k_t = k_t(l_t, i_t) \in \natz$.
We identify $(\ßl, \ßi)$ with a single index $\ßk$,
whose $t$-th component is given by $k_t(l_t, i_t)$.
Consequently,
we can regard $\liset$ as a subset $\{\ßk \mid \ßx_\ßk \in \sgset\}$
of $\nat^d$.
We will switch between the notations whenever appropriate.
All statements that are formulated in the $\ßk$ notation are
valid for both the nodal and the hierarchical basis
(i.e., for all tensor product bases).

In the following, $k_t$ denotes the $t$-th component of a $d$-vector $\ßk$
as usual.
With $\ßk_{-t}$, we denote the $(d-1)$-vector that is obtained from $\ßk$
by omitting the $k$-th component,
i.e., $\ßk_{-t} := (k_1, \dotsc, k_{t-1}, k_{t+1}, \dotsc, k_d)$.
For a $j$-tuple $T = (t_1, \dotsc, t_j) \in \{1, \dotsc, d\}^j$,
we define $\ßk_T$ to be the $j$-vector $(k_{t_1}, \dotsc, k_{t_j})$
that only contains the entries of the dimensions listed in $T$.
Accordingly, $\ßk_{-T}$ is defined as the $(d-j)$-vector
that contains the entries of the remaining dimensions
(sorted by the dimension $t$).
We define $a:b := (a, a + 1, \dotsc, b)$ as an indexing shortcut ($a \le b$).

\section{Hierarchization on Full Grids (Unidirectional Principle)}
\label{sec:42fullGrids}

If $\Omega^\sparse$ is a full grid $\fgset{\ßl}$
(see \cref{sec:21nodalSpaces}),
the well-known \emph{\up}
can be used to apply $L$ to input data $\ßu$.
Its idea is to apply the corresponding one-dimensional operators on the
one-dimensional subgrids (the \emph{poles}) of $\Omega^\sparse$,
which is repeated for all dimensions.
In this section, we first formulate the \up for
general linear operators $L$ and then prove its correctness for
the case $L = A^{-1}$ of hierarchization.
The correctness for the general case of arbitrary tensor product operators
will follow from the considerations in \cref{sec:444weaklyFundamentalSplines}.

\paragraph{Unidirectional principle and its correctness}

The \up is given in \cref{alg:unidirectionalPrinciple}.
The algorithm is given a permutation $(t_1, \dotsc, t_d)$ of $(1, \dotsc, d)$
that specifies the order of dimensions in which the \up should be applied.
We denote with $L_{t_j}$ the one-dimensional version of $L$
applied in dimension $t_j$ ($j = 1, \dotsc, d$).
With the following invariant, we can prove the correctness of the \up.

\begin{algorithm}
  \begin{algorithmic}[1]
    \Function{$\ßy =$ unidirectionalPrinciple}{%
      $K$, $(t_1, \dotsc, t_d)$, $\ßu$%
    }
      \State{$\ßy^{(0)} \gets \ßu$}
      \For{$j = 1, \dotsc, d$}
        \State{%
          Define equivalence relation $\sim$ on $K$
          as $\ßk \sim \ßk' \iff \ßk_{-t_j} = \ßk'_{-t_j}$%
        }
        \For{$K_\mathrm{pole} \in K/{\sim}$}
          \State{%
            $(y_\ßk^{(j)})_{\ßk \in K_\mathrm{pole}} \gets
            L_{t_j}\Big[(y_\ßk^{(j-1)})_{\ßk \in K_\mathrm{pole}}\Big]$%
          }
          \Comment{apply 1D operator on pole}%
          \label{line:algUnidirectionalPrinciple1}
        \EndFor{}
      \EndFor{}
      \State{$\ßy \gets \ßy^{(d)}$}
    \EndFunction{}
  \end{algorithmic}
  \caption{%
    Application of a tensor product operator $L$ with
    the unidirectional principle.
    Inputs are the set $K$ of grid indices,
    the permutation $(t_1, \dotsc, t_d)$ specifying the order in which
    the one-dimensional operators $L_{t_j}$ should be applied, and
    the vector $\ßu = (u_\ßk)_{\ßk \in K}$ of input data.
    The output is the vector $\ßy = (y_\ßk)_{\ßk \in K}$
    of output data.%
  }%
  \label{alg:unidirectionalPrinciple}%
\end{algorithm}

\begin{proposition}[%
  invariant of \cref{alg:unidirectionalPrinciple} for hierarchization%
]
  \label{prop:invariantUnidirectionalPrinciple}
  Let $T$ be the hierarchization operator on a full grid,
  i.e.,
  $T = A^{-1}$,
  $\ßu = (f_\ßk)_{\ßk \in K}$,
  $\ßy = (\alpha_\ßk)_{\ßk \in K}$,
  $T_{t_j}$ is the 1D interpolation operator $(A_{t_j})^{-1}$, and
  $K = \{\ß0, \dotsc, \ß2^\ßl\}$
  corresponds to a full grid $\fgset{\ßl}$ of level $\ßl$.
  After iteration $j$ of \cref{alg:unidirectionalPrinciple}
  ($j = 1, \dotsc, d$), it holds for $T := (t_1, \dotsc, t_j)$
  \begin{equation}
    \sum_{\ßk_T=\ß0}^{\ß2^{\ßl_T}}
    \alpha^{(j)}_{(\ßk_T,\ßk'_{-T})} \basis{\ßk_T}(\vgp{\ßk'_T})
    = f_{\ßk'},\quad
    \ßk' = \ß0, \dotsc, \ß2^\ßl,
  \end{equation}
  where $(\ßk_T,\ßk'_{-T})$ is defined to be the index $\ßk''$
  with $\ßk''_T := \ßk_T$ and $\ßk''_{-T} := \ßk'_{-T}$.
\end{proposition}

\begin{proof}
  We prove the assertion by induction over $j = 1, \dotsc, d$.
  We set $T' := (t_1, \dotsc, t_{j-1})$,
  $T := (t_1, \dotsc, t_{j-1}, t_j)$,
  and we exploit the tensor product structure of the basis
  to write the \lhs of the assertion for $j$
  and arbitrary $\ßk' = \ß0, \dotsc, \ß2^\ßl$ as
  \begin{align}
    \sum_{\ßk_T=\ß0}^{\ß2^{\ßl_T}}
    \alpha^{(j)}_{(\ßk_T,\ßk'_{-T})} \basis{\ßk_T}(\vgp{\ßk'_T})
    &= \sum_{\ßk_{T'}=\ß0}^{\ß2^{\ßl_{T'}}}
    \basis{\ßk_{T'}}(\vgp{\ßk'_{T'}}) \cdot
    \sum_{k_{t_j}=0}^{2^{l_{t_j}}}
      \alpha^{(j)}_{(\ßk_T,\ßk'_{-T})} \basis{k_{t_j}}(\gp{k'_{t_j}}).\\
    \intertext{%
      If we choose the equivalence class
      $K_\mathrm{pole} := \clint{(\ßk_T,\ßk'_{-T})}_\sim$
      ($\ßk_T$ arbitrary, $\sim$ as defined in iteration $j$),
      then the sum over $k_{t_j}$ equals
      $\sum_{\ßk \in K_\mathrm{pole}}
      \alpha^{(j)}_\ßk \basis{k_{t_j}}(\gp{k'_{t_j}})
      %= ((L_{t_j})^{-1}
      %\Big[(\alpha_\ßk^{(j)})_{\ßk \in K_\mathrm{pole}}\Big])_{k'_{t_j}}
      %= ((\alpha_\ßk^{(j-1)})_{\ßk \in K_\mathrm{pole}})_{k'_{t_j}}
      = \alpha^{(j-1)}_{(\ßk_{T'},\ßk'_{-T'})}$
      by the 1D interpolation operator $T_{t_j}$
      (\cref{line:algUnidirectionalPrinciple1} of
      \cref{alg:unidirectionalPrinciple}).
      We can conclude that the \lhs equals%
    }
    &= \sum_{\ßk_{T'}=\ß0}^{\ß2^{\ßl_{T'}}}
    \alpha^{(j-1)}_{(\ßk_{T'},\ßk'_{-T'})}
    \basis{\ßk_{T'}}(\vgp{\ßk'_{T'}}),
  \end{align}
  which, by the induction hypothesis, equals $f_{\ßk'}$ as desired
  (if $j > 1$).
  The same reasoning for $(\ast\ast)$ can be used
  to establish the base case for $j = 1$.
\end{proof}

\begin{shortcorollary}
  \Cref{alg:unidirectionalPrinciple}
  is correct for hierarchization on full grids.
\end{shortcorollary}

\begin{proof}
  We apply \cref{prop:invariantUnidirectionalPrinciple} to obtain
  $\sum_{\ßk=\ß0}^{\ß2^\ßl}
  \alpha^{(j)}_\ßk \basis{\ßk}(\vgp{\ßk'})
  = f_{\ßk'}$
  for all $\ßk' = \ß0, \dotsc, \ß2^\ßl$, i.e.,
  the $\alpha^{(j)}_\ßk$ are the correct interpolation coefficients
  according to \eqref{eq:hierarchizationProblem}.
\end{proof}

\paragraph{Complexity}

We compare the complexity of the \up for hierarchization compared
to directly solving the system \eqref{eq:hierarchizationSLE} of
linear equations.
If we assume that $L$ and $L_{t_j}$ apply Gaussian elimination to
solve the multivariate and univariate systems, respectively,
then directly solving \eqref{eq:hierarchizationSLE} takes
$\calO(N^3 d)$ time (in 1D basis evaluations) and $\calO(N^2)$ memory.
In contrast, the unidirectional principle only

\blindtext{}

\section{Hierarchization on Dimensionally Adaptive Sparse Grids}
\label{sec:43dimAdaptive}

\blindtext{}



\subsection{Combination Technique and Its Combinatorial Proof}
\label{sec:431combiTechniqueProof}

\todo{mention Delvos/Schempp}

\todo{mention Wasilkowski}

\blindtext{}

\paragraph{Formal description and outline of a combinatorial proof}

In the following, we give a formal description of the
sparse grid combination technique and we outline a combinatorial proof
for its correctness.
We discuss a high-level explanation of the proofs in this section
The proofs themselves can be found in \cref{sec:proofCombiTechnique}
as most of them are rather technical.
For simplicity,
we formulate the combination technique and its proof for regular
sparse grids (see \cref{sec:231regularSG}).
However, the main ideas of the proof chain are also applicable
for dimensionally adaptive sparse grids
(see \cref{sec:232dimensionallyAdaptiveSG}).

\begin{theorem}[sparse grid combination technique]
  \label{thm:combiTechnique}
  Let $\liset := \{(\*l, \*i) \mid
  \normone{\*l} \le n,\; \*i \in \hiset{\*l}\}$
  correspond to the regular sparse grid
  $\regsgset{n}{d}{}$ and let $(\fcnval{\*l,\*i})_{(\*l,\*i) \in \liset}$
  be given function values on $\regsgset{n}{d}{}$.
  If we define
  \begin{itemize}
    \item
    the combined sparse grid interpolant $\regsgintp{n}{d}{\ct}$ via
    \eqref{eq:combiTechnique}, i.e.,
    \begin{equation}
      \regsgintp{n}{d}{\ct}
      = \sum_{q=0}^{d-1} (-1)^q \binom{d-1}{q} \sum_{\normone{\*l'} = n-q}
      \fgintp{\*l'},
    \end{equation}
    where $\fgintp{\*l'} \in \ns{\*l'}$ is the full grid interpolant
    of $\objfun$ with level $\*l'$, and
    
    \item
    the hierarchical sparse grid interpolant $\regsgintp{n}{d}{}$
    via \eqref{eq:hierarchizationProblem} and
    \eqref{eq:hierarchizationInterpolant}
  \end{itemize}
  and assume that the hierarchical splitting equation
  \eqref{eq:hierSplittingMV} holds,
  then the combined and the hierarchical sparse grid interpolants coincide:
  \begin{equation}
    \regsgintp{n}{d}{\ct}
    = \regsgintp{n}{d}{}.
  \end{equation}
\end{theorem}

\begin{proof}[Proof (sketch)]
  Let $\gp{\*l,\*i} \in \regsgset{n}{d}{}$ be an arbitrary
  point of the regular sparse grid.
  First, we split the inner sum of $\regsgintp{n}{d}{\ct}(\gp{\*l,\*i})$
  into levels $\*l'$ whose full grid sets $\fgset{\*l'}$
  contain $\gp{\*l,\*i}$ and levels whose full grid sets
  do not contain $\gp{\*l,\*i}$:
  \begin{subequations}
    \begin{align}
      \regsgintp{n}{d}{\ct}(\gp{\*l,\*i})
      &= \sum_{q=0}^{d-1} (-1)^q \binom{d-1}{q} \cdot \left(
      \sum_{\substack{\normone{\*l'} = n - q\\\fgset{\*l'} \ni \gp{\*l,\*i}}}
      \fgintp{\*l'}(\gp{\*l,\*i}) +
      \sum_{\substack{\normone{\*l'} = n - q\\\fgset{\*l'} \notni \gp{\*l,\*i}}}
      \fgintp{\*l'}(\gp{\*l,\*i})\right).\\
      \intertext{%
        The summands $\fgintp{\*l'}(\gp{\*l,\*i})$ of the first inner sum
        equal $\fcnval{\*l,\*i}$ due to full grid interpolation
        property \eqref{eq:interpFullGridMV}.
        Therefore, the first inner sum is equal to the product
        $\fcnval{\*l,\*i}$ with the number of summands:%
      }
      \label{eq:combiTechniqueSplitSum}
      \begin{split}
        &= \fcnval{\*l,\*i} \cdot \sum_{q=0}^{d-1} (-1)^q \binom{d-1}{q} \cdot
        \setsize{
          \{\*l' \mid \normone{\*l'} = n - q,\; \fgset{\*l'} \ni \gp{\*l,\*i}\}
        } +{}\\
        &\qquad\sum_{q=0}^{d-1} (-1)^q \binom{d-1}{q} \cdot
        \sum_{\substack{\normone{\*l'} = n - q\\\fgset{\*l'} \notni \gp{\*l,\*i}}}
        \fgintp{\*l'}(\gp{\*l,\*i}).
      \end{split}
    \end{align}
  \end{subequations}
  We will prove that the sum of the first $\sum$ sign
  equals one (see \cref{prop:combiTechniqueOne})
  and the sum of the second $\sum$ sign
  equals zero (see \cref{prop:combiTechniqueZero}).
  Consequently, we infer
  \begin{equation}
    \regsgintp{n}{d}{\ct}(\gp{\*l,\*i})
    = \fcnval{\*l,\*i},
  \end{equation}
  i.e., $\regsgintp{n}{d}{\ct}$ interpolates $f$ at $\regsgset{n}{d}{}$.
  Note that $\regsgintp{n}{d}{\ct}$ is contained in $\regsgspace{n}{d}{}$,
  if the hierarchical splitting equation \eqref{eq:hierSplittingMV} holds,
  as
  \begin{equation}
    \fgintp{\*l'} \in
    \ns{\*l'}
    = \bigoplus_{\*l''=\*0}^{\*l'} \hs{\*l''}
    \subset \regsgspace{n}{d}{},\quad
    \normone{\*l'} \le n,
  \end{equation}
  due to $\normone{\*l''} \le \normone{\*l'} \le n$
  for $\*l'' \le \*l'$.%
  \footnote{%
    This argumentation can be straightforwardly adapted
    for general dimensionally adaptive sparse grids
    with downward closed level sets as mentioned in
    \cref{sec:232dimensionallyAdaptiveSG}.%
  }
  As both $\regsgintp{n}{d}{\ct}$ and $\regsgintp{n}{d}{}$
  are contained in $\regsgspace{n}{d}{}$ and
  interpolate $f$ on $\regsgset{n}{d}{}$, they must coincide
  due to the uniqueness of sparse grid interpolation
  (linear independence of the hierarchical basis functions).
\end{proof}

\paragraph{Inclusion-exclusion principle}

It remains to prove that the first sum in \eqref{eq:combiTechniqueSplitSum}
is indeed one and that the second sum vanishes.
The first statement is a direct consequence of the so-called
\term{inclusion-exclusion principle}.
In its simplest form, the idea of the principle is that the cardinality
of the union of two finite subsets $A, B$ of some set is given by
\begin{equation}
  \setsize{A \cup B}
  = \setsize{A} + \setsize{B} - \setsize{A \cap B}.
\end{equation}
This equation states that we have to first count (include)
the elements in $A$ and then in $B$,
but as we have counted the elements of $A \cap B$ twice,
we have to subtract (exclude) its cardinality again.

The setting is similar for the combination technique.
If we add all grids in \cref{fig:combinationTechnique}
on the green diagonal, then every point whose index is not odd
will be counted multiple times.
By subtracting the number of occurrences of the points on the
red diagonal,
the result of the ``weighted counting'' is exactly one.
The following proposition, whose proof is of purely combinatorial nature,
generalizes this argument to higher dimensions:

\begin{restatable}[inclusion-exclusion principle]{%
  proposition%
}{%
  propCombiTechniqueOne%
}
  \label{prop:combiTechniqueOne}
  For every $\gp{\*l,\*i} \in \regsgset{n}{d}{}$, we have
  \begin{equation}
    \label{eq:combiTechniqueOne}
    \sum_{q=0}^{d-1} (-1)^q \binom{d-1}{q} \cdot
    \setsize{
      \{\*l' \mid \normone{\*l'} = n - q,\; \fgset{\*l'} \ni \gp{\*l,\*i}\}
    }
    = 1.
  \end{equation}
\end{restatable}

\begin{proof}
  See \cref{sec:proofCombiTechnique}.
\end{proof}

\paragraph{Canceling out function values}

The second statement about the vanishing second sum in
\eqref{eq:combiTechniqueSplitSum} is harder to prove.
It states that the contributions $\fgintp{\*l'}$ of levels $\*l'$
that do not contain a specific point $\gp{\*l,\*i}$ cancel out,
which may seem quite surprising.
The key observation is as follows:
The values of $\fgintp{\*l'}, \fgintp{\*l''}$ for two levels
$\*l', \*l''$ are the same at $\gp{\*l,\*i}$,
if the non-equal entries $l'_t, l''_t$ of the levels are
greater or equal to $l_t$.

For a higher-level explanation,
note that the statement $l'_t \ge l_t$ is equivalent to
$\gp{l_t,i_t} \in \fgset{l'_t}$.
%i.e., the projection of the grid point $\gp{\*l,\*i}$ onto the
%$t$-th dimension is contained in the projection of $\fgset{\*l'}$.
Both $\fgintp{\*l'}, \fgintp{\*l''}$ interpolate at
$\gp{\*l,\*i}$ when projected onto the $t$-th dimension,
so their contribution to $\fgintp{\*l'}(\gp{\*l,\*i})$ and
$\fgintp{\*l''}(\gp{\*l,\*i})$ must be the same.
Although there may be dimensions $t$ for which
$\gp{l_t,i_t} \notin \fgset{l'_t}$,
these dimensions do not matter if $l'_t = l''_t$,
as the univariate restrictions of $\fgintp{\*l'}, \fgintp{\*l''}$
interpolate the same data and they are evaluated at the same point
$\gp{l_t,i_t}$.

Let us formalize these considerations by defining an
equivalence relation on the set of levels, such that the values of
$\fgintp{\*l'}$ at $\gp{\*l,\*i}$ are constant
on the equivalence classes.

\begin{definition}[relation ``$\sim$'']
  \label{def:combiTechniqueEquivalenceRelation}
  Let $\gp{\*l,\*i} \in \regsgset{n}{d}{}$ be fixed and
  \begin{equation}
    \label{eq:combiTechniqueSpecialLevelSet}
    L
    := \{\*l' \mid \ex{q=0,\dotsc,d-1}{
      \normone{\*l'} = n - q,\; \fgset{\*l'} \notni \gp{\*l,\*i}
    }\}
  \end{equation}
  be the set of levels that do not contain $\gp{\*l,\*i}$.
  We define a relation ``$\sim$'' on $L$ as follows:
  For $\*l', \*l'' \in L$, we set $\*l' \sim \*l''$ if and only if
  \begin{equation}
    \fa{t \notin T_{\*l',\*l''}}{\min\{l'_t, l''_t\} \ge l_t},\quad
    T_{\*l',\*l''}
    := \{t \mid l'_t = l''_t\}.
  \end{equation}
\end{definition}

\begin{restatable}{%
  shortlemma%
}{%
  lemmaCombiTechniqueIdenticalValues%
}
  \label{lemma:combiTechniqueIdenticalValues}
  Let $\*l', \*l'' \in L$ with $\*l' \sim \*l''$.
  Then, $\fgintp{\*l'}(\gp{\*l,\*i})
  = \fgintp{\*l''}(\gp{\*l,\*i})$.
\end{restatable}

\begin{proof}
  See \cref{sec:proofCombiTechnique}.
\end{proof}

By exploiting on the tensor product structure of the basis functions,
the proof shows an even stronger version, which is shown
in \cref{fig:combiTechniqueProof}:
The components $\fgintp{\*l'}$ and $\fgintp{\*l''}$ are equal
on the $m$-dimensional subspace through $\gp{\*l,\*i}$
parallel to the $m$ coordinates in $T_{\*l',\*l''}$.
The lemma allows to group summands in the second sum of
\eqref{eq:combiTechniqueSplitSum} by function values.
Hence, it remains to count the number of levels for
each equivalence class of ``$\sim$''.
Therefore, we need a characterization of the structure
of the equivalence classes:

\begin{SCfigure}
  \includegraphics{combiTechniqueProof_1}%
  \caption[%
    Canceling out function values in the proof of the combination technique%
  ]{%
    Nodal subspaces $\ns{\*l}$ contributing to the combination
    technique solution for the two-dimensional regular sparse grid
    $\regsgspace{n}{d}{}$ of level $n = 3$.
    After picking a point $\gp{\*l,\*i} \in \regsgspace{n}{d}{}$
    (\emph{cross}, here $\*l = (2, 1)$, $\*i = (1, 1)$),
    the set $L$ of levels whose grids do not contain $\gp{\*l,\*i}$
    \emph{(colored subspaces)}
    decompose into three disjoint equivalence classes
    \emph{(colors)} given by the relation ``$\sim$''.
    In every equivalence class $L_0 \in \eqclasses{L}{\sim}$,
    the interpolants $\fgintp{\*l'}$ ($\*l' \in L_0$)
    equal on an affine subspace
    \emph{(dark lines}), which contains $\gp{\*l,\*i}$.
    Due to the combination coefficients,
    the contribution to the combined solution
    vanishes per equivalence class.%
  }%
  \label{fig:combiTechniqueProof}%
\end{SCfigure}

\begin{restatable}[characterization of the equivalence classes of ``$\sim$'']{%
  lemma%
}{%
  lemmaCombiTechniqueCharacterization%
}
  \label{lemma:combiTechniqueCharacterization}
  Let $L_0 \in \eqclasses{L}{\sim}$ be an equivalence class of ``$\sim$''.
  If we define
  \begin{equation}
    T_{L_0}
    := \{t \mid \exfa{l^\ast_t}{\*l' \in L_0}{l'_t = l^\ast_t}\}
  \end{equation}
  as the set of dimensions in which all levels in $L_0$
  have the same entry, then
  \begin{equation}
    L_0
    = \{\*l' \in L \mid
    \fa{t \in T_{L_0}}{l'_t = l^\ast_t},\;
    \fa{t \notin T_{L_0}}{l'_t \ge l_t}\}.
  \end{equation}
\end{restatable}

\begin{proof}
  See \cref{sec:proofCombiTechnique}.
\end{proof}

The lemma states that every equivalence class $L_0$ is exactly the
set of the levels whose entries are equal in some dimensions
(which are described with $T_{L_0}$)
and whose entries are greater or equal than $l_t$ in all the other dimensions.
While this statement may seem intuitively correct,
the proof is rather technical.
Finally, we are now able to show that the second sum in
\eqref{eq:combiTechniqueSplitSum} vanishes:

\begin{restatable}[function value cancellation]{%
  proposition%
}{%
  propCombiTechniqueZero%
}
  \label{prop:combiTechniqueZero}
  For every $\gp{\*l,\*i} \in \regsgset{n}{d}{}$, we have
  \begin{equation}
    \sum_{q=0}^{d-1} (-1)^q \binom{d-1}{q} \cdot
    \sum_{\substack{\normone{\*l'} = n - q\\\fgset{\*l'} \notni \gp{\*l,\*i}}}
    \fgintp{\*l'}(\gp{\*l,\*i})
    = 0.
  \end{equation}
\end{restatable}

\begin{proof}
  See \cref{sec:proofCombiTechnique}.
\end{proof}

The proof essentially first counts the number of possible levels in
an equivalence class and then applies known combinatorial identities
to prove that the sum must vanish.
This proves \thmref{thm:combiTechnique}.



\subsection{Hierarchization with the Combination Technique}
\label{sec:432hierarchizationCombiTechnique}

As shown in \cref{alg:combiTechnique},
it is straightforward to hierarchize function values
$\fcnval{\*l,\*i}$ on dimensionally adaptive sparse grids
with the combination technique.
In \cref{line:algCombiTechnique1},
the hierarchical surpluses corresponding to the full grid
interpolant $\fgintp{\*l'} \in \ns{\*l'}$ have to be computed
(see \eqref{eq:interpFullGridMV}).
As shown in \cref{sec:42fullGrids}, we can easily calculate these
surpluses with the unidirectional principle in
\cref{alg:unidirectionalPrinciple}.
The surpluses are then combined with the same combination formula
as for the combination technique.
Note that it is imperative to employ the hierarchical basis functions
$\basis{\*l,\*i}$ with $\*l = \*0, \dotsc, \*l'$ and $\*i \in I_{\*l}$
and not the nodal basis
(which would be $\basis{\*l',\*i'}$ with $\*i' = \*0, \dotsc, \*2^{\*l'}$).

\begin{algorithm}
  \begin{algorithmic}[1]
    \Function{$\vlinout =$ combinationTechnique}{%
      $n$, $d$, $\vlinin$%
    }
      \For{$q = 0, \dotsc, d - 1$}
        \For{$\*l' \in \natz^d$ with $\normone{\*l'} = n - q$}
          \State{%
            Let $(\surplus{\*l,\*i}^{(\*l')})_{
              \*l = \*0, \dotsc, \*l'\!,\, \*i \in \hiset{\*l}
            }$ be such that
            $\sum_{\*l=\*0}^{\*l'} \sum_{\*i \in \hiset{\*l}}
            \surplus{\*l,\*i}^{(\*l')} \basis{\*l,\*i} \equiv
            \fgintp{\*l'}$%
          }
          \label{line:algCombiTechnique1}
          \State{%
            $\surplus{\*l,\*i}^{(\*l')} \gets 0$
            for all $(\*l,\*i) \in \liset$
            with $\lnot(\*l \le \*l')$%
          }
          \Comment{extend surpluses}%
          \label{line:algCombiTechnique2}
        \EndFor{}
      \EndFor{}
      \State{%
        $\linout{\*l,\*i}
        = \sum_{q=0}^{d-1} (-1)^q \binom{d-1}{q}
        \sum_{\normone{\*l'} = n-q} \surplus{\*l,\*i}^{(\*l')}$
        for all $(\*l, \*i) \in \liset$%
      }
      \Comment{combine surpluses}%
    \EndFunction{}
  \end{algorithmic}
  \caption[%
    Hierarchization with the combination technique%
  ]{%
    Application of the hierarchization operator $\linop = \intpmat^{-1}$
    with the combination technique.
    For simplicity,
    the algorithm is described for regular sparse grids,
    but it can be generalized to arbitrary dimensionally adaptive sparse grids.
    Inputs are the level $n$ and dimensionality $d$ of the
    regular sparse grid and
    the vector $\vlinin = (\linin{\*l,\*i})_{(\*l,\*i) \in \liset}$
    of input data (function values $\fcnval{\*l,\*i}$ at the grid points),
    where $\liset$ is the set of all feasible level-index pairs $(\*l,\*i)$,
    i.e., $\normone{\*l} \le n$, $\*i \in \hiset{\*l}$.
    The output is the vector
    $\vlinout = (\linout{\*l,\*i})_{(\*l,\*i) \in \liset}$
    of output data (hierarchical surpluses $\surplus{\*l,\*i}$).%
  }%
  \label{alg:combiTechnique}%
\end{algorithm}

Of course, the proof of the correctness of \cref{alg:combiTechnique}
relies on the correctness of the combination technique
(see \cref{thm:combiTechnique}).
When determining the combination coefficients correctly
\cite{Wasilkowski95Explicit}, the algorithm can even be applied to
all dimensionally adaptive sparse grids.
The proof of the following proposition can be generalized accordingly.

\begin{proposition}[correctness of \cref{alg:combiTechnique}]
  \label{prop:correctnessAlgCombiTechnique}
  \Cref{alg:combiTechnique}
  is correct for hierarchization on regular sparse grids.
\end{proposition}

\begin{proof}
  According to \cref{line:algCombiTechnique1} of \cref{alg:combiTechnique},
  the full grid interpolants $\fgintp{\*l'}$ can be written as
  \begin{equation}
    \fgintp{\*l'}
    = \sum_{\normone{\*l} \le n} \sum_{\*i \in \hiset{\*l}}
    \surplus{\*l,\*i}^{(\*l')} \basis{\*l,\*i}
  \end{equation}
  where the surpluses have been extended with zero by
  \cref{line:algCombiTechnique2}.
  \Thmref{thm:combiTechnique} now allows to write the hierarchical
  interpolant $\regsgintp{n}{d}{}$ in terms of the full grid components:
  \begin{subequations}
    \begin{align}
      \
      \regsgintp{n}{d}{}
      = \regsgintp{n}{d}{\ct}
      &= \sum_{q=0}^{d-1} (-1)^q \binom{d-1}{q} \sum_{\normone{\*l'} = n-q}
      \fgintp{\*l'}\\
      &= \sum_{q=0}^{d-1} (-1)^q \binom{d-1}{q} \sum_{\normone{\*l'} = n-q}
      \sum_{\normone{\*l} \le n} \sum_{\*i \in \hiset{\*l}}
      \surplus{\*l,\*i}^{(\*l')} \basis{\*l,\*i}.\\
      \intertext{%
        By rearranging the sums, we obtain%
      }
      \label{eq:propCorrectnessAlgCombiTechnique1}
      &= \sum_{\normone{\*l} \le n} \sum_{\*i \in \hiset{\*l}}
      \underbrace{
        \left(\sum_{q=0}^{d-1} (-1)^q \binom{d-1}{q} \sum_{\normone{\*l'} = n-q}
        \surplus{\*l,\*i}^{(\*l')}\right)
      }_{= \linout{\*l,\*i}}
      \basis{\*l,\*i},
    \end{align}
  \end{subequations}
  where $\linout{\*l,\*i}$ is the $(\*l,\*i)$-th entry of the output vector
  of \cref{alg:combiTechnique}.
  Note that the hierarchical interpolant $\regsgintp{n}{d}{}$
  can be written as
  $\regsgintp{n}{d}{} = \sum_{\normone{\*l} \le n} \sum_{\*i \in \hiset{\*l}}
  \surplus{\*l,\*i} \basis{\*l,\*i}$
  (see \eqref{eq:regularSGInterpolant}),
  where the surpluses $\surplus{\*l,\*i}$ are unique due to the
  linear independence of the hierarchical basis.
  As \eqref{eq:propCorrectnessAlgCombiTechnique1}
  equals $\regsgintp{n}{d}{}$ and has the same form,
  the coefficients $\linout{\*l,\*i}$
  (which are the output of \cref{alg:combiTechnique})
  must coincide with the surpluses $\surplus{\*l,\*i}$.
\end{proof}



\subsection{Hierarchization with Residual Interpolation}
\label{sec:433residualInterpolation}

Another possible method to hierarchize function values on
dimensionally adaptive sparse grids is the
\term{method of residual interpolation}.
The advantage over the combination technique is that
it only needs to operate on the so-called \term{active nodal spaces}.
In contrast, the combination technique needs to perform computations
on additional non-active nodal subspaces
(for the regular sparse grid case:
summands with $q \ge 1$ in \eqref{eq:combiTechnique}).

\Cref{alg:residualInterpolation} describes the procedure of the method,
given the levels $L$ contained in the sparse grid
(see \eqref{eq:dimensionallyAdaptiveSG})
and function values $\vlinin$ corresponding to the grid points.
The list $\*l^{(1)}, \dotsc, \*l^{(m)}$ of active nodal spaces
in \cref{line:algResidualInterpolation1} is determined by the condition
\begin{equation}
  \bigcup_{j=1}^m \{\*l \in \natz^d \mid \*l \le \*l^{(j)}\} = L,\quad
  \fa{j_1 \not= j_2}{\lnot(\*l^{(j_1)} \le \*l^{(j_2)})}.
\end{equation}
This means that the corresponding sparse grid $\sgset$
is the (non-disjoint) union of the full grid sets $\fgset{\*l^{(j)}}$
($j = 1, \dotsc, m$)
and no full grid set is contained is another
(no full grid set can be omitted without
removing points from the union $\sgset$).

\begin{algorithm}
  \begin{algorithmic}[1]
    \Function{$\vlinout =$ residualInterpolation}{%
      $\levelset$, $\vlinin$%
    }
      % renames:
      % * m --> j
      % * f^(m) --> r^(j)
      % * alpha^(m) --> y^(j)
      % * tilde(g)^(m) --> r_l^(j-1)
      % * beta^(m) --> alpha^(j)
      % * tilde(f)^(m) --> f^{s,j}
      \State{%
        $r^{(0)}(\gp{\*l,\*i}) \gets \fcnval{\*l,\*i}$
        for all $(\*l,\*i) \in \liset$%
      }
      \State{%
        Get list $\*l^{(1)}, \dotsc, \*l^{(m)}$
        of active nodal spaces (see text)%
        %(such that $\bigcup_{j=1}^m \{\*l \mid \*l \le \*l^{(j)}\} = L$)%
      }
      \label{line:algResidualInterpolation1}
      \State{%
        Sort $\*l^{(1)}, \dotsc, \*l^{(m)}$ by decreasing level sum%
        %($\normone{\*l^{(m_1)}} \ge \normone{\*l^{(m_2)}}$
        %for all $m_1 \le m_2$)%
      }
      \For{$j = 1, \dotsc, m$}
        \State{%
          Let $r_{\*l^{(j)}}^{(j-1)} \in \ns{\*l^{(j)}}$ be the
          interpolant of $r^{(j-1)}$ on $\fgset{\*l^{(j)}}$%
          %(see \eqref{eq:interpFullGridMV})%
        }
        \State{%
          Let $(\surplus{\*l,\*i}^{(j)})_{(\*l,\*i) \in \liset}$ be such that
          $\sum_{\*l=\*0}^{\*l^{(j)}} \sum_{\*i \in \hiset{\*l}}
          \surplus{\*l,\*i}^{(j)} \basis{\*l,\*i}
          \equiv r_{\*l^{(j)}}^{(j-1)}$%
        }
        \Comment{res. interpolation}%
        \label{line:algResidualInterpolation2}
        \State{%
          $r^{(j)}(\gp{\*l,\*i}) \gets
          r^{(j-1)}(\gp{\*l,\*i}) - r_{\*l^{(j)}}^{(j-1)}(\gp{\*l,\*i})$
          for all $(\*l,\*i) \in \liset$%
        }
        \Comment{new residuals}%
      \EndFor{}
      \State{%
        $\vlinout \gets \sum_{j=1}^{m} \vsurplus^{(j)}$
        (where $\surplus{\*l,\*i}^{(j)} = 0$
        $(\*l,\*i) \in \liset$
        if $\lnot(\*l \le \*l^{(j)})$)%
      }
      \Comment{combine surpluses}%
    \EndFunction{}
  \end{algorithmic}
  \caption[%
    Hierarchization with residual interpolation%
  ]{%
    Application of the hierarchization operator $\linop = \intpmat^{-1}$
    with residual interpolation
    for dimensionally adaptive sparse grids.
    Inputs are the set $\levelset$ of levels that are part of
    the sparse grid (see \eqref{eq:dimensionallyAdaptiveSG}) and
    the vector $\vlinin = (\linin{\*l,\*i})_{(\*l,\*i) \in \liset}$
    of input data (function values $\fcnval{\*l,\*i}$ at the grid points),
    where $\liset$ is the set of all feasible level-index pairs $(\*l,\*i)$,
    i.e., $\*l \in \levelset$, $\*i \in \hiset{\*l}$.
    The output is the vector
    $\vlinout = (\linout{\*l,\*i})_{(\*l,\*i) \in \liset}$
    of output data (hierarchical surpluses $\surplus{\*l,\*i}$).%
  }%
  \label{alg:residualInterpolation}%
\end{algorithm}

The principle of \Cref{alg:residualInterpolation} is maintaining
a vector $(r^{(j)}(\gp{\*l,\*i}))_{(\*l,\*i) \in \liset}$ of residuals
and subsequently interpolating the residual data on the active nodal spaces.
Again, note that it is necessary to compute the coefficients
$\surplus{\*l,\*i}^{(j)}$ in the hierarchical basis, despite interpolating
on the nodal space.
In the Appendix, we prove the following invariant of the algorithm that
can be used to show its correctness:

\begin{restatable}[invariant of \cref{alg:residualInterpolation}]{%
  proposition%
}{%
  propInvariantResidualInterpolation%
}
  \label{prop:invariantResidualInterpolation}
  % renames:
  % * m --> j
  % * m* --> j, m --> j'
  % * f^(m) --> r^(j)
  % * alpha^(m) --> y^(j)
  % * tilde(g)^(m) --> r_{l^(j)}^(j-1)
  % * beta^(m) --> alpha^(j)
  % * tilde(f)^(m) --> f^{s,j}
  For $j = 1, \dotsc, m$, it holds
  \begin{subequations}
    \label{eq:propInvariantResidualInterpolationStatements}
    \begin{align}
      \label{eq:propInvariantResidualInterpolation1}
      r_{\*l^{(j)}}^{(j-1)}(\gp{\*l,\*i})
      &= 0,
      &&\*l \le \*l^{(j')},\; \*i \in \hiset{\*l},
      &j'
      &= 1, \dotsc, j - 1,\\
      \label{eq:propInvariantResidualInterpolation2}
      r^{(j)}(\gp{\*l,\*i})
      &= 0,
      &&\*l \le \*l^{(j')},\; \*i \in \hiset{\*l},
      &j'
      &= 1, \dotsc, j,\\
      \label{eq:propInvariantResidualInterpolation3}
      r^{(j)}(\gp{\*l,\*i})
      &= \fcnval{\*l,\*i} - f^{\sparse,(j)}(\gp{\*l,\*i}),
      &&\*l \in L,\; \*i \in \hiset{\*l},&&
    \end{align}
  \end{subequations}
  where
  \begin{equation}
    \label{eq:propInvariantResidualInterpolation4}
    f^{\sparse,(j)}
    := \sum_{\*l' \in \levelset} \sum_{\*i' \in \hiset{\*l'}}
    \left(\sum_{j'=1}^{j} \surplus{\*l',\*i'}^{(j')}\right) \basis{\*l',\*i'}.
  \end{equation}
\end{restatable}

\begin{proof}
  See \cref{sec:proofResidualInterpolation}.
\end{proof}

\begin{corollary}[correctness of \cref{alg:residualInterpolation}]
  \Cref{alg:residualInterpolation} is correct for hierarchization
  on dimensionally adaptive sparse grids.
\end{corollary}

\begin{proof}
  Let $\*l \in \levelset$ and $\*i \in \hiset{\*l}$.
  By construction of the active nodal spaces,
  there exists some $j' \in \{1, \dotsc, m\}$ such that $\*l \le \*l^{(j')}$.
  By \cref{prop:invariantResidualInterpolation}, we obtain
  for $j = m$
  \begin{subequations}
    \begin{align}
      \sum_{\*l' \in L} \sum_{\*i' \in \hiset{\*l'}}
      \smash{
        \underbrace{
          \left(\sum_{j''=1}^{m} \surplus{\*l',\*i'}^{(j'')}\right)
        }_{= \linout{\*l',\*i'}}
      }
      \basis{\*l',\*i'}(\gp{\*l,\*i})
      &\quad\;
      \mathclap{\overset{\eqref{eq:propInvariantResidualInterpolation4}}{=}}
      \quad\;
      f^{\sparse,(m)}(\gp{\*l,\*i})
      \overset{\eqref{eq:propInvariantResidualInterpolation3}}{=}
      \fcnval{\*l,\*i} - r^{(m)}(\gp{\*l,\*i})\\
      &\quad\;
      \mathclap{\overset{\eqref{eq:propInvariantResidualInterpolation2}}{=}}
      \quad\;
      \fcnval{\*l,\*i}.
    \end{align}
  \end{subequations}
  As the hierarchical interpolant $\sgintp$
  (see \eqref{eq:hierarchizationInterpolant})
  has the same form
  $\sum_{\*l' \in \levelset} \sum_{\*i' \in \hiset{\*l'}}
  \surplus{\*l',\*i'} \basis{\*l',\*i'}$ as the \lhs
  with unique surpluses $\surplus{\*l',\*i'}$ such that the function values
  are interpolated (see \eqref{eq:hierarchizationProblem}),
  the coefficients $\linout{\*l,\*i}$
  (which are the output of \cref{alg:residualInterpolation})
  must coincide with the surpluses $\surplus{\*l,\*i}$.
\end{proof}

\section{Spatially Adaptive Sparse Grids}

\todo{write}

\subsection{Iterative Application of the Unidirectional Principle}

\todo{write}

\subsection{Lagrange Bases and Fundamental Splines}

\todo{write}

\subsubsection{Hierarchization with Lagrange Bases}

\todo{write}

\subsubsection{Canonical Way to Obtain Lagrange Bases}

\todo{write}

\subsubsection{Fundamental Splines}

\todo{write}

\subsubsection{Modified Fundamental Splines}

\todo{write}

\subsection{Weakly Fundamental Splines}

\todo{write}

\subsubsection{Motivation and Definition}

\todo{write}

\subsubsection{Hermite Hierarchization}

\todo{write}

\subsubsection{Inserting Chain Points}

\todo{write}

\cite{zenger91}

\blindmathpaper


\cleardoublepage
