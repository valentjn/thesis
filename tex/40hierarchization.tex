\setdictum{%
  Who are you? How did you get in my house?%
}{%
  Donald E. Knuth about one-based array indices in algorithms
  (according to \texttt{xkcd}\footnotemark)%
}

\chapter{Algorithms for B-Splines on Sparse Grids}
\footnotetext{\url{https://xkcd.com/163/}}



As we have seen in the last chapter,
hierarchical B-splines constitute an interesting basis on sparse grids,
enabling smooth approximations of objective functions.
When correctly incorporating boundary conditions into the hierarchical basis,
the basis can exactly represent polynomials,
which is a prerequisite for higher approximation orders.

However, the new basis functions pose new algorithmic challenges
as we are usually not able to apply the sparse grid algorithms
that were designed for hat functions.
We learn in this chapter that the main problem is the larger support
of higher-order B-splines (degree $p > 1$) when compared to degree $p = 1$.
Unfortunately, it is not possible to simply ``transform'' the basis
functions $\varphi_{l,i}^p$ to shrink the size of their support
to $\supp \varphi_{l,i}^1$.

In this chapter, we give an overview of algorithms for B-splines
on sparse grids.
Some algorithms are already known from the literature,
while others are new.
In any case, we give correctness results for every algorithm.
We take hierarchization as the example problem for our algorithms,
but the algorithms can be generalized to any linear operator.
Furthermore, most algorithms are not tailored for B-splines $\varphi_{l,i}^p$,
but applicable for general tensor product basis functions $\varphi_{l,i}$.
The types of feasible approaches for algorithms
depend on the ``regularity'' of the sparse grid at hand
(full grid/dimensionally adaptive sparse grid/spatially adaptive sparse grid).
Naturally, the more assumptions the grid satisfies, the faster and
easier will the corresponding algorithms be.

\Cref{sec:41problem} explains hierarchization as our example problem
and defines the notation used in this chapter.
The remaining three sections treat the different cases of grids:
First, \cref{sec:42fullGrids} deals with full grids to formalize and repeat
the well-known unidirectional principle.
Second, \cref{sec:43dimAdaptive} studies algorithms for
sparse grids that have been generated dimensionally adaptively.
Third, \cref{sec:44spatAdaptive} treats the for us most important case
of arbitrary (spatially adaptive) sparse grids.



\section{The Hierarchization Problem}
\label{sec:41problem}

Let $\sgset \subset \clint{0, 1}^d$ be a general (sparse) grid that
may be spatially adaptive, i.e.,
of the form $\setsize{\sgset} = \{\gp{\*l,\*i} \mid (\*l, \*i) \in \liset\}$,
where $\liset$ is a set of level-index pairs $(\*l, \*i)$ with $\*l \in \natz$
and $\*i \in \hiset{\*l}$ such that
$\ngp := \setsize{\sgset} = \setsize{\liset} < \infty$
(see \cref{sec:233spatiallyAdaptiveSG}).
The \term{hierarchization problem} is finding
\term{hierarchical surpluses}
$(\surplus{\*l',\*i'})_{(\*l',\*i') \in \liset} \in \real^{\ngp}$ such that
\begin{equation}
  \label{eq:hierarchizationProblem}
  \sum_{\mathclap{(\*l', \*i') \in \liset}} \surplus{\*l',\*i'}
  \basis{\*l',\*i'}(\gp{\*l,\*i}) = \fcnval{\*l,\*i}
  \quad\text{for all}\quad
  (\*l, \*i) \in \liset,
\end{equation}
where $(\fcnval{\*l,\*i})_{(\*l,\*i) \in \liset} \in \real^{\ngp}$ is a given set of
function values $\objfun(\gp{\*l,\*i})$ at the grid points $\gp{\*l,\*i}$.
The hierarchical surpluses then define the interpolant $\sgintp$ as
\begin{equation}
  \sgintp\colon \clint{\*0, \*1} \to \real,\quad
  \sgintp :=
  \sum_{\mathclap{(\*l', \*i') \in \liset}} \surplus{\*l',\*i'}
  \basis{\*l',\*i'},
\end{equation}
which interpolates $\objfun$ at the grid points $\gp{\*l,\*i}$ of $\sgset$.

The basis functions $\basis{\*l',\*i'}$ are
arbitrary tensor product functions.
We explicitly allow $\basis{\*l',\*i'}$ to be a non-hierarchical
nodal basis, in which case $\*l'$ is constant and
$\sgset$ is usually a full grid.
Strictly speaking, the problem is then a \term{interpolation problem}
and the $\surplus{\*l',\*i'}$ are \term{interpolation coefficients}.
However, we still apply the terms
``hierarchization'' and ``hierarchical surpluses'' in this case
to keep the terminology consistent.

\paragraph{Hierarchization as a linear operator}

The example of hierarchization can be generalized
to arbitrary linear operators
\begin{equation}
  \linop\colon \real^{\ngp} \to \real^{\ngp},\quad
  \vlinin \mapsto \vlinout = \linop[\vlinin],
\end{equation}
where $\linop$ depends on the grid $\sgset$ at hand.
Input $\vlinin$ and output $\vlinout$ are scalar-valued data%
%\footnote{%
%  The setting could be generalized even more,
%  for example to vector-valued data.%
%  We restrict ourselves to the scalar-valued case,
%  keeping hierarchization as our main application in mind.%
%}
\begin{equation}
  \vlinin = (\linin{\*l,\*i})_{(\*l,\*i) \in \liset} \in \real^{\ngp},\quad
  \vlinout = (\linout{\*l,\*i})_{(\*l,\*i) \in \liset} \in \real^{\ngp},
\end{equation}
which give one scalar per grid point $\gp{\*l,\*i} \in \sgset$.
For the case of hierarchization,
$\linop$ is the inverse of the \term{interpolation matrix}
$\intpmat \in \real^{\ngp \times \ngp}$:
\begin{subequations}
  \label{eq:hierarchizationSLE}
  \begin{equation}
    \linop = \intpmat^{-1},\quad
    \intpmat = (\basis{\*l',\*i'}(\gp{\*l,\*i}))_%
    {(\*l,\*i),(\*l',\*i') \in \liset},\quad
    \vlinin = (\fcnval{\*l,\*i})_{(\*l,\*i) \in \liset},\quad
    \vlinout = (\surplus{\*l',\*i'})_{(\*l',\*i') \in \liset}.
  \end{equation}
  This means that we can determine the $\surplus{\*l',\*i'}$ by solving
  the $\ngp \times \ngp$ system of linear equations
  \begin{equation}
    \vlinout = \linop[\vlinin]
    \quad\iff\quad
    \intpmat \cdot (\surplus{\*l',\*i'})_{(\*l',\*i') \in \liset}
    = (\fcnval{\*l,\*i})_{(\*l,\*i) \in \liset}.
  \end{equation}
\end{subequations}

\paragraph{Complexity of B-spline hierarchization}

As noted in \cite{Valentin18Fundamental},
hierarchization on sparse grids with hierarchical B-splines
$\basis{\*l,\*i}^\*p$ of degree $\*p$
as basis functions $\basis{\*l,\*i}$ is a tedious task.
The corresponding linear system \eqref{eq:hierarchizationSLE} is in general
non-symmetric
(i.e., $\basis{\*l',\*i'}^\*p(\gp{\*l,\*i}) \not=
\basis{\*l,\*i}^\*p(\gp{\*l',\*i'})$) and densely populated.
This is because the matrix entry in the $(\*l,\*i)$-th row and
$(\*l',\*i')$-th column vanishes if and only if
\begin{equation}
  \gp{\*l,\*i} \notin \interiorsupp \basis{\*l',\*i'}^\*p
  \iff
  \ex{t = 1, \dotsc, d}{
    \gp{l_t,i_t} \notin
    \opintscaled{
      \gp{l'_t,i'_t} - \tfrac{p_t+1}{2} \ms{l'_t},\,
      \gp{l'_t,i'_t} + \tfrac{p_t+1}{2} \ms{l'_t}
    }
  },
\end{equation}
where $\interiorsupp$ is the interior of the support
\cite{Valentin18Fundamental}.
For coarse levels $\*l'$, the mesh size $\ms{l'_t}$ is large in
every dimension $t$, which implies that $\interiorsupp \basis{\*l',\*i'}^\*p$
contains most of the grid points.
In contrast to the hat function case ($\*p = \*1$),
the value of $\surplus{\*l',\*i'}$ depends not only on
$\fcnval{\*l,\*i}$ and the data of its $3^d - 1$ neighboring grid points
on the boundary of $\supp \basis{\*l',\*i'}^\*1$,
but potentially on the data of the whole grid.

This prohibits the use of the \up
(which we will discuss in the next \cref{sec:42fullGrids})
on sparse grids with hierarchical B-splines.
Consequently, we have to solve the linear system
\eqref{eq:hierarchizationSLE}, which is significantly more time-consuming,
as it takes between $\landauOmega{\ngp^2 d}$ and $\landauO{\ngp^3 d}$ time
(via Gaussian elimination).
In addition, if we use an explicit solver for the linear system,
we additionally have to store an $\ngp \times \ngp$ matrix in memory.
However, a grid of size $\ngp = \num{50000}$ would already exceed the memory
of a \SI{16}{\gibi\byte} workstation
(if we explicitly store the full matrix in double precision).
In comparison, for the hat function basis,
the \up only requires $\landauO{\ngp d}$ time and $\landauO{\ngp}$ memory.

\paragraph{Notation}

We do not need the hierarchical level-index information $(\*l, \*i)$ in
$\sgset$, $\liset$, $\vlinin$, and $\vlinout$
for most of the considerations in this chapter.
Therefore, we assume that in each dimension $t$, the level-index pairs
$(l_t, i_t)$ ($l_t \in \natz$, $i_t \in \hiset{l_t}$)
are continuously enumerated by a single index $k_t = k_t(l_t, i_t) \in \natz$.
We identify $(\*l, \*i)$ with a single index $\*k$,
whose $t$-th component is given by $k_t(l_t, i_t)$.
Consequently,
we can regard $\liset$ as a subset $\{\*k \mid \*x_\*k \in \sgset\}$
of $\nat^d$.
We will switch between the notations whenever appropriate.
All statements that are formulated in the $\*k$ notation are
valid for both the nodal and the hierarchical basis
(i.e., for all tensor product bases).

\usenotation{k}
In the following, $k_t$ denotes the $t$-th component of a $d$-vector $\*k$
as usual.
\usenotation{kt10}
With $\*k_{-t}$, we denote the $(d-1)$-vector that is obtained from $\*k$
by omitting the $k$-th component,
i.e., $\*k_{-t} := (k_1, \dotsc, k_{t-1}, k_{t+1}, \dotsc, k_d)$.
\usenotation{kT20}
For a $j$-tuple $T = (t_1, \dotsc, t_j) \in \{1, \dotsc, d\}^j$,
we define $\*k_T$ to be the $j$-vector $(k_{t_1}, \dotsc, k_{t_j})$
that only contains the entries of the dimensions listed in $T$.
\usenotation{kT30}
Accordingly, $\*k_{-T}$ is defined as the $(d-j)$-vector
that contains the entries of the remaining dimensions
(sorted by the dimension $t$).
We define $\*k_{\range{a}{b}} := (k_a, k_{a+1}, \dotsc, k_b)$
as an indexing shortcut ($a \le b$).

\section{Hierarchization on Full Grids (Unidirectional Principle)}
\label{sec:42fullGrids}

If $\sgset$ is a full grid $\fgset{\*l}$
(see \cref{sec:21nodalSpaces}),
the well-known \emph{\up}
can be used to apply $\linop$ to input data $\vlinin$.
As shown in \cref{fig:unidirectionalPrinciple}, the idea of the \up
is to apply the corresponding one-dimensional operators on the
one-dimensional subgrids (the \emph{poles}) of $\sgset$,
which is repeated for all dimensions.
In this section, we first formulate the \up for
general linear operators $\linop$ and then prove its correctness for
the case $\linop = \intpmat^{-1}$ of hierarchization.
The correctness for the general case of arbitrary tensor product operators
will follow from the considerations in \cref{sec:45spatAdaptiveUP}.

\begin{figure}
  \includegraphics{unidirectionalPrinciple_1}%
  \caption[%
    Unidirectional principle%
  ]{%
    Application of a linear operator $\linop$
    on two-dimensional sparse grid data with the unidirectional principle.
    First, the univariate operator $\linop_1$ is applied on
    the input data $\vlinin$
    along poles of the first dimension $x_1$ \emph{(left)}.
    All grid points of the same color are part of the same pole
    (equivalence classes of ``$\samepole{t}$'' in
    \cref{alg:unidirectionalPrinciple}).
    Second, the univariate operator $\linop_2$ is applied on the
    resulting intermediate data $\vlinout[(1)]$
    along poles of the second dimension $x_2$ \emph{(center)}.
    This gives the final values $\vlinout = \linop[\vlinin]$ \emph{(right)}.%
  }%
  \label{fig:unidirectionalPrinciple}%
\end{figure}

\paragraph{Unidirectional principle and its correctness}

The \up is given in \cref{alg:unidirectionalPrinciple}.
The algorithm is given a permutation $(t_1, \dotsc, t_d)$ of $(1, \dotsc, d)$
that specifies the order of dimensions in which the \up should be applied.
We denote with $\linop_{t_j}$ the one-dimensional version of $\linop$
applied in dimension $t_j$ ($j = 1, \dotsc, d$).
With the following invariant, we can prove the correctness of the \up.

\begin{algorithm}
  \begin{algorithmic}[1]
    \Function{$\vlinout =$ unidirectionalPrinciple}{%
      $\liset$, $(t_1, \dotsc, t_d)$, $\vlinin$%
    }
      \State{$\vlinout[(0)] \gets \vlinin$}
      \For{$j = 1, \dotsc, d$}
        \State{%
          Define equivalence relation $\samepole{t_j}$ on $\liset$
          as $\*k \samepole{t_j} \*k' \iff \*k_{-t_j} = \*k'_{-t_j}$%
        }
        \For{$\liset_\mathrm{pole} \in \eqclasses{\liset}{\samepole{t_j}}$}
          \State{%
            $(\linout[(j)]{\*k})_{\*k \in \liset_\mathrm{pole}} \gets
            \linop_{t_j}
            \Big[(\linout[(j-1)]{\*k})_{\*k \in \liset_\mathrm{pole}}\Big]$%
          }
          \Comment{apply 1D operator on pole}%
          \label{line:algUnidirectionalPrinciple1}
        \EndFor{}
      \EndFor{}
      \State{$\vlinout \gets \vlinout[(d)]$}
    \EndFunction{}
  \end{algorithmic}
  \caption[%
    Unidirectional principle%
  ]{%
    Application of a tensor product operator $\linop$ with
    the unidirectional principle.
    Inputs are the set $\liset$ of grid indices,
    the permutation $(t_1, \dotsc, t_d)$ specifying the order in which
    the one-dimensional operators $\linop_{t_j}$ should be applied, and
    the vector $\vlinin = (\linin{\*k})_{\*k \in \liset}$ of input data.
    The output is the vector $\vlinout = (\linout{\*k})_{\*k \in \liset}$
    of output data.%
  }%
  \label{alg:unidirectionalPrinciple}%
\end{algorithm}

\begin{proposition}[%
  invariant of \cref{alg:unidirectionalPrinciple} for hierarchization%
]
  \label{prop:invariantUnidirectionalPrinciple}
  Let $\linop$ be the hierarchization operator on a full grid,
  i.e.,
  $\linop = \intpmat^{-1}$,
  $\vlinin = (\fcnval{\*k})_{\*k \in \liset}$,
  $\vlinout = (\surplus{\*k})_{\*k \in \liset}$,
  $\linop_{t_j}$ is the 1D interpolation operator $(\intpmat_{t_j})^{-1}$, and
  $\liset = \{\*0, \dotsc, \*2^\*l\}$
  corresponds to a full grid $\fgset{\*l}$ of level $\*l$.
  After iteration $j$ of \cref{alg:unidirectionalPrinciple}
  ($j = 1, \dotsc, d$), it holds for $T := (t_1, \dotsc, t_j)$
  \begin{equation}
    \sum_{\*k_T=\*0}^{\*2^{\*l_T}}
    \surplus[(j)]{(\*k_T,\*k'_{-T})} \basis{\*k_T}(\gp{\*k'_T})
    = \fcnval{\*k'},\quad
    \*k' = \*0, \dotsc, \*2^\*l,
  \end{equation}
  where $(\*k_T,\*k'_{-T})$ is defined to be the index $\*k''$
  with $\*k''_T := \*k_T$ and $\*k''_{-T} := \*k'_{-T}$.
\end{proposition}

\begin{proof}
  We prove the assertion by induction over $j = 1, \dotsc, d$.
  We set $T' := (t_1, \dotsc, t_{j-1})$,
  $T := (t_1, \dotsc, t_{j-1}, t_j)$,
  and we exploit the tensor product structure of the basis
  to write the \lhs of the assertion for $j$
  and arbitrary $\*k' = \*0, \dotsc, \*2^\*l$ as
  \begin{align}
    \sum_{\*k_T=\*0}^{\*2^{\*l_T}}
    \surplus[(j)]{(\*k_T,\*k'_{-T})} \basis{\*k_T}(\gp{\*k'_T})
    &= \sum_{\*k_{T'}=\*0}^{\*2^{\*l_{T'}}}
    \basis{\*k_{T'}}(\gp{\*k'_{T'}}) \cdot
    \sum_{k_{t_j}=0}^{2^{l_{t_j}}}
    \surplus[(j)]{(\*k_T,\*k'_{-T})} \basis{k_{t_j}}(\gp{k'_{t_j}}).\\
    \intertext{%
      If we choose the equivalence class
      $\liset_\mathrm{pole} := \clint{(\*k_T,\*k'_{-T})}_{\samepole{t_j}}$
      ($\*k_T$ arbitrary),
      then the sum over $k_{t_j}$ equals
      $\sum_{\*k \in \liset_\mathrm{pole}}
      \surplus[(j)]{\*k} \basis{k_{t_j}}(\gp{k'_{t_j}})
      %= ((\linop_{t_j})^{-1}
      %\Big[(\surplus[(j)]{\*k})_{\*k \in \liset_\mathrm{pole}}\Big])_{k'_{t_j}}
      %= ((\surplus[(j-1)]{\*k})_{\*k \in \liset_\mathrm{pole}})_{k'_{t_j}}
      = \surplus[(j-1)]{(\*k_{T'},\*k'_{-T'})}$ ($\ast$)
      by the 1D interpolation operator $T_{t_j}$
      (\cref{line:algUnidirectionalPrinciple1} of
      \cref{alg:unidirectionalPrinciple}).
      We can conclude that the \lhs equals%
    }
    &= \sum_{\*k_{T'}=\*0}^{\*2^{\*l_{T'}}}
    \surplus[(j-1)]{(\*k_{T'},\*k'_{-T'})}
    \basis{\*k_{T'}}(\gp{\*k'_{T'}}),
  \end{align}
  which, by the induction hypothesis, equals $\fcnval{\*k'}$ as desired
  (if $j > 1$).
  The same reasoning for $(\ast)$ can be used
  to establish the base case for $j = 1$.
\end{proof}

\begin{shortcorollary}[correctness of \cref{alg:unidirectionalPrinciple}]
  \label{cor:algUnidirectionalPrincipleCorrectness}
  \Cref{alg:unidirectionalPrinciple}
  is correct for hierarchization on full grids.
\end{shortcorollary}

\begin{proof}
  We apply \cref{prop:invariantUnidirectionalPrinciple} to obtain
  $\sum_{\*k=\*0}^{\*2^\*l}
  \surplus[(j)]{\*k} \basis{\*k}(\gp{\*k'})
  = \fcnval{\*k'}$
  for all $\*k' = \*0, \dotsc, \*2^\*l$, i.e.,
  the $\surplus[(j)]{\*k}$ are the correct interpolation coefficients
  according to \eqref{eq:hierarchizationProblem}.
\end{proof}

\paragraph{Complexity}

We compare the complexity of the \up for hierarchization compared
to directly solving the system \eqref{eq:hierarchizationSLE} of
linear equations.
If we assume that $d$ is constant and that
$\linop$ and $\linop_{t_j}$ apply Gaussian elimination to
solve the multivariate and univariate systems, respectively,
then directly solving \eqref{eq:hierarchizationSLE} takes
$\landauO{\ngp^2 (\ngp + d)}$ time%
\footnote{%
  $\landauO{\ngp^2 d}$ for calculating $\intpmat$ and
  $\landauO{\ngp^3}$ for solving the system.
}
and
$\landauO{\ngp^2}$ memory.
In contrast, the \up only requires
$\landauO{\ngp \sum_t \ngp_t^2}$ time%
\footnote{%
  There are $\ngp/\ngp_t$ poles in the
  $t$-th iteration of \cref{alg:unidirectionalPrinciple}.
  Each pole requires the solution of an $\ngp_t \times \ngp_t$ linear system,
  which takes $\landauO{\ngp_t^3}$ time.
},
if $\ngp_t$ is the grid size
$\setsize{\{k_t \mid \*k \in \liset\}}$ in dimension $t = 1, \dotsc, d$,
and $\landauO{\max_t N_t^2}$ memory.
The dependency from the univariate grid sizes $\ngp_t$ instead of $\ngp$
makes the \up significantly less computationally expensive.
As already mentioned,
the \up is even more efficient for the piecewise linear case
where the univariate interpolation operators can be applied
in-place.
Hence, it only needs $\landauO{Nd}$ time and
$\landauO{N}$ memory in this case.

\section{Hierarchization on Dimensionally Adaptive Sparse Grids}
\label{sec:43dimAdaptive}

\blindtext{}



\subsection{Combination Technique and Combinatorial Proof}
\label{sec:431combiTechniqueProof}

\todo{mention Delvos/Schempp}

\todo{mention Wasilkowski}

\blindtext{}

\paragraph{Formal description and outline of a combinatorial proof}

In the following, we give a formal description of the
sparse grid combination technique and we outline a combinatorial proof
for its correctness.
We discuss a high-level explanation of the proofs in this section
The proofs themselves can be found in \cref{sec:proofCombiTechnique}
as most of them are rather technical.
For simplicity,
we formulate the combination technique and its proof for regular
sparse grids (see \cref{sec:231regularSG}).
However, the main ideas of the proof chain are also applicable
for dimensionally adaptive sparse grids
(see \cref{sec:232dimensionallyAdaptiveSG}).

\begin{theorem}[sparse grid combination technique]
  Let $\liset := \{(\*l, \*i) \mid
  \normone{\*l} \le n,\; \*i \in \hiset{\*l}\}$
  correspond to the regular sparse grid
  $\regsgset{n}{d}{}$ and let $(\fcnval{\*l,\*i})_{(\*l,\*i) \in \liset}$
  be given function values on $\regsgset{n}{d}{}$.
  If we define
  \begin{itemize}
    \item
    the combined sparse grid interpolant $\regsgintp{n}{d}{\ct}$ via
    \eqref{eq:combiTechnique}, i.e.,
    \begin{equation}
      \regsgintp{n}{d}{\ct}
      = \sum_{q=0}^{d-1} (-1)^q \binom{d-1}{q} \sum_{\normone{\*l} = n-q}
      \fgintp{\*l},
    \end{equation}
    where $\fgintp{\*l} \in \ns{\*l}$ is the full grid interpolant
    of $f$ with level $\*l$, and
    
    \item
    the hierarchical sparse grid interpolant $\regsgintp{n}{d}{}$
    via \eqref{eq:hierarchizationProblem} and
    \eqref{eq:hierarchizationInterpolant},
  \end{itemize}
  then the combined and the hierarchical sparse grid interpolants coincide:
  \begin{equation}
    \regsgintp{n}{d}{}
    = \regsgintp{n}{d}{\ct}.
  \end{equation}
\end{theorem}

\begin{proof}[Proof (sketch)]
  \blindtext{}
\end{proof}

\begin{restatable}[inclusion-exclusion principle]{%
  proposition%
}{%
  propCombiTechniqueOne%
}
  \label{prop:combiTechniqueOne}
  For every $\gp{\*l,\*i} \in \regsgset{n}{d}{}$, we have
  \begin{equation}
    \label{eq:combiTechniqueOne}
    \sum_{q=0}^{d-1} (-1)^q \binom{d-1}{q} \cdot
    \setsize{
      \{\*l' \mid \normone{\*l'} = n - q,\; \fgset{\*l'} \ni \gp{\*l,\*i}\}
    }
    = 1.
  \end{equation}
\end{restatable}

\begin{proof}
  See \cref{sec:proofCombiTechnique}.
\end{proof}

\begin{definition}[relation ``$\sim$'']
  \label{def:combiTechniqueEquivalenceRelation}
  Let $\gp{\*l,\*i} \in \regsgset{n}{d}{}$ be fixed and
  \begin{equation}
    \label{eq:combiTechniqueSpecialLevelSet}
    \levelset
    := \{\*l' \mid \ex{q=0,\dotsc,d-1}{
      \normone{\*l'} = n - q,\; \fgset{\*l'} \notni \gp{\*l,\*i}
    }\}
  \end{equation}
  be the set of levels that do not contain $\gp{\*l,\*i}$.
  We define a relation ``$\sim$'' on $L$ as follows:
  For $\*l', \*l'' \in L$, we set $\*l' \sim \*l''$ if and only if
  \begin{equation}
    \fa{t \notin T_{\*l',\*l''}}{\min\{l'_t, l''_t\} \ge l_t},\quad
    T_{\*l',\*l''}
    := \{t \mid l'_t = l''_t\}.
  \end{equation}
\end{definition}

\begin{restatable}{%
  shortlemma%
}{%
  lemmaCombiTechniqueIdenticalValues%
}
  \label{lemma:combiTechniqueIdenticalValues}
  Let $\*l', \*l'' \in L$ with $\*l' \sim \*l''$.
  Then, $\fgintp{\*l'}(\gp{\*l,\*i})
  = \fgintp{\*l''}(\gp{\*l,\*i})$.
\end{restatable}

\begin{proof}
  See \cref{sec:proofCombiTechnique}.
\end{proof}

\todo{add notation for equivalence class, replace}

\begin{restatable}[characterization of the equivalence classes of ``$\sim$'']{%
  lemma%
}{%
  lemmaCombiTechniqueCharacterization%
}
  \label{lemma:combiTechniqueCharacterization}
  Let $L_0 \in L/{\sim}$ be an equivalence class of ``$\sim$''.
  If we define
  \begin{equation}
    T_{L_0}
    := \{t \mid \exfa{l^\ast_t}{\*l' \in L_0}{l'_t = l^\ast_t}\}
  \end{equation}
  as the set of dimensions in which all levels in $L_0$
  have the same entry, then
  \begin{equation}
    L_0
    = \{\*l' \in L \mid
    \fa{t \in T_{L_0}}{l'_t = l^\ast_t},\;
    \fa{t \notin T_{L_0}}{l'_t \ge l_t}\}.
  \end{equation}
\end{restatable}

\begin{proof}
  See \cref{sec:proofCombiTechnique}.
\end{proof}

\begin{restatable}[function value cancellation]{%
  proposition%
}{%
  propCombiTechniqueZero%
}
  \label{prop:combiTechniqueZero}
  For every $\gp{\*l,\*i} \in \regsgset{n}{d}{}$, we have
  \begin{equation}
    \sum_{q=0}^{d-1} (-1)^q \binom{d-1}{q} \cdot
    \sum_{\substack{\normone{\*l'} = n - q\\\fgset{\*l'} \notni \gp{\*l,\*i}}}
    \fgintp{\*l'}(\gp{\*l,\*i})
    = 0.
  \end{equation}
\end{restatable}

\begin{proof}
  See \cref{sec:proofCombiTechnique}.
\end{proof}

\blindtext{}



\subsection{Hierarchization with the Combination Technique}
\label{sec:432hierarchizationCombiTechnique}

\blindtext{}

\subsection{%
  Hierarchization by Successive Applications of the Unidirectional Principle%
}
\label{sec:433hierarchizationSuccessiveUP}

\blindtext{}

\section{Hierarchization on Spatially Adaptive Sparse Grids}
\label{sec:44spatAdaptive}

\blindtext{}



\subsection{Unidirectional Principle for Linear-Type Basis Functions}
\label{sec:441linearType}

\blindtext{}



\subsection{Iterative Application of the Unidirectional Principle}
\label{sec:442iterativeUnidirectionalPrinciple}

\blindtext{}



\subsection{Lagrange Bases and Fundamental Splines}
\label{sec:443fundamentalSplines}

\blindtext{}

\subsubsection{Hierarchization with Lagrange Bases}

\blindtext{}

\subsubsection{Canonical Way to Obtain Lagrange Bases}

\blindtext{}

\subsubsection{Fundamental Splines}

\blindtext{}

\subsubsection{Modified Fundamental Splines}

\blindtext{}



\subsection{Weakly Fundamental Splines}
\label{sec:444weaklyFundamentalSplines}

\blindtext{}

\subsubsection{Motivation and Definition}

\blindtext{}

\subsubsection{Hermite Hierarchization}

\blindtext{}

\subsubsection{Inserting Chain Points}

\blindtext{}


\cleardoublepage
