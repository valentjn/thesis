\setdictum{%
  There is a fine line between wrong and visionary.
  Unfortunately, you have to be a visionary to see it!%
}{%
  Sheldon Cooper%
}

\chapter{Introduction}

\cite{zenger91}

Hello World! \texttt{Hello World!}

\nocite{*}
\TODO{don't cite everything}

Umlaute: äöüß

Die Software \sgpp ist sehr cool.

\section{Bla}

This is\TODO{write} defined\TODO{write} as $a := 2b$.
This is the function $f$ (which is defined as $y =: f(x)$).

\begin{gather}
  X \times Y\\
  A \cdot \vec{x} = \vec{b}\\
  \min_{\vec{x} \in [0, 1]} \int_\Omega f(\vec{x}, \vec{y}) d\vec{y}\\
  4(a+b)f(x)g(x)h(x)p(x)(c+d)fghf^\prime g^\prime h^\prime\\
  f(x)\cos(x)g(x)\\
  \mathrm{f}(x)\cos(x)\mathrm{g}(x)
\end{gather}

\begin{table}
  \begin{tabular}{llll}
    \toprule
    \textbf{Header 1}&\textbf{Header 2}&\textbf{Header 3}&\textbf{Header 4}\\
    \midrule
    bla&bla&bla&bla\\
    bla&bla&bla&bla\\
    bla&bla&bla&bla\\
    \bottomrule
  \end{tabular}
  \caption{This is a test table.}
  \label{tbl:test}
\end{table}

\begin{figure}
  \begin{subfigure}{60mm}
    \includegraphics{bspl_hier_1}
    \caption{Dummy caption A.}
    \label{fig:test1a}
  \end{subfigure}
  \begin{subfigure}{60mm}
    \includegraphics{bspl_hier_1}
    \caption{Dummy caption B.}
    \label{fig:test1b}
  \end{subfigure}
  \caption{This is a test caption.}
  \label{fig:test1}
\end{figure}

\begin{figure}
  \includegraphics{bspl_hier_2}
  \caption{%
    This is a test caption.
    This is a test caption.
    This is a test caption.
    This is a test caption.
    This is a test caption.
    This is a test caption.
    This is a test caption.
    This is a test caption.
    This is a test caption.%
  }
  \label{fig:test2}
\end{figure}

\begin{algorithm}
  \begin{algorithmic}[1]
    \Statex{}
    \Function{$a =$ GetAffectedBasisFunctions}{$X$, $\vec{\alpha}$, $\vec{x}$, $t$, $\vec{\ell}$, $\vec{j}$, $b$}
    \If{$x_{\vec{\ell},\vec{j}} \notin X$}
    \Return{$0$}
    \Comment{nichts tun, falls Gitterpunkt nicht vorhanden}
    \EndIf\vspace{-0.25em}
    \If{$t = d$}
    \State{$a \gets \alpha_{\vec{\ell},\vec{j}} \cdot (b \cdot \varphi_{\ell_d,j_d}(x_d))$}
    \Comment{letzte Dimension: Summanden zu Ergebnis addieren}
    \color{green}
    \If{$\vec{x}_{\vec{\ell},\vec{j}}^{\text{(rn($d$))}} \in X$}
    $a \gets a + \alpha_{\vec{\ell},\vec{j}}^{\text{(rn($d$))}} \cdot
    (b \cdot \varphi_{\ell_d,j_d}^{\text{(rn($d$))}}(x_d))$
    \EndIf
    \label{line:getaffected1}
    \If{$\vec{x}_{\vec{\ell},\vec{j}}^{\text{(ln($d$))}} \in X$}
    $a \gets a + \alpha_{\vec{\ell},\vec{j}}^{\text{(ln($d$))}} \cdot
    (b \cdot \varphi_{\ell_d,j_d}^{\text{(ln($d$))}}(x_d))$
    \EndIf\vspace{-0.4em}
    \label{line:getaffected2}
    \color{black}
    %         \If{$\text{\Call{RightNeighbor}{$\vec{\ell}$, $\vec{j}$}} \in X$}
    %             $a \gets a + \alpha_{\vec{\ell},\vec{j}} \cdot (b \cdot \varphi_{\ell_d,j_d}(x_d))$
    %         \EndIf
    \Else
    \State{$a \gets \text{\Call{GABF}{$X$, $\vec{\alpha}$, $\vec{x}$, $t+1$,
          $\vec{\ell}$, $\vec{j}$, $b \cdot \varphi_{\ell_t,j_t}(x_t)$}}$}
    \Comment{nächste Dimension}
    \label{line:getaffected5}
    \color{green}
    \If{$\vec{x}_{\vec{\ell},\vec{j}}^{\text{(rn($t$))}} \in X$}
    $a \gets a + \text{\Call{GABF}{$X$, $\vec{\alpha}$, $\vec{x}$, $t+1$,
        $\vec{\ell}$, $\vec{j}^{\text{(rn($t$))}}$,
        $b \cdot \varphi_{\ell_t,j_t}^{\text{(rn($t$))}}(x_t)$}}$
    \EndIf
    \label{line:getaffected3}
    \If{$\vec{x}_{\vec{\ell},\vec{j}}^{\text{(ln($t$))}} \in X$}
    $a \gets a + \text{\Call{GABF}{$X$, $\vec{\alpha}$, $\vec{x}$, $t+1$,
        $\vec{\ell}$, $\vec{j}^{\text{(ln($t$))}}$,
        $b \cdot \varphi_{\ell_t,j_t}^{\text{(ln($t$))}}(x_t)$}}$
    \EndIf
    \label{line:getaffected4}
    \color{black}
    \EndIf\vspace{-0.25em}
    \If{$x_t > j_t h_{\ell_t}$}
    $a \gets a + \text{\Call{GABF}{$X$, $\vec{\alpha}$, $\vec{x}$, $t$,
        $\vec{\ell}^{\text{(rc($t$))}}$, $\vec{j}^{\text{(rc($t$))}}$, $b$}}$
    \Comment{nächster Level}
    \label{line:getaffected6}
    \Else{}
    $a \gets a + \text{\Call{GABF}{$X$, $\vec{\alpha}$, $\vec{x}$, $t$,
        $\vec{\ell}^{\text{(lc($t$))}}$, $\vec{j}^{\text{(lc($t$))}}$, $b$}}$
    \EndIf\vspace{-0.15em}
    \label{line:getaffected7}
    \State{\Return{$a$}}
    \EndFunction
  \end{algorithmic}
  \caption{Approximative Auswertung von Linearkombinationen auf dünnen Gittern,
    Zeilen \ref*{line:getaffected1}, \ref*{line:getaffected2},
    \ref*{line:getaffected3}, \ref*{line:getaffected4} nicht für
    stückweise lineare Basisfunktionen,\\
    \emph{input:}
    Gitter $X = \{\vec{x}_i\}_i$,
    Koeffizienten $\vec{\alpha} = (\alpha_i)_i$,
    Auswertungspunkt $\vec{x} \in [0, 1]^d$,
    aktuelle Dimension $t \in \{1, \dotsc, d\}$ (anfangs $1$),
    Level und Index $(\vec{\ell}, \vec{j})$ des aktuellen Punkts
    (für randlose Gitter anfangs $(\vec{e}, \vec{e})$) und
    aktuelles Produkt $b$ von 1D-Auswertungen (anfangs 1),\\
    \emph{output:}
    $a \approx \widetilde{f}(\vec{x}) = \sum_{k=1}^N \alpha_k \varphi_k(\vec{x})$
    (für stückweise lineare Funktionen sogar $a = \widetilde{f}(\vec{x})$)}%
  \label{alg:getaffected}
\end{algorithm}

\cref{tbl:test}

\cref{fig:test1}

\cref{fig:test1a}

\cref{fig:test1b}

\cref{fig:test2}

\cref{alg:getaffected}

\begin{theorem}[TODO Theorem]
  \blindtext
\end{theorem}

\begin{lemma}[TODO Lemma]
  TODO
\end{lemma}

\blindtext

\begin{definition}[TODO Definition]%
  \blindtext
\end{definition}

\blindmathpaper
