\chapter{Proofs}

\section{%
  Proof of \texorpdfstring{%
    \Cref{prop:invariantCoarseBoundary}%
  }{%
    Proposition \ref{prop:invariantCoarseBoundary}%
  }%
}
\label{sec:proofInvariantCoarseBoundary}

\begin{proof}
  First, we show that every inserted level $\ßl' \in \NN_0^t$ in the inner loop
  can be found on the right-hand side of~\eqref{eq:coarseInvariant}.
  If $\ßl' := (\ßl, 0)$
  is inserted for some $\ßl \in L^{(t-1)}$,
  then we have $\norm{\mathop{\vec{\max}}(\ßl, \ß1)}_1 \le n+d+t-b$ or
  $\ßl = \ß0$ by line \ref{line:algCoarseBoundary1}.
  In the first case, we have
  \begin{equation}
    \norm{\mathop{\vec{\max}}(\ßl', \ß1)}_1
    = \norm{\mathop{\vec{\max}}(\ßl, \ß1)}_1 + 1
    \le n - d + t - b + 1,
  \end{equation}
  and in the second case $\ßl' = \vec{0}$.
  In either case, $\ßl'$ is contained in the RHS of \eqref{eq:coarseInvariant}.
  
  If $\ßl' := (\ßl, l_t)$ is inserted
  for some $\ßl \in L^{(t-1)}$ and
  $l_t \in \{1, \dotsc, l^\ast\}$, then there are,
  depending on whether $\ßl \in \NN^{t-1}$ or not, two cases:
  \begin{itemize}
    \item
    If $\ßl \in \NN^{t-1}$, then $\ßl' \in \NN^{t-1}$ as well and
    $\norm{\ßl'}_1 \le \norm{\ßl}_1 + l^\ast = n - d + t$
    due to line \ref{line:algCoarseBoundary2},
    i.e., $\ßl'$ is contained in the first set of the RHS of
    \eqref{eq:coarseInvariant}.
    
    \item
    If $\ßl \notin \NN^{t-1}$, then $\ßl' \notin \NN^{t-1}$ as well and
    $\norm{\mathop{\vec{\max}}(\ßl', \ß1)}_1
    \le \norm{\mathop{\vec{\max}}(\ßl, \ß1)}_1 + l^\ast
    = n - d + t - b + 1$
    due to line \ref{line:algCoarseBoundary3},
    i.e., $\ßl$ is contained in the second set of the RHS of
    \eqref{eq:coarseInvariant}.
  \end{itemize}
  Thus, all levels that the algorithm inserts into $L^{(t)}$
  can be found on the RHS of \eqref{eq:coarseInvariant}.
  
  It remains to prove that all levels on the RHS of~\eqref{eq:coarseInvariant}
  are inserted by the algorithm into $L^{(t)}$ eventually.
  We prove this by induction over $t = 1, \dotsc, d$.
  For $t = 1$, the right-hand side (RHS) of \eqref{eq:coarseInvariant} equals
  $\{l \in \NN_0 \mid l \le n-d+1\}$, which is just $L^{(1)}$.
  For the induction step $(t - 1) \to t$, we assume
  the validity of the induction hypothesis
  \begin{equation}
    \label{eq:proofCoarseInductionHypothesis}
    \begin{aligned}
    L^{(t-1)}
    &= \{\ßl \in \NN^{t-1} \mid
    \norm{\ßl}_1 \le n-d+t-1\} \dotcup {}\\
    &\hphantom{{}={}}
    \big(\{\ßl \in \NN_0^{t-1} \setminus \NN^{t-1} \mid
    \norm{\mathop{\vec{\max}}(\ßl, \ß1)}_1 \le n-d+t-b\} \cup
    \{\vec{0}\}\big).
    \end{aligned}
  \end{equation}
  
  \todo{check beginning from here: n --> n-d+1, l' --> l, l --> l'}
  First, let $\ßl' = (\ßl, l_t)$ be in the first set of the RHS,
  i.e., $\norm{\ßl'}_1 \le n - d + t$ and $l_t > 0$.
  Note that $\ßl$ will be encountered in the inner loop, as
  \begin{equation*}
    \ßl \in \NN^{t-1},\quad
    \norm{\ßl}_1
    = \norm{\ßl'}_1 - l_t
    \le (n-d+t) - 1
    = n - d + t - 1
  \end{equation*}
  implies $\ßl \in L^{(t-1)}$ by the induction
  hypothesis~\eqref{eq:proofCoarseInductionHypothesis}.
  Since $N_{\ßl} = 0$ and
  \begin{equation*}
    l_t
    = \norm{\ßl'}_1 - \norm{\ßl}_1
    \le n-d+t - \norm{\ßl}_1
    = u_{\ßl},
  \end{equation*}
  this $\ßl'$ is inserted into $L^{(t)}$ during the innermost loop.
  
  Now, we take $\ßl = (\ßl', l_t)$
  in the second set of the RHS, i.e.,
  $\norm{\ßl}_1 + N_{\ßl} \le n+t-b$ and
  $N_{\ßl} > 0$.
  Again, there are two cases:
  \begin{enumerate}
    \item
    $l_t > 0$:
    This implies $N_{\ßl'} = N_{\ßl} > 0$ and
    $\ßl'$ will be reached in the loop 2b) since
    \begin{equation*}
      \ßl' \in \NN_0^{t-1} \setminus \NN^{t-1},\quad
      \norm{\ßl'}_1 + N_{\ßl'}
      = (\norm{\ßl}_1 - l_t) + N_{\ßl}
      \le n+t-b-l_t
      \le n+t-1-b
    \end{equation*}
    implies $\ßl' \in L^{(t-1)}$ by the induction
    hypothesis~\eqref{eq:proofCoarseInductionHypothesis}.
    Also,
    \begin{equation*}
      l_t
      = \norm{\ßl}_1 - \norm{\ßl'}_1
      \le (n+t-b - N_{\ßl}) - \norm{\ßl'}_1
      = n+t-b - (\norm{\ßl'}_1 + N_{\ßl'})
      = u_{\ßl'},
    \end{equation*}
    i.e., $\ßl$ gets added to $L^{(t)}$
    by the innermost loop.
    
    \item
    $l_t = 0$:
    This implies
    $\norm{\ßl'}_1 = \norm{\ßl}_1$ and
    $N_{\ßl'} = N_{\ßl} - 1$.
    $\ßl'$ will be reached in the loop 2b) since
    \begin{equation*}
      \ßl' \in \NN_0^{t-1} \setminus \NN^{t-1},\quad
      \norm{\ßl'}_1 + N_{\ßl'}
      = \norm{\ßl}_1 + (N_{\ßl} - 1)
      \le n + t - 1 - b,
    \end{equation*}
    if $N_{\ßl'} > 0$ and
    \begin{equation*}
      \ßl' \in \NN^{t-1},\quad
      \norm{\ßl'}_1
      = \norm{\ßl}_1
      \le n + t - b - N_{\ßl}
      \le n + t - 2,
    \end{equation*}
    if $N_{\ßl'} = 0$
    (due to $N_{\ßl} = 1$ in this sub-case and $b \ge 1$).
    In both of these sub-cases,
    we have $\ßl' \in L^{(t-1)}$ by the induction
    hypothesis~\eqref{eq:proofCoarseInductionHypothesis}. Also,
    \begin{equation*}
      \norm{\ßl'}_1 + (N_{\ßl'} + 1)
      = \norm{\ßl}_1 + N_{\ßl}
      \le n + t - b.
    \end{equation*}
    i.e., $\ßl$ gets added to $L^{(t)}$
    by the if clause.
  \end{enumerate}
  
  Finally, we take $\ßl = (\vec{0}, 0) \in \NN_0^t$
  in the third set of the RHS.
  This level is appended to $L^{(t)}$, because
  $\ßl' = \vec{0} \in \NN_0^{t-1}$ is in $L^{(t-1)}$ by 
  induction hypothesis~\eqref{eq:proofCoarseInductionHypothesis} and $N_{\ßl'} = t - 1$
  (i.e., the second statement in the if clause is fulfilled).
\end{proof}
