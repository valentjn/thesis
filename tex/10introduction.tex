\setdictum{%
  There is a fine line between wrong and visionary.
  Unfortunately, you have to be a visionary to see it\dots%
}{%
  Sheldon Cooper%
}

\chapter{Introduction}

\blindtext{}

\paragraph{Motivation}

\blindtext{}

\paragraph{Related Work}

\blindtext{}

\paragraph{Original contribution}

This thesis is written to be largely self-contained.
Therefore, it is necessary that some introductory definitions and
results are repeated from the literature,
which is properly attributed in the respective chapters.
In addition, some new results have already been published in accordance
with the regulations for PhD theses at the University of Stuttgart.
Whenever a publication of the author of this thesis and his supervisor
is co-authored by collaborators,
the original contribution of the author is highlighted 
at the beginning of the respective chapters or sections.

\paragraph{Notation}

The notation of this thesis tries to be intuitive and suggestive.
For example, vectors are written in bold face, which allows for
very similar formulas for the univariate and the multivariate case
(e.g., $\sum_{l'=0}^l \sum_{i' \in I_{l'}}
\alpha_{l',i'} \varphi_{l',i'}$ becomes
$\sum_{\ßl'=\ß0}^\ßl \sum_{\ßi' \in I_{\ßl'}}
\alpha_{\ßl',\ßi'} \varphi_{\ßl',\ßi'}$).
Other necessary notation is introduced in the text when needed.
If a symbol or an abbreviation is unclear,
it is likely explained in the glossary at the beginning of the thesis.

\paragraph{Outline}

\blindtext{}

\cleardoublepage
