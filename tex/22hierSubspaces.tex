\section{Hierarchical Basis and Hierarchical Subspace}
\label{sec:22hierSubspaces}

The dimension of the nodal space $\ns{\ßl}$ is given by
\begin{equation}
  \label{eq:dimensionFG}
  \dim \ns{\ßl}
  = \setsize{\fgset{\ßl}}
  = \prod_{t=1}^d (2^{l_t} + 1).
\end{equation}
If we choose the same level $n \in \NNz$ in all dimensions,
then the dimension of $\ns{n,d}$ and the
number of grid points grow at least as fast as
$2^{nd} = (\ms{n}^{-1})^d$.
This exponential dependency between $\dim \ns{n,d}$ and $d$ is known as the
\term{curse of dimensionality}
\multicite{Bungartz04Sparse,Pflueger10Spatially}.
The curse makes interpolation on $\ns{\ßl}$ computationally infeasible
for dimensionalities $d > 4$,
as we would have to calculate and store $\dim \ns{\ßl}$ coefficients $\interpcoeff{\ßi}$.%
%\footnote{%
%  The number of necessary basis evaluations to evaluate the interpolant once
%  would not be as large, as most types of basis functions
%  (like hat functions $\basis{l,i}^1$ and higher-order B-splines)
%  are locally supported.%
%}



\subsection{Hierarchical Splitting in the Univariate Case}
\label{sec:221hierUV}

\paragraph{Hierarchical subspaces}

In order to reduce the computational effort,
we first split $\ns{\ßl}$ into smaller subspaces and then identify
which subspaces are most important and which subspaces can be omitted
at the cost of a slightly larger error.
In the univariate case, the key observation is that a grid point of a level $l$
can be written as a grid point of a higher level~$l'$:
\begin{equation}
  \label{eq:rewriteGridPoint}
  \gp{l,i} = \gp{l',i'},\quad
  l' \ge l,\quad
  i' = 2^{l'-l} i.
\end{equation}
Conversely, this implies that every grid point $\gp{l,i}$ of level $l \ge 1$
and index $i \ge 1$ can be uniquely written
as a grid point of a coarser level $l'$ and an odd index $i'$:
\begin{equation}
  \gp{l,i} = \gp{l',i'},\quad
  l' = l - \left[\log_2(\xor(i, i-1) + 1) - 1\right],\quad
  i' = i/2^{l-l'},
\end{equation}
% https://mathoverflow.net/a/29973
where $\xor$ is the bitwise ``exclusive or'' function.
The term in square brackets is the exponent of the
highest power of two that divides $i$.
As shown in \cref{fig:pointSplittingUniform},
this implies that $\fgset{l}$ decomposes into
\begin{equation}
  \fgset{l}
  = \bigdotcup_{l'=0}^l \{\gp{l',i'} \mid i' \in \his{l'}\},\quad
  \his{l'} :=
  \begin{cases}
    \{i' = 0, \dotsc, 2^{l'} \mid \text{$i'$ odd}\},&l' > 0,\\
    \{0, 1\},&l' = 0,
  \end{cases}
\end{equation}
where $\dotcup$ indicates the disjoint union.
We call the spaces spanned by the basis functions that correspond to the
joined sets \term{hierarchical subspaces} $\hs{l}$:
\begin{equation}
  \hs{l}
  := \spn\{\basis{l,i} \mid i \in \his{l}\}.
\end{equation}

\begin{figure}
  \includegraphics{pointSplitting_1}%
  \caption{%
    The set of grid points $\fgset{l}$ of level $l = 4$ \emph{(top)}
    decomposes into hierarchical grids of level $l' \le l$,
    whose grid points $\gp{l',i'}$ have odd indices $i' \in \his{l'}$
    ($\gp{0,0}$ being the only exception).%
  }
  \label{fig:pointSplittingUniform}
\end{figure}

\paragraph{Hierarchical splitting}

For the hat function basis $\basis{l,i}^1$ and other basis types,
we can prove that the corresponding nodal space
decomposes into the direct sum of all
hierarchical subspaces of coarser levels or the same level, i.e.,
\begin{equation}
  \label{eq:hierSplittingUV}
  \ns{l}
  \overset{?}{=} \bigoplus_{l'=0}^l \hs{l'},
\end{equation}
We call this relation \term{hierarchical splitting}.
Here, the direct sum $\oplus$ is
the normal vector space sum with the additional indication
that the dimension of the sum $\sum_{l'=0}^l \hs{l'}$ is the sum
of the dimensions of the summands $\hs{l'}$
(analogously to $\setsize{\fgset{l}} = \sum_{l'=0}^l \setsize{\fgset{l'}}$,
where $\fgset{l}$ is the disjoint union of the sets $\fgset{l'}$).
In general, \eqref{eq:hierSplittingUV} may not be true.
The following lemma provides a characterization,
which can be used to prove \eqref{eq:hierSplittingUV} for hat functions.
The hierarchical hat function basis is shown in \cref{fig:hierarchicalHat}.

\begin{figure}
  \subcaptionbox{%
    Basis functions $\basis{l',i'}^1$ ($l' \le l$, $i' \in \his{l'}$)
    and grid points $\gp{l',i'}$ \emph{(dots)}.
    The domain is the unit interval $\clint{0, 1}$.%
  }[75mm]{%
    \includegraphics{hierarchicalHat_1}%
  }%
  \hfill%
  \subcaptionbox{%
    Piecewise linear interpolant $\fgintp{l}^1$ as a linear combination
    of the hierarchical hat functions \emph{(stacked)}.
    The two boundary functions are combined to a single function
    \emph{\textcolor{C0}{(blue)}} for simplicity.%
  }[75mm]{%
    \includegraphics{hierarchicalHat_2}%
  }%
  \caption{Univariate hierarchical hat functions up to level $l = 3$.}
  \label{fig:hierarchicalHat}
\end{figure}

\begin{lemma}[univariate hierarchical splitting characterization]
  \label{lemma:hierSplittingUV}
  Equivalent to relation \eqref{eq:hierSplittingUV} is the satisfaction of
  both of the following conditions:
  \begin{itemize}
    \item
    The hierarchical subspaces $\hs{l'}$ ($l' \le l$) are subspaces of $\ns{l}$.
    
    \item
    The hierarchical functions
    $\basis{l',i'}$ ($l' \le l$, $i' \in \his{l'}$) are linearly independent.
  \end{itemize}
\end{lemma}
\begin{proof}
  The first condition is equivalent to $\sum_{l'=0}^l \hs{l'} \subset \ns{l}$.
  The second condition is equivalent to
  $\dim \sum_{l'=0}^l \hs{l'} = \sum_{l'=0}^l \dim \hs{l'}$,
  i.e., to the directness of the sum.
  Therefore, the logical conjunction of both is equivalent to
  $\bigoplus_{l'=0}^l \hs{l'} \subset \ns{l}$.
  If the sum is direct,
  the dimension of the sum is equal to $2 + \sum_{l'=1}^l 2^{l'-1} = 2^l + 1$
  (due to $\dim \hs{l'} = \setsize{\his{l'}} = 2^{l'-1}$ for $l' > 0$ and
  $\dim \hs{l'} = 2$ for $l' = 0$),
  which is also the dimension of $\ns{l}$.
  The only subspace of $\ns{l}$ that has the same dimension as $\ns{l}$ is $\ns{l}$ itself,
  so we infer $\bigoplus_{l'=0}^l \hs{l'} = \ns{l}$.
\end{proof}
\begin{corollary}
  The hierarchical splitting \eqref{eq:hierSplittingUV}
  holds for the hat function basis.
\end{corollary}
\begin{proof}
  The first condition of \cref{lemma:hierSplittingUV}
  is satisfied as piecewise linear splines of level~$l'$
  are also piecewise linear splines of higher levels $l \ge l'$.
  The linear independence for the second condition can be proved by induction
  over $l$:
  If a linear combination of $\basis{l',i'}^1$ ($l' \le l$, $i' \in \his{l'}$)
  vanishes everywhere, then the coefficients of level $l$ must be zero,
  as otherwise the basis functions $\basis{l,i'}^1$ ($i' \in \his{l}$) would
  introduce kinks at $\gp{l,i'}$, which the zero function does not have.
  This means that we have a zero linear combination of $\basis{l',i'}^1$ for
  $l' \le l - 1$, $i' \in \his{l'}$,
  and by the induction hypothesis, the other coefficients also vanish.
\end{proof}



\subsection{Hierarchical Splitting in the Multivariate Case}
\label{sec:222hierMV}

Multivariate hierarchical subspaces of level $\ßl$
are defined analogously to the 1D case:
\begin{equation}
  \hs{\ßl}
  := \spn\{\basis{\ßl,\ßi} \mid \ßi \in \his{\ßl}\},\quad
  \his{\ßl}
  := \his{l_1} \times \dotsb \times \his{l_d}.
\end{equation}
The splitting \eqref{eq:hierSplittingUV} can now be generalized to the
multivariate case:
\begin{equation}
  \label{eq:hierSplittingMV}
  \ns{\ßl}
  \overset{?}{=} \bigoplus_{\ßl'=0}^\ßl \hs{\ßl'},
\end{equation}
Again, this relation does not hold in general.
A multivariate counterpart of \thmref{lemma:hierSplittingUV} can be used
to prove that \eqref{eq:hierSplittingMV} holds if
the corresponding 1D relation \eqref{eq:hierSplittingUV} holds for all dimensions:
\begin{lemma}[multivariate hierarchical splitting characterization]
  \label{lemma:hierSplittingMV}
  Equivalent to relation \eqref{eq:hierSplittingMV} is the satisfaction of
  both of the following conditions:
  \begin{itemize}
    \item
    The hierarchical subspaces $\hs{\ßl'}$ ($\ßl' \le \ßl$) are subspaces of $\ns{\ßl}$.
    
    \item
    The basis functions $\basis{\ßl',\ßi'}$ ($\ßl' \le \ßl$, $\ßi' \in \his{\ßl'}$)
    are linearly independent.
  \end{itemize}
\end{lemma}
\begin{proof}
  If the sum is direct, then its dimension is given by
  \begin{equation}
    \hspace*{-5mm}
    \dim \sum_{\ßl'=0}^\ßl \hs{\ßl'}
    = \sum_{l_1'=0}^{l_1} \dotsb \sum_{l_d'=0}^{l_d}
    \prod_{t=1}^d \dim \hs{l_t'}
    = \prod_{t=1}^d \sum_{l_t'=0}^{l_t} \dim \hs{l_t'}
    = \prod_{t=1}^d (2^{l_t} + 1)
    = \dim \ns{\ßl}
    \hspace*{-5mm}
  \end{equation}
  using \eqref{eq:dimensionFG}.
  The rest is analogous to the proof of \cref{lemma:hierSplittingUV}.
\end{proof}
\begin{proposition}[from univariate to multivariate splitting]
  \label{prop:splittingUVToMV}
  If univariate splitting \eqref{eq:hierSplittingUV} holds for every dimension,
  then the multivariate splitting \eqref{eq:hierSplittingMV} holds as well.
\end{proposition}
\begin{proof}
  We check the two conditions of \cref{lemma:hierSplittingMV}
  given the two univariate conditions of \cref{lemma:hierSplittingUV}:
  \begin{enumerate}
    \item
    The hierarchical basis functions $\basis{\ßl',\ßi'}$
    of $\hs{\ßl'}$ ($\ßl' \le \ßl$, $\ßi' \in \his{\ßl'}$) are tensor products
    of functions $\basis{l_t',i_t'}$.
    According to the first condition of \cref{lemma:hierSplittingUV},
    each $\basis{l_t',i_t'}$ can be written as a linear combination of
    the nodal basis $\basis{l_t,i_t}$ ($i_t = 0, \dotsc, 2^{l_t}$).
    We can expand the tensor product to a linear combination
    of tensor products of the univariate nodal basis functions.
    Therefore, $\basis{\ßl',\ßi'}$ is a linear combination of
    multivariate nodal functions, i.e., $\basis{\ßl',\ßi'} \in \ns{\ßl}$.
    As this is true for all $\ßi' \in \his{\ßl'}$, we obtain
    $\hs{\ßl'} \subset \ns{\ßl}$.
    
    \item
    The linear independence of the hierarchical functions $\basis{\ßl',\ßi'}$
    ($\ßl' \le \ßl$, $\ßi' \in \his{\ßl'}$) can be shown completely analogously
    to the proof of \thmref{lemma:tensorProductLinearIndependence}.\qedhere
  \end{enumerate}
\end{proof}
\begin{corollary}
  \label{cor:hierSplittingHat}
  The multivariate hierarchical splitting \eqref{eq:hierSplittingMV}
  holds for the hat function basis.
\end{corollary}
