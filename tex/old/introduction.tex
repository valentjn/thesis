
\chapter{Introduction}
\label{cha:introduction}

\begin{gather}
  X \times Y\\
  A \cdot \vec{x} = \vec{b}\\
  \min_{\vec{x} \in [0, 1]} \int_\Omega f(\vec{x}, \vec{y}) d\vec{y}
\end{gather}

% -------------------------------------------------------------------------------------

\chapter{Uncertainty quantification}
\label{sec:uncert-quant}

\begin{chapquote}{Lewis Carroll, \textit{Alice in Wonderland}}
  ``Begin at the beginning,'' the King said, gravely, ``and go on till you
  come to an end; then stop.''
\end{chapquote}

\begin{itemize}
\item quantities of interest
\item sources of uncertainty
\end{itemize}

\blindtext
\blindtext
\blindtext
\blindtext
\blindtext
\blindtext
\blindtext
\blindtext
\blindtext
\blindtext
\blindtext
\blindtext
\blindtext
\blindtext
\blindtext
\blindtext
% -------------------------------------------------------------------------------------

\chapter{Sparse Grids}
\label{cha:sparse-grids}

\begin{figure}
\centering
\begin{minipage}{0.48\textwidth}
\includegraphics[width=0.90\columnwidth]{bsplhier1}
\end{minipage}
\begin{minipage}{0.48\textwidth}
\includegraphics[width=0.90\columnwidth]{bsplhier2}
\end{minipage}
\caption{Subspace tableau and resulting sparse grid in 2D on level 3.}
\end{figure}

\cite{zenger91sparse}
\cite{zenger91sparse, Adams12Rigorous}

% -------------------------------------------------------------------------------------

\chapter{Polynomial chaos expansion}
\label{cha:polyn-chaos-expans}


% -------------------------------------------------------------------------------------

\chapter{Input modeling}
\label{cha:input-modeling}

\begin{itemize}
\item random variables
\end{itemize}

\blindtext
\blindtext
\blindtext
\blindtext
\blindtext
\blindtext
\blindtext
\blindtext
\blindtext
\blindtext
\blindtext
\blindtext
\blindtext
\blindtext
\blindtext
\blindtext

\section{Density Estimation}
\label{sec:}

\blindtext
\blindtext
\blindtext
\blindtext
\blindtext
\blindtext
\blindtext
\blindtext
\blindtext
\blindtext
\blindtext
\blindtext
\blindtext
\blindtext
\blindtext
\blindtext

\subsection{Kernel density estimators}
\label{sec:kern-dens-estim}

\blindtext
\blindtext
\blindtext
\blindtext
\blindtext
\blindtext
\blindtext
\blindtext
\blindtext
\blindtext
\blindtext
\blindtext
\blindtext
\blindtext
\blindtext
\blindtext


\begin{itemize}
\item general approach, motivation via smoothing histograms
\item special case: Gaussian kernels
\end{itemize}

\begin{equation}
  p(x) = \frac{1}{n} \sum_{i = 0}^n K_i(x)
\end{equation}

\subsection{Sparse Grid Density Estimation}
\label{sec:sparse-grid-density}

\begin{itemize}
\item general approach
\item \todo{positivity}
\end{itemize}

\section{On probabilistic transformations}
\label{sec:on-prob-transformations}

\begin{itemize}
\item Nataf transformation
  \begin{itemize}
  \item numerical issues with cdf/ppf
  \item computation of corrected correlation matrix via
    Gauss-quadrature and bisection
  \item equal to Rosenblatt for correlated RVs
  \item density estimation needed just for the marginals
  \end{itemize}
\item Rosenblatt transformation
  \begin{itemize}
  \item general approach if cdf and conditional cdf is defined
  \item show Rosenblatt for KDE (refer to Sandia) and SGDE (refer to
    Benjamins work in the context of sampling)
  \end{itemize}
\end{itemize}

\section{Numerical Results}
\label{sec:im-numerical-results}

% -------------------------------------------------------------------------------------

\chapter{Uncertainty propagation}
\label{cha:uncert-prop}

\section{Problem formulation}
\label{sec:problem-formulation}

\section{Adaptive sparse grid collocation}
\label{sec:adaptive-sparse-grid}

\subsection{$hp$-refinement}
\label{sec:hp-refinement}

\begin{itemize}
\item piecewise polynomial basis
\item refinement criteria
\item \todo{adding grid points instead of refining}
\item \todo{sublinearity}
\end{itemize}

\subsection{Probabilistic analysis}
\label{sec:probabi-analysis}

\begin{itemize}
\item moment estimation: quadrature (analytic, monte carlo)
\item sobol indices: based on moment estimation
\item risk analysis: based on Monte Carlo quadrature
\end{itemize}

\section{Non-intrusive polynomial chaos expansion}
\label{sec:non-intr-polyn}

\begin{itemize}
\item definition
\item higher-
\end{itemize}

\subsection{Askey scheme}
\label{sec:askey-scheme}

\begin{itemize}
\item present case for independent marginals from a family of
  distributions
\end{itemize}

\subsection{Arbitrary expansion}
\label{sec:arbitrary-expansion}

\begin{itemize}
\item present case for arbitrary distributions
\item marginals: three-term-recurrence
\item orthogonalization: transformation matrix based on moments of
  distributions; implementation: Cholesky factor of Gramian matrix or,
  better, QR of Vandermonde like marix
\end{itemize}

\subsection{Sampling rules}
\label{sec:sampling-rules}

\begin{itemize}
\item probabilistic transformation of optimal samplings in independent
  case
\item Leja sequences
\item root search of orthogonal polynomial (out of scope)
\end{itemize}

\section{Numerical Results}
\label{sec:up-numerical-results}

% -------------------------------------------------------------------------------------

\chapter{Data assimilation}
\label{cha:data-assimilation}

\section{Markov-Chain Monte Carlo}
\label{sec:markov-chain-monte}


\section{Numerical Results}
\label{sec:da-numerical-results}


\begin{itemize}
\item Bayesian inverse problem: definition
\item application of leja + apce and sgde + asgc
\end{itemize}

% -------------------------------------------------------------------------------------

\chapter{Summary and Outlook}
\label{cha:summary-outlook}



%%% --------------------------------------------------------------------
%%% Local Variables:
%%% mode: latex
%%% TeX-master: "../thesis"
%%% End:
