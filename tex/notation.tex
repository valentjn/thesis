% black-board letters
% define \bbN and \NN (N = arbitrary letter)
\newcommand*{\defbb}[1]{%
  \newnamecommand{bb#1}{\mathbb{#1}}%
  \newnamecommand{#1#1}{\mathbb{#1}}%
}
%\defbb{N}
\defbb{R}
\defbb{Z}

\newnotationcommand{\NN}{\mathbb{N}}{N}{$\NN$}{
  Natural numbers without zero ($1, 2, 3, \dotsc$)
}
\newnotationcommand{\NNz}{\mathbb{N}_0}{N0}{$\NNz$}{
  Natural numbers with zero ($\NN \cup \{0\}$)
}

\newnotation{l}{$l$}{Level $\in \NN_0$}
\newnotation{i}{$i$}{Index $= 0, \dotsc, 2^l$}

\newnotationcommand[1]{\gp}{x_{#1}}{xli}{$x_{l,i}$}{
  Grid point $:= i \cdot h_l$
}

\newnotationcommand[1]{\vgp}{\ßx_{#1}}{xli!}{$\vgp{\ßl,\ßi}$}{
  Multivariate grid point $:= \ßi \cdot \ßh_\ßl$
}

% calligraphic letters
% define \calC (C = arbitrary letter)
\newcommand*{\defcal}[1]{\newnamecommand{cal#1}{\mathcal{#1}}}
\defcal{C}
\defcal{O}

% defined terms
\newcommand*{\term}[1]{\emph{#1}}

% mathematic operators
\DeclareMathOperator{\intsupp}{\interior{\supp}}
\makecommandnotation{\intsupp}{supp!}{$\intsupp$}{
  Bla bla blubb
}
\DeclareMathOperator{\spn}{span}
\DeclareMathOperator{\supp}{supp}
\DeclareMathOperator{\xor}{xor}

% subsets
\renewcommand*{\subset}{\subseteq}

% differential for integral/derivatives
\newcommand*{\diff}{\mathop{}\!\mathrm{d}}
\newcommand*{\dx}{\diff{}x}
\newcommand*{\partialdiff}{\mathop{}\!\partial}

% vectors
\renewcommand*{\vec}[1]{{\boldsymbol{#1}}}
\newcommand*{\ß}[1]{\vec{#1}}
\newcommand*{\veclog}{\mathop{\vec{\log}}}
\newcommand*{\vecmax}{\mathop{\vec{\max}}}

% quantors
\newcommand*{\fa}[2]{\forall_{#1}\;\;#2}
\newcommand*{\ex}[2]{\exists_{#1}\;\;#2}

% norm
\newcommand*{\norm}[1]{\lVert{#1}\rVert}
\newcommand*{\bignorm}[1]{\left\lVert{#1}\right\rVert}

% superscripts
\newcommand*{\clenshawcurtis}{\mathrm{cc}}
\newcommand*{\modified}{\mathrm{mod}}
\newcommand*{\ntrl}{\mathrm{nat}}
\newcommand*{\notaknot}{\mathrm{nak}}
\newcommand*{\sparse}{\mathrm{s}}

% disjoint union
\newcommand*{\dotcup}{\mathbin{\dot{\cup}}}
\DeclareMathOperator*{\bigdotcup}{\dot{\bigcup}}

% interior and boundary
\newcommand*{\interior}[1]{\mathring{#1}}
\newcommand*{\bndry}[1]{\mathop{}\!\partial#1}

% open interval
\newcommand*{\openinterval}[1]{\mathopen]#1\mathclose[}
\newcommand*{\halfopeninterval}[1]{\mathopen[#1\mathclose[}
\newcommand*{\openintervalscaled}[1]{\left]#1\right[}
