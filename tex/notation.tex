% black-board letters
% define \bbN and \NN (N = arbitrary letter)
\newcommand*{\defbb}[1]{%
  \newnamecommand{bb#1}{\mathbb{#1}}%
  \newnamecommand{#1#1}{\mathbb{#1}}%
}
%\defbb{N}
\defbb{R}
\defbb{Z}

\newnotationcommand{\NN}{\mathbb{N}}{N}{$\NN$}{
  Natural numbers without zero ($1, 2, 3, \dotsc$)
}

\newnotationcommand{\NNz}{\mathbb{N}_0}{N0}{$\NNz$}{
  Natural numbers with zero ($\NN \cup \{0\}$)
}

\newnotation{l}{$l$}{
  Level $\in \NN_0$
}

\newnotation{l!}{$\ßl$}{
  Multivariate level $\in \NNz^d$
}

\newnotation{i}{$i$}{
  Index $= 0, \dotsc, 2^l$
}

\newnotation{i!}{$\ßi$}{
  Multivariate index $= \ß0, \dotsc, \ß2^\ßl$
}

\newnotation{i02l!}{$\ßi = \ß0, \dotsc, \ß2^\ßl$}{
  For all $\ßi$ with $0 \le i_t \le 2^{l_t}$ for all $t = 1, \dotsc, d$
}

\newnotation{d}{$d$}{
  Dimensionality $\in \NN$
}

\newnotation{0!}{$\ß0$}{
  $(0, \dotsc, 0) \in \NNz^d$
}

\newnotation{1!}{$\ß1$}{
  $(1, \dotsc, 1) \in \NN^d$
}

\newnotationcommand[1]{\gp}{x_{#1}}{xli}{$\gp{l,i}$}{
  Grid point $:= i \cdot \ms{l}$
}

\newnotationcommand[1]{\vgp}{\ßx_{#1}}{xli!}{$\vgp{\ßl,\ßi}$}{
  Multivariate grid point $:= \ßi \cdot \vms{ßl}$
}

\newnotationcommand[1]{\ms}{h_{#1}}{hl}{$\ms{l}$}{
  Mesh size $:= 2^{-l}$
}

\newnotationcommand[1]{\vms}{\ßh_{#1}}{hl!}{$\vms{ßl}$}{
  Multivariate mesh size $:= \ß2^{-\ßl}$
}

\newnotationcommand[1]{\basis}{\varphi_{#1}}{phili}{$\basis{l,i}$}{
  Hierarchical basis function of level $l$, index $i$
}

\newnotation{phili!}{$\basis{\ßl,\ßi}$}{
  Multivariate hierarchical basis function of level $\ßl$, index $\ßi$
}

\newnotation{.1}{$\cdot^1$}{
  Superscript for ``Piecewise linear
  (basis function/function space/interpolant)''
}

\newnotation{.p}{$\cdot^p$}{
  Superscript for ``B-splines of degree $p$
  (basis function/function space/interpolant)''
}

\newnotationcommand[1]{\ns}{V_{#1}}{Vl}{$\ns{l}$}{
  Nodal space of level $l$
}

\newnotation{Vl!}{$\ns{\ßl}$}{
  Multivariate nodal space of level $\ßl$
}

\newnotationcommand[1]{\fgintp}{f_{#1}}{fl}{$\fgintp{l}$}{
  Full grid interpolant of $f$ in $\ns{l}$
}

\newnotation{fl!}{$\fgintp{\ßl}$}{
  Full grid interpolant of $f$ in $\ns{\ßl}$
}

\newnotationcommand[1]{\clint}{[#1]}{ab!c}{$\clint{\ßa, \ßb}$}{
  Closed hyper-rectangle
  $:= \clint{a_1, b_1} \times \dotsb \times \clint{a_d, b_d}$
  with $\clint{a_t, b_t} := \{x_t \in \RR \mid a_t \le x_t \le b_t\}$
}

\newcommand*{\clintscaled}[1]{\left[#1\right]}

\newnotationcommand[1]{\normone}{\norm{#1}_1}{||.||1}{$\normone{\cdot}$}{
  $\ell_1$ norm $\normone{\ßx} := \sum_{t=1}^d x_t$
}

%\newgsymbol{ci}{$c_i$}{Coefficients of a linear combination}%
%\newgsymbol{ci!}{$c_\ßi$}{Coefficients of a linear combination}%

% calligraphic letters
% define \calC (C = arbitrary letter)
\newcommand*{\defcal}[1]{\newnamecommand{cal#1}{\mathcal{#1}}}
\defcal{C}
\defcal{O}

% defined terms
\newcommand*{\term}[1]{\emph{#1}}

% mathematic operators
\DeclareMathOperator{\intsupp}{\interior{\supp}}
\makecommandnotation{\intsupp}{supp!}{$\intsupp$}{
  Bla bla blubb
}

\DeclareMathOperator{\spn}{span}
\makecommandnotation{\spn}{span}{$\spn$}{
  Linear span (set of all linear combinations)
}

\DeclareMathOperator{\supp}{supp}

\DeclareMathOperator{\xor}{xor}

% subsets
\renewcommand*{\subset}{\subseteq}

% differential for integral/derivatives
\newcommand*{\diff}{\mathop{}\!\mathrm{d}}
\newcommand*{\dx}{\diff{}x}
\newcommand*{\partialdiff}{\mathop{}\!\partial}

% vectors
\renewcommand*{\vec}[1]{{\boldsymbol{#1}}}
\newcommand*{\ß}[1]{\vec{#1}}
\newcommand*{\veclog}{\mathop{\vec{\log}}}
\newcommand*{\vecmax}{\mathop{\vec{\max}}}

% quantors
\newcommand*{\fa}[2]{\forall_{#1}\;\;#2}
\newcommand*{\ex}[2]{\exists_{#1}\;\;#2}

% norm
\newcommand*{\norm}[1]{\lVert{#1}\rVert}
\newcommand*{\bignorm}[1]{\left\lVert{#1}\right\rVert}

% superscripts
\newcommand*{\clenshawcurtis}{\mathrm{cc}}
\newcommand*{\modified}{\mathrm{mod}}
\newcommand*{\ntrl}{\mathrm{nat}}
\newcommand*{\notaknot}{\mathrm{nak}}
\newcommand*{\sparse}{\mathrm{s}}

% disjoint union
\newcommand*{\dotcup}{\mathbin{\dot{\cup}}}
\DeclareMathOperator*{\bigdotcup}{\dot{\bigcup}}

% interior and boundary
\newcommand*{\interior}[1]{\mathring{#1}}
\newcommand*{\bndry}[1]{\mathop{}\!\partial#1}

% open interval
\newcommand*{\openinterval}[1]{\mathopen]#1\mathclose[}
\newcommand*{\halfopeninterval}[1]{\mathopen[#1\mathclose[}
\newcommand*{\openintervalscaled}[1]{\left]#1\right[}
