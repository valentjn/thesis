\section{Homogenization and the Two-Scale Approach}
\label{sec:61homogenization}

\paragraph{Density function}

We roughly follow the presentation given in
\multicite{%
  Huebner14Mehrdimensionale,%
  Valentin14Hierarchische,%
  Valentin16Hierarchical%
}.
Let $\domain \subset \real^{\dimdomain}$ be the optimization domain
(which later contains the optimal shape) and
$\dens\colon \domain \to \clint{0, 1}$ a function,
which is called \term{density function}.
Usually, we assume $\dimdomain = 2$ or $\dimdomain = 3$,
although the method can be generalized to
arbitrary dimensionalities $\dimdomain \in \nat$.
The function values $\dens(\tilde{\*x})$ tell if $\tilde{\*x}$
is contained in the object (value of one) or not (value of zero).
The approach of \term{homogenization} now also allows values between
zero and one, giving the density of the material in that point.

\paragraph{Optimization of compliance values}

Furthermore, for each density function $\dens$,
let $\compliance(\dens)$ be an objective function value.
In our setting,
we exert one force $\force$ (or multiple forces),
measure the resulting deformation of the object, and
compute the \term{compliance} (i.e., the inverse of the stiffness) as
the objective function value $\compliance(\dens)$:
\begin{equation}
  \compliance(\dens)
  = \int_{\domain} \innerprod[2]{\force}{\displacement_{\dens}(\tilde{\*x})}
  \diff\tilde{\*x},
\end{equation}
where $\displacement_{\dens}\colon \domain \to \real^{\dimdomain}$
is the \term{displacement function}, depending on the density
\cite{Huebner14Mehrdimensionale}.
The problem of topology optimization is to find the density function
with minimal objective function value:
\begin{equation}
  \label{eq:topoOptProblemContinuous}
  \min_{\dens} \compliance(\dens).
\end{equation}
Often, there are trivial solutions, if we do not impose additional conditions.
For example, when minimizing compliance function values,
choosing $\dens :\equiv 1$ will most likely lead to the shape with the
highest stiffness and, thus, the smallest displacement and compliance value.
Therefore, we introduce the following volume constraint:
\begin{equation}
  \frac{\voldens{\dens}{\domain}}{\vol{\domain}} \le \densub,\quad
  \voldens{\dens}{\domain}
  := \int_{\domain} \dens(\tilde{\*x}) \diff\tilde{\*x},\quad
  \vol{\domain}
  := \voldens{1}{\domain},
\end{equation}
where $\vol{\domain} = \int_{\domain} 1 \diff\tilde{\*x}$
is the volume of the domain and
$\densub \in \clint{0, 1}$ is an upper bound on the volume fraction.

\paragraph{Discretization and two-scale approach}

Of course, we cannot solve the problem \eqref{eq:topoOptProblemContinuous}
directly on computers,
as there are infinitely many density functions $\dens$.
To be able to discretize the domain $\domain \subset \real^{\dimdomain}$
more easily, we assume that $\domain$ is some hyper-rectangle
$\clint{\tilde{a}_1, \tilde{b}_1} \times \dotsb \times
\clint{\tilde{a}_{\dimdomain}, \tilde{b}_{\dimdomain}}$;
if it is not, we can replace $\domain$ with its bounding box.
We can then split $\domain$ into $M_1 \times \dotsb \times M_{\dimdomain}$
equally sized and axis-aligned sub-hyper-rectangles,
which we call \term{macro-cells}.
In the \term{two-scale approach},
we assume the material of the macro-cells to be
repetitions of infinitesimally small periodic structures
(i.e., they are identical for each macro-cell),
called \term{micro-cells}.
These micro-cells have a specific shape, which is parametrized by $d$ so-called
\term{micro-cell parameters} $x_1, \dotsc, x_d$.
The parameters are assumed to be normalized to values in the
unit interval $\clint{0, 1}$.
For instance, in $\dimdomain = 2$ dimensions,
this shape could be an axis-aligned cross of two bars
with thicknesses $x_1$ and $x_2$.

\paragraph{Elasticity tensors}

Note that while the shape of all micro-cells in one macro-cell is identical,
the micro-cell parameters corresponding to different macro-cells will differ.
This enables varying densities in different regions of $\domain$.
We denote the parameters of the $j$-th macro-cell
with $\*x^{(j)} = (x_1^{(j)}, \dotsc, x_d^{(j)}) \in \clint{0, 1}^d$,
where $j = 1, \dotsc, N$ and
$M := M_1 \dotsm M_{\dimdomain}$ is the number of macro-cells.
With linear elasticity,
one can compute so-called \term{elasticity tensors} $\etensor_j$,
which can be written as
symmetric $(3 \times 3)$ or $(6 \times 6)$ matrices
(for $\dimdomain = 2$ or $\dimdomain = 3$ dimensions, respectively).%
\footnote{%
  In general, the elasticity tensor is a fourth-order tensor in
  $\real^{\dimdomain \times \dimdomain \times \dimdomain \times \dimdomain}$.
  One can reduce the size of the tensor by exploiting various symmetries
  \cite{Huebner14Mehrdimensionale}
  to obtain $6$ or $21$ stiffness coefficients.%
}
The elasticity tensors encode information about the material properties
of the different macro-cells.
They are usually computed as the solution of a \fem problem,
the \term{micro-problem}.
Once all elasticity tensors $\etensor_j$ are known,
we can compute the compliance value corresponding to the macro-shape
by solving another \fem problem, the \term{macro-problem}.

The new optimization problem following from the
two-scale discretization process has the form
\begin{equation}
  \min J(\*x^{(1)}, \dotsc, \*x^{(M)}),\quad
  \*x^{(1)}, \dotsc, \*x^{(M)} \in \clint{0, 1}^d
  \quad\text{s.t.}\quad
  \frac{1}{M} \sum_{j=1}^M \volcell{\*x^{(j)}}
  \le \volfracub,
\end{equation}
where $\volcell{\*x^{(j)}}$ is the fraction of the material volume of a
micro-cell of the $j$-th macro-cell with micro-cell parameter $\*x^{(j)}$.

\blindtext{}











































