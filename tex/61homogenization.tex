\section{Homogenization and the Two-Scale Approach}
\label{sec:61homogenization}

We roughly follow the presentation given in
\multicite{%
  Huebner14Mehrdimensionale,%
  Valentin14Hierarchische,%
  Valentin16Hierarchical%
}.
Let $\domain \subset \real^\dimdomain$ be the optimization domain
(which later contains the optimal shape) and
$\densfcn\colon \domain \to \clint{0, 1}$ a function,
which is called \term{density function}.
Usually, we assume $\dimdomain = 2$ or $\dimdomain = 3$,
although the method can be generalized to
arbitrary dimensionalities $\dimdomain \in \nat$.
The function values $\densfcn(\tilde{\*x})$ tell if $\tilde{\*x}$
is contained in the object (value of one) or not (value of zero).
The approach of \term{homogenization} now also allows values between
zero and one, giving the density of the material in that point.

Furthermore, for each density function $\densfcn$,
let $\compliance(\densfcn)$ be an objective function value.
In our setting,
we exert one force or multiple forces on specific points,
measure the resulting deformation of the object, and
compute the \term{compliance} (i.e., the inverse of the stiffness) as
the objective function value $\compliance(\densfcn)$.
The problem of topology optimization is to find the density function
with minimal objective function value:
\begin{equation}
  \label{eq:topoOptProblemContinuous}
  \min_{\densfcn} \compliance(\densfcn).
\end{equation}
Often, there are trivial solutions, if we do not impose additional conditions.
For example, when minimizing compliance function values,
choosing $\densfcn :\equiv 1$ will most likely lead to the shape with the
highest stiffness.
Therefore, we introduce the following volume constraint:
\begin{equation}
  \frac{\voldens{\densfcn}{\domain}}{\vol{\domain}} \le \volfracub,\quad
  \voldens{\densfcn}{\domain}
  := \int_{\domain} \densfcn(\tilde{\*x}) \diff\tilde{\*x},\quad
  \vol{\domain}
  := \voldens{1}{\domain},
\end{equation}
where $\vol{\domain} = \int_{\domain} 1 \diff\tilde{\*x}$
is the volume of the domain and
$\volfracub \in \clint{0, 1}$ is an upper bound on the volume fraction.

Of course, we cannot solve the problem \eqref{eq:topoOptProblemContinuous}
directly on computers,
as there are infinitely many density functions $\densfcn$.
To be able to discretize the domain $\domain \subset \real^\dimdomain$
more easily, we assume that $\domain$ is some hyper-rectangle
$\clint{\tilde{a}_1, \tilde{b}_1} \times \dotsb \times
\clint{\tilde{a}_\dimdomain, \tilde{b}_\dimdomain}$;
if it is not, we can replace $\domain$ with its bounding box.
We can then split $\domain$ into $N_1 \times \dotsb \times N_\dimdomain$
equally sized and axis-aligned sub-hyper-rectangles,
which we call \term{macro-cells}.
In the \term{two-scale approach},
we assume the material of the macro-cells to be
repetitions of infinitesimally small periodic structures
(i.e., they are identical for each macro-cell),
called \term{micro-cells}.
These micro-cells have a specific shape, which is parametrized by $d$ so-called
\term{micro-cell parameters} $x_1, \dotsc, x_d$.
The parameters are assumed to be normalized to values in the
unit interval $\clint{0, 1}$.
For instance, in $\dimdomain = 2$ dimensions,
this shape could be an axis-aligned cross of two bars
with thicknesses $x_1$ and $x_2$.

Note that while the shape of all micro-cells in one macro-cell is identical,
the micro-cell parameters corresponding to different macro-cells will differ.
This enables varying densities in different regions of $\domain$.
We denote the parameters of the $j$-th macro-cell
with $x_1^{(j)}, \dotsc, x_d^{(j)} \in \clint{0, 1}$
($j = 1, \dotsc, N_1 \dotsm N_\dimdomain$).

\blindtext{}











































