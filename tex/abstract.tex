\addchap{Abstract/\foreignlanguage{ngerman}{Kurzzusammenfassung}}

\printornamentsfalse
\section*{Abstract}
\printornamentstrue

In simulation technology, computationally expensive objective functions
are often replaced by cheap surrogates,
which can be obtained by interpolation.
Full grid interpolation methods suffer from the
so-called curse of dimensionality,
rendering them infeasible if the parameter domain of the function
is higher-dimensional (four or more parameters).
Sparse grids constitute a discretization method that does not suffer from the
curse, while the approximation quality deteriorates only insignificantly.
However, conventional basis functions such as piecewise linear functions
are not smooth (continuously differentiable).
Hence, these basis functions are unsuitable for applications
in which gradients are required.
One example for such an application is gradient-based optimization,
in which the availability of gradients greatly improves the speed of
convergence and the accuracy of results.
This thesis demonstrates that hierarchical B-splines on sparse grids are
well-fitted for obtaining smooth interpolants for higher dimensionalities.
The thesis is organized in two main parts:
The first part constructs new B-spline bases on sparse grids and studies
their implications on theory and algorithms.
The second part considers three real-world applications in optimization:
topology optimization, biomechanical continuum-mechanics, and
dynamic portfolio choice models in finance.
As the results show, the optimization problems of these applications
can be solved accurately and efficiently with hierarchical B-splines on
sparse grids.
% 205 words

\pagebreak

\begin{otherlanguage}{ngerman}
  \section*{Kurzzusammenfassung}
  
  In der Simulationstechnologie werden zeitaufwendige Zielfunktionen
  oft durch einfache Surrogate ersetzt, die durch Interpolation
  gewonnen werden können.
  Vollgitter-Interpolationsmethoden leiden unter dem
  sogenannten Fluch der Dimensionalität,
  der sie unbrauchbar macht, falls der Parameterbereich der Funktion
  höherdimensional ist (vier oder mehr Parameter).
  Dünne Gitter sind eine Diskretisierungsmethode, die nicht unter
  dem Fluch leidet, aber die Approximationsqualität nur leicht verschlechtert.
  Leider sind konventionelle Basisfunktionen wie die stückweise
  lineare Funktionen nicht glatt (stetig differenzierbar).
  Daher sind sie für Anwendungen unbrauchbar, in denen Gradienten
  benötigt werden.
  Ein Beispiel für eine solche Anwendung ist gradientenbasierte Optimierung,
  in der die Verfügbarkeit von Gradienten die Konvergenzgeschwindigkeit und
  die Ergebnisgenauigkeit deutlich verbessert.
  Diese Dissertation demonstriert, dass hierarchische B-Splines auf
  dünnen Gittern gut geeignet sind,
  um glatte Interpolierende für höhere Dimensionalitäten zu erhalten.
  Die Dissertation ist in zwei Hauptbereiche gegliedert:
  Der erste Teil konstruiert neue B-Spline-Basen auf dünnen Gittern und
  untersucht ihre Implikationen bezüglich Theorie und Algorithmen.
  Der zweite Teil behandelt drei Realwelt-Anwendungen:
  Topologieoptimierung, biomechanische Kontinuumsmechanik und
  Modelle der dynamischen Portfolio-Wahl in der Finanzmathematik.
  Die Ergebnisse zeigen, dass die Optimierungsprobleme dieser
  Anwendungen durch hierarchische B-Splines auf dünnen Gittern
  genau und effizient gelöst werden können.
  % 184 Wörter
\end{otherlanguage}

\cleardoublepage
