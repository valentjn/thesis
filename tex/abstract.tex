\addchap{Abstract/\foreignlanguage{ngerman}{Kurzzusammenfassung}}

\section*{Abstract}

In simulation technology, computationally expensive objective
functions are often replaced
by cheap surrogates, which can be obtained by interpolation.
Full-grid interpolation methods are affected by the curse of dimensionality,
rendering them infeasible if the function has four or more parameters.
Sparse grids are an interpolation method that does not suffer from the
curse while not much deteriorating the approximation quality.
However, conventional basis functions such as piecewise linear functions
are not smooth (continuously differentiable).
Thus, they are unsuitable for applications in which gradients are required,
for example, gradient-based optimization.
This thesis shows that hierarchical B-splines are well-fitted for obtaining
smooth interpolants in higher dimensionalities.
The thesis considers numerical and algorithmic aspects of the new basis
as well as three different real-world applications in optimization:
topology optimization, biomechanical continuum-mechanics, and
dynamic portfolio choice models in finance.
The results show that the optimization problems in these applications
can be solved accurately and efficiently with hierarchical B-splines on sparse grids.

\begin{otherlanguage}{ngerman}
  \section*{Kurzzusammenfassung}
  
  In der Simulationstechnologie werden zeitaufwendige Zielfunktionen
  oft durch einfache Surrogate ersetzt, die durch Interpolation
  gewonnen werden können.
  Vollgitter-Interpolationsmethoden sind vom Fluch der
  Dimensionalität betroffen, der sie für Funktionen mit vier oder mehr
  Parametern unbrauchbar macht.
  Dünne Gitter sind eine Interpolationsmethode, die nicht unter
  dem Fluch leidet, aber den Approximationsfehler nicht stark verschlechtert.
  Leider sind konventionelle Basisfunktionen wie die stückweise
  lineare Funktionen nicht glatt (stetig differenzierbar).
  Daher sind sie für Anwendungen unbrauchbar, in denen Gradienten
  benötigt werden, zum Beispiel gradientenbasierte Optimierung.
  Diese Dissertation zeigt, dass hierarchische B-Splines gut
  geeignet sind, um glatte Interpolierende für höhere
  Dimensionalitäten zu erhalten.
  Die Dissertation behandelt sowohl numerische und algorithmische
  Aspekte der neuen Basis als auch drei verschiedene
  Realwelt-Anwendungen:
  Topologieoptimierung, biomechanische Kontinuumsmechanik und
  Modelle der dynamischen Portfolio-Wahl in der Finanzmathematik.
  Die Ergebnisse zeigen, dass die Optimierungsprobleme in diesen
  Anwendungen durch hierarchische B-Splines auf dünnen Gittern
  gut und effizient gelöst werden können.
\end{otherlanguage}
