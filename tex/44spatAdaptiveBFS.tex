\longsection{%
  Hierarchization on Spatially Adaptive Sparse Grids with
  Breadth-First Search%
}{%
  Hierarchization with Breadth-First Search%
}
\label{sec:44spatAdaptiveBFS}

Unfortunately, we cannot apply the algorithms presented in the last
sections for spatially adaptive sparse grids with
hierarchical B-splines.
The reason is that the algorithms relied on the final interpolant $\sgintp$
being a linear combination of full grid solutions $\fgintp{\*l}$,
which is only possible for dimensionally adaptive sparse grids.
Consequently, the problem of hierarchization becomes significantly
harder if we operate on spatially adaptive sparse grids.
An exception is the case of piecewise linear basis functions ($p = 1$),
where we are still able to apply the \up,
as we will show in \cref{sec:45spatAdaptiveUP}.

In this section, we study one approach to hierarchize on
spatially adaptive sparse grids,
namely transforming the hierarchical basis to so-called fundamental bases
to enable a \bfs algorithm for hierarchization.
Another approach, which is based on the \up, will be shown in the next section.

The approach in this section has already been published as
\cite{Valentin18Fundamental}.
Again, please note that while B-splines are our target application,
the considerations in this chapter are fully independent on the choice of
basis functions $\basis{\*l,\*i}$ 
(as long as they have tensor product structure).
Although we do not state it explicitly, it is possible to employ
different types of basis functions $\basis{l_t,i_t}$ in the different
dimensions, e.g., B-splines of different degrees to enable $p$-adaptivity.

\subsection{Hierarchization with Breadth-First Search on Fundamental Bases}
\label{sec:441BFSFundamentalBases}

\paragraph{Fundamental property}

As we have already discussed in \cref{sec:41problem},
the main cause of the difficulty of hierarchizing with B-splines
$\bspl{\*l}{\*i}{}$ is their overlapping support,
which they need for their approximation order.
In the univariate case,
the idea is to transform the B-spline basis to obtain
new basis functions $\fundbasis{l',i'}\colon \clint{0, 1} \to \real$
($l' \in \natz$, $i' \in \hiset{l'}$) which satisfy
\begin{subequations}
  \label{eq:fundamentalProperty}
  \begin{alignat}{3}
    \label{eq:fundamentalProperty1}
    \fundbasis{l',i'}(\gp{l,i})
    &= 0,\qquad
    &&l < l',\quad
    &&i \in \hiset{l}\\
    \label{eq:fundamentalProperty2}
    \fundbasis{l',i'}(\gp{l',i})
    &= \kronecker{i}{i'},\qquad
    &&&&i \in \hiset{l'}.
  \end{alignat}
\end{subequations}
We call \eqref{eq:fundamentalProperty} \term{fundamental property}
and functions $\fundbasis{l',i'}$
that fulfill this property \term{fundamental basis functions}.
\todo{check ``that'' comma}
The first \cref{eq:fundamentalProperty1} ensures that
basis functions of level $l'$ vanish at
grid points of coarser levels $l < l'$.
The second \cref{eq:fundamentalProperty2} requires the
basis functions $\fundbasis{l',i'}$
to additionally vanish at all grid points of the same level $l'$
with different index $i \not= i'$.
An example for fundamental basis functions are
the piecewise linear B-splines $\bspl{l,i}{1}{}$.

The statement that $\fundbasis{l',i'}(\gp{l',i'})$ should equal $1$
is not an additional restriction:
The value $\fundbasis{l',i'}(\gp{l',i'})$ must be non-zero;
otherwise, the interpolation problem \eqref{eq:interpFullGridUV}
of the nodal functions $\fundbasis{l',i'}$ ($i' = 0, \dotsc, 2^{l'}$)
is ill-posed, as the interpolation coefficients are not uniquely
determined anymore.
If $\fundbasis{l',i'}(\gp{l',i'})$ cannot vanish,
then we can just replace $\fundbasis{l',i'}$ with
$\fundbasis{l',i'}/\fundbasis{l',i'}(\gp{l',i'})$ to obtain
$\fundbasis{l',i'}(\gp{l',i'}) = 1$.

\paragraph{Triangular interpolation matrix}

The main motivation for enforcing the fundamental property
is the fact it results in the univariate hierarchization matrix $\intpmat$
being triangular, if rows and columns are arranged
according to monotonously increasing levels.
To be more precise,
let $k = k(l, i)$ an continuously enumerated index of
the level-index pairs $(l, i)$,
such that
\begin{equation}
  k(l, i) \le k(l', i') \implies l \le l',\quad
  (l, i), (l', i') \in K,
\end{equation}
where $\liset$ is the level-index set of the
spatially adaptive sparse grid at hand
(see \cref{sec:233spatiallyAdaptiveSG}).
If $k = k(k, i)$ and $k' = k(l', i')$, then
the $(k, k')$-th entry of $\intpmat$ vanishes
for $k < k'$:
\begin{equation}
  \label{eq:fundamentalTriangularUV}
  \intpmatentry{k}{k'}
  = \fundbasis{k'}(\gp{k})
  = 0,\quad
  k < k',
\end{equation}
as $k < k'$ implies that either
$l < l'$ (apply \eqref{eq:fundamentalProperty1})
or $l = l'$, but $i \not= i'$ (apply \eqref{eq:fundamentalProperty2}).
Consequently, $\intpmat$ is in lower triangular form
with diagonal entries
$\intpmatentry{k}{k} = \fundbasis{k}(\gp{k}) = 1$
according to \eqref{eq:fundamentalProperty2}.
This means that the hierarchization system \eqref{eq:hierarchizationSLE}
can easily be solved via forward substitution.

\paragraph{Multivariate case}

For the multivariate case of $d \in \nat$ dimensions,
we define as usual tensor product versions
$\fundbasis{\*l',\*i'}$ of the univariate fundamental bases
$\fundbasis{l'_t,i'_t}$ ($t = 1, \dotsc, d$).
\Cref{eq:fundamentalProperty} then implies
\begin{equation}
  \label{eq:fundamentalPropertyImplicationMV}
  \fundbasis{\*l',\*i'}(\gp{\*l,\*i}) \not= 0
  \implies
  \falarge{t = 1, \dotsc, d}{
    \left[(l'_t < l_t) \lor ((l'_t, i'_t) = (l_t, i_t))\right]
  },\quad
  (\*l, \*i), (\*l',\*i') \in \liset.
  \hspace*{-1mm}
\end{equation}
This means that every basis function
$\fundbasis{\*l',\*i'}$ can only be non-zero
at the grid points $\gp{\*l,\*i}$ that, in every dimension $t$,
are in a strictly finer level $l_t$ or
have the same level-pair $(l_t, i_t)$ as the basis function.

We can generalize the relation \eqref{eq:fundamentalTriangularUV} about the
triangular structure of the interpolation matrix to the multivariate case:
We assume that $k = k(\*l, \*i)$ is a single scalar and
continuously enumerated index of the level-index pairs $(\*l, \*i) \in K$
such that
\begin{equation}
  k(\*l, \*i) \le k(\*l', \*i') \implies \normone{\*l} \le \normone{\*l'},\quad
  (\*l, \*i), (\*l', \*i') \in K,
\end{equation}
i.e., we sort row indices $k = k(\*l, \*i)$ and
column indices $k' = k(\*l', \*i')$ of $\intpmat$
by level sum $\normone{\cdot}$.
As a consequence,
$\intpmat$ is in lower block-triangular form:
\begin{subequations}
  \label{eq:fundamentalTriangularMV}
  \begin{alignat}{2}
    \intpmatentry{k}{k'}
    &= \fundbasis{k'}(\vgp{k})
    = 0,\quad
    &\normone{\*l}
    &< \normone{\*l'},\\
    \intertext{%
      as $\normone{\*l} < \normone{\*l'} \implies \ex{t}{l_t < l'_t}$
      and using \eqref{eq:fundamentalProperty1}.
      Additionally, the diagonal blocks are unit matrices due to%
    }
    \intpmatentry{k}{k'}
    &= \fundbasis{k'}(\vgp{k})
    \overset{(\ast)}{=} \kronecker{(\*l,\*i)}{(\*l',\*i')},\quad
    &\normone{\*l}
    &= \normone{\*l'},
  \end{alignat}
\end{subequations}
since $\normone{\*l} = \normone{\*l'}$ implies that
either $\left[\ex{t = 1, \dotsc, d}{l_t < l'_t}\right]$ or $\*l = \*l'$.
In the former case, both sides of $(\ast)$ vanish
(according to \eqref{eq:fundamentalProperty1}),
and in the latter case, 
both sides equal $\kronecker{\*i}{\*i'}$
(according to \eqref{eq:fundamentalProperty2}).

\paragraph{Forward substitution}

The triangular structure of $\intpmat$ implies that
it is possible to determine the surpluses $\surplus{\*l,\*i}$
via forward substitution:

\begin{lemma}[forward substitution]
  \label{lemma:forwardSubstitution}
  The hierarchical surpluses $\surplus{\*l,\*i}$, which
  are determined by \eqref{eq:hierarchizationSLE}
  with respect to $\fundbasis{\*l,\*i}$, satisfy
  \begin{equation}
    \surplus{\*l,\*i}
    = \fcnval{\*l,\*i} -
    \sum_{\substack{(\*l',\*i')\in\liset\\\normone{\*l'} < \normone{\*l}}}
    \surplus{\*l',\*i'} \fundbasis{\*l',\*i'}(\gp{\*l,\*i}),\quad
    (\*l, \*i) \in \liset.
  \end{equation}
\end{lemma}

\begin{proof}
  The linear system \eqref{eq:hierarchizationSLE} is given by
  \begin{subequations}
    \begin{align}
      \fcnval{\*l,\*i}
      &= \sum_{(\*l',\*i') \in \liset}
      \surplus{\*l',\*i'} \fundbasis{\*l',\*i'}(\gp{\*l,\*i}),\quad
      (\*l, \*i) \in \liset.\\
      \intertext{%
        According to \eqref{eq:fundamentalTriangularMV},
        all summands with $\normone{\*l'} > \normone{\*l}$ vanish
        and from the summands with $\normone{\*l'} = \normone{\*l}$,
        only the summand $(\*l, \*i)$ remains
        with $\fundbasis{\*l,\*i}(\gp{\*l,\*i}) = 1$:%
      }
      &= \surplus{\*l,\*i} +
      \sum_{\substack{(\*l',\*i')\in\liset\\\normone{\*l'} < \normone{\*l}}}
      \surplus{\*l',\*i'} \fundbasis{\*l',\*i'}(\gp{\*l,\*i}),
    \end{align}
  \end{subequations}
  as desired.
\end{proof}

\paragraph{Breadth-first search}

Exploiting this lemma, we can formulate
a hierarchization algorithm (see \cref{alg:BFS})
that applies forward substitution by \bfs in the \dagr of
the spatially adaptive sparse grid $\sgset$.
The nodes of the \dagr are the level-index pairs $(\*l, \*i) \in \liset$.
An edge connects $(\*l, \*i)$ to $(\*l', \*i')$,
if $(\*l, \*i)$ is a direct ancestor of $(\*l', \*i')$, i.e., if
\begin{equation}
  \label{eq:directAncestor}
  \exlarge{t = 1, \dotsc, d}{
    l'_t = l_t + 1,\;
    i'_t \in
    \begin{cases}
      \{1\},&l_t = 0,\\
      \{2i_t - 1, 2i_t + 1\},&l_t > 0.
    \end{cases}
  }
\end{equation}

\begin{algorithm}
  \begin{algorithmic}[1]
    \Function{$\vlinout =$ breadthFirstSearch}{%
      $\liset$, $\vlinin$%
    }
      \State{$\vlinout \gets \vlinin$}
      \State{$K_\mathrm{p} \gets \{\*i \mid \*i \in \{0, 1\}^d\}$}
      \Comment{set of processed points}%
      \State{%
        $q \gets \text{FIFO queue with same contents as } K_\mathrm{p}$%
      }
      \Comment{points to be processed next}%
      \While{$q \not= \emptyset$}
        \State{$(\*l', \*i') \gets q.\texttt{Pop}()$}
        \Comment{obtain next point}%
        \For{%
          $\{(\*l, \*i) \in \liset \setminus \{(\*l', \*i')\} \mid
          \fa{t=1,\dotsc,d}{(l'_t < l_t) \lor ((l'_t, i'_t) = (l_t, i_t))}\}$%
        }
          \State{%
            $\linout{\*l,\*i} \gets \linout{\*l,\*i} -
            \linout{\*l',\*i'} \fundbasis{\*l',\*i'}(\gp{\*l,\*i})$%
          }
          \Comment{%
            update surpluses according to \cref{lemma:forwardSubstitution}%
          }%
        \EndFor{}
        \For{%
          $\{(\*l, \*i) \in \liset \setminus \liset_\mathrm{p} \mid
          \text{$(\*l, \*i)$ direct child of $(\*l', \*i')$}\}$%
        }
          \State{$q.\texttt{Push}((\*l, \*i))$}
          \Comment{add children to queue}%
          \State{%
            $\liset_\mathrm{p} \gets
            \liset_\mathrm{p} \cup \{(\*l, \*i)\}$%
          }
          \Comment{mark as processed}%
        \EndFor{}
      \EndWhile{}
    \EndFunction{}
  \end{algorithmic}
  \caption[%
    Hierarchization with breadth-first search (BFS)%
  ]{%
    Hierarchization with breadth-first search
    on spatially adaptive sparse grids with fundamental bases.
    Inputs are the set $\liset$ of level-index pairs of the
    sparse grid (see \eqref{eq:spatiallyAdaptiveSG}) and
    the vector $\vlinin = (\linin{\*l,\*i})_{(\*l,\*i) \in \liset}$
    of input data (function values $\fcnval{\*l,\*i}$ at the grid points),
    where $\liset$ is the set of all feasible level-index pairs $(\*l,\*i)$,
    i.e., $\normone{\*l} \le n$, $\*i \in \hiset{\*l}$.
    The output is the vector
    $\vlinout = (\linout{\*l,\*i})_{(\*l,\*i) \in \liset}$
    of output data (hierarchical surpluses $\surplus{\*l,\*i}$).%
    %
  }%
  \label{alg:BFS}%
\end{algorithm}

\paragraph{Correctness}

\blindtext{}

\paragraph{Complexity}

\blindtext{}



\subsection{Constructing Fundamental Bases}
\label{sec:442constructingFundamentalBases}

\blindtext{}

\paragraph{Hierarchical fundamental transformation}

\blindtext{}

\paragraph{Nodal fundamental transformation}

\blindtext{}

\paragraph{Translation-invariant fundamental transformation}

\blindtext{}



\subsection{Fundamental Splines}
\label{sec:443fundamentalSplines}

\blindtext{}

\paragraph{Definition}

\blindtext{}

\paragraph{Properties}

\blindtext{}

\paragraph{Modified fundamental splines}

\blindtext{}
