\setdictum{%
  Money. A social life. A shave.\\
  A Ph.D.\ student needs not such things.%
}{%
  Mike Slackenerny
  (PHD Comics\footnotemark)%
}

\longchapter{%
  Application 1: Topology Optimization%
}{%
  Application 1:\texorpdfstring{\\}{ }Topology Optimization%
}{%
  Application 1: Topology Optimization%
}
\footnotetext{\url{http://phdcomics.com/comics/archive.php?comicid=40}}

\initial{0em}{N}{ow, we want to investigate}
the first application, which is the context of topology optimization.
In engineering,
the area of topology optimization is a generalization of the
task of shape optimization, where the shape of a component has to be
determined such that some objective function value $\objfun(\*x)$ is optimal
(minimal or maximal).
For example, a bridge over a valley can be built in the shape of a
parabolic arc.
The task of shape optimization is then to choose the coefficients of the
parabola such that the bridge's stability is maximized,
possibly with the constraint that the volume occupied by the bridge
does not exceed a certain value (to save construction costs) or
that the size of the resulting passage meets some size requirements
(e.g., at least \SI{20}{\meter} wide and \SI{6}{\meter} tall).

However, the framework of topology optimization is more general as
it does not prescribe the topology
of the shapes in the search space.
In the example, it may well be that a viaduct-type bridge with three arcs
is more stable while occupying less volume.
We are not able to find such a bridge in the previous example,
as we have restricted the search space to single-arc bridges.
In topology optimization, the topology%
\footnote{%
  Two objects are considered ``topologically different'',
  if their number of ``holes'' (which are each called ``genus'') differ.
  This stems from the fact that in the field of mathematical topology,
  the genus of a topological space is a topological invariant, i.e.,
  the genus is invariant under homeomorphism.
  If the genera of two topological spaces differ, then they cannot be
  homeomorphic.%
}
is not fixed by the user
and chosen by the optimization algorithm (in a hopefully optimal way).

B-splines have already been used for
shape optimization \cite{Martin16Formoptimierung} and
topology optimization \multicite{Qian13Topology,Zhang17Topology}.
Sparse grids have been employed for
topology optimization \cite{Huebner14Mehrdimensionale} as well.
In this chapter, we want to combine these two numerical tools
to perform topology optimization using B-splines on sparse grids.
The two most common approaches for topology optimization are
the level-set method and the homogenization method.
The level-set method describes the boundary of the object
as the zero level set of a function (the level-set function)
and uses a \pde to iteratively transport this function and,
consequently, the object's boundary \cite{Allaire04Topology}.
We want to focus on the second method: the method of homogenization.

The chapter is structured as follows:
\Cref{sec:61homogenization} explains how the homogenization method works.
In \cref{sec:62tensors}, we discuss the details of applying B-splines on
sparse grids to this method.
We set up different micro-cell models and scenarios in \cref{sec:63models},
before reviewing numerical results in \cref{sec:64results}.
The results in this chapter have been obtained in collaboration with
Prof.\ Dr.\ Michael Stingl and Daniel Hübner
(both FAU Erlangen-Nürnberg, Germany).
\todo{mention paper if published}
The author of this thesis contributed the parts related to
interpolation and sparse grids, while the collaborators at FAU
studied the engineering and optimization parts of the joint project
(for example, they provided optimization scenarios and
assessed the quality of the results).

\section{Homogenization and the Two-Scale Approach}
\label{sec:61homogenization}

\paragraph{Density function}

We roughly follow the presentation given in
\multicite{%
  Huebner14Mehrdimensionale,%
  Valentin14Hierarchische,%
  Valentin16Hierarchical%
}.
Let $\domain \subset \real^{\dimdomain}$ be the optimization domain
(which later contains the optimal shape) and
$\densfcn\colon \domain \to \clint{0, 1}$ a function,
which is called \term{density function}.
Usually, we assume $\dimdomain = 2$ or $\dimdomain = 3$,
although the method can be generalized to
arbitrary dimensionalities $\dimdomain \in \nat$.
The function values $\densfcn(\tilde{\*x})$ tell if $\tilde{\*x}$
is contained in the object (value of one) or not (value of zero).
The approach of \term{homogenization} now also allows values between
zero and one, giving the density of the material in that point.

\paragraph{Optimization of compliance values}

Furthermore, for each density function $\densfcn$,
let $\compliance(\densfcn)$ be an objective function value.
In our setting,
we exert one force or multiple forces on specific points,
measure the resulting deformation of the object, and
compute the \term{compliance} (i.e., the inverse of the stiffness) as
the objective function value $\compliance(\densfcn)$.
The problem of topology optimization is to find the density function
with minimal objective function value:
\begin{equation}
  \label{eq:topoOptProblemContinuous}
  \min_{\densfcn} \compliance(\densfcn).
\end{equation}
Often, there are trivial solutions, if we do not impose additional conditions.
For example, when minimizing compliance function values,
choosing $\densfcn :\equiv 1$ will most likely lead to the shape with the
highest stiffness.
Therefore, we introduce the following volume constraint:
\begin{equation}
  \frac{\voldens{\densfcn}{\domain}}{\vol{\domain}} \le \volfracub,\quad
  \voldens{\densfcn}{\domain}
  := \int_{\domain} \densfcn(\tilde{\*x}) \diff\tilde{\*x},\quad
  \vol{\domain}
  := \voldens{1}{\domain},
\end{equation}
where $\vol{\domain} = \int_{\domain} 1 \diff\tilde{\*x}$
is the volume of the domain and
$\volfracub \in \clint{0, 1}$ is an upper bound on the volume fraction.

\paragraph{Discretization and two-scale approach}

Of course, we cannot solve the problem \eqref{eq:topoOptProblemContinuous}
directly on computers,
as there are infinitely many density functions $\densfcn$.
To be able to discretize the domain $\domain \subset \real^{\dimdomain}$
more easily, we assume that $\domain$ is some hyper-rectangle
$\clint{\tilde{a}_1, \tilde{b}_1} \times \dotsb \times
\clint{\tilde{a}_{\dimdomain}, \tilde{b}_{\dimdomain}}$;
if it is not, we can replace $\domain$ with its bounding box.
We can then split $\domain$ into $M_1 \times \dotsb \times M_{\dimdomain}$
equally sized and axis-aligned sub-hyper-rectangles,
which we call \term{macro-cells}.
In the \term{two-scale approach},
we assume the material of the macro-cells to be
repetitions of infinitesimally small periodic structures
(i.e., they are identical for each macro-cell),
called \term{micro-cells}.
These micro-cells have a specific shape, which is parametrized by $d$ so-called
\term{micro-cell parameters} $x_1, \dotsc, x_d$.
The parameters are assumed to be normalized to values in the
unit interval $\clint{0, 1}$.
For instance, in $\dimdomain = 2$ dimensions,
this shape could be an axis-aligned cross of two bars
with thicknesses $x_1$ and $x_2$.

\paragraph{Elasticity tensors}

Note that while the shape of all micro-cells in one macro-cell is identical,
the micro-cell parameters corresponding to different macro-cells will differ.
This enables varying densities in different regions of $\domain$.
We denote the parameters of the $j$-th macro-cell
with $\*x^{(j)} = (x_1^{(j)}, \dotsc, x_d^{(j)}) \in \clint{0, 1}^d$,
where $j = 1, \dotsc, N$ and
$M := M_1 \dotsm M_{\dimdomain}$ is the number of macro-cells.
With linear elasticity,
one can compute so-called \term{elasticity tensors} $\etensor_j$,
which can be written as
symmetric $3 \times 3$ or $6 \times 6$ matrices
(for $\dimdomain = 2$ or $3$ dimensions, respectively).%
\footnote{%
  In general, the elasticity tensor is a fourth-order tensor in
  $\real^{\dimdomain \times \dimdomain \times \dimdomain \times \dimdomain}$.
  One can reduce the size of the tensor by exploiting various symmetries
  \cite{Huebner14Mehrdimensionale}
  to obtain $6$ or $21$ stiffness coefficients.%
}
The elasticity tensors encode information about the material properties
of the different macro-cells.
They are usually computed as the solution of a \fem problem,
the \term{micro-problem}.
Once all elasticity tensors $\etensor_j$ are known,
we can compute the compliance value corresponding to the macro-shape
by solving another \fem problem, the \term{macro-problem}.

The new optimization problem following from the
two-scale discretization process has the form
\begin{equation}
  \min J(\*x^{(1)}, \dotsc, \*x^{(M)}),\quad
  \*x^{(1)}, \dotsc, \*x^{(M)} \in \clint{0, 1}^d
  \quad\text{s.t.}\quad
  \frac{1}{M} \sum_{j=1}^M \volcell{\*x^{(j)}}
  \le \volfracub,
\end{equation}
where $\volcell{\*x^{(j)}}$ is the fraction of the material volume of a
micro-cell of the $j$-th macro-cell with micro-cell parameter $\*x^{(j)}$.

\blindtext{}












































\section{Approximating Elasticity Tensors}
\label{sec:62tensors}

\blindtext{}

\section{Micro-Cell Models and Optimization Scenarios}
\label{sec:63models}

\minitoc{62mm}{3}

\noindent
In the following, we present the different micro-cell models
and optimization scenarios for which we perform numerical
experiments in the following section.



\subsection{Micro-Cell Models}
\label{sec:631models}

We use the various micro-cell models that are depicted in \cref{fig:microCell}.
%\todo{mention paper if published}
The models differ in the spatial dimensionality $\dimobjdomain$ (two or three)
and the number $d$ (two to five) of micro-cell parameters
$\*x \in \clint{\*0, \*1} = \clint{0, 1}^d$.
Note that the presented models are only some examples.
One can easily design complicated micro-cell models
with larger numbers of parameters.

\begin{figure}
  \subcaptionbox{%
    \lefthphantom{2D cross}{(2D-C, $d = 2$)}\\(2D-C, $d = 2$)%
    \label{fig:microCell_1}%
  }[31mm]{%
    \includegraphics{microCell_1}%
  }%
  \hfill%
  \subcaptionbox{%
    2D framed cross\\(2D-FC, $d = 4$)%
    \label{fig:microCell_2}%
  }[31mm]{%
    \includegraphics{microCell_3}%
  }%
  \hfill%
  \subcaptionbox{%
    2D sheared cross\\\rlap{\hspace*{11mm}{(2D-SC, $d = 3$)}}%
    \label{fig:microCell_3}%
  }[41mm]{%
    \includegraphics{microCell_2}%
  }%
  \hfill%
  \subcaptionbox{%
    2D sheared framed cross (2D-SFC, $d = 5$)%
    \label{fig:microCell_4}%
  }[37.5mm]{%
    \hspace*{-45mm}%
    \rlap{\includegraphics{microCell_4}}%
  }\\[2mm]%
  \subcaptionbox{%
    3D cross (3D-C, $d = 3$)%
    \label{fig:microCell_5}%
  }[44mm]{%
    \includegraphics{microCell_5}%
  }%
  \qquad%
  \subcaptionbox{%
    3D sheared cross (3D-SC, $d = 5$)%
    \label{fig:microCell_6}%
  }[52mm]{%
    \includegraphics{microCell_6}%
  }%
  \caption[Types of micro-cell models]{%
    Types of micro-cell models in two dimensions \emph{(top row)}
    and three dimensions \emph{(bottom row)}.%
  }%
  \label{fig:microCell}%
\end{figure}

\paragraph{Orthogonal (non-sheared) models in two dimensions}

The basic component of the four two-dimensional models
is a square with a cross (\cref{fig:microCell_1})
of two axis-aligned orthogonal bars,
whose widths are determined by two micro-cell parameters $x_1$ and $x_2$.
The micro-cell parameters are ratios of the bar widths
to the edge lengths of the micro-cell
(although the actual micro-cells are infinitesimally small).
This results in the \emph{cross model.}
For the \term{framed cross model} (\cref{fig:microCell_2}),
we add a diagonal cross with orthogonal bars
of widths $x_3$ and $x_4$ (horizontally measured).
To simplify the boundary treatment,
we shift the contents of the framed cross micro-cell by
\SI{50}{\percent} of the micro-cell's edge lengths in both directions,
such that previous corners of the micro-cell correspond to the new center.
%This is possible because the micro-cells are periodic.

\paragraph{Sheared models in two dimensions}

Both of these models can be extended by shearing.
The idea is to increase the stability of the resulting macro-structure
with respect to forces that act at angles other than
\ang{0} and \ang{90} (cross model) or
\ang{0}, \ang{90}, and \ang{45} (framed cross model).
If we just rotated the crosses in the micro-cells,
then the micro-structure would not be periodic.
Instead, we shear the whole micro-cell in the horizontal direction,
where the shearing angle $\theta$ is an additional micro-cell parameter,
which gives us another degree of freedom.%
\footnote{%
  To be more precise, the angle $\theta$ corresponds to an
  additional micro-cell parameter $x_3$ (sheared cross) or
  $x_5$ (sheared framed cross) that is determined by normalization
  from $\clint{-0.35\pi, 0.35\pi}$, i.e., $\theta/(0.7\pi) + 1/2$.%
}
This results in the \term{sheared cross model} (\cref{fig:microCell_3})
and \term{sheared framed crossmodel } (\cref{fig:microCell_4})
with three and five micro-parameters each.

\paragraph{Models in three dimensions}

The two-dimensional cross model can be transferred to three
spatial dimensions by just adding another bar in the new dimension.
Each of the three bars has square cross-section with given edge lengths
$x_1$, $x_2$, or $x_3$, respectively,
resulting in the \term{3D cross model} with three micro-cell parameters
(\cref{fig:microCell_5}).
By shearing in the two horizontal directions,
we obtain two new degrees of freedom $\theta_1$ and $\theta_2$
(shearing angles).
The emerging \term{3D sheared cross model} has five micro-cell parameters
(\cref{fig:microCell_6}).



\subsection{Test Scenarios}
\label{sec:632scenarios}

%\paragraph{Scenario settings}

For test scenarios to benchmark the performance of the new method,
we take a subset of the scenarios given in \cite{Valdez17Topology},
which reviews more than 100 papers on topology optimization
to determine the most common test scenarios in the field.
The geometry and the boundary conditions of the
four scenarios (two for each 2D and 3D)
are given in \cref{fig:topoOptScenario} (dimensions in meters).
In contrast to \cite{Valdez17Topology},
we only use single-point loads
(i.e., not loads applied to some line segment)
for implementational reasons.
The upper bound on the density (see \cref{sec:611homogenization})
is $\densub = \SI{50}{\percent}$ for the 2D scenarios and
$\densub = \SI{10}{\percent}$ for the 3D scenarios.
As in \cite{Sigmund01Line} and for simplicity reasons,
we apply a force $\force$ with unit value
(i.e., $\norm[2]{\force} = \SI{1}{\newton}$)
and we use a hypothetical material with
a Young modulus (stiffness) of \SI{1}{\pascal} and
a Poisson ratio (transversal expansion to axial compression) of $0.3$.

\begin{figure}
  \subcaptionbox{%
    2D cantilever%
    \label{fig:topoOptScenario_1}%
  }[72mm]{%
    \includegraphics{topoOptScenario2D_1}%
  }%
  \hfill%
  \smash{\raisebox{-22mm}{\subcaptionbox{%
    2D L-shape%
    \label{fig:topoOptScenario_2}%
  }[72mm]{%
    \includegraphics{topoOptScenario2D_2}%
  }}}%
  \\[7mm]%
  \subcaptionbox{%
    3D cantilever%
    \label{fig:topoOptScenario_3}%
  }[72mm]{%
    \includegraphics{topoOptScenario3D_1}%
  }%
  \hfill%
  \subcaptionbox{%
    3D center-load%
    \label{fig:topoOptScenario_4}%
  }[72mm]{%
    \includegraphics{topoOptScenario3D_2}%
  }%
  \caption[Test scenarios in topology optimization]{%
    Test scenarios in topology optimization in
    two and three spatial dimensions.
    Shown are
    the domains $\objdomain$,
    load points,
    locations of homogeneous Dirichlet boundary conditions
    \emph{\textcolor{C1}{(red)},} and
    exemplary optimal structures \emph{\textcolor{C0}{(blue)}.}%
  }%
  \label{fig:topoOptScenario}%
\end{figure}

%\paragraph{Material}
%
%As in \cite{Valdez17Topology}, we use the ASTM A-36 steel,
%which is ``one of the most commonly used material in
%real-world structural mechanics'' according to \cite{Valdez17Topology},
%as test material.
%Its material properties are as follows \cite{Valdez17Topology}:
%$\text{Young modulus} = \SI{211}{\giga\pascal}$,
%$\text{Poisson ratio} = 0.29$,
%$\text{density} = \SI{7874}{\kilogram\per\meter\cubed}$, and
%$\text{yield stress} = \SI{220}{\mega\pascal}$.

\section{Numerical Results}
\label{sec:64results}

\minitoc[-5mm]{72mm}{3}

\noindent
In this final section of the chapter,
we study optimal results of the test scenarios and
analyze interpolation and optimization errors
for topology optimization with B-spline surrogates on sparse grids.



\subsection{Methodology}
\label{sec:641methodology}

For simplicity, in the following,
we combine the functions to be interpolated,
i.e., the Cholesky factor
$\cholfactor\colon \clint{\*0, \*1} \to \real^{6 \times 6}$ and
the micro-cell density $\denscell\colon \clint{\*0, \*1} \to \real$,
to one single objective function
$\*\objfun\colon \clint{\*0, \*1} \to \real^{m+1}$.

\paragraph{Overview over offline and online phase}

Our method is divided into an offline phase and an online phase,
both of which are sketched in \cref{fig:topoOptPhases}.
The offline phase comprises
generating the spatially adaptive sparse grid
$\sgset = \{\gp{\*l_k,\*i_k} \mid k = 1, \dotsc, \ngp\}$,
solving corresponding micro-problems,
computing the Cholesky factors, and
hierarchizing the Cholesky factor entries and micro-cell densities
to obtain the sparse grid interpolant $\*\sgintp$.
Each optimization iteration of the online phase consists of
evaluating the interpolant $\*\sgintp$
for each micro-cell parameter $\*x^{(j)}$ ($j = 1, \dotsc, M$),
reconstructing the elasticity tensor $\etensor[\chol,\sparse]$ from
the Cholesky factors $\cholfactor[\sparse]$, and
solving the macro-problem to retrieve the approximated compliance value
$\compliance[\sparse](\*x^{(1)}, \dotsc, \*x^{(M)})$.%
\footnote{%
  In addition, the partial derivatives
  $\partialdiff{} \etensor[\chol,\sparse]/\partialdiff{} x_t$
  ($t = 1, \dotsc, d$)
  are evaluated using \cref{eq:choleskyFactorDerivative}.
  This is required as we employ gradient-based optimization.%
}
The superscript in $\compliance[\sparse]$ indicates that
we do not use the exact elasticity tensors to compute the compliance value.

\begin{figure}
  \tikzset{
    myCircle/.style={
      circle,
      fill=mittelblau!30,
      draw=mittelblau,
      inner sep=0.5mm,
    }
  }%
  \subcaptionbox{%
    Offline phase (without the actual grid generation).%
  }[149mm]{%
    \begin{tikzpicture}
      \node[myCircle] (points) at (0mm,0mm) {%
        $
          \begin{matrix}
            \gp{\*l_1,\*i_1},\\
            \dots,\\
            \gp{\*l_{\ngp},\*i_{\ngp}}
          \end{matrix}
        $%
      };
      \node[myCircle] (elasticityTensors) at (43mm,0mm) {%
        $
          \begin{matrix}
            \etensor(\gp{\*l_1,\*i_1}),\\
            \dots,\\
            \etensor(\gp{\*l_{\ngp},\*i_{\ngp}})
          \end{matrix}
        $%
      };
      \node[myCircle] (choleskyFactors) at (80mm,0mm) {%
        $
          \begin{matrix}
            \cholfactor(\gp{\*l_1,\*i_1}),\\
            \dots,\\
            \cholfactor(\gp{\*l_{\ngp},\*i_{\ngp}})
          \end{matrix}
        $%
      };
      \node[myCircle] (choleskyInterpolant) at (118mm,0mm) {%
        $
          \begin{matrix}
            \cholfactor[\sparse]\colon \clint{\*0, \*1}\\
            {} \to \real^{6 \times 6}
          \end{matrix}
        $%
      };
      \draw[->,draw=C0] (points) -- node[above] {%
        \footnotesize{}micro-problem%
      } (elasticityTensors);
      \draw[->,draw=C0] (elasticityTensors) -- node[above] {%
        \footnotesize{}%
        $
          \tr{\cholfactor} \cholfactor = \etensor
        $\vphantom{p}%
      } (choleskyFactors);
      \draw[->,draw=C0] (choleskyFactors) -- node[above] {%
        \footnotesize{}interpolate%
      } (choleskyInterpolant);
    \end{tikzpicture}%
  }%
  \\[2mm]%
  \subcaptionbox{%
    Online phase (one iteration of the optimizer).%
  }[149mm]{%
    \begin{tikzpicture}
      \node[myCircle] (points) at (0mm,0mm) {%
        $
          \begin{matrix}
            \*x^{(1)},\\
            \dots,\\
            \*x^{(M)}
          \end{matrix}
        $%
      };
      \node[myCircle] (choleskyFactors) at (34mm,0mm) {%
        $
          \begin{matrix}
            \cholfactor[\sparse](\*x^{(1)}),\\
            \dots,\\
            \cholfactor[\sparse](\*x^{(M)})
          \end{matrix}
        $%
      };
      \node[myCircle] (elasticityTensors) at (83mm,0mm) {%
        $
          \begin{matrix}
            \etensor[\chol,\sparse](\*x^{(1)}),\\
            \dots,\\
            \etensor[\chol,\sparse](\*x^{(M)})
          \end{matrix}
        $%
      };
      \node[myCircle] (complianceValue) at (129.5mm,0mm) {%
        $
          \begin{matrix}
            \compliance[\sparse](\*x^{(1)},\\
            \dotsc,\\
            \*x^{(M)})
          \end{matrix}
        $%
      };
      \draw[->,draw=C0] (points) -- node[above] {%
        \footnotesize{}evaluate\vphantom{p}%
      } (choleskyFactors);
      \draw[->,draw=C0] (choleskyFactors) -- node[above] {%
        \footnotesize{}%
        $
          \etensor[\chol,\sparse]
          = \tr{(\cholfactor[\sparse])} \cholfactor[\sparse]
        $\vphantom{p}%
      } (elasticityTensors);
      \draw[->,draw=C0] (elasticityTensors) -- node[above] {%
        \footnotesize{}macro-problem%
      } (complianceValue);
    \end{tikzpicture}%
  }%
  \caption[Offline and online phase for topology optimization]{%
    Offline and online phase for topology optimization.
    The interpolation of the micro-cell density with $\denscell^{\sparse}$
    (see \cref{sec:622BSplines}) has been omitted for brevity.%
  }%
  \label{fig:topoOptPhases}%
\end{figure}

\paragraph{Generation of spatially adaptive sparse grids}

We use the classical surplus-based refinement criterion
(see, e.g., \cite{Pflueger10Spatially})
to generate the spatially adaptive sparse grids
as show in \cref{alg:topoOptGridGeneration}.
The difference to common surrogate settings is that the objective function
$\*f\colon \clint{\*0, \*1} \to \real^{m+1}$
is not scalar-valued, but matrix-valued.
As the components of $\cholfactor$ cannot be evaluated individually,
the adaptivity criterion has to consider all entries at once
to avoid performing unnecessary evaluations.
We use the surpluses in the piecewise linear hierarchical basis,
as their absolute values correlate with the second mixed derivative
of the objective function due to \eqref{eq:surplusIntegral}.

\begin{algorithm}
  \begin{algorithmic}[1]
    \Function{$\sgset = \texttt{offlinePhase}$}{%
      $\*\objfun$, $n$, $b$, $\*c$, $l_{\max}$, $\varepsilon$,
      $\ngp_{\mathrm{refine}}$%
    }
      \State{$\sgset \gets \coarseregsgset{n}{d}{b}$}
      \Comment{initial regular sparse grid}%
      \While{\True}
        \State{$\ngp \gets \setsize{\sgset}$}
        \Comment{number of grid points}%
        \State{%
          Let $(\surplus{\*l_{k'},\*i_{k'}})_{k' = 1, \dotsc, \ngp}$
          satisfy $
            \fa{k = 1, \dotsc, \ngp}{
              \sum_{k'=1}^{\ngp} \vsurplus_{\*l_{k'},\*i_{k'}}
              \bspl{\*l_{k'},\*i_{k'}}{1}(\gp{\*l_k,\*i_k})
              = \*\objfun(\gp{\*l_k,\*i_k})
            }
          $%
        }
        \ForOneLine{$k = 1, \dotsc, \ngp$}{%
          $\beta_k \gets \tr{\*c} \abs{\vsurplus_{\*l,\*i}}$%
        }
        \Comment{combine surpluses to a scalar value}%
        \State{%
          $
            \liset^\ast \gets \{
              k = 1, \dotsc, \ngp \mid
              \ex{\gp{\*l,\*i} \notin \sgset}{
                \gp{\*l_k,\*i_k} \to \gp{\*l,\*i}
              },\,
              \norm[\infty]{\*l_k} < l_{\max},\,
              \abs{\surplus{\*l_k,\*i_k}} > \varepsilon
            \}
          $%
        }
        \IfOneLine{$\liset^\ast = \emptyset$}{\Break{}}
        \Comment{stop when there are not refinable grid points left}%
        \State{%
          Refine $\le \ngp_{\mathrm{refine}}$ of the points
          $\{\gp{\*l_k,\*i_k} \in \sgset \mid k \in \liset^\ast\}$
          with largest $\abs{\beta_k}$%
        }
      \EndWhile{}
    \EndFunction{}
  \end{algorithmic}
  \caption[%
    Generation of spatially adaptive sparse grids for topology optimization%
  ]{%
    Generation of spatially adaptive sparse grids for topology optimization.
    Inputs are
    the objective function $\*f\colon \clint{\*0, \*1} \to \real^{m+1}$
    (combination of Cholesky factors of elasticity tensors and
    micro-cell densities),
    the level $n \ge d$ and boundary parameter $b \in \natz$ of the
    initial regular sparse grid,
    the vector $\*c \in \real^d$ of coefficients with which the
    absolute values of the entries of the surpluses are combined,
    the maximal level $l_{\max} \in \nat$,
    the refinement threshold $\varepsilon \in \posreal$, and
    the number $\ngp_{\mathrm{refine}} \in \nat$ of points to refine
    in each iteration.
    Output is the spatially adaptive sparse grid $\sgset$.%
  }%
  \label{alg:topoOptGridGeneration}%
\end{algorithm}

\paragraph{Parameter bounds}

In the micro-cell models presented in \cref{sec:631models},
extreme micro-cell parameters near zero or one may cause problems
with the resulting elasticity tensors.
For instance, the elasticity tensor entries corresponding to
the 2D cross model are discontinuous near the lines $x_1 = 1$ or $x_2 = 1$
\multicite{Huebner14Mehrdimensionale,Valentin14Hierarchische}.
This is due to the fact the micro-cell is completely filled with material
on these lines,
independent of the other micro-cell parameter that is not one.
Similar issues occur for the other models and the shearing angles.
Hence, we have to restrict the range of the feasible micro-cell parameters,
i.e., the sparse grid points $\*x$ are still defined on the unit hyper-cube
$\clint{\*0, \*1}$,
but the actual micro-cell parameters $\xscaled$ are retrieved by an
affine transformation $\xscaled := \*a + (\*b - \*a) \*x$.
For the models in \cref{sec:631models},
we restrict the bar widths to $\clint{0.01, 0.99}$ and
the shearing angles to $\clint{-0.35\pi, 0.35\pi}$.

\paragraph{Software, algorithms, and domain discretization}

The micro-problems and macro-problems are solved with the \fem
implemented in the \fem software package CFS++ \cite{Kaltenbacher10Advanced}.%
\footnote{%
  \url{http://www.lse.uni-erlangen.de/cfs/}%
}
The micro-problems were discretized by dividing the micro-cells into
$128 \times 128$ elements (models in two dimensions) or
$16 \times 16 \times 16$ elements (models in three dimensions).
Constructing the sparse grids (offline phase) was done via a MATLAB code,
while the evaluations of the interpolants (online phase) were performed
by the sparse grid toolbox \sgpp \cite{Pflueger10Spatially}.%
\footnote{%
  \url{http://sgpp.sparsegrids.org/}%
}
For the solution of the emerging optimization problems,
a sequential quadratic programming method was employed
(see \cref{sec:513gradientBasedConstrained}).



\subsection{Error Sources}
\label{sec:642errorSources}

There are multiple sources that contribute to the numerical error
of our method:

\begin{enumerate}
  \item
  Discretization of the micro-problem
  (i.e., the elasticity tensors $\etensor$ are inaccurate)
  
  \item
  Sparse grid interpolation
  (i.e., $\etensor[\sparse] \not= \etensor$)
  
  \item
  Reconstruction of elasticity tensors with Cholesky factors
  (i.e., $\etensor[\chol,\sparse] \not= \etensor[\sparse]$)
  
  \item
  Discretization of the macro-problem
  (i.e., the compliance $\compliance$ is inaccurate)
  
  \item
  Optimization
  (i.e., the global minimum found by the optimizer is inaccurate)
  
  \item
  Floating-point rounding errors
  (i.e., arithmetical operations are inaccurate)
\end{enumerate}

\noindent
Errors of type~6 are always present and will not be analyzed in this chapter.
Errors of type~1 and~4 are intrinsic to the homogenization approach
and will also not be discussed here.
%The errors of type 2 and 5 have already been analyzed in
%\cref{chap:50optimization}.
Therefore, in the remainder of this chapter,
we will analyze the errors of types~2 and~3 (\cref{sec:643interpolation})
and of type~5 (\cref{sec:644optimization}).



\subsection{Interpolation Error}
\label{sec:643interpolation}

\begin{figure}
  \subcaptionbox{%
    TODO%
  }[72mm]{%
    \includegraphics{topoOptInterpolationPointwise_1}%
  }%
  \hfill%
  \subcaptionbox{%
    TODO%
  }[72mm]{%
    \includegraphics{topoOptInterpolationPointwise_2}%
  }%
  \hfill%
  \includegraphics{topoOptInterpolationPointwise_3}%
  \caption[TODO]{%
    TODO%
  }%
  \label{fig:topoOptInterpolationErrorPointwise}%
\end{figure}

\begin{figure}
  \hspace*{5mm}%
  \includegraphics{topoOptInterpolation_3}%
  \hfill%
  \raisebox{0.5mm}{\includegraphics{topoOptInterpolation_4}}%
  \\[2mm]%
  \subcaptionbox{%
    TODO%
  }[72mm]{%
    \includegraphics{topoOptInterpolation_1}%
  }%
  \hfill%
  \subcaptionbox{%
    TODO%
  }[72mm]{%
    \includegraphics{topoOptInterpolation_2}%
  }%
  \caption[TODO]{%
    TODO%
  }%
  \label{fig:topoOptInterpolationErrorBasisFunctions}%
\end{figure}

\dummytext[4]{}



\subsection{Optimized Structures and Optimization Error}
\label{sec:644optimization}

\begin{table}
  \setnumberoftableheaderrows{1}%
  \begin{tabular}{%
    >{\kern\tabcolsep}=l<{\kern5mm}*{6}{+c}<{\kern\tabcolsep}%
  }
    \toprulec
    \headerrow
    Scenario&       2D-C&   2D-FC&  2D-SC&  2D-SFC& 3D-C&   3D-SC\\
    \midrulec
    2D cantilever&  XX.XXX& XX.XXX& XX.XXX& XX.XXX& ---&    ---\\
    2D L shape&     XX.XXX& XX.XXX& XX.XXX& XX.XXX& ---&    ---\\
    \midrulec
    3D cantilever&  ---&    ---&    ---&    ---&    XX.XXX& XX.XXX\\
    3D center load& ---&    ---&    ---&    ---&    XX.XXX& XX.XXX\\
    \bottomrulec
  \end{tabular}
  \caption[Optimal compliance values for different micro-cell models]{%
    Optimal compliance values for the different scenarios
    and micro-cell models (maximum number $\ngpMax = \num{10000}$
    of sparse grid points).
    The columns correspond to the micro-cell models as presented
    in \cref{fig:microCell}:
    2D cross,
    2D framed cross,
    2D shared cross,
    2D shared framed cross,
    3D cross, and
    3D sheared cross.
    The highlighted entries indicate the best choice
    of micro-cell models for a given scenario.%
  }%
  \label{tbl:TODO1}%
\end{table}

\begin{table}
  \setnumberoftableheaderrows{1}%
  \begin{tabular}{%
      >{\kern\tabcolsep}=l<{\kern5mm}*{5}{+c}<{\kern\tabcolsep}%
    }
    \toprulec
    \headerrow
    Scenario&
    $\bspl{\*l,\*i}{1}$&
    $\bspl{\*l,\*i}{3}$&
    $\bspl[\nak]{\*l,\*i}{3}$&
    $\bspl{\*l,\*i}{5}$&
    $\bspl[\nak]{\*l,\*i}{5}$\\
    \midrulec
    2D cantilever&  XX.XXX& XX.XXX& XX.XXX& XX.XXX& XX.XXX\\
    2D L shape&     XX.XXX& XX.XXX& XX.XXX& XX.XXX& XX.XXX\\
    \midrulec
    3D cantilever&  XX.XXX& XX.XXX& XX.XXX& XX.XXX& XX.XXX\\
    3D center load& XX.XXX& XX.XXX& XX.XXX& XX.XXX& XX.XXX\\
    \bottomrulec
  \end{tabular}
  \caption[Optimal compliance values for different basis functions]{%
    Optimal compliance values for the different scenarios
    and basis functions (maximum number $\ngpMax = \num{10000}$
    of sparse grid points).
    The highlighted entries indicate the best choice
    of basis functions for a given scenario.%
  }%
  \label{tbl:TODO2}%
\end{table}

\todo{try different basis types}

\dummytext[12]{}


\cleardoublepage
