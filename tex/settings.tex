% don't use sans-serif font for headings
\setkomafont{disposition}{\normalcolor\bfseries}

% don't use sans-serif font for dictums
\addtokomafont{dictum}{\rmfamily}

% make dictums larger, remove dictum rule (led to errors),
% align text on the right
\renewcommand*{\dictumwidth}{0.6667\textwidth}
\renewcommand*{\dictumrule}{}
\renewcommand*{\raggeddictumtext}{\raggedleft}

% all-caps sans-serif page header
\setkomafont{pageheadfoot}{%
  \normalfont\normalcolor\sffamily\fontsize{9}{10}\selectfont\lsstyle%
}

% ... but don't use all-caps for page number in footer
% (caution: the "\textls*[0]" is a hack for deactivating letter spacing)
\setkomafont{pagenumber}{\normalfont\normalsize\textls*[0]}

% change font size for captions
\makeatletter
\DeclareCaptionFormat{mycaptionformat}{%
  \small\@hangfrom{#1#2}%
  \advance\caption@parindent\hangindent
  \advance\caption@hangindent\hangindent
  \caption@@par#3\par%
}
\makeatother

% change appearance of caption label
\DeclareCaptionLabelFormat{mycaptionlabelformat}{%
  \sffamily\bfseries\fontsize{9.5}{12}\selectfont\textls*{\MakeUppercase{#1}} #2%
}

% set caption format
\captionsetup{
  % use custom format
  format=mycaptionformat,
  % use custom label format
  labelformat=mycaptionlabelformat,
  % separate number and caption with \quad
  labelsep=quad,
}

% change font size for sub-captions
\makeatletter
\DeclareCaptionFormat{mysubcaptionformat}{%
  \fontsize{10}{12}\selectfont\@hangfrom{#1#2}%
  \advance\caption@parindent\hangindent
  \advance\caption@hangindent\hangindent
  \caption@@par#3\par%
}
\makeatother

% change appearance of sub-caption label
\DeclareCaptionLabelFormat{mysubcaptionlabelformat}{%
  \sffamily\bfseries\fontsize{8.8}{11}\selectfont\textls*{\MakeUppercase{#2}}%
}

% set sub-caption format
\captionsetup[sub]{
  % use custom format
  format=mysubcaptionformat,
  % use custom label format
  labelformat=mysubcaptionlabelformat,
  % separate number and caption with \quad
  labelsep=quad,
}

% automatically center all float and subfigure contents
\makeatletter
\g@addto@macro{\@floatboxreset}{\centering}
\apptocmd\subcaption@minipage{\centering}{}{}
\makeatother

% change font size in float environments
\makeatletter
\g@addto@macro{\figure}{\small}
\g@addto@macro{\table}{\small}
\g@addto@macro{\algorithm}{\small}
\makeatother

% recalculate page margins as the line spread has changed
\recalctypearea

% vertically center ":" in ":="
\mathtoolsset{centercolon}

% get name of current label (chapter, section, subsection, ...)
\makeatletter
\newcommand*{\currentname}{\@currentlabelname}
\makeatother

% warn for every usage of \blindtext
\let\oldblindtext\blindtext
\renewcommand*{\blindtext}{%
  \GenericWarning{}{%
    LaTeX Warning (\thesubsection\space\currentname): \string\blindtext%
  }%
  \oldblindtext%
}

% location of graphics files
\graphicspath{{../gfx/}}

% chapter heading: make number bigger
\renewcommand*{\chapterformat}{%
  \mbox{%
    \chapappifchapterprefix{\nobreakspace}%
    \scalebox{3.2}{\thechapter\autodot}%
    \IfUsePrefixLine{}{\enskip}%
  }%
}

% chapter heading: align number and text at the baseline of the last line
% of the text, add some more space between number and text
\renewcommand*{\chapterlinesformat}[3]{%
  \begin{tabularx}{\textwidth}{@{}l@{}X@{}}%
    % chapter number
    #2%
    % add hspace between number and text, but only if \chapter (not \chapter*)
    % has been used (i.e., if the chapter number #2 is not empty)
    \if\relax\detokenize{#2}\relax\else\hspace*{5mm}\fi&%
    % text of heading, [b] is for alignment, \linewidth is width of X cell,
    % \raggedchapter for not justifying headings
    \parbox[b]{\linewidth}{\raggedchapter\setstretch{1}#3}%
  \end{tabularx}%
}

% check mode
\iftoggle{checkmode}{
  % show overfull boxes
  \overfullrule=1mm
  % show underfull vboxes
  \directlua{dofile("detect_underfull.lua")}
  % whitelist is in hyphenation_whitelist.txt:
  % one word per line with dashes (-) indicating the places where
  % hyphenation is whitelisted
  \LuaCheckHyphen{whitelist=hyphenation_whitelist.txt}
}{}

% meta-data variables
\newcommand*{\thetitle}{%
  B-Splines for Sparse Grids:\texorpdfstring{\\}{}
  Algorithms and Application to
  Higher-Dimensional Optimization%
}
\newcommand*{\theauthor}{Julian Valentin}
\newcommand*{\thedate}{\TODO{insert date}}

% define line colors (mix between MATLAB and matplotlib colors)
\definecolor{C0}{rgb}{0.000,0.447,0.741}
\definecolor{C1}{rgb}{0.850,0.325,0.098}
\definecolor{C2}{rgb}{0.929,0.694,0.125}
\definecolor{C3}{rgb}{0.494,0.184,0.556}
\definecolor{C4}{rgb}{0.466,0.674,0.188}
\definecolor{C5}{rgb}{0.301,0.745,0.933}
\definecolor{C6}{rgb}{0.635,0.078,0.184}
\definecolor{C7}{rgb}{0.887,0.465,0.758}
\definecolor{C8}{rgb}{0.496,0.496,0.496}

% define university CD colors
\definecolor{anthrazit}{RGB}{62,68,76}
\definecolor{mittelblau}{RGB}{0,81,158}
\definecolor{helllblau}{RGB}{0,190,255}

% set up hyperref
\hypersetup{
  % set metadata
  pdftitle={\thetitle},
  pdfauthor={\theauthor},
  pdfcreator={LaTeX, KOMA-Script, hyperref},
  % underline links instead of putting a framed box around them
  pdfborderstyle={/S/U/W 1},
  % set link colors
  citebordercolor=C1,
  filebordercolor=C1,
  linkbordercolor=C1,
  menubordercolor=C1,
  runbordercolor=C1,
  urlbordercolor=C0,
}

% location of *.bib file
\addbibresource{../bib/bibliography.bib}

% insert colon after author names
\renewcommand*{\labelnamepunct}{: }

% hide urldate field
\AtEveryBibitem{\clearfield{urlyear}}

% make paper titles italic, remove quotation marks
\DeclareFieldFormat[article]{title}{\mkbibemph{#1}}
\DeclareFieldFormat[article]{journaltitle}{#1}
\DeclareFieldFormat[thesis]{title}{\mkbibemph{#1}}
\DeclareFieldFormat[inbook]{title}{\mkbibemph{#1}}
\DeclareFieldFormat[inbook]{booktitle}{#1}
\DeclareFieldFormat[incollection]{title}{\mkbibemph{#1}}
\DeclareFieldFormat[incollection]{booktitle}{#1}
\DeclareFieldFormat[inproceedings]{title}{\mkbibemph{#1}}
\DeclareFieldFormat[inproceedings]{booktitle}{#1}
\DeclareFieldFormat[unpublished]{title}{\mkbibemph{#1}}

% fix title capitalization to sentence case (all lowercase)
\DeclareFieldFormat{titlecase}{\MakeSentenceCase*{#1}}

% ... but don't change journal titles
\renewbibmacro*{journal}{%
  \iffieldundef{journaltitle}{}{%
    \printtext[journaltitle]{%
      \printfield[noformat]{journaltitle}%
      \setunit{\subtitlepunct}%
      \printfield[noformat]{journalsubtitle}%
    }%
  }%
}

% suppress "In:" before journal names
\renewbibmacro*{in:}{}

% separate authors with semicolon, suppress "and" in author names
\renewcommand{\multinamedelim}{\addsemicolon\space}
\renewcommand*{\finalnamedelim}{\addsemicolon\space}

% make names bold
\renewcommand*{\mkbibnamefamily}[1]{\textbf{#1}}

\DeclareNameAlias{sortname}{last-first}
\DeclareNameAlias{default}{last-first}

% define custom heading to fix non-uppercase page header, make URLs smaller
\defbibheading{myheading}[\bibname]{%
  \addchap{#1}%
  \markboth{BIBLIOGRAPHY}{BIBLIOGRAPHY}%
  \let\oldtexttt\texttt%
  \renewcommand*{\texttt}[1]{\oldtexttt{\scriptsize##1}}%
}

% add custom bibliography post note (URLs last checked on ...)
\defbibnote{mypostnote}{%
  All URLs have last been checked on \thedate.%
  \renewcommand*{\texttt}[1]{\oldtexttt{##1}}%
}

% omit "URL:" and "DOI:" separators to save space
\DeclareFieldFormat{url}{\url{#1}}
\DeclareFieldFormat{doi}{%
  \ifhyperref{\href{https://doi.org/#1}{\nolinkurl{#1}}}{\nolinkurl{#1}}%
}

% change font size of bibliography
\renewcommand*{\bibfont}{\small}

% set single line spacing for bibliography
\renewcommand*{\bibsetup}{\setstretch{1}}

% break URLs after every character (most importantly for bibliography)
\luaexec{require("breakurl")}
\let\oldhref\href
\renewcommand*{\url}[1]{\luadirect{breakurl(\luastringN{#1}, \luastringN{#1})}}
\renewcommand*{\href}[2]{\luadirect{breakurl(\luastringN{#1}, \luastringN{#2})}}

% use single spacing in table of contents
\AfterTOCHead[toc]{\setstretch{1}}

% define new glossary style based on long (uses longtable)
% in which the column widths are fixed and space before/after the
% table is removed
\newglossarystyle{longfixed}{%
  \setglossarystyle{long}%
  \renewenvironment{theglossary}%
    {%
      \setlength{\LTpre}{0pt}%
      \setlength{\LTpost}{0pt}%
      \begin{spacing}{1}%
        \begin{longtable}{@{}p{0.2\textwidth}@{}p{0.8\textwidth}@{}}%
    }%
    {%
        \end{longtable}%
      \end{spacing}%
    }%
}

% use new glossary style
\setglossarystyle{longfixed}

% generate table of symbols and acronyms
\makeglossaries
