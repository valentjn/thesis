% ======================================================================
% Meta-Data
% ======================================================================

% meta-data variables
\newcommand*{\thetitle}{%
  B-Splines for Sparse Grids:\texorpdfstring{\\}{}
  Algorithms and Application to\texorpdfstring{\\}{}
  Higher-Dimensional Optimization%
}
\newcommand*{\theauthor}{Julian Valentin}
\newcommand*{\thebirthplace}{Stuttgart}
\newcommand*{\thedefensedate}{\todo{insert date of defense}}
\newcommand*{\thedate}{\todo{insert date}}
\newcommand*{\theyear}{2018}
\newcommand*{\theinstitute}{Institut für Parallele und Verteilte Systeme}
\newcommand*{\theadvisor}{Prof.\ Dr.\ Dirk Pflüger}
\newcommand*{\theexamineri}{\todo{insert 1st examiner}}
\newcommand*{\theexaminerii}{\todo{insert 2nd examiner}}
\newcommand*{\theexamineriii}{\todo{insert 3rd examiner}}

% ======================================================================
% KOMA-Script and Text Area
% ======================================================================

% options for KOMA-Script
\KOMAoptions{
  % choose font size (PO: should be between 12pt and 14pt)
  fontsize=12pt,
  % suppress "very small head height detected" warning
  DIV=12,
  % don't end section numbers with period despite appendices
  numbers=noendperiod,
  % set linespacing of header and footer to 1.5
  % (otherwise, the header will "jump" back and forth from normal
  % pages and pages with single line spacing, e.g., table of contents)
  onpsinit=\onehalfspacing,
  % use \section for headings of list of figures/tables/algorithms
  listof=leveldown,
}

% binding offset that will be substracted from inner margins
\newcommand*{\bindingoffset}{10mm}

% set page margins
\geometry{
  bindingoffset=\bindingoffset,
  inner=15mm,
  outer=30mm,
  top=20mm,
  bottom=30mm,
  includehead=true,
}

% ======================================================================
% Graphics
% ======================================================================

% location of graphics files
\graphicspath{{../gfx/}}

% TikZ libraries
\usetikzlibrary{
  % arrow types
  arrows.meta,
  % coordinate calculations
  calc,
  % zig-zag lines
  decorations.pathmorphing,
  % curly braces
  decorations.pathreplacing,
  % color gradients
  fadings,
  % drop shadows
  shadows,
}

% ======================================================================
% Cross-References
% ======================================================================

% define abbreviated names for cross-references
\crefname{algorithm}{Alg.}{Algorithms}
\Crefname{algorithm}{Algorithm}{Algorithms}

\crefname{chapter}{Chap.}{Chapters}
\Crefname{chapter}{Chapter}{Chapters}

\crefname{corollary}{Cor.}{Corollaries}
\Crefname{corollary}{Corollary}{Corollaries}

\crefname{definition}{Def.}{Definitions}
\Crefname{definition}{Definition}{Definitions}

\crefname{equation}{Eq.}{Equations}
\Crefname{equation}{Equation}{Equations}

\crefname{figure}{Fig.}{Figures}
\Crefname{figure}{Figure}{Figures}

\crefname{lemma}{Lemma}{Lemmas}
\Crefname{lemma}{Lemma}{Lemmas}

\crefname{line}{Line}{Lines}
\Crefname{line}{Line}{Lines}

\crefname{proposition}{Prop.}{Propositions}
\Crefname{proposition}{Proposition}{Propositions}

\crefname{section}{Sec.}{Sections}
\Crefname{section}{Section}{Sections}

\crefname{table}{Tab.}{Tables}
\Crefname{table}{Table}{Tables}

\crefname{theorem}{Thm.}{Theorems}
\Crefname{theorem}{Theorem}{Theorems}

% reference theorems with their name appended
\newcommand*{\thmref}[1]{\hyperlink{#1}{\cref*{#1}~(\nameref*{#1})}}
\newcommand*{\Thmref}[1]{\hyperlink{#1}{\Cref*{#1}~(\nameref*{#1})}}

% ======================================================================
% Tables of Contents, Figures, Tables, Algorithms, and Theorems
% ======================================================================

% remove indentation for lists of figures and tables
\DeclareTOCStyleEntry[indent=0em]{tocline}{figure}
\DeclareTOCStyleEntry[indent=0em]{tocline}{table}

% declare algorithm float environment, create list of algorithms,
% use same formatting as for figures and tables
\DeclareNewTOC[
  type=algorithm,
  name=Algorithm,
  listname={List of Algorithms},
  float,
  floattype=4,
  floatpos=tp,
  counterwithin=chapter,
  tocentrynumwidth=2.3em,
  tocentryindent=0em,
]{loa}

% create list of theorems,
% use same formatting as for figures, tables, and algorithms
% (thmtools's thm-listof.sty is unfortunately a little buggy,
% as it causes the flip book images to jump; using KOMA-Script for
% all lists is more consistent and less error-prone)
\DeclareNewTOC[
  type=mytheorem,
  name=Theorem,
  listname={List of Theorems},
  tocentrynumwidth=2.3em,
  tocentryindent=0em,
]{lop}

% add code that adds the theorem to the *.lop file to hook
\makeatletter
\addtotheorempostheadhook{%
  \addxcontentsline{lop}{mytheorem}{\csname ll@\thmt@envname\endcsname}%
}
\makeatother

% use single spacing in
% tables of contents, figures, tables, algorithms, and theorems
\AfterTOCHead{\begin{spacing}{1}}
\AfterStartingTOC{\end{spacing}}

% include subsubsections in table of contents, but without numbers
\setcounter{tocdepth}{\subsubsectionnumdepth}

% increase space of page numbers
% (default values of 1.55em and 2.55em are too small for three-digit numbers)
\makeatletter
\renewcommand*{\@pnumwidth}{2em}
\renewcommand*{\@tocrmarg}{3em}
\makeatother

% ======================================================================
% Headings
% ======================================================================

% don't use sans-serif font for headings, use single spacing
\setkomafont{disposition}{\normalcolor\bfseries\setstretch{1}}

% format chapter headings with custom commands (below)
\renewcommand*{\chapterformat}{\chapternumber{\thechapter\autodot}}
\renewcommand*{\chapterlinesformat}[3]{#2\chaptertitle{#3}}

% scale chapter number, bottom alignment
\newcommand*{\chapternumber}[1]{%
  \begin{minipage}[b]{0.15\linewidth}%
    \chapappifchapterprefix{\nobreakspace}%
    \scalebox{3.2}{#1\vphantom{0123456789}}%
    \IfUsePrefixLine{}{\enskip}%
  \end{minipage}%
}

% left-align chapter title
\newcommand*{\chaptertitle}[1]{%
  \leavevmode\smash{%
    \begin{minipage}[b]{0.85\linewidth}%
      \raggedchapter#1%
    \end{minipage}%
  }%
}

% set space before and after chapter heading
\RedeclareSectionCommand[beforeskip=8em,afterskip=5em]{chapter}

% Commands for long chapter/section titles:
% If the title is too long, then the mark in the page header will
% overflow and wrap into the next line, which looks ugly.
% \chaptermark/\sectionmark is supposed to fix that. However, if you just
% call it after \chapter/\section, then it will only affect the marks on
% the following pages, not the one on which the heading is defined.
% Therefore, we have to call \chaptermark/\sectionmark twice and also
% supply the optional parameter (otherwise, there will be errors as LaTeX
% tries to add the marks to the table of contents).
% Use the commands like \longchapter{heading title}{mark title}.
\newcommand*{\longchapter}[2]{\chapter[#1]{#1\chaptermark{#2}}\chaptermark{#2}}
\newcommand*{\longsection}[2]{\section[#1]{#1\sectionmark{#2}}\sectionmark{#2}}

% dirty hack to automatically output ornaments before every section heading
\RedeclareSectionCommand[
  font={%
    \hfill%
    \textcolor{black!30}{\pgfornament[width=100mm]{88}}%
    \hfill\null\\[1.625ex plus .5ex minus -.1ex]%
    \Large%
  },
  beforeskip=-1.625ex plus -.5ex minus .1ex,
]{section}

% dirty hack to automatically output ornaments before every subsection heading
\RedeclareSectionCommand[
  font={%
    \hfill%
    \textcolor{black!30}{%
      \pgfornament[height=3mm]{2}\quad%
      \pgfornament[height=3mm]{24}\quad%
      \pgfornament[height=3mm,symmetry=v]{2}%
    }%
    \hfill\null\\[1.3ex plus .5ex minus -.1ex]%
    \Large%
  },
  beforeskip=-1.3ex plus -.5ex minus .1ex,
]{subsection}

% decrease skip before \paragraph and \subparagraph headings
\RedeclareSectionCommand[beforeskip=2ex plus 1ex minus .2px]{paragraph}
\RedeclareSectionCommand[beforeskip=1ex plus 1ex minus .2px]{subparagraph}

\makeatletter

% append period to \paragraph headings
\let\paragraphwithoutperiod\paragraph
\newcommand*{\paragraphwithperiodwithstar}[1]{\paragraphwithoutperiod*{#1.}}
\newcommand*{\paragraphwithperiod}[1]{\paragraphwithoutperiod{#1.}}
\renewcommand*{\paragraph}{%
  \@ifstar{\paragraphwithperiodwithstar}{\paragraphwithperiod}%
}

% append period to \subparagraph headings
\let\subparagraphwithoutperiod\subparagraph
\newcommand*{\subparagraphwithperiodwithstar}[1]{\subparagraphwithoutperiod*{#1.}}
\newcommand*{\subparagraphwithperiod}[1]{\subparagraphwithoutperiod{#1.}}
\renewcommand*{\subparagraph}{%
\@ifstar{\subparagraphwithperiodwithstar}{\subparagraphwithperiod}%
}

\makeatother

% increase indentation of paragraphs (default: 1em = \quad)
\setlength{\parindent}{2em}

% ======================================================================
% Dictums
% ======================================================================

% don't use sans-serif font for dictums
\addtokomafont{dictum}{\rmfamily}

% format dictums with custom command
\renewcommand*{\dictum}[2][]{\cleanchapterquote{#2}{#1}}

% shortcut for setting the dictum for a chapter
\newcommand*{\setdictum}[2]{%
  \setchapterpreamble{%
    \dictum[#2]{#1}%
    \chapterheadendvskip%
  }%
}

% fancy quotes (taken from cleanthesis.sty, slightly adapted)
\newcommand*{\hugequote}{%
  \fontsize{75}{80}\selectfont%
  \hspace*{-0.6em}\color{lightgray}%
  \textit{``}%
  \vskip -0.8em%
}

\newcommand*{\cleanchapterquote}[2]{%
  \begin{minipage}{\textwidth}%
    \begin{flushright}
      \begin{minipage}{0.54\textwidth}%
        \begin{flushleft}
          {\hugequote}\textit{#1}
        \end{flushleft}
        \begin{flushright}
          \small--- #2
        \end{flushright}
      \end{minipage}%
    \end{flushright}
  \end{minipage}%
  \bigskip
}

% ======================================================================
% Page Headers and Footers
% ======================================================================

% all-caps sans-serif page header
\setkomafont{pageheadfoot}{%
  \normalfont\normalcolor\sffamily%
  \fontsize{9.5}{12}\selectfont\lsstyle%
}
\setkomafont{pagenumber}{\usekomafont{pageheadfoot}\bfseries}

% move page number to page header
\clearscrheadfoot
\lehead[\pagemark]{\pagemark}
\rehead[]{\headmark}
\lohead[]{\headmark}
\rohead[\pagemark]{\pagemark}
\lefoot[]{}
\rofoot[]{}

% prepend "Chapter X: " in front of the chapter name in page headers
\renewcommand*{\chaptermarkformat}{Chapter \thechapter: }

% ======================================================================
% Initials
% ======================================================================

% height of initial in text lines
\setcounter{DefaultLines}{3}
% no indent right of initial
\setlength{\DefaultNindent}{0em}
% enlarge initial by 10% (grows to the top, aligned on baseline of 3rd line)
\renewcommand*{\DefaultLoversize}{0.08}
% move initial to the left by 10% of its width
\renewcommand*{\DefaultLhang}{0.1}
% use Zallman Caps for initial
\renewcommand*{\LettrineFontHook}{\Zallmanfamily}
% don't use small caps for rest of first word
\renewcommand*{\LettrineTextFont}{\normalfont}

% ======================================================================
% Framed Boxes
% ======================================================================

% define mdframed style for floats
\mdfdefinestyle{floatmdfstyle}{
  innerleftmargin=7pt,
  innerrightmargin=7pt,
  innertopmargin=7pt,
  innerbottommargin=6pt,
  roundcorner=10pt,
  linewidth=1pt,
  linecolor=mittelblau,
  backgroundcolor=mittelblau!10,
  shadow=true,
  shadowcolor=black!20,
  % the documentation of mdframed lies about the default values of
  % splittopskip and splitbottomskip
  splittopskip=15pt,
  splitbottomskip=5pt,
}

% define mdframed style for theorems
\mdfdefinestyle{thmmdfstyle}{
  innerleftmargin=12pt,
  innerrightmargin=7pt,
  innertopmargin=5pt,
  innerbottommargin=6pt,
  linewidth=1pt,
  linecolor=mittelblau,
  backgroundcolor=mittelblau!10,
  % the documentation of mdframed lies about the default values of
  % splittopskip and splitbottomskip
  splittopskip=15pt,
  splitbottomskip=5pt,
  % thick line on left side
  extra={%
    \draw[mittelblau,line width=4pt]
    ($(O)+(2pt,0pt)$) -- ($(O|-P)+(2pt,0pt)$);%
  },
  % "torn page" effect at bottom of the box before page break
  firstextra={
    \begin{scope}
      \clip ($(O)+(-0.5pt,-2pt)$) rectangle ($(P|-O)+(0.5pt,2pt)$);
      \draw[
        white,line width=3pt,decorate,
        decoration={random steps,segment length=2pt,amplitude=1.5pt}
      ]
      ($(O)+(-2pt,0pt)$) -- ($(P|-O)+(2pt,0pt)$);
    \end{scope}
  },
  % "torn page" effect at top of the box after page break
  secondextra={
    \begin{scope}
      \clip ($(O|-P)+(-0.5pt,-2pt)$) rectangle ($(P)+(0.5pt,2pt)$);
      \draw[
        white,line width=3pt,decorate,
        decoration={random steps,segment length=2pt,amplitude=1.5pt}
      ]
      ($(O|-P)+(-2pt,0pt)$) -- ($(P)+(2pt,0pt)$);
    \end{scope}
  },
}

% ======================================================================
% Floats
% ======================================================================

\makeatletter
% make "tp" to default position of figure and table
\def\fps@figure{tp}
\def\fps@table{tp}

% automatically center all float and subfigure contents
\g@addto@macro{\@floatboxreset}{\centering}
\apptocmd\subcaption@minipage{\centering}{}{}

% change font size in float environments
\g@addto@macro{\figure}{\small}
\g@addto@macro{\table}{\small}
\g@addto@macro{\algorithm}{\small}
\makeatother

% wrap floats in mdframed boxes
\apptocmd{\figure}{\begin{mdframed}[style=floatmdfstyle]}{}{}
\pretocmd{\endfigure}{\end{mdframed}}{}{}
\apptocmd{\table}{\begin{mdframed}[style=floatmdfstyle]}{}{}
\pretocmd{\endtable}{\end{mdframed}}{}{}
\apptocmd{\algorithm}{\begin{mdframed}[style=floatmdfstyle]}{}{}
\pretocmd{\endalgorithm}{\end{mdframed}}{}{}

% wrap SCfigure/SCtable in mdframed boxes
\makeatletter
\xpatchcmd{\endSC@float}{\endSC@FLOAT\@tempdima}{%
  \setlength{\@tempdima}{\@tempdima-17pt}\endSC@FLOAT\@tempdima%
}{}{}
\xpatchcmd{\endSC@FLOAT}{\@FLOAT{\SC@captype}}{%
  \@FLOAT{\SC@captype}\begin{mdframed}[style=floatmdfstyle]%
}{}{}
\xpatchcmd{\endSC@FLOAT}{\end@FLOAT}{\end{mdframed}\end@FLOAT}{}{}
\makeatother

% don't move floats to own page (float page) when they are tall,
% the default value is 0.5, which is a bit small
\renewcommand*{\floatpagefraction}{0.7}

% space between floats and text
\setlength{\textfloatsep}{40pt}

% ======================================================================
% Float Captions
% ======================================================================

% change font size for captions, add horizontal rule
\makeatletter
\DeclareCaptionFormat{mycaptionformat}{%
  \small\@hangfrom{#1#2}%
  \advance\caption@parindent\hangindent
  \advance\caption@hangindent\hangindent
  \caption@@par#3\par%
}
\makeatother

\newcommand*{\formatcaption}[1]{%
  \sffamily\bfseries\fontsize{9.5}{12}\selectfont%
  \textls*{\textcolor{mittelblau}{\MakeUppercase{#1}}}%
}

\newcommand*{\formatsubcaption}[1]{%
  \sffamily\bfseries\fontsize{8.8}{11}\selectfont%
  \textls*{\textcolor{mittelblau}{\MakeUppercase{#1}}}%
}

% change appearance of caption label
\DeclareCaptionLabelFormat{mycaptionlabelformat}{%
  \formatcaption{#1 #2}%
}

% special format for unnumbered figures (e.g., cover figure)
\DeclareCaptionLabelFormat{mycaptionlabelformatunnumbered}{%
  \formatcaption{#1}%
}

% set caption format
\captionsetup{
  % use custom format
  format=mycaptionformat,
  % use custom label format
  labelformat=mycaptionlabelformat,
  % separate number and caption with \quad
  labelsep=quad,
}

% don't indent caption text in SCfigure (figures with side caption)
\DeclareCaptionFormat{mysidecaptionformat}{\small#1#2\\#3\par}
\AtBeginEnvironment{SCfigure}{\captionsetup{format=mysidecaptionformat}}
\AtBeginEnvironment{SCtable}{\captionsetup{format=mysidecaptionformat}}

% use maximum space for SCfigure captions (sidecap's default is that
% the caption can only be as wide as the figure itself)
\renewcommand*{\sidecaptionrelwidth}{50}

% change font size for sub-captions
\makeatletter
\DeclareCaptionFormat{mysubcaptionformat}{%
  \fontsize{10}{12}\selectfont\@hangfrom{#1#2}%
  \advance\caption@parindent\hangindent
  \advance\caption@hangindent\hangindent
  \caption@@par#3\par%
}
\makeatother

% change appearance of sub-caption label
\DeclareCaptionLabelFormat{mysubcaptionlabelformat}{%
  \formatsubcaption{#2}%
}

% set sub-caption format
\captionsetup[sub]{
  % use custom format
  format=mysubcaptionformat,
  % use custom label format
  labelformat=mysubcaptionlabelformat,
  % separate number and caption with \quad
  labelsep=quad,
}

% ======================================================================
% Tables
% ======================================================================

% fixed-width column types with left, center, and right alignment
\newcolumntype{L}[1]{%
  >{\raggedright\let\newline\\\arraybackslash\hspace{0pt}}p{#1}%
}
\newcolumntype{C}[1]{%
  >{\centering\let\newline\\\arraybackslash\hspace{0pt}}p{#1}%
}
\newcolumntype{R}[1]{%
  >{\raggedleft\let\newline\\\arraybackslash\hspace{0pt}}p{#1}%
}

% ======================================================================
% Algorithms
% ======================================================================

% disable ligatures ("fi" etc.) in monospace font
\DisableLigatures{encoding=*,family=tt*}

% line number format
\algrenewcommand{\alglinenumber}[1]{\footnotesize\color{anthrazit}{\texttt{#1}}}
\algrenewcommand{\algorithmicrequire}{\textbf{Input:}}
\algrenewcommand{\algorithmicensure}{\textbf{Output:}}

% function name
\algrenewcommand{\textproc}{}

% bold keywords
\algnewcommand{\Break}{\textbf{break}}
\algnewcommand{\Continue}{\textbf{continue}}
\algnewcommand{\True}{\textbf{true}}
\algnewcommand{\False}{\textbf{false}}
\algnewcommand{\Null}{\textbf{null}}
\algrenewcommand{\algorithmicend}{\textbf{end}}
\algrenewcommand{\algorithmicdo}{\textbf{do}}
\algrenewcommand{\algorithmicwhile}{\textbf{while}}
\algrenewcommand{\algorithmicfor}{\textbf{for}}
\algrenewcommand{\algorithmicforall}{\textbf{for all}}
\algrenewcommand{\algorithmicloop}{\textbf{loop}}
\algrenewcommand{\algorithmicrepeat}{\textbf{repeat}}
\algrenewcommand{\algorithmicuntil}{\textbf{until}}
\algrenewcommand{\algorithmicprocedure}{\textbf{procedure}}
\algrenewcommand{\algorithmicfunction}{\textbf{function}}
\algrenewcommand{\algorithmicif}{\textbf{if}}
\algrenewcommand{\algorithmicthen}{\textbf{then}}
\algrenewcommand{\algorithmicelse}{\textbf{else}}
\algrenewcommand{\algorithmicreturn}{\textbf{return}}
\algnewcommand{\algorithmicforever}{\textbf{for ever}}
\algdef{S}[FOR]{ForEver}{\algorithmicforever\ \algorithmicdo}
\algnewcommand{\algorithmicgoto}{\textbf{go to}}
\algnewcommand{\Goto}[1]{\algorithmicgoto\ \cref*{#1}}

% small monospace font in algorithms
\makeatletter
\g@addto@macro\ALG@beginalgorithmic{\small\ttfamily}
\makeatother

% comment format
\algrenewcomment[1]{\hfill$\rightsquigarrow$ {\normalfont\emph{#1}}}

% set indentation to two characters
\algrenewcommand{\algorithmicindent}{\widthof{AB}}

% ======================================================================
% Theorems
% ======================================================================

% patch mdframed such that it doesn't violate \flushbottom
% (mdframed produces ragged page bottoms without this)
\makeatletter
\xpatchcmd{\mdf@put@frame@i}{\hrule \@height\z@ \@width\hsize\vfill}{}{}{}
\xpatchcmd{\mdf@put@frame@i}{\hrule \@height\z@ \@width\hsize\vfill}{}{}{}
\xpatchcmd{\mdf@put@frame@i}{\hrule \@height\z@ \@width\hsize\vfill}{}{}{}
\makeatother

% ignore "bad break" warnings by mdframed, they don't seem to make sense
% (they claim that the box after the page break would be empty,
% even if it isn't)
\WarningFilter{mdframed}{You got a bad break}

% define theorem styles (format theorem "head" with sans-serif caps)
\declaretheoremstyle[
  headformat={\formatcaption{\NAME{} \NUMBER}\hspace{0.7em}\NOTE\\},
  notefont={\normalfont\normalsize},
  headpunct={},
]{mythmdefstyle}

% same as before, but italic body font
\declaretheoremstyle[
  headformat={\formatcaption{\NAME{} \NUMBER}\hspace{0.7em}\NOTE\\},
  notefont={\normalfont\normalsize},
  headpunct={},
  bodyfont={\itshape},
]{mythmplainstyle}

% same as before, but no note and line break after head
\declaretheoremstyle[
  headformat={\formatcaption{\NAME{} \NUMBER}\hspace{0.7em}\\},
  notefont={\normalfont\normalsize},
  headpunct={},
  bodyfont={\itshape},
]{mythmshortplainstyle}

% theorem environments
\declaretheorem[
  name={Definition},
  numberwithin=chapter,
  mdframed={style=thmmdfstyle},
  style=mythmdefstyle,
]{definition}
\declaretheorem[
  name={Theorem},
  numberlike=definition,
  mdframed={style=thmmdfstyle},
  style=mythmplainstyle,
]{theorem}
\declaretheorem[
  name={Proposition},
  numberlike=definition,
  mdframed={style=thmmdfstyle},
  style=mythmplainstyle,
]{proposition}
\declaretheorem[
  name={Lemma},
  numberlike=definition,
  mdframed={style=thmmdfstyle},
  style=mythmplainstyle,
]{lemma}
\declaretheorem[
  name={Lemma},
  numberlike=definition,
  mdframed={style=thmmdfstyle},
  style=mythmshortplainstyle,
]{shortlemma}
\declaretheorem[
  name={Corollary},
  numberlike=definition,
  mdframed={style=thmmdfstyle},
  style=mythmplainstyle,
]{corollary}
\declaretheorem[
  name={Corollary},
  numberlike=definition,
  mdframed={style=thmmdfstyle},
  style=mythmshortplainstyle,
]{shortcorollary}

% ======================================================================
% Mathematics
% ======================================================================

% vertically center ":" in ":="
\mathtoolsset{centercolon}

% automatically replace "l" with \ell in math mode
\mathcode`l="8000
\makeatletter
\begingroup
\lccode`\~=`\l
\DeclareMathSymbol{\lsb@l}{\mathalpha}{letters}{`l}
\lowercase{\gdef~{\ifnum\the\mathgroup=\m@ne \ell \else \lsb@l \fi}}%
\endgroup
\makeatother

% change QED symbol to filled square
\renewcommand*{\qedsymbol}{\textcolor{mittelblau}{\blacksquare}}

% redefine own proof environment (prohibits \qedhere)
\renewenvironment{proof}[1][Proof]{%
  \noindent{\formatcaption{#1}\hspace{1em}}%
}{\qed\par}

% ======================================================================
% Blindtext
% ======================================================================

% get name of current label (chapter, section, subsection, ...)
\makeatletter
\newcommand*{\currentname}{\@currentlabelname}
\makeatother

% insert TODO and warn for every usage of \blindtext
\let\oldblindtext\blindtext
\renewcommand*{\blindtext}{%
  \todo{write}
  \textcolor{gray}{\oldblindtext}%
}

% ======================================================================
% Dynamic Commands
% ======================================================================

% calculate difference between page numbers
\newcommand*{\pagedifference}[2]{%
  \number\numexpr\getpagerefnumber{#2}-\getpagerefnumber{#1}\relax%
}

% custom TODO command with warning
% (all packages that were tried produced problems)
\newcommand*{\todo}[1]{%
  \GenericWarning{}{%
    LaTeX Warning (\thesubsection\space\currentname): TODO "#1"%
  }%
  \textcolor{red}{TODO: #1}%
}

% set up versioning information
\luaexec{require("version")}
\edef\gitCommitHash{\luadirect{getGitCommitHash()}}
\edef\gitCommitTimeShort{\luadirect{getGitCommitTimeShort()}}
\edef\gitCommitTimeLong{\luadirect{getGitCommitTimeLong()}}
\edef\currentTimeShort{\luadirect{getCurrentTimeShort()}}
\edef\currentTimeLong{\luadirect{getCurrentTimeLong()}}
\edef\compileCounter{\luadirect{getAndIncreaseCompileCounter()}}

% command for checking if file is in includeonly or not
\makeatletter
\newcommand*{\isincluded}[1]{%
  \@tempswatrue
  \if@partsw
    \@tempswafalse
    \edef\reserved@b{#1}%
    \@for\reserved@a:=\@partlist\do
    {\ifx\reserved@a\reserved@b\@tempswatrue\fi}%
  \fi
  \if@tempswa\expandafter\@firstoftwo\else\expandafter\@secondoftwo\fi
}
\makeatother

% ======================================================================
% Colors
% ======================================================================

% define line colors (mix between MATLAB and matplotlib colors)
\definecolor{C0}{rgb}{0.000,0.447,0.741}
\definecolor{C1}{rgb}{0.850,0.325,0.098}
\definecolor{C2}{rgb}{0.929,0.694,0.125}
\definecolor{C3}{rgb}{0.494,0.184,0.556}
\definecolor{C4}{rgb}{0.466,0.674,0.188}
\definecolor{C5}{rgb}{0.301,0.745,0.933}
\definecolor{C6}{rgb}{0.635,0.078,0.184}
\definecolor{C7}{rgb}{0.887,0.465,0.758}
\definecolor{C8}{rgb}{0.496,0.496,0.496}

% define university CD colors
\definecolor{anthrazit}{RGB}{62,68,76}
\definecolor{mittelblau}{RGB}{0,81,158}
\definecolor{helllblau}{RGB}{0,190,255}

% ======================================================================
% Check Mode
% ======================================================================

% check mode
\iftoggle{checkMode}{
  % show overfull boxes
  \overfullrule=1mm
  % whitelist is in hyphenation_whitelist.txt:
  % one word per line (all lowercase) with dashes (-) indicating the
  % places where hyphenation is permitted
  \LuaCheckHyphen{whitelist=hyphenation_whitelist.txt}
}{}

% ======================================================================
% Flip Book
% ======================================================================

% prevent "Label `1' multiply defined" warnings when using
% KOMA-Script's \ifthispageodd together with \includeonly
\makeatletter
\newcounter{scbookpg}
\renewcommand*{\is@thispageodd}{%
  \@bsphack
  \begingroup
    %\@tempcnta=\scr@tpo
    %\advance\@tempcnta by\@ne
    \stepcounter{scbookpg}%
    \xdef\scr@tpo{\thescbookpg}%
    \protected@write\@auxout{\let\arabic\relax}{%
      \string\new@tpo@label{\scr@tpo}{\arabic{page}}}%
    \expandafter\ifx\csname tpo@\scr@tpo\endcsname\relax
      \protect\G@refundefinedtrue
      \ClassWarning{\KOMAClassName}{%
        odd/even page label number \scr@tpo\space undefined}%
      \edef\@tempa{\the\value{page}}%
    \else
      \edef\@tempa{\csname tpo@\scr@tpo\endcsname}%
    \fi
    \ifodd\number\@tempa
      \aftergroup\thispagewasoddtrue
    \else
      \aftergroup\thispagewasoddfalse
    \fi
  \endgroup
  \@esphack
}
\makeatletter

% set up flip book
\newcounter{currentpagenumber}
\iftoggle{flipBookMode}{
  \iftoggle{partialCompileMode}{}{
    % start flip book after second title page
    \setcounter{currentpagenumber}{-4}
  }
  \newcount\flipbookindex
  
  \AddEverypageHook{%
    \stepcounter{currentpagenumber}%
    \flipbookindex=\numexpr\thecurrentpagenumber/2\relax%
    \ifnum\number\flipbookindex>0%
      \begin{tikzpicture}[remember picture,overlay,scale=1]
        \ifthispageodd{
          \node[anchor=south east,inner sep=0pt,xshift=0mm,yshift=-3mm]
          at (current page.south east) [above left] {%
            \includegraphics[scale=0.8]{flipBookSG_\number\flipbookindex}%
          };
        }{
          \node[anchor=south west,inner sep=0pt,xshift=-6mm,yshift=2mm]
          at (current page.south west) [above right] {%
            \includegraphics{flipBookBSpline_\number\flipbookindex}%
          };
        }
      \end{tikzpicture}%
    \fi%
  }
}{}

% ======================================================================
% Watermarks
% ======================================================================

% set up watermarks
\iftoggle{draftMode}{
  \AddEverypageHook{%
    \begin{tikzpicture}[remember picture,overlay,scale=1]
      \node[
        opacity=0.5,anchor=center,xshift=0mm,yshift=7mm,inner sep=0pt,
      ] at (current page.south) [above] {
        \parbox{100mm}{%
          \begin{center}%
            \onehalfspacing\bfseries\color{black}%
            Draft \texttt{\normalsize{}v\compileCounter{}}
            (\currentTimeShort)\\%
            Commit \texttt{\normalsize{}\gitCommitHash{}}
            (\gitCommitTimeShort)%
          \end{center}%
        }%
      };
      \ifthispageodd{
        \node[
          opacity=0.5,anchor=north east,xshift=-15mm,yshift=-38.5mm,
          inner sep=0pt,text width=4.51mm,align=right,
        ] at (current page.north east) [below left] {%
          \ttfamily\onehalfspacing%
          1\\2\\3\\4\\5\\6\\7\\8\\9\\10\\%
          11\\12\\13\\14\\15\\16\\17\\18\\19\\20\\%
          21\\22\\23\\24\\25\\26\\27\\28\\29\\30\\%
          31\\32\\33\\34\\35\\36\\37\\%
        };
      }{
        \node[
          opacity=0.5,anchor=north west,xshift=15mm,yshift=-38.5mm,color=black,
          inner sep=0pt,text width=4.51mm,align=right,
        ] at (current page.north west) [below right] {%
          \ttfamily\onehalfspacing%
          1\\2\\3\\4\\5\\6\\7\\8\\9\\10\\%
          11\\12\\13\\14\\15\\16\\17\\18\\19\\20\\%
          21\\22\\23\\24\\25\\26\\27\\28\\29\\30\\%
          31\\32\\33\\34\\35\\36\\37\\%
        };
      }
    \end{tikzpicture}%
  }
}{}

% ======================================================================
% PDF Meta-Data and Links
% ======================================================================

% set up hyperref
\hypersetup{
  % set metadata
  pdftitle={\thetitle},
  pdfauthor={\theauthor},
  pdfcreator={LaTeX, KOMA-Script, hyperref},
  % underline links instead of putting a framed box around them
  pdfborderstyle={/S/U/W 1},
  % set link colors
  citebordercolor=C1,
  filebordercolor=C1,
  linkbordercolor=C1,
  menubordercolor=C1,
  runbordercolor=C1,
  urlbordercolor=C0,
  % prepend bookmarks with section number
  bookmarksnumbered,
  % open bookmark tree on start
  bookmarksopen,
}

% include parentheses of \eqref in hyperlink
\makeatletter
\renewcommand*{\eqref}[1]{\hyperref[{#1}]{\textup{\tagform@{\ref*{#1}}}}}
\makeatother

% ======================================================================
% Bibliography
% ======================================================================

% break URLs after every character (most importantly for bibliography)
\luaexec{require("breakurl")}
\let\oldhref\href
\renewcommand*{\url}[1]{\luadirect{breakurl(\luastringN{#1}, \luastringN{#1})}}
\renewcommand*{\href}[2]{\luadirect{breakurl(\luastringN{#1}, \luastringN{#2})}}

% location of *.bib file
\addbibresource{../bib/bibliography.bib}

% don't append "+" sign to BibLaTeX citations in alphabetic style
\renewcommand*{\labelalphaothers}{}

% insert colon after author names in bibliography
\renewcommand*{\labelnamepunct}{: }

% hide urldate field
\AtEveryBibitem{\clearfield{urlyear}}

% make paper titles italic, remove quotation marks
\DeclareFieldFormat[article]{title}{\mkbibemph{#1}}
\DeclareFieldFormat[article]{journaltitle}{#1}
\DeclareFieldFormat[thesis]{title}{\mkbibemph{#1}}
\DeclareFieldFormat[inbook]{title}{\mkbibemph{#1}}
\DeclareFieldFormat[inbook]{booktitle}{#1}
\DeclareFieldFormat[incollection]{title}{\mkbibemph{#1}}
\DeclareFieldFormat[incollection]{booktitle}{#1}
\DeclareFieldFormat[inproceedings]{title}{\mkbibemph{#1}}
\DeclareFieldFormat[inproceedings]{booktitle}{#1}
\DeclareFieldFormat[unpublished]{title}{\mkbibemph{#1}}

% fix title capitalization to sentence case (all lowercase),
\DeclareFieldFormat{titlecase}{\MakeTitleCase{#1}}

% .. but don't change journal titles, book titles, and so on
\newrobustcmd{\MakeTitleCase}[1]{%
  \ifthenelse{%
    \ifcurrentfield{booktitle}\OR\ifcurrentfield{booksubtitle}%
    \OR\ifcurrentfield{maintitle}\OR\ifcurrentfield{mainsubtitle}%
    \OR\ifcurrentfield{journaltitle}\OR\ifcurrentfield{journalsubtitle}%
    \OR\ifcurrentfield{issuetitle}\OR\ifcurrentfield{issuesubtitle}%
    \OR\ifentrytype{book}\OR\ifentrytype{mvbook}\OR\ifentrytype{bookinbook}%
    \OR\ifentrytype{booklet}\OR\ifentrytype{suppbook}%
    \OR\ifentrytype{collection}\OR\ifentrytype{mvcollection}%
    \OR\ifentrytype{suppcollection}\OR\ifentrytype{manual}%
    \OR\ifentrytype{periodical}\OR\ifentrytype{suppperiodical}%
    \OR\ifentrytype{proceedings}\OR\ifentrytype{mvproceedings}%
    \OR\ifentrytype{reference}\OR\ifentrytype{mvreference}%
    \OR\ifentrytype{report}\OR\ifentrytype{thesis}%
  }{%
    #1%
  }{%
    \MakeSentenceCase*{#1}%
  }%
}

% suppress "In:" before journal names
\renewbibmacro*{in:}{}

% separate authors with semicolon, suppress "and" in author names
\renewcommand{\multinamedelim}{\addsemicolon\space}
\renewcommand*{\finalnamedelim}{\addsemicolon\space}

% make names bold
\renewcommand*{\mkbibnamefamily}[1]{\textbf{#1}}

\DeclareNameAlias{sortname}{last-first}
\DeclareNameAlias{default}{last-first}

% define custom heading to fix non-uppercase page header, make URLs smaller
\defbibheading{myheading}[\bibname]{%
  \chapter{#1}%
  \markboth{BIBLIOGRAPHY}{BIBLIOGRAPHY}%
  \let\oldtexttt\texttt%
  \renewcommand*{\texttt}[1]{\oldtexttt{\scriptsize##1}}%
}

% add custom bibliography post note (URLs last checked on ...)
\defbibnote{mypostnote}{%
  All URLs have last been checked on \thedate.%
  \renewcommand*{\texttt}[1]{\oldtexttt{##1}}%
}

% omit "URL:", "DOI:", and "ISBN:" separators to save space,
% link DOIs and ISBNs
\DeclareFieldFormat{url}{\url{#1}}
\DeclareFieldFormat{doi}{%
  \ifhyperref{\href{https://doi.org/#1}{\nolinkurl{#1}}}{\nolinkurl{#1}}%
}
\DeclareFieldFormat{isbn}{%
  \ifhyperref{%
    \href{https://www.amazon.com/s/?field-keywords=#1}{\nolinkurl{#1}}%
  }{\nolinkurl{#1}}%
}

% change font size of bibliography
\renewcommand*{\bibfont}{\small}

% set single line spacing for bibliography
\renewcommand*{\bibsetup}{\singlespacing}

% include brackets in \cite hyperlink
\DeclareCiteCommand{\cite}{\usebibmacro{prenote}}{%
  \usebibmacro{citeindex}%
  \printtext[bibhyperref]{\mkbibbrackets{\usebibmacro{cite}}}%
}{\multicitedelim}{\usebibmacro{postnote}}

% declare special \multicite command which produce [Abc01; Def02]
% (instead of [Abc01]; [Def02])
\DeclareCiteCommand{\multicite}[\mkbibbrackets]{\usebibmacro{prenote}}{%
  \usebibmacro{citeindex}\usebibmacro{cite}%
}{\multicitedelim}{\usebibmacro{postnote}}

% ======================================================================
% Glossary
% ======================================================================

\makeatletter

% \newcommand with variable command names
\newcommand*{\newnamecommand}{\@star@or@long\new@name@command}
\newcommand*{\new@name@command}[1]{\expandafter\new@command\csname #1\endcsname}

\iftoggle{partialCompileMode}{
  % define dummy commands that work as if partialCompileMode was disabled
  \newcommand*{\newnotation}[3]{}
  \newcommand*{\newnotationcommand}[6][0]{\newcommand*{#2}[#1]{#3}}
  \newcommand*{\newnotationcommandoptarg}[7][]{\newcommand*{#3}[#2][#1]{#4}}
  \newcommand*{\makecommandnotation}[4]{}
  \newcommand*{\usenotation}[1]{\ignorespaces}
  
  \newcommand*{\newgacronym}[3][]{%
    \if\relax\detokenize{#1}\relax%
      \newnamecommand{#2}{\MakeUppercase{#2}\xspace}%
      \newnamecommand{#2s}{\MakeUppercase{#2}s\xspace}%
      \newnamecommand{#2c}{\MakeUppercase{#2}\xspace}%
      \newnamecommand{#2sc}{\MakeUppercase{#2}s\xspace}%
    \else%
      \newnamecommand{#2}{#1\xspace}%
      \newnamecommand{#2s}{#1s\xspace}%
      \newnamecommand{#2c}{\makefirstuc{#1}\xspace}%
      \newnamecommand{#2sc}{\makefirstuc{#1}s\xspace}%
    \fi%
  }
  \newcommand*{\hidegacronym}[1]{}
}{
  % define new glossary style based on long (uses longtable)
  % in which the column widths are fixed and space before/after the
  % table is removed
  \newglossarystyle{longraggedfixed}{
    % base glossary style
    \setglossarystyle{long}
    
    % custom flag to check if we're in the glossary
    \newif\ifinglossary
    
    \renewenvironment{theglossary}%
    {%
      % remove space before/after glossary
      \setlength{\LTpre}{0pt}%
      \setlength{\LTpost}{0pt}%
      %
      % set custom flag that we are now in the glossary
      \inglossarytrue%
      %
      % set font size
      \small%
      %
      % use single spacing
      \begin{spacing}{1}%
        % table header
        \begin{longtable}{@{}L{0.15\textwidth}@{}L{0.85\textwidth}@{}}%
          \textbf{Symbol}\vspace{2mm}&%
          \textbf{Meaning\hfill{}Page with First Occurrence}\vspace{2mm}%
          \endhead%
    }%
    {%
      % table footer
        \end{longtable}%
      \end{spacing}%
      % set custom flag that we are now outside the glossary
      \inglossaryfalse%
    }
  
    % add dotted leaders between entry description and page number
    % (called for every row of the table)
    \renewcommand*{\glossentry}[2]{%
      \glsentryitem{##1}\glstarget{##1}{\glossentryname{##1}}&%
      \glossentrydesc{##1}\leavevmode\kern3pt\leaders\hbox{%
        \hspace{0.2237em}.\hspace{0.2237em}%
      }\hfill\kern0pt%
      ##2%
      \tabularnewline%
    }
  }
  
  % use new glossary style
  \setglossarystyle{longraggedfixed}
  
  % add field to glossary entries that stores whether the entry
  % has been used at least once in the document
  \glsaddstoragekey{myused}{false}{\glsmyused}
  \glsaddstoragekey{myisacronym}{false}{\glsmyisacronym}
  
  % \newnotation{sortKey}{entrySymbol}{entryText}
  % adds notation to the glossary without command,
  % only to be used with \usenotation
  \newcommand*{\newnotation}[3]{%
    \newglossaryentry{#1}{text={},sort={#1},name={#2},description={#3}}%
  }
  
  % \newnotationcommand[numberOfArgs]{\commandName}
  %   {commandDefinition}{sortKey}{entrySymbol}{entryText}
  % adds notation to the glossary and creates a command
  \newcommand*{\newnotationcommand}[6][0]{%
    \newcommand*{#2}[#1]{#3}%
    \makecommandnotation{#2}{#4}{#5}{#6}%
  }
  
  % \newnotationcommandoptarg[defaultValue]{numberOfArgs}{\commandName}
  %   {commandDefinition}{sortKey}{entrySymbol}{entryText}
  % is like \newnotationcommand, but with an optional argument for
  % the command to be created
  \newcommand*{\newnotationcommandoptarg}[7][]{%
    \newcommand*{#3}[#2][#1]{#4}%
    \makecommandnotation{#3}{#5}{#6}{#7}%
  }
  
  % \makecommandnotation{\commandName}{sortKey}{entrySymbol}{entryText}
  % adds notation to the glossary and modifies \commandName such that
  % the glossary entry is referenced on every use of \commandName
  % (except if used in the glossary itself)
  \newcommand*{\makecommandnotation}[4]{%
    \newglossaryentry{#2}{text={},sort={#2},name={#3},description={#4}}%
    % only reference entry if outside glossary
    % (otherwise, the page number of the glossary would be displayed)
    \pretocmd{#1}{\ifinglossary\else\usenotation{#2}\fi}{}{}%
  }
  
  % \usenotation{sortKey}
  % explicitly "uses" the notation given by sortKey without printing anything
  \newcommand*{\usenotation}[1]{%
    \glsfieldgdef{#1}{myused}{true}%
    \glsdisp{#1}{\relax}\ignorespaces%
  }
  
  % shortcut for \newacronym, automatically generating
  % a corresponding shortcut for \gls
  \newcommand*{\newgacronym}[3][]{%
    \if\relax\detokenize{#1}\relax%
      \newacronym[sort={Ü#2},description={\makefirstuc{#3}}]%
      {Ü#2}{\MakeUppercase{#2}}{#3}%
    \else%
      \newacronym[sort={Ü#2},description={\makefirstuc{#3}}]{Ü#2}{#1}{#3}%
    \fi%
    \glsfieldgdef{Ü#2}{myisacronym}{true}%
    \newnamecommand{#2}{\glsfieldgdef{Ü#2}{myused}{true}\gls{Ü#2}\xspace}%
    \newnamecommand{#2s}{\glsfieldgdef{Ü#2}{myused}{true}\glspl{Ü#2}\xspace}%
    \newnamecommand{#2c}{\glsfieldgdef{Ü#2}{myused}{true}\Gls{Ü#2}\xspace}%
    \newnamecommand{#2sc}{\glsfieldgdef{Ü#2}{myused}{true}\Glspl{Ü#2}\xspace}%
  }
  
  % hide specific acronyms (that are only used once) in glossary
  \newignoredglossary{hidden}
  \newcommand*{\hidegacronym}[1]{\glsmoveentry{Ü#1}{hidden}}
  
  % spell out acronyms at beginning of chapters and sections
  \newcommand*{\resetacronyms}{%
    \forglsentries{\mysymbolname}{%
      \ifthenelse{\equal{\glsmyisacronym{\mysymbolname}}{true}}{
        \glsreset{\mysymbolname}%
      }{}%
    }%
  }
  
  \addtokomafont{chapter}{\resetacronyms}
  \addtokomafont{section}{\resetacronyms}
  
  % warn for every symbol that has not been used
  \AtEndDocument{%
    \forglsentries{\mysymbolname}{%
      \ifthenelse{\equal{\glsmyused{\mysymbolname}}{false}}{
        \GenericWarning{}{%
          LaTeX Warning: Symbol ``\mysymbolname'' defined, but unused%
        }%
      }{}%
    }%
  }
  
  % generate table of symbols and acronyms
  \makeglossaries
}

% add debug info about definitions of glossary entries
\iftoggle{showGlossaryDefinitionsMode}{
  \let\oldnewgsymbol\newgsymbol
  \renewcommand*{\newgsymbol}[3]{%
    \ifx\@onlypreamble\@notprerr%
      \textcolor{C1}{Defining ``#2'' as ``#3''. }%
    \fi%
    \oldnewgsymbol{#1}{#2}{#3}%
  }
}{}

\makeatother
