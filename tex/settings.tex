% don't use sans-serif font for headings
\setkomafont{disposition}{\normalcolor\bfseries}

% don't use sans-serif font for dictums
\addtokomafont{dictum}{\rmfamily}

% make dictums larger, remove dictum rule (led to errors),
% align text on the right
\renewcommand*{\dictumwidth}{0.6667\textwidth}
\renewcommand*{\dictumrule}{}
\renewcommand*{\raggeddictumtext}{\raggedleft}

% small-caps page header
\setkomafont{pageheadfoot}{\normalfont\normalcolor\scshape}

% recalculate page margins as the line spread has changed
\recalctypearea

% location of graphics files
\graphicspath{{../gfx/}}

% chapter heading: make number bigger
\renewcommand*{\chapterformat}{%
  \mbox{%
    \chapappifchapterprefix{\nobreakspace}%
    \scalebox{3.5}{\thechapter\autodot}%
    \IfUsePrefixLine{}{\enskip}%
  }%
}

% chapter heading: align number and text at the baseline of the last line
% of the text, add some more space between number and text
\renewcommand*{\chapterlinesformat}[3]{%
  \begin{tabularx}{\textwidth}{@{}l@{}X@{}}%
    % chapter number
    #2%
    % add hspace between number and text, but only if \chapter (not \chapter*)
    % has been used (i.e., if the chapter number #2 is not empty)
    \if\relax\detokenize{#2}\relax\else\hspace*{5mm}\fi&%
    % text of heading, [b] is for alignment, \linewidth is width of X cell,
    % \raggedchapter for not justifying headings
    \parbox[b]{\linewidth}{\raggedchapter\setstretch{1.05}#3}%
  \end{tabularx}%
}

% check mode
\iftoggle{checkmode}{
  % show overfull boxes
  \overfullrule=1mm
  % show underfull vboxes
  \directlua{dofile("detect_underfull.lua")}
  % whitelist is in hyphenation_whitelist.txt:
  % one word per line with dashes (-) indicating the places where
  % hyphenation is whitelisted
  \LuaCheckHyphen{whitelist=hyphenation_whitelist.txt}
}{}

% location of *.bib file
\addbibresource{../bib/bibliography.bib}

% meta-data variables
\newcommand*{\thetitle}{%
  B-Splines for Sparse Grids:\texorpdfstring{\\}{}
  Algorithms and Application to
  Higher-Dimensional Optimization%
}
\newcommand*{\theauthor}{Julian Valentin}
\newcommand*{\thedate}{TODO(date)}

% set up hyperref
\hypersetup{
  pdftitle={\thetitle},
  pdfauthor={\theauthor},
  pdfcreator={LaTeX, KOMA-Script, hyperref},
}

% change bibliography heading (default: "Bibliography")
\DefineBibliographyStrings{english}{bibliography={References}}

% change font size of bibliography
\renewcommand*{\bibfont}{\small}

% define dummy glossary
% (Needed for SCons because it automatically executes makeindex
% twice; once for *.glo and once for *.acn. *.glo is the default
% one and we don't have a *.acn, so we have to generate an empty one.)
\newglossary[alg]{dummyglossary}{acr}{acn}{Empty Dummy Glossary}

% define new glossary style based on long (uses longtable)
% in which the column widths are fixed and space before/after the
% table is removed
\newglossarystyle{longfixed}{%
  \setglossarystyle{long}%
  \renewenvironment{theglossary}%
    {%
      \setlength{\LTpre}{0pt}%
      \setlength{\LTpost}{0pt}%
      \begin{longtable}{@{}p{0.2\textwidth}@{}p{0.8\textwidth}@{}}}%
    {\end{longtable}}%
}

% use new glossary style
\setglossarystyle{longfixed}

% generate table of symbols and acronyms
\makeglossaries
