% use new right page
\cleardoublepage

% set page margins
\newgeometry{
  bindingoffset=\bindingoffset,
  inner=20mm,
  outer=40mm,
  top=35mm,
  bottom=35mm,
}

% don't display page number
\thispagestyle{empty}

\begin{spacing}{1}
  % no indentation of paragraphs
  \setlength{\parindent}{0pt}
  
  {\LARGE\itshape\theauthor\par}
  
  \vspace{4em}
  
  {%
    \fontsize{40}{48}\selectfont\sffamily\bfseries%
    \hspace*{-0.1em}\textls*{B-SPLINES FOR}\\%
    \hspace*{-0.1em}\textls*{SPARSE GRIDS}%
    \par%
  }
  
  \vspace{2em}
  
  {\LARGE Algorithms and Application to\\Higher-Dimensional Optimization\par}
  
  \vfill
  
  \begin{center}
    \includegraphics[height=90mm]{title_1}
  \end{center}
  
  \vfill
  
  {%
    \raisebox{-0.6\height}{%
      \includegraphics[height=12mm]{logoUniversityEnglishGray}%
    }\hfill%
    \raisebox{-0.5\height}{%
      \includegraphics[height=12mm]{logoSimTechGray}%
    }\hfill%
    \raisebox{-0.5\height}{%
      \includegraphics[height=13mm]{logoIPVSGray}%
    }\hfill%
    \raisebox{-0.5\height}{%
      \includegraphics[height=15mm]{logoSGSGray}%
    }\par%
  }
\end{spacing}

% restore original margin sizes
\restoregeometry

\pagebreak

% don't display page number
\thispagestyle{empty}

\vspace*{\fill}

\begin{minipage}{0.865\textwidth}%
  \begin{center}
    \begin{minipage}{0.65\textwidth}%
      \begin{flushleft}
        {\hugequote}%
        \textit{%
          It seems that it is not enough to have a good idea or insight.
          One needs, like Schoenberg, the appreciation and courage to
          develop the idea systematically, make its objects mathematically
          presentable by giving them names, and give them much exposure in
          many papers.%
        }
      \end{flushleft}
      \begin{flushright}
        \small--- Carl de~Boor \cite{Boor16Comment}
      \end{flushright}
    \end{minipage}%
  \end{center}
\end{minipage}

\vspace*{\fill}

{
  % no indentation of paragraphs
  \setlength{\parindent}{0pt}%
  \small
  
  % cover figure caption with special name and label format (without number)
  \captionsetup[figure]{
    name={Cover Figure},
    labelformat=mycaptionlabelformatunnumbered,
    hypcap=false,
  }
  \captionof{figure}[]{%
    A regular sparse grid in two dimensions \emph{(dots)} as a subset
    of the full grid \emph{(mesh, bottom)}
    with a bicubic B-spline \emph{(mesh, top)}.%
  }

  \vspace{1em}
  
  % version information
  \begin{tabular}{@{}l@{}l@{}}
    Compiled as version \texttt{v\compileCounter{}}%
    \hphantom*{~}&on \currentTimeLong.\\
    Committed as \texttt{\gitCommitHash{}}%
    \hphantom*{~}&on \gitCommitTimeLong.
  \end{tabular}%
  
  \vspace{1em}
  
  \begin{tabular}{%
    @{}p{0.2\textwidth}@{}p{0.8\textwidth}@{}%
  }
    \includegraphics[height=10mm]{licenseBadge}&
    \raisebox{3.8mm}{%
      \parbox{\linewidth}{%
        Copyright \copyright{} \theyear{} \theauthor{}.
        This work is licensed under the
        \term{%
          \href{https://creativecommons.org/licenses/by-sa/4.0/}{%
            Creative Commons Attribution-ShareAlike 4.0
            International License%
          }%
        }.%
      }%
    }
  \end{tabular}
  
  \vspace{1em}
  
  Although this thesis was written with uttermost care,
  it cannot be ruled out that it contains errors.
  Please send any corrections and mistakes to
  \href{mailto:thesis@bsplines.org}{\texttt{thesis@bsplines.org}}.
}
