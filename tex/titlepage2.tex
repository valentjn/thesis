% use new right page
\cleardoublepage

% don't display page number
\thispagestyle{empty}

\begin{spacing}{1}
  % no indentation of paragraphs
  \setlength{\parindent}{0pt}
  
  \vspace*{2em}
  
  {\LARGE\itshape\theauthor\par}
  
  \vspace{4em}
  
  {%
    \fontsize{40}{48}\selectfont\sffamily\bfseries%
    \hspace*{-0.1em}\textls*{B-SPLINES FOR}\\%
    \hspace*{-0.1em}\textls*{SPARSE GRIDS}%
    \par%
  }
  
  \vspace{2em}
  
  {\LARGE Algorithms and Application to\\Higher-Dimensional Optimization\par}
  
  \vfill
  
  \begin{center}
    \includegraphics[height=80mm]{title_1}
  \end{center}
  
  \vfill
  
  {%
    \raisebox{-0.6\height}{\includegraphics[height=12mm]{logo_us_english_gray}}\hfill%
    \raisebox{-0.5\height}{\includegraphics[height=12mm]{logo_simtech_gray}}\hfill%
    \raisebox{-0.5\height}{\includegraphics[height=13mm]{logo_ipvs_gray}}\hfill%
    \raisebox{-0.5\height}{\includegraphics[height=15mm]{logo_sgs_gray}}\par%
  }
\end{spacing}

\pagebreak

% don't display page number
\thispagestyle{empty}

\vspace*{\fill}

\begin{minipage}{0.865\textwidth}%
  \begin{center}
    \begin{minipage}{0.65\textwidth}%
      \begin{flushleft}
        {\hugequote}%
        \textit{%
          It seems that it is not enough to have a good idea or insight.
          One needs, like Schoenberg, the appreciation and courage to develop the idea
          systematically, make its objects mathematically presentable by giving them
          names, and give them much exposure in many papers.%
        }
      \end{flushleft}
      \begin{flushright}
        \small--- Carl de Boor \cite{boor16}
      \end{flushright}
    \end{minipage}%
  \end{center}
\end{minipage}

\vspace*{\fill}

\begin{tabular}{@{}l@{}l@{}}
  Compiled as version \getAndIncreaseCompileCounter{}\hphantom*{~}&on \currentTimeLong\\
  Committed as \gitCommitHash{}\hphantom*{~}&on \gitCommitTimeLong
\end{tabular}
