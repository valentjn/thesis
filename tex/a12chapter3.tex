\section{Proofs for Chapter 3}

\subsection{%
  Proof of \texorpdfstring{%
    \Cref{prop:hierBSplineLinearlyIndependent}%
  }{%
    Proposition \ref{prop:hierBSplineLinearlyIndependent}%
  }%
}
\label{sec:proofHierBSplineLinearlyIndependent}

\propHierBSplineLinearlyIndependent*

\begin{proof}
  The proof is rigorous for the common low B-spline degrees of
  $p \in \{1, 3, 5, 7\}$.
  For higher degrees, the proof has to be viewed as a sketch.
  
  We follow the presentation in \cite{Valentin16Hierarchical} and
  prove the assertion by induction over $l \in \NN_0$.
  For $l = 0$, the B-splines $\varphi_{0,i'}^p$ with $i' \in \{0, 1\}$
  are linearly independent.
  For the induction step $(l-1) \to l$, let
  \begin{equation}
    \sum_{l'=0}^l \sum_{i' \in I_{l'}} \alpha_{l',i'} \varphi_{l',i'}^p
    \equiv 0
  \end{equation}
  be a linear combination of the zero function.
  We separate the summands of level $l$
  from the summands of coarser levels $l' < l$:
  \begin{equation}
    \label{eq:hierBSplineLinearlyIndependent3}
    \sum_{i \in I_l} \alpha_{l,i} \varphi_{l,i}^p
    =: g_1 \equiv g_2 :=
    -\sum_{l'=0}^{l-1} \sum_{i' \in I_{l'}} \alpha_{l',i'} \varphi_{l',i'}^p.
  \end{equation}
  The right-hand side $g_2$ is smooth in every grid point
  $x_{l,i}$ ($i \in I_l$) of level $l$,
  since these grid points are not knots of the hierarchical B-splines
  $\varphi_{l',i'}^p$ ($i' \in I_{l'}$) of level $l' < l$.
  This implies that the left-hand $g_1$ side must be smooth there as well:
  \begin{alignat}{3}
    &&
    \partialdiff_-^p g_1(x_{l,i'})
    &= \partialdiff_+^p g_1(x_{l,i'}),\quad
    &&i' \in I_l,\\
    \label{eq:hierBSplineLinearlyIndependent1}
    \iff\quad&&
    \sum_{i \in I_l} \alpha_{l,i}
    \partialdiff_-^p \varphi_{l,i}^p(x_{l,i'})
    &= \sum_{i \in I_l} \alpha_{l,i}
    \partialdiff_+^p \varphi_{l,i}^p(x_{l,i'}),\quad
    &&i' \in I_l,
  \end{alignat}
  where $\partialdiff_-^p$ and $\partialdiff_+^p$ denote the left and right
  derivative of order $p$, respectively.
  By repeated application of \eqref{eq:cardinalBSplineDerivative},
  one can show
  \begin{equation}
    \partialdiff_-^p b^p(k + 1)
    = (-1)^k \binom{p}{k}
    = \partialdiff_+^p b^p(k),\quad
    k \in \ZZ,
  \end{equation}
  where $\binom{p}{k} = 0$ for $k < 0$ or $k > p$
  \cite{Hoellig13Approximation}.
  We can insert this relation into
  \eqref{eq:hierBSplineLinearlyIndependent1}
  and use \eqref{eq:uniformHierarchicalBSplineUV} to obtain
  \begin{alignat}{2}
    \sum_{i \in I_l} \alpha_{l,i} (-1)^{k-1} \binom{p}{k-1}
    &= \sum_{i \in I_l} \alpha_{l,i} (-1)^k \binom{p}{k},\quad
    &&i' \in I_l,\quad
    k := \frac{p+1}{2} + i' - i.
  \end{alignat}
  As $\binom{p}{k-1} + \binom{p}{k} = \binom{p+1}{k}$
  and $(-1)^k$ is constant for $i \in I_l$ when $i'$ is fixed,
  this is equivalent to
  \begin{equation}
    \label{eq:hierBSplineLinearlyIndependent2}
    \sum_{i \in I_l} \alpha_{l,i}
    \binom{p+1}{\frac{p+1}{2} + i' - i} = 0,\quad
    i' \in I_l.
  \end{equation}
  
  This is a quadratic linear system of equations
  whose system matrix
  $A(p)$ is a banded symmetric Toeplitz%
  \footnote{%
    A Toeplitz matrix is a matrix $A$ whose entries $a_{k,j}$
    solely depend on $k - j$, i.e.,
    $a_{k,j} = c_{k-j}$ for some vector $\ßc$.%
  }
  matrix of size
  $2^{l-1} \times 2^{l-1}$ with bandwidth $\lceil\tfrac{p-1}{4}\rceil$.
  %For $p = 1, 3, 5, 7$, the top rows $a(p)$ of $A(p)$,
  %which fully determines a symmetric Toeplitz matrix, is given as follows:
  %\begin{align}
  %  a(1) &= (\hphantom{7}2, \hphantom{2}0, 0, 0, \dotsc, 0),\\
  %  a(3) &= (\hphantom{7}6, \hphantom{2}1, 0, 0, \dotsc, 0),\\
  %  a(5) &= (20, \hphantom{2}6, 0, 0, \dotsc, 0),\\
  %  a(7) &= (70, 28, 1, 0, \dotsc, 0).
  %\end{align}
  %For $p = 1, 3, 5, 7$, $A(p)$ is as follows:
  %\begin{equation}
  %  \begin{gathered}
  %    A(1) =
  %    \begin{pmatrix}
  %      2&&\\
  %      &\ddots&\\
  %      &&2
  %    \end{pmatrix},\quad
  %    A(3) =
  %    \begin{pmatrix}
  %      6&1&&\\
  %      1&\ddots&\ddots\\
  %      &\ddots&\ddots&1\\
  %      &&1&6
  %    \end{pmatrix},\\
  %    A(5) =
  %    \begin{pmatrix}
  %      20&6&&\\
  %      6&\ddots&\ddots\\
  %      &\ddots&\ddots&6\\
  %      &&6&20
  %    \end{pmatrix},\quad
  %    A(7) =
  %    \begin{pmatrix}
  %      70&28&1&&\\
  %      28&\ddots&\ddots&\ddots&\\
  %      1&\ddots&\ddots&\ddots&1\\
  %      &\ddots&\ddots&\ddots&28\\
  %      &&1&28&70
  %    \end{pmatrix},
  %  \end{gathered}
  %\end{equation}
  The non-zero values of $A(p)$ are tabulated for $p = 1, 3, 5, 7$
  in \cref{tbl:proofHierBSplineLinearlyIndependent}.
  The corresponding matrices are diagonally dominant and therefore regular.
  For higher B-spline degrees $p$, the regularity of $A(p)$ has
  to be shown differently.
  
  Due to the regularity of $A(p)$, we infer from
  \eqref{eq:hierBSplineLinearlyIndependent2} that
  $\alpha_{l,i'} = 0$ for $i' \in I_l$.
  Due to \eqref{eq:hierBSplineLinearlyIndependent3},
  we obtain
  a linear combination of the zero function with the hierarchical
  B-splines of level $< l$, i.e.,
  \begin{equation}
  \sum_{l'=0}^{l-1} \sum_{i' \in I_{l'}} \alpha_{l',i'} \varphi_{l',i'}^p
  = 0,
  \end{equation}
  which implies $\alpha_{l',i'} = 0$ for all $l' = 0, \dotsc, l - 1$
  by the induction hypothesis.
  Thus, the hierarchical functions $\varphi_{l',i'}^p$
  ($l' \le l$, $i' \in I_{l'}$) are linearly independent.
\end{proof}

\begin{table}
  \begin{tabular}{l@{\hspace{7mm}}ccc}
    \toprule
    &$k = 0$&$k = 1$&$k = 2$\\
    \midrule
    $p = 1$&2&&\\
    $p = 3$&6&1&\\
    $p = 5$&20&6&\\
    $p = 7$&70&28&1\\
    \bottomrule
  \end{tabular}%
  \caption{%
    Non-zero values $(A(p))_{j,j+k}$ of the diagonals of $A(p)$
    obtained in \eqref{eq:hierBSplineLinearlyIndependent2}.%
  }%
  \label{tbl:proofHierBSplineLinearlyIndependent}
\end{table}
