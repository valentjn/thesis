% convention for sort keys:
% @ --> digits
% Ä --> Greek letters
% Ë --> superscripts
% Ö --> symbols that are not letters or digits
% Ü --> acronyms

\newcommand*{\ifempty}[2]{\if\relax\detokenize{#1}\relax#2\fi}
\newcommand*{\ifnotempty}[2]{\if\relax\detokenize{#1}\relax\else#2\fi}
\newcommand*{\appendwithcomma}[1]{\ifnotempty{#1}{,#1}}

% ======================================================================
% Number Sets
% ======================================================================

\newnotationcommand{\nat}{\mathbb{N}}{N30}{$\nat$}{
  Natural numbers without zero ($1, 2, 3, \dotsc$)
}

\newnotationcommand{\natz}{\mathbb{N}_0}{N40}{$\natz$}{
  Natural numbers with zero ($\nat \cup \{0\}$)
}

\newnotationcommand{\integer}{\mathbb{Z}}{Z}{$\integer$}{
  Integer numbers ($\dotsc, -2, -1, 0, 1, 2, \dotsc$)
}

\newnotationcommand{\real}{\mathbb{R}}{R}{$\real$}{
  Real numbers
}

\newnotationcommand{\posreal}{\mathbb{R}_{{} > 0}}{R>0}{$\posreal$}{
  Positive real numbers
}

\newnotationcommand{\nonnegreal}{\mathbb{R}_{{} \ge 0}}{R>=0}{$\nonnegreal$}{
  Non-negative real numbers
}

% ======================================================================
% Basic Single-Letter Symbols (Level, Index, ...)
% ======================================================================

\newnotation{l}{$\*l$}{
  Hierarchical level $\*l \in \natz^d$
}

\newnotation{i}{$\*i$}{
  Hierarchical index $\*i = \*0, \dotsc, \*2^\*l$
  (i.e., $\*i \in \natz^d$ with $0 \le i_t \le 2^{l_t}$
  for all $t = 1, \dotsc, d$)
}

\newnotation{k}{$\*k$}{
  Continuously enumerated index $\in \natz^d$
  of the hierarchical level-index pairs $(\*l, \*i) \in \liset$
  %(one possible choice is that $k_t$ the zero-based index of $(l_t, i_t)$ in
  %$(0,0), (0,1), (1,1), (2,1), (2,3), (3,1), (3,3), (3,5), (3,7), \dotsc$)
}

\newnotation{d}{$d$}{
  Dimensionality $\in \nat$
}

\newnotation{n10}{$n$}{
  Level $\in \natz$ of full or sparse grid
}

\newnotationcommand{\ngp}{N}{N20}{$\ngp$}{
  Number of grid points
  for a finite set $\sgset \subset \clint{\*0, \*1}$ of grid points
}

\newnotationcommandoptarg{1}{\vlinin}{%
  \*u\ifnotempty{#1}{^{#1}}}{u}{$\vlinin$%
}{
  Input of the linear operator $\linop$
}

\newcommand*{\linin}[2][]{u_{#2}\ifnotempty{#1}{^{#1}}}

\newnotationcommandoptarg{1}{\vlinout}{%
  \*y\ifnotempty{#1}{^{#1}}}{y}{$\vlinout$%
}{
  Output of the linear operator $\linop$
}

\newcommand*{\linout}[2][]{y_{#2}\ifnotempty{#1}{^{#1}}}

% ======================================================================
% Vector Notation
% ======================================================================

\newnotation{@0}{$\*0$}{
  $(0, \dotsc, 0) \in \natz^d$
}

\newnotation{@1}{$\*1$}{
  $(1, \dotsc, 1) \in \nat^d$
}

\newnotation{kt10}{$\*k_{-t}$}{
  Vector of all dimensions except the $t$-th
  %($\*k_{-t}
  %:= (k_1, \dotsc, k_{t-1}, k_{t+1}, \dotsc, k_d)$)
}

\newnotation{kT20}{$\*k_T$}{
  Vector of all dimensions that are contained in $T \in \{1, \dotsc, d\}^j$
  %($\*k_T
  %:= (k_{t_1}, \dotsc, k_{t_j}$ with $T = (t_1, \dotsc, t_j)$)
}

\newnotation{kT30}{$\*k_{-T}$}{
  Vector of all dimensions that are not contained in $T \in \{1, \dotsc, d\}^j$
  %($\*k_{-T} := \*k_{T'}$ for $T = (t_1, \dotsc, t_j)$ and
  %$T' = (t'_1, \dotsc, t'_{d-j})$ with
  %$\{t_1, \dotsc, t_j\} \cup \{t'_1, \dotsc, t'_{d-j}\} = \{1, \dotsc, d\}$ and
  %$t'_1 < \dotsb t'_{d-j}$)
}

\newnotationcommand[2]{\range}{{{#1}:{#2}}}{abrange}{$\*k_{\range{a}{b}}$}{
  Vector of all dimensions that are contained in $a, a + 1, \dotsc, b$
}

\newnotationcommand[1]{\stdbasis}{\*e_{#1}}{et}{$\stdbasis{t}$}{
  $t$-th standard basis vector $:= (0, \dotsc, 0, 1, 0, \dotsc, 0) \in \real^d$
}

\newnotationcommandoptarg[\*k]{2}{\chain}{#1^{(#2)}}{kt}{$\chain{t_{j'}}$}{
  Point on chain $(\chain{t_1}, \dotsc, \chain{t_j})$
  between $\*k'$ and $\*k''$
}

\newcommand*{\chainuv}[1]{\chain[k]{#1}}

% ======================================================================
% Grid Points
% ======================================================================

\newnotationcommand[1]{\gp}{%
  \containsvector{#1}{\*x}{x}_{#1}%
}{xli}{$\gp{\*l,\*i}$}{
  Grid point $\gp{\*l,\*i} := \*i \cdot \ms{\*l}$
}

\newcommand*{\vgp}[1]{\*x_{#1}}

\newnotationcommand[1]{\ccgp}{%
  \containsvector{#1}{\*x}{x}_{#1}^\cc%
}{xlicc}{$\ccgp{\*l,\*i}$}{
  Clenshaw--Curtis grid point
}

% ======================================================================
% Mesh Size
% ======================================================================

\newnotationcommand[1]{\ms}{%
  \containsvector{#1}{\*h}{h}_{#1}%
}{hl}{$\ms{\*l}$}{
  Mesh size $\ms{\*l} := \*2^{-\*l}$
}

% ======================================================================
% Superscripts
% ======================================================================

\newcommand*{\cc}{\mathrm{cc}}
\newcommand*{\modified}{\mathrm{mod}}
\newcommand*{\ntrl}{\mathrm{nat}}
\newcommand*{\nak}{\mathrm{nak}}
\newcommand*{\sparse}{\mathrm{s}}
\newcommand*{\ct}{\mathrm{ct}}
\newcommand*{\hft}{\mathrm{hft}}
\newcommand*{\tift}{\mathrm{tift}}
\newcommand*{\ext}{\mathrm{ext}}
\newcommand*{\fs}{\mathrm{fs}}
\newcommand*{\wfs}{\mathrm{wfs}}
\newcommand*{\chol}{\mathrm{chol}}
\newcommand*{\reference}{\mathrm{ref}}
\newcommand*{\opt}{\mathrm{opt}}

\newnotation{Ë1}{$\cdot^1$}{
  Superscript for ``Piecewise linear''
  %(basis function/function space/interpolant)''
}

\newnotation{Ëp}{$\cdot^p$}{
  Superscript for ``B-splines of degree $p$''
  %(basis function/function space/interpolant)''
}

\newnotation{Ës}{$\cdot^\sparse$}{
  Superscript for ``Sparse grid''
  %(grid point set/function space/interpolant)''
}

\newnotation{Ësb}{$\cdot^{\sparse(b)}$}{
  Superscript for ``Sparse grid with boundary parameter $b$''
  %(grid point set/function space/interpolant)''
}

\newnotation{Ëmod}{$\cdot^\modified$}{
  Superscript for ``Modified''
  %(basis function/function space/interpolant)
  %for grids without boundary points''
}

\newnotation{Ëcc}{$\cdot^\cc$}{
  Superscript for ``Clenshaw--Curtis''
  %(grid point/grid point set/basis
  %function/function space/interpolant)''
}

\newnotation{Ënak}{$\cdot^\nak$}{
  Superscript for ``Not-a-knot''
  %(knot sequence/basis function/function space/interpolant)''
}

\newnotation{Ënat}{$\cdot^\ntrl$}{
  Superscript for ``Natural''
  %(basis function/function space/interpolant)''
}

\newnotation{Ëhft}{$\cdot^\hft$}{
  Superscript for ``Hierarchical fundamental transformation''
  %(basis function/function space/interpolant)''
}

\newnotation{Ëtift}{$\cdot^\tift$}{
  Superscript for ``Translation-invariant fundamental transformation''
  %(basis function/function space/interpolant)''
}

\newnotation{Ëfs}{$\cdot^\fs$}{
  Superscript for ``Fundamental spline''
  %(basis function/function space/interpolant)''
}

\newnotation{Ëwfs}{$\cdot^\wfs$}{
  Superscript for ``Weakly fundamental spline''
  %(basis function/function space/interpolant)''
}

\newnotation{Ëchol}{$\cdot^\chol$}{
  Superscript for ``Cholesky''
}

\newnotation{Ëref}{$\cdot^\reference$}{
  Superscript for ``Reference solution''
}

\newnotation{Ëopt}{$\cdot^\opt$}{
  Superscript for ``Optimal''
}

% ======================================================================
% Intervals
% ======================================================================

\newnotationcommand[1]{\clint}{[#1]}{Öab10}{$\clint{\*a, \*b}$}{
  Closed hyper-rectangle
  $:= \clint{a_1, b_1} \times \dotsb \times \clint{a_d, b_d}$
  with $\clint{a_t, b_t} := \{x_t \in \real \mid a_t \le x_t \le b_t\}$
}

\newnotationcommand[1]{\opint}{\mathopen]#1\mathclose[}{Öab20}{%
  $\opint{\*a, \*b}$%
}{
  Open hyper-rectangle
  $:= \opint{a_1, b_1} \times \dotsb \times \opint{a_d, b_d}$
  with $\opint{a_t, b_t} := \{x_t \in \real \mid a_t < x_t < b_t\}$
}

\newnotationcommand[1]{\hopint}{\mathopen[#1\mathclose[}{Öab30}{%
  $\hopint{\*a, \*b}$%
}{
  Half-open hyper-rectangle
  $:= \hopint{a_1, b_1} \times \dotsb \times \hopint{a_d, b_d}$
  with $\hopint{a_t, b_t} := \{x_t \in \real \mid a_t \le x_t < b_t\}$
}

\hidenextnotation
\newnotationcommand[1]{\pohint}{\mathopen]#1\mathclose]}{Öab40}{%
  $\pohint{\*a, \*b}$%
}{
  Half-open hyper-rectangle
  $:= \pohint{a_1, b_1} \times \dotsb \times \pohint{a_d, b_d}$
  with $\pohint{a_t, b_t} := \{x_t \in \real \mid a_t < x_t \le b_t\}$
}

\newcommand*{\clintscaled}[1]{\left[#1\right]}
\newcommand*{\opintscaled}[1]{\left]#1\right[}
\newcommand*{\hopintscaled}[1]{\left[#1\right[}
\newcommand*{\pohintscaled}[1]{\left]#1\right]}

% ======================================================================
% Norms
% ======================================================================

\newnotationcommand[1]{\normone}{\norm[1]{#1}}{Önorm1}{$\normone{\cdot}$}{
  $\ell_1$ norm $\normone{\*x} := \sum_{t=1}^d x_t$
}

\newcommand*{\Ltwo}{L^2}

\newnotationcommand[1]{\normLtwo}{\norm[\Ltwo]{#1}}{ÖnormL2}{%
  $\normLtwo{\cdot}$%
}{
  $\Ltwo$ norm $\normLtwo{\objfun} := \sqrt{\int_\Omega \objfun(x)^2 \dx}$
  of a function $\objfun\colon \Omega \to \real$
}

\newcommand*{\normLtwoscaled}[1]{\normscaled[\Ltwo]{#1}}

% ======================================================================
% Relation Symbols
% ======================================================================

\let\equivorig\equiv
\let\equiv\undefined
\newnotationcommand{\equiv}{\equivorig}{Öequiv}{$\equivorig$}{
  Equality of functions everywhere on their domain
  (e.g., $\objfun \equiv 0$ means $\objfun(x) = 0$ for all feasible $x$)
}

% ======================================================================
% Grid Sets
% ======================================================================

\newcommand*{\gridset}[3][]{%
  \if\relax\detokenize{#1}\relax\Omega\else#1{\Omega}\fi_{#2}^{#3}%
}

\newcommand*{\interiorgrid}[1]{\mathring{#1}}
\newcommand*{\bndrygrid}[1]{\mathop{}\!\partial#1}

\newcommand*{\interiorsgset}{\gridset[\interiorgrid]{}{\sparse}}
\newcommand*{\bndrysgset}{\gridset[\bndrygrid]{}{\sparse}}

\makecommandnotation{\interiorgrid}{ÄΩso}{$\interiorsgset$}{
  Set $:= \sgset \cap \opint{\*0, \*1}$ of
  interior grid points
  for a finite set $\sgset \subset \clint{\*0, \*1}$ of grid points
}

\let\bndrygridorig\bndrygrid
\newcommand*{\bndrysgsetorig}{\gridset[\bndrygridorig]{}{\sparse}}
\makecommandnotation{\bndrygrid}{ÄΩsd}{$\bndrysgsetorig$}{
  Set $:= \sgset \cap \bndrydomain{\clint{\*0, \*1}}
  = \sgset \setminus \interiorsgset$ of
  boundary sparse grid points
  for a finite set $\sgset \subset \clint{\*0, \*1}$ of grid points
}

\newnotationcommandoptarg{2}{\fgset}{%
  \Omega_{#2}\ifnotempty{#1}{^{#1}}%
}{ÄΩl}{$\fgset{\*l}$}{
  Set of full grid points of level $\*l$
}

\newnotationcommandoptarg{1}{\sgset}{%
  \gridset{}{\sparse\appendwithcomma{#1}}%
}{ÄΩs}{$\sgset$}{
  Arbitrary sparse grid (possibly spatially adaptive)
}

\newnotationcommandoptarg{3}{\regsgset}{%
  \gridset{#2,#3}{\sparse\appendwithcomma{#1}}%
}{ÄΩnds}{$\regsgset{n}{d}$}{
  Set of regular sparse grid points of level $n$ with dimensionality $d$
}

\newnotationcommandoptarg{4}{\coarseregsgset}{%
  \gridset{#2,#3}{\sparse(#4)\appendwithcomma{#1}}%
}{ÄΩndsb}{$\coarseregsgset{n}{d}{b}$}{
  Set of regular sparse grid points of level $n$ with dimensionality $d$
  and boundary parameter $b$
}

\newcommand*{\interiorregsgset}[3][]{%
  \gridset[\interiorgrid]{#2,#3}{\sparse\appendwithcomma{#1}}%
}

\newcommand*{\bndryregsgset}[3][]{%
  \gridset[\bndrygrid]{#2,#3}{\sparse\appendwithcomma{#1}}%
}

% ======================================================================
% Function Spaces
% ======================================================================

\newcommand*{\gridspace}[3][]{%
  \if\relax\detokenize{#1}\relax V\else#1{V}\fi_{#2}^{#3}%
}

\newnotationcommandoptarg{2}{\ns}{%
  V_{#2}\ifnotempty{#1}{^{#1}}%
}{Vl}{$\ns{\*l}$}{
  Nodal space of level $\*l$
}

\newcommand*{\nsbspl}[3][]{\ns[#3\appendwithcomma{#1}]{#2}}

\newnotation{Vnd}{$\ns{n,d}$}{
  Multivariate nodal space
  $:= \ns{n \cdot \*1}$ of level $n$ with dimensionality $d$
}

\newnotationcommandoptarg{2}{\hs}{%
  W_{#2}\ifnotempty{#1}{^{#1}}%
}{Wl}{$\hs{\*l}$}{
  Hierarchical subspace of level $\*l$
}

\newcommand*{\hsbspl}[3][]{\hs[#3\appendwithcomma{#1}]{#2}}

\newnotationcommand{\sgspace}{\gridspace{}{\sparse}}{Vs}{$\sgspace$}{
  Arbitrary sparse grid space (possibly spatially adaptive)
}

\newnotationcommandoptarg{3}{\regsgspace}{%
  \gridspace{#2,#3}{\sparse\appendwithcomma{#1}}%
}{Vnds}{$\regsgspace{n}{d}$}{
  Regular sparse grid space of level $n$ with dimensionality $d$
}

\newnotationcommand[2]{\nonunifsplspace}{S_{#1}^{#2}}{Sxip}{%
  $\nonunifsplspace{\knotseq}{p}$%
}{
  Spline space of degree $p$ with knots $\knotseq$
}

\newnotationcommand[2]{\restrictedsplspace}{S_{#1}^{#2}}{Slp}{%
  $\restrictedsplspace{l}{p}$%
}{
  Spline space $:= \nonunifsplspace{\nodalknotseq{l}{p}}{p}$
  (spanned by the uniform nodal B-splines of level $l$, degree $p$)
}

\newnotationcommand[2]{\wholesplspace}{S_{#1}^{#2,\clint{0,1}}}{Slp01}{%
  $\wholesplspace{l}{p}$%
}{
  Spline space of degree $p$ on the grid
  $\{\gp{l,i} \mid i = 0, \dotsc, 2^l\}$ of level $l$
}

\newnotationcommand[2]{\naksplspace}{S_{#1}^{#2,\nak}}{Slpnak}{%
  $\naksplspace{l}{p}$%
}{
  Spline space $:= \nonunifsplspace{\nodalknotseq[\nak]{l}{p}}{p}$
  (spanned by the nodal not-a-knot B-splines of level $l$, degree $p$)
}

\newnotationcommand[1]{\polyspace}{P^{#1}}{Pp}{$\polyspace{\*p}$}{
  Space of all $d$-variate polynomials of
  coordinate degree $\le \*p$ on $\clint{\*0, \*1}$
}

\newnotationcommand[2]{\restrictspace}{#1|_{#2}}{V|D}{%
  $\restrictspace{V}{D}$%
}{
  Restriction $:= \{\restrictfcn{\objfun}{D} \mid \objfun \in V\}$
  onto a sub-domain $D$ for a function space $V$
}

% ======================================================================
% Landau Notation
% ======================================================================

\newnotationcommand[1]{\landauO}{\mathcal{O}(#1)}{O}{$\landauO{\objfun(x)}$}{
  Big-$\mathcal{O}$ Landau notation
}

\newnotationcommand[1]{\landauOmega}{\Omega(#1)}{ÄΩ}{%
  $\landauOmega{\objfun(x)}$%
}{
  Big-$\Omega$ Landau notation
}

\newnotationcommand[1]{\landauTheta}{\Theta(#1)}{ÄΘ}{%
  $\landauTheta{\objfun(x)}$%
}{
  Big-$\Theta$ Landau notation
}

% ======================================================================
% Unary Operators and Matrices
% ======================================================================

\makecommandnotation{\dim}{dim}{$\dim$}{
  Vector space dimension
}

\newnotationcommand[1]{\abs}{|#1|}{Öabs}{$\abs{\cdot}$}{
  Absolute value of a scalar
}

\newnotationcommand[1]{\setsize}{|#1|}{Öcount}{$\setsize{\cdot}$}{
  Number of elements of a set
}

\newcommand*{\bndrydomain}[1]{\mathop{}\!\partial#1}
\let\bndrydomainorig\bndrydomain
\makecommandnotation{\bndrydomain}{ÄΩd}{$\bndrydomainorig{\Omega}$}{
  Topological boundary of a set $\Omega \subset \real^d$
}

\newnotationcommand{\linop}{\mathfrak{L}}{L2}{$\linop$}{
  Linear operator $\linop\colon \real^{\ngp} \to \real^{\ngp}$
  on grid point data
}

\newnotationcommand{\intpmat}{\mat{A}}{A}{$\intpmat$}{
  Interpolation matrix with entries $\basis{\*l',\*i'}(\gp{\*l,\*i})$
}

\newcommand*{\intpmatuv}[1]{\mat{A}^{(#1)}}

\newnotationcommand[1]{\upop}{\linop^{(#1)}}{Lt1}{%
  $\upop{t_1, \dotsc, t_j}$%
}{
  Operator $\upop{t_1, \dotsc, t_j}\colon \real^{\ngp} \to \real^{\ngp}$
  corresponding to the application of the unidirectional principle in the
  dimensions $t_1, \dotsc, t_j$ on the $d$-dimensional sparse grid
}

\newnotationcommand[2]{\upopuv}{\linop^{(#1),#2}}{LtKpole}{%
  $\upopuv{t}{\lisetpole}$%
}{
  One-dimensional operator $\upopuv{t}{\lisetpole}\colon
  \real^{\setsize{\lisetpole}} \to \real^{\setsize{\lisetpole}}$
  corresponding to the application of the unidirectional principle in the
  $t$-th dimension on a pole $\lisetpole$ of the sparse grid
}

\newcommand*{\linopinv}{\linop^{-1}}
\newcommand*{\intpmatinv}{\intpmat^{-1}}
\newcommand*{\intpmatuvinv}[1]{(\intpmatuv{#1})^{-1}}
\newcommand*{\upopinv}[1]{(\upop{#1})^{-1}}
\newcommand*{\upopuvinv}[2]{(\linop^{(#1),#2})^{-1}}

\newnotationcommand{\idop}{\mathrm{id}}{id}{$\idop$}{
  Identity operator $\idop\colon \real^{\ngp} \to \real^{\ngp}$
}

\newnotationcommand[2]{\intpmatentry}{A_{#1,#2}}{Akk'}{$\intpmatentry{k}{k'}$}{
  $(k, k')$-th entry of $\intpmat$
}

\DeclareMathOperator{\supp}{supp}
\let\supporig\supp
\makecommandnotation{\supp}{supp}{$\supporig$}{
  Support of a function (i.e., the closure of the non-zero set)
}

\DeclareMathOperator{\interiorsupp}{\mathring{\supporig}}
\let\interiorsupporig\interiorsupp
\makecommandnotation{\interiorsupp}{suppo}{$\interiorsupporig$}{
  Interior of the support of a function
  (i.e., the set where the function does not vanish)
}

\DeclareMathOperator{\spn}{span}
\let\spnorig\spn
\makecommandnotation{\spn}{span}{$\spnorig$}{
  Linear span (set of all linear combinations)
}

\makecommandnotation{\deg}{deg}{$\deg \objfun$}{
  Degree of the polynomial $\objfun$
}

\DeclareMathOperator*{\argmin}{arg\,min}
\let\argminorig\argmin
\hidenextnotation
\makecommandnotation{\argmin}{argmin}{$\argminorig$}{
  Point where the minimum is attained
}

\DeclareMathOperator*{\argmax}{arg\,max}
\let\argmaxorig\argmax
\hidenextnotation
\makecommandnotation{\argmax}{argmax}{$\argmaxorig$}{
  Point where the maximum is attained
}

\newcommand*{\floor}[1]{\lfloor#1\rfloor}

\newnotationcommand[1]{\ceil}{\lceil#1\rceil}{Öceil}{%
  $\floor{\cdot}$, $\ceil{\cdot}$%
}{
  Floor/ceiling function
  (greatest/smallest integer smaller/greater or equal than $\cdot$)
}

\hidenextnotation
\newnotationcommandoptarg{3}{\deriv}{%
  \frac{\diff\ifnotempty{#1}{^{#1}}}{\diff{}#2\ifnotempty{#1}{^{#1}}} #3%
}{Öderivative}{$\deriv[q]{x}{\objfun}$}{
  $q$-th derivative of $\objfun$ with respect to $x$
}

\hidenextnotation
\newnotationcommandoptarg{3}{\tderiv}{%
  \tfrac{\diff\ifnotempty{#1}{^{#1}}}{\diff{}#2\ifnotempty{#1}{^{#1}}} #3%
}{Öderivativesmall}{$\tderiv[q]{x}{\objfun}$}{
  $q$-th derivative of $\objfun$ with respect to $x$
}

\hidenextnotation
\newnotationcommandoptarg{3}{\partialderiv}{%
  \frac{\partialdiff\ifnotempty{#1}{^{#1}}}{#2} #3%
}{Öderivative}{$\partialderiv{\partialdiff{}x}{\objfun}$}{
  Partial derivative of $\objfun$ with respect to $x$
}

\hidenextnotation
\newnotationcommandoptarg{3}{\tpartialderiv}{%
  \tfrac{\partialdiff\ifnotempty{#1}{^{#1}}}{#2} #3%
}{Öderivativesmall}{$\tpartialderiv{\partialdiff{}x}{\objfun}$}{
  Partial derivative of $\objfun$ with respect to $x$
}

\newnotationcommand[2]{\gradient}{%
  \mathop{\*\nabla_{\hspace*{-0.1em}#1}} #2%
}{Ögradient}{$\gradient{\*x}{\objfun}$}{
  Gradient of a function $\objfun$ with respect to $\*x$
}

\newnotationcommand[2]{\hessian}{%
  \mathop{\*\nabla_{\hspace*{-0.1em}#1}^2} #2%
}{Öhessian}{$\hessian{\*x}{\objfun}$}{
  Hessian of a function $\objfun$ with respect to $\*x$
}

\newnotationcommand[1]{\tr}{#1^\mathrm{T}}{ËT}{$\tr{{\cdot}}$}{
  Transpose of a vector or matrix
}

\newnotationcommand{\eye}{\mat{I}}{I}{$\eye$}{
  Identity matrix $\eye \in \real^{n \times n}$
}

\newnotationcommand[1]{\nonnegpart}{(#1)_+}{Ë.+}{$\nonnegpart{\cdot}$}{
  Non-negative part $\nonnegpart{x} := \max(x, 0)$
  of $x \in \real$
}

\newnotationcommandoptarg{2}{\expectation}{%
  \mathbb{E}\ifnotempty{#1}{_{#1}}\bracket*{#2}%
}{ËEX}{$\expectation{X}$}{
  Expectation of the random variable $X$
}

% ======================================================================
% Binary Operators
% ======================================================================

\let\oplusorig\oplus
\let\oplus\undefined
\newnotationcommand{\oplus}{\oplusorig}{Öoplus}{$\oplus$}{%
  Direct sum of vector spaces (vector space sum in which the dimension of
  the sum equals the sum of the summands' dimensions)
}

\newnotationcommand{\convolution}{\ast}{Öconvolution}{$\convolution$}{
  Convolution $f \ast g$ of two functions $f$ and $g$
}

\newnotationcommand[2]{\kronecker}{\delta_{#1,#2}}{Äδij}{$\kronecker{i}{j}$}{
  Kronecker delta $:= 1$ if $i = j$ and $:= 0$ otherwise,
  defined for arbitrary objects $i, j$ that can be compared via ``$=$''
}

\DeclareMathOperator{\xor}{xor}
\let\xororig\xor
\hidenextnotation
\makecommandnotation{\xor}{xor}{$\xororig$}{
  Bitwise ``exclusive or''
}

\newnotationcommand[2]{\restrictfcn}{#1|_{#2}}{fD}{$\restrictfcn{\objfun}{D}$}{
  Restriction $\restrictfcn{\objfun}{D}\colon D \to \real$,
  $\restrictfcn{\objfun}{D}(x) := \objfun(x)$,
  onto a sub-domain $D$ for a function $\objfun$
}

\newnotationcommand{\dotcup}{\mathbin{\dot{\cup}}}{Öunion}{$\dotcup$}{
  Disjoint union of sets (union where the pairwise intersection of
  the joined sets is empty)
}

\DeclareMathOperator*{\bigdotcup}{\dot{\bigcup}}

\newnotationcommand{\eq}{\sim}{Öeq}{$\eq$}{
  Equivalence relation where
  $a \eq b$ denotes that two elements $a, b$ are equivalent
}

\newnotationcommand[1]{\samepole}{\eq_{#1}}{Ösimt}{$\samepole{t}$}{
  Pole equivalence relation with respect to dimension $t = 1, \dotsc, d$
  ($\*k \samepole{t} \*k'$ if $\*k$ and $\*k'$ are on the same pole
  $\lisetpole \in \eqclasses{\liset}{\samepole{t}}$ in dimension $t$)
}

\newnotationcommand[2]{\eqclasses}{{#1}/{#2}}{Öeqclasses}{%
  $\eqclasses{\cdot}{\eq}$%
}{
  Set of equivalence classes for an equivalence relation $\eq$ on
  a set $\cdot$
}

\newnotationcommand[2]{\eqclass}{[#1]_{#2}}{Öeqclass}{%
  $\eqclass{\cdot}{\eq}$%
}{
  Equivalence class of $\cdot$ (set of all elements equivalent to $\cdot$)
  with respect to an equivalence relation $\eq$
}

% ======================================================================
% Level/Index Sets
% ======================================================================

\newnotationcommand[1]{\hiset}{I_{#1}}{Il}{$\hiset{\*l}$}{
  Set of (odd) indices for hierarchical basis functions
}

\newnotationcommand{\liset}{K}{K}{$\liset$}{
  Finite set of hierarchical level-index pairs $(\*l, \*i)$
  or finite subset of continuous indices $\*k \in \natz^d$
}

\newnotationcommandoptarg{1}{\lisetpole}{%
  \liset_\mathrm{pole}\ifnotempty{#1}{^{#1}}%
}{Kpole}{$\lisetpole$}{
  Pole $\lisetpole \in \eqclasses{\liset}{\samepole{t}}$
  in some dimension $t$ ($\lisetpole$ is a subset of $\liset$)
}

\newnotationcommand{\levelset}{L}{L1}{$\levelset$}{
  Finite subset $\levelset \subset \natz^d$ of levels
}

\newnotationcommandoptarg{4}{\coarselevelset}{%
  \levelset_{#2,#3}^{\sparse(#4)\appendwithcomma{#1}}%
}{Lndbs}{$\coarselevelset{n}{d}{b}$}{
  Finite subset $\coarselevelset{n}{d}{b} \subset \natz^d$ of levels
  for the grid $\coarseregsgset{n}{d}{b}$
}

\newnotationcommandoptarg{2}{\relindexset}{%
  J_{#2}\ifnotempty{#1}{^{#1}}%
}{Jl}{$\relindexset{l}$}{
  Set of relevant indices for the translation-invariant
  fundamental transformation
}

% ======================================================================
% Interpolation Values and Coefficients
% ======================================================================

\newnotationcommandoptarg{2}{\fcnval}{%
  f\ifnotempty{#1}{^{#1}}(\gp{#2})%
}{fli}{$\fcnval{\*l,\*i}$}{
  Value of the objective function $\objfun$ at the grid point $\gp{\*l,\*i}$
}

\newnotationcommandoptarg{2}{\surplus}{%
  \alpha_{#2}\ifnotempty{#1}{^{#1}}%
}{Äαli}{$\surplus{\*l,\*i}$}{
  Hierarchical surpluses (coefficients of a linear combination of hierarchical
  basis functions)
}

\newcommand*{\vsurplus}{\*\alpha}
\newcommand*{\surplustilde}[2][]{\tilde{\alpha}_{#2}\ifnotempty{#1}{^{#1}}}

\newnotationcommandoptarg{2}{\interpcoeff}{%
  c_{#2}\ifnotempty{#1}{^{#1}}%
}{cli}{$\interpcoeff{\*l,\*i}$}{
  Full grid interpolation coefficients
  (coefficients of a linear combination of nodal basis functions)
}

\newcommand*{\interpcoefftilde}[2][]{\tilde{c}_{#2}\ifnotempty{#1}{^{#1}}}

\newcommand*{\fundsplcoeff}[3][]{c_{#2,#3}\ifnotempty{#1}{^{#1}}}
\newcommand*{\fundsplcutoff}[1]{n_{#1}}

\newcommand*{\wfundsplcoeff}[3][]{c_{#2,#3}\ifnotempty{#1}{^{#1}}}

% ======================================================================
% Special Numbers
% ======================================================================

\newnotationcommand{\econst}{\mathrm{e}}{e}{$\econst$}{
  Euler constant $\econst = \exp(1)$
}

\newnotationcommand{\ngpMax}{\ngp_\mathrm{max}}{Nmax}{$\ngpMax$}{
  Maximum number of grid points when generating spatially adaptive sparse grids
}

% ======================================================================
% Special Points
% ======================================================================

\newnotationcommandoptarg{1}{\xopt}{%
  \ifempty{#1}{\*x^\opt}\ifnotempty{#1}{x^\opt_{#1}}%
}{xopt}{$\xopt$}{
  Solution of an optimization problem of the form
  $\xopt = \argmin \objfun(\*x)$
}

\newnotationcommandoptarg{1}{\xoptappr}{%
  \ifempty{#1}{\*x^{\opt,\ast}}%
  \ifnotempty{#1}{x^{\opt,\ast}_{#1}}%
}{xopt*}{$\xoptappr$}{
  Approximation for $\xopt = \argmin \objfun(\*x)$
}

\newnotationcommandoptarg{1}{\xoptscaled}{%
  \ifempty{#1}{\bar{\*x}^\opt}%
  \ifnotempty{#1}{\bar{x}^\opt_{#1}}%
}{xoptbar}{$\xoptscaled$}{
  Scaled solution of an optimization problem
}

\newnotationcommandoptarg{1}{\xscaled}{%
  \ifempty{#1}{\bar{\*x}}\ifnotempty{#1}{\bar{x}_{#1}}%
}{xbar}{$\xscaled$}{
  Scaled $\xscaled \in \clint{a, b}$
  for the definition of test functions
}

% ======================================================================
% Special Functions
% ======================================================================

\newnotationcommandoptarg{2}{\basis}{%
  \varphi_{#2}\ifnotempty{#1}{^{#1}}%
}{Äφli}{$\basis{\*l,\*i}$}{
  Hierarchical basis function of level $\*l$, index $\*i$
}

\newnotationcommand[1]{\fundbasis}{%
  \varphi_{#1}^\mathrm{f}%
}{Äφlif}{$\fundbasis{\*l,\*i}$}{
  Fundamental hierarchical basis function of level $\*l$, index $\*i$
  (satisfying the fundamental property)
}

\newnotationcommand[1]{\wfundbasis}{%
  \varphi_{#1}^\mathrm{wf}%
}{Äφliwf}{$\wfundbasis{\*l,\*i}$}{
  Weakly fundamental hierarchical basis function of level $\*l$, index $\*i$
  (satisfying the weakly fundamental property)
}

\newnotationcommandoptarg{1}{\parentfcn}{%
  \varphi\ifnotempty{#1}{^{#1}}%
}{Äφ}{$\parentfcn$}{
  Parent function $\parentfcn\colon \clint{0, 1} \to \real$
  (if $\parentfcn$ is a parent function of $\basis{l,i}$,
  then the $\basis{l,i}$ are scaled and translated versions of $\parentfcn$)
}

\newnotationcommandoptarg{3}{\bspl}{%
  \varphi_{#2}^{#3\appendwithcomma{#1}}%
}{Äφlip}{$\bspl{\*l,\*i}{\*p}$}{
  Hierarchical B-spline basis function of
  level $\*l$, index $\*i$, degree $\*p$
}

\newnotationcommand[1]{\cardbspl}{b^{#1}}{bp}{$\cardbspl{p}$}{
  Cardinal B-spline of degree $p$
}

\newnotationcommandoptarg{2}{\parentbspl}{%
  \parentfcn[#2\appendwithcomma{#1}]%
}{Äφp}{$\parentbspl{p}$}{
  Centralized cardinal B-spline
  $\parentbspl{p} := \cardbspl{p}({\cdot} + \tfrac{p+1}{2})$
  of degree $p$
}

\newnotationcommandoptarg{2}{\parentfundspl}{%
  \parentfcn[#2,\fs\appendwithcomma{#1}]%
}{Äφpfs}{$\parentfundspl{p}$}{
  Fundamental spline of degree $p$ (parent function)
}

\newnotationcommandoptarg{2}{\parentwfundspl}{%
  \parentfcn[#2,\wfs\appendwithcomma{#1}]%
}{Äφpwfs}{$\parentwfundspl{p}$}{
  Weakly fundamental spline of degree $p$ (parent function)
}

\newnotationcommand[2]{\nonunifbspl}{b_{#1}^{#2}}{bkξp}{%
  $\nonunifbspl{k,\knotseq}{p}$%
}{
  Non-uniform B-spline of degree $p$ with knots $\knotseq$ and index $k$
}

\newnotationcommand{\spl}{s}{s}{$s$}{
  Spline (piecewise polynomial)
}

\newnotationcommand[1]{\charfun}{\chi_{#1}}{ÄχA}{$\charfun{A}$}{
  Characteristic function of the set $A \subset \real$
}

\newnotationcommand{\objfun}{f}{f}{$\objfun$}{
  Objective function $\objfun\colon \clint{\*0, \*1} \to \real$
}

\newnotationcommandoptarg{1}{\ineqconfun}{%
  \ifempty{#1}{\*g}\ifnotempty{#1}{g_{#1}}%
}{g}{$\ineqconfun$}{
  Inequality constraint function
  $\ineqconfun\colon \clint{\*0, \*1} \to \real^{m_{\ineqconfun}}$
  (optimal solutions $\*x$ must satisfy $\ineqconfun(\*x) \le \*0$)
}

\newnotationcommandoptarg{1}{\eqconfun}{
  \ifempty{#1}{\*h}\ifnotempty{#1}{h_{#1}}%
}{h}{$\eqconfun$}{
  Equality constraint function
  $\eqconfun\colon \clint{\*0, \*1} \to \real^{m_{\eqconfun}}$
  (optimal solutions $\*x$ must satisfy $\eqconfun(\*x) \le \*0$)
}

\newnotationcommand{\objfunscaled}{\bar{\objfun}}{fbar}{$\objfunscaled$}{
  Scaled objective function $\objfunscaled\colon \clint{\*a, \*b} \to \real$
}

\newnotationcommandoptarg{1}{\ineqconfunscaled}{%
  \ifempty{#1}{\bar{\*g}}\ifnotempty{#1}{\bar{g}_{#1}}%
}{gbar}{$\ineqconfunscaled$}{
  Scaled inequality constraint function
  $\ineqconfunscaled\colon \clint{\*a, \*b} \to \real^{m_{\ineqconfun}}$
}

\newnotationcommandoptarg{1}{\eqconfunscaled}{
  \ifempty{#1}{\bar{\*h}}\ifnotempty{#1}{\bar{h}_{#1}}%
}{hbar}{$\eqconfunscaled$}{
  Scaled equality constraint function
  $\eqconfunscaled\colon \clint{\*a, \*b} \to \real^{m_{\eqconfun}}$
}

\newnotationcommand[1]{\testobjfunscaled}{%
  \bar{\objfun}_{\mathrm{#1}}%
}{fbarTest}{$\testobjfunscaled{Test}$}{
  Scaled test objective function
  $\testobjfunscaled{Test}\colon \clint{\*a, \*b} \to \real$
  with name ``Test''
}

\newnotationcommandoptarg{2}{\testineqconfunscaled}{%
  \ifempty{#1}{\bar{\*g}_{\mathrm{#2}}}%
  \ifnotempty{#1}{\bar{g}_{\mathrm{#2},#1}}%
}{gbarTest}{$\testineqconfunscaled{Test}$}{
  Scaled test inequality constraint function
  $\testineqconfunscaled{Test}\colon
  \clint{\*a, \*b} \to \real^{m_{\ineqconfun}}$
  with name ``Test''
}

\newnotationcommandoptarg{2}{\testeqconfunscaled}{
  \ifempty{#1}{\bar{\*h}_{\mathrm{#2}}}%
  \ifnotempty{#1}{\bar{h}_{\mathrm{#2},#1}}%
}{hbarTest}{$\testeqconfunscaled{Test}$}{
  Scaled test equality constraint function
  $\testeqconfunscaled{Test}\colon
  \clint{\*a, \*b} \to \real^{m_{\eqconfun}}$
  with name ``Test''
}

\newnotationcommand[1]{\fgintp}{f_{#1}}{fl}{$\fgintp{\*l}$}{
  Full grid interpolant of $\objfun$ in $\ns{\*l}$
}

\newnotationcommandoptarg{1}{\sgintp}{%
  f^{\sparse\appendwithcomma{#1}}%
}{fs}{$\sgintp$}{
  Sparse grid interpolant of $\objfun$ in $\sgspace$
  (on some grid $\sgset$)
}

\newnotationcommandoptarg{3}{\regsgintp}{%
  f_{#2,#3}^{\sparse\appendwithcomma{#1}}%
}{fnds}{$\regsgintp{n}{d}$}{
  Regular sparse grid interpolant of $\objfun$ in $\regsgspace{n}{d}$
}

\newnotationcommandoptarg{2}{\lagrangepoly}{%
  L_{#2}\ifnotempty{#1}{^{#1}}%
}{Lli}{$\lagrangepoly{l,i}$}{
  Lagrange polynomial of level $l$, index $i$
}

% ======================================================================
% Spline Knots
% ======================================================================

\newnotationcommand[1]{\knot}{\xi_{#1}}{Äξ10}{$\knot{k}$}{
  Knot $\in \real$ of a spline space
  (point where the splines are not smooth)
}

\newnotationcommand{\knotseq}{\*\xi}{Äξ20}{$\knotseq$}{
  Knot sequence $:= (\knot{0}, \dotsc, \knot{m+p})$
}

\newnotationcommandoptarg{3}{\nodalknot}{%
  \xi_{#2}^{#3\appendwithcomma{#1}}%
}{Äξlkp}{$\nodalknot{l,k}{p}$}{
  Knot for uniform nodal B-splines of level $l$, degree $p$
  ($k$-th entry of $\nodalknotseq{l}{p}$)
}

\newnotationcommandoptarg{3}{\nodalknotseq}{%
  \*\xi_{#2}^{#3\appendwithcomma{#1}}%
}{Äξlp}{$\nodalknotseq{l}{p}$}{
  Knot sequence for uniform nodal B-splines of level $l$, degree $p$
}

% ======================================================================
% Domains
% ======================================================================

\newnotationcommand[2]{\spldomain}{D_{#1}^{#2}}{D10}{%
  $\spldomain{\knotseq}{p}$%
}{
  Spline interpolation domain of $\nonunifsplspace{\knotseq}{p}$
}

\newnotationcommand[2]{\rspldomain}{D_{#1}^{#2}}{D20}{%
  $\rspldomain{l}{p}$%
}{
  Spline interpolation domain of $\restrictedsplspace{l}{p}$
  (uniform nodal B-splines of level $l$, degree $p$)
}

% ======================================================================
% Application 0: Fuzzy Extension Principle
% ======================================================================

\newnotationcommandoptarg{2}{\fuzzy}{%
  \tilde{#2}\ifnotempty{#1}{^{#1}}%
}{xtilde}{$\fuzzy{x}$}{
  Fuzzy set
}

\newnotationcommandoptarg{2}{\memfun}{%
  \mu_{\fuzzy[#1]{#2}}%
}{ĵtilde}{$\memfun{x}$}{
  Membership function $\memfun{x}\colon X \to \clint{0, 1}$ of the
  fuzzy set $\fuzzy{x}$
}

\newnotationcommandoptarg{3}{\acut}{%
  (\tilde{#2}\ifnotempty{#1}{_{#1}})_{#3}%
}{xtildeα}{$\acut{x}{\alpha}$}{
  $\alpha$-cut of $\fuzzy{x}$ for $\alpha \in \clint{0, 1}$
}

% ======================================================================
% Application 1: Topology Optimization
% ======================================================================

\newnotationcommand{\dimdomain}{\tilde{d}}{dtilde}{$\dimdomain$}{
  Dimensionality of $\domain$
}

\newnotationcommand{\domain}{\tilde{\Omega}}{ÄΩtilde}{$\domain$}{
  Domain $\domain \subset \real^{\dimdomain}$
}

\newnotationcommand{\dens}{\varrho}{Äρ}{$\dens$}{
  Density in $\clint{0, 1}$
}

\newnotationcommand{\compliance}{J}{Jρ}{$\compliance(\dens)$}{
  Compliance of the density function $\dens$
}

\newnotationcommand[1]{\vol}{\mathrm{vol}(#1)}{volΩ}{$\vol{\domain}$}{
  Volume (Lebesgue measure) of $\domain \subset \real^{\dimdomain}$
}

\newnotationcommand[2]{\voldens}{\mathrm{vol}_{#1}(#2)}{volρΩ}{
  $\voldens{\dens}{\domain}$
}{%
  Volume of $\domain \subset \real^{\dimdomain}$ with respect to the
  density function $\dens$
}

\newnotationcommand{\densub}{\varrho^\ast}{r*}{$\densub$}{
  Upper bound on the volume fraction
  $\voldens{\dens}{\domain}/\vol{\domain}$
}

\newnotationcommand{\densmean}{\bar{\varrho}}{rbar}{$\densmean$}{
  Mean density of all macro-cells%
}

\newnotationcommand{\force}{\*F}{F}{$\force$}{
  Force $\force \in \real^{\dimdomain}$
}

\newnotationcommand{\displacement}{\*u}{u2}{$\displacement$}{
  Displacement $\displacement\colon \domain \to \real^{\dimdomain}$
}

\newnotationcommandoptarg{1}{\etensor}{%
  \mat{E}\ifnotempty{#1}{^{#1}}%
}{E}{$\etensor$}{
  Elasticity tensor in
  $\real^{3 \times 3}$ (two dimensions) or in
  $\real^{6 \times 6}$ (three dimensions)
}

\newcommand*{\etensorentry}[2][]{E_{#2}\ifnotempty{#1}{^{#1}}}

\newnotationcommandoptarg{1}{\cholfactor}{%
  \mat{R}\ifnotempty{#1}{^{#1}}%
}{R2}{$\cholfactor$}{
  Cholesky factor of $\etensor$ in
  $\real^{3 \times 3}$ (two dimensions) or in
  $\real^{6 \times 6}$ (three dimensions)
}

% ======================================================================
% Application 3: Dynamic Portfolio Choice Models
% ======================================================================

\newnotationcommand{\state}{\*x}{x}{$\state_t$}{
  State variables $\in \clint{\*0, \*1}$ of time $t$
}

\newnotationcommand{\policy}{\*p}{p}{$\policy_t$}{
  Policy variables $\in \real^{m_{\policy}}$ of time $t$
}

\newnotationcommand{\stochastic}{\*\omega}{Äω}{$\stochastic_t$}{
  Stochastic variables $\in \stochdomain$ of time $t$
}

\newnotationcommand{\stochdomain}{\Omega}{ÄΩ}{$\stochdomain_t$}{
  Domain of the stochastic variables $\stochastic_t$ of time $t$
}

\newnotationcommand{\wealth}{w}{w}{$\wealth_t$}{
  Wealth of time $t$ (available money)
}

\newnotationcommand{\consume}{c}{c}{$\consume_t$}{
  Consumption of time $t$
}

\newnotationcommand{\bond}{b}{b}{$\bond_t$}{
  Bond investment of time $t$
}

\newnotationcommand{\stock}{s}{s}{$\stock_t$}{
  Stock investment of time $t$
}

\newnotationcommand{\valuefcn}{J}{J}{$\valuefcn_t$}{
  Value function
  $\valuefcn_t\colon \clint{\*0, \*1} \to \real$ of time $t$
}

\newnotationcommand{\optpolicyfcn}{\policy^\opt}{popt}{$\optpolicyfcn_t$}{
  Optimal policy function
  $\optpolicyfcn_t\colon \clint{\*0, \*1} \to \real^{m_{\policy}}$ of time $t$
}

\newnotationcommand{\statefcn}{\psi}{Äψ}{$\statefcn_t$}{
  State transition function
  $\statefcn_t\colon \clint{\*0, \*1} \times \real^{m_{\policy}} \times
  \stochdomain_t \to \clint{\*0, \*1}$ of time $t$
}

\newnotationcommand{\utility}{u}{u}{$\utility(\consume_t)$}{
  Utility of consuming $\consume_t$
}

\newnotationcommand{\riskav}{\gamma}{Äγ}{$\riskav$}{
  Risk aversion $\in \real \setminus \{1\}$
}

\newnotationcommand{\patience}{\varrho}{Äρ}{$\patience$}{
  Patience factor $\in \hopint{0, 1}$
}

% ======================================================================
% Miscellaneous
% ======================================================================

% defined terms
\newcommand*{\term}[1]{\emph{#1}}

% check if argument contains \* (vector)
\newcommand*{\containsvector}[3]{%
  \protect\IfSubStr{\detokenize{#1}}{\detokenize{\*}}{#2}{#3}%
}

% subsets/supersets
\renewcommand*{\subset}{\subseteq}
\renewcommand*{\supset}{\supseteq}

% differential for integral/derivatives
\newcommand*{\diff}{\mathop{}\!\mathrm{d}}
\newcommand*{\dx}{\diff{}x}
\newcommand*{\partialdiff}{\mathop{}\!\partial}

% vectors
\renewcommand*{\vec}[1]{{\boldsymbol{#1}}}
\def\*#1{\vec{#1}}
\newcommand*{\veclog}{\mathop{\vec{\log}}}
\newcommand*{\vecmax}{\mathop{\vec{\max}}}

% matrices
\newcommand*{\mat}[1]{{\boldsymbol{#1}}}

% quantors
\newcommand*{\fa}[2]{\forall_{#1}\;#2}
\newcommand*{\ex}[2]{\exists_{#1}\;#2}
\newcommand*{\fafa}[3]{\forall_{#1} \forall_{#2}\;#3}
\newcommand*{\faex}[3]{\forall_{#1} \exists_{#2}\;#3}
\newcommand*{\exfa}[3]{\exists_{#1} \forall_{#2}\;#3}
\newcommand*{\exex}[3]{\exists_{#1} \exists_{#2}\;#3}
\newcommand*{\falarge}[2]{\forall_{#1}\;\;#2}
\newcommand*{\exlarge}[2]{\exists_{#1}\;\;#2}
\newcommand*{\fafalarge}[3]{\forall_{#1} \forall_{#2}\;\;#3}
\newcommand*{\faexlarge}[3]{\forall_{#1} \exists_{#2}\;\;#3}
\newcommand*{\exfalarge}[3]{\exists_{#1} \forall_{#2}\;\;#3}
\newcommand*{\exexlarge}[3]{\exists_{#1} \exists_{#2}\;\;#3}

% delimiters
\DeclarePairedDelimiter{\braced}{\{}{\}}
\DeclarePairedDelimiter{\bracket}{[}{]}
\DeclarePairedDelimiter{\paren}{(}{)}

% inner product
\newcommand*{\innerprod}[3][]{%
  \langle{#2,#3}\rangle\ifnotempty{#1}{_{#1}}
}
\newcommand*{\innerprodscaled}[3][]{%
  \left\langle{#2,#3}\right\rangle\ifnotempty{#1}{_{#1}}
}

% norm
\newcommand*{\norm}[2][]{%
  \lVert{#2}\rVert\ifnotempty{#1}{_{#1}}%
}
\newcommand*{\normscaled}[2][]{%
  \left\lVert{#2}\right\rVert\ifnotempty{#1}{_{#1}}%
}

% sum with large summation index
\newcommand*{\largesum}[2][]{%
  \;\;\sum_{\mathclap{#2}}\ifnotempty{#1}{^{\mathclap{#1}}}\;\;%
}

% equivalent of \notin for \ni
\newcommand*{\notni}{\not\ni}

% for the BFS algorithm
\newcommand*{\pop}{\texttt{pop}}
\newcommand*{\push}{\texttt{push}}

% ticks/crosses for tables
\newcommand*{\yes}{%
  \tikz[x=0.7em,y=0.7em]{
    \draw[C4,line width=1.4pt] (0,0.3) -- (0.3,0) -- (1,1);
  }%
}
\newcommand*{\no}{%
  \tikz[x=0.7em,y=0.7em]{
    \draw[C1,line width=1.4pt] (0,0) -- (1,1) (0,1) -- (1,0);
  }%
}

% SG++
\newcommand*{\sgpp}{SG\textsuperscript{++}\xspace}
