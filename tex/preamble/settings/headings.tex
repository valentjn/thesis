% ======================================================================
% Headings
% ======================================================================

% don't use sans-serif font for headings, use single spacing
\setkomafont{disposition}{\normalcolor\bfseries\setstretch{1}}

% format chapter headings with custom commands (below)
\renewcommand*{\chapterformat}{\thechapter}
\renewcommand*{\chapterlinesformat}[3]{\mychapterlinesformat{#2}{#3}}

% check if chapter with number or without number
\newcommand*{\mychapterlinesformat}[2]{%
  \if\relax\detokenize{#1}\relax%
    \chaptertitlenonumber{#2}%
  \else%
    \begin{minipage}{\textwidth}%
      \centering%
      \chapternumber{#1}\\[1em]%
      \chaptertitle{#2}%
    \end{minipage}%
  \fi%
}

% put chapter number in circle
\newcommand*{\chapternumber}[1]{%
  \scalebox{3.0}{%
    \tikz{
      \fill[
        shading=axis,
        left color=mittelblau,
        right color=hellblau,
        shading angle=135,
        inner sep=0mm,
      ] (0,0) circle (5mm);
      \node[
        white,
        inner sep=0mm,
      ] at (0,0) {%
        \ifthenelse{\equal{#1}{3}}{\hspace*{0.4mm}}{}%
        \ifthenelse{\equal{#1}{5}}{\hspace*{0.5mm}}{}%
        \ifthenelse{\equal{#1}{7}}{\hspace*{-0.2mm}}{}%
        \ifthenelse{\equal{#1}{B}}{\hspace*{0.5mm}}{}%
        \ifthenelse{\equal{#1}{C}}{\hspace*{-0.5mm}}{}%
        \ifthenelse{\equal{#1}{D}}{\hspace*{0.5mm}}{}%
        \ifthenelse{\equal{#1}{E}}{\hspace*{0.2mm}}{}%
        \textbf{#1}%
      };
    }%
  }%
}

% if chapter with number: blue chapter title
\newcommand*{\chaptertitle}[1]{\textcolor{mittelblau}{#1}}
% if chapter without number: black chapter title
\newcommand*{\chaptertitlenonumber}[1]{#1}

% set space before and after chapter heading
\RedeclareSectionCommand[beforeskip=5em,afterskip=5em]{chapter}

% Commands for long chapter/section titles:
% If the title is too long, then the mark in the page header will
% overflow and wrap into the next line, which looks ugly.
% \chaptermark/\sectionmark is supposed to fix that. However, if you just
% call it after \chapter/\section, then it will only affect the marks on
% the following pages, not the one on which the heading is defined.
% Therefore, we have to call \chaptermark/\sectionmark twice and also
% supply the optional parameter (otherwise, there will be errors as LaTeX
% tries to add the marks to the table of contents).
% Use the commands like \longchapter{TOC title}{heading title}{mark title}.
\newcommand*{\longchapter}[3]{%
  \chapter[#1]{#2\chaptermark{#3}}\chaptermark{#3}%
}
\newcommand*{\longsection}[3]{%
  \section[#1]{#2\sectionmark{#3}}\sectionmark{#3}%
}
\newcommand*{\addlongchap}[3]{%
  \addchap[#1]{#2\chaptermark{#3}}%
}

% flag to print ornaments, disable if desired
% (in places where you don't want to have ornaments)
\newif\ifprintornaments
\printornamentstrue

% dirty hack to automatically output ornaments before every section heading
\RedeclareSectionCommand[
  font={%
    \ifprintornaments%
      \includegraphics[width=\textwidth]{ornament_1}%
      \\[1.625ex plus .5ex minus -.1ex]%
      \pagebreak[0]%
    \else%
      \vspace{1.625ex plus .5ex minus -.1ex}%
    \fi%
    \Large%
  },
  beforeskip=-1.625ex plus -.5ex minus .1ex,
]{section}

% dirty hack to automatically output ornaments before every subsection heading
\RedeclareSectionCommand[
  font={%
    \ifprintornaments%
      \hfill\includegraphics{ornament_2}%
      \hfill\null\\[1.3ex plus .5ex minus -.1ex]%
      \pagebreak[0]%
    \else%
      \vspace{1.3ex plus .5ex minus -.1ex}%
    \fi%
    \large%
  },
  beforeskip=-1.3ex plus -.5ex minus .1ex,
]{subsection}

% decrease skip before \paragraph and \subparagraph headings
\RedeclareSectionCommand[beforeskip=2ex plus 1ex minus .2px]{paragraph}
\RedeclareSectionCommand[beforeskip=1ex plus 1ex minus .2px]{subparagraph}

% append period to \paragraph headings
\let\paragraphwithoutperiod\paragraph
\newcommand*{\paragraphwithperiodwithstar}[1]{\paragraphwithoutperiod*{#1.}}
\newcommand*{\paragraphwithperiod}[1]{\paragraphwithoutperiod{#1.}}
\renewcommand*{\paragraph}{%
  \@ifstar{\paragraphwithperiodwithstar}{\paragraphwithperiod}%
}

% append period to \subparagraph headings
\let\subparagraphwithoutperiod\subparagraph
\newcommand*{\subparagraphwithperiodwithstar}[1]{\subparagraphwithoutperiod*{#1.}}
\newcommand*{\subparagraphwithperiod}[1]{\subparagraphwithoutperiod{#1.}}
\renewcommand*{\subparagraph}{%
  \@ifstar{\subparagraphwithperiodwithstar}{\subparagraphwithperiod}%
}

% increase indentation of paragraphs (default: 1em = \quad)
\setlength{\parindent}{2em}
