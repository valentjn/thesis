\section{The Hierarchization Problem}
\label{sec:41problem}

\blindtext{}

Let $\Omega^\sparse \subset [0, 1]^d$ be a sparse grid that
may be spatially adaptive (see \cref{sec:23sparseGrids})
and $K := \{(\ßl, \ßi) \mid \ßx_{\ßl,\ßi} \in \Omega^\sparse\}$
with $N := |\Omega^\sparse| = |K| < \infty$.
The \term{hierarchization problem} is finding
$(\alpha_{\ßl',\ßi'})_{(\ßl',\ßi') \in K} \in \RR^N$ such that
\begin{equation}
  \label{eq:hierarchizationProblem}
  \sum_{\mathclap{(\ßl', \ßi') \in K}} \alpha_{\ßl',\ßi'}
  \varphi_{\ßl',\ßi'}(\ßx_{\ßl,\ßi}) = f_{\ßl,\ßi}
  \quad\text{for all}\quad
  (\ßl, \ßi) \in K,
\end{equation}
where $(f_{\ßl,\ßi})_{(\ßl,\ßi) \in K} \in \RR^N$ is a given set of
function values $f(\ßx_{\ßl,\ßi})$ at the grid points $\ßx_{\ßl,\ßi}$.

\blindtext{}

\paragraph{Hierarchization as a linear operator}

The example of hierarchization can be generalized
to arbitrary linear operators
\begin{equation}
  L\colon \RR^N \to \RR^N,\quad
  \ßu \mapsto \ßy = L[\ßu],
\end{equation}
where $L$ depends on the grid $\Omega^\sparse$ at hand.
Input $\ßu$ and output $\ßy$ are scalar-valued data sets%
\footnote{%
  The setting could be generalized even more,
  for example to vector-valued data.
  We restrict ourselves to the scalar-valued case,
  keeping hierarchization as our main application in mind.%
}
\begin{equation}
  \ßu = (u_{\ßl,\ßi})_{(\ßl,\ßi) \in K} \in \RR^N,\quad
  \ßy = (y_{\ßl,\ßi})_{(\ßl,\ßi) \in K} \in \RR^N,
\end{equation}
which give one scalar per grid point $\ßx_{\ßl,\ßi} \in \Omega^\sparse$.
For the case of hierarchization,
$L$ is the inverse of the \term{interpolation matrix} $A \in \RR^{N \times N}$:
\begin{equation}
  L = A^{-1},\quad
  A = (\varphi_{\ßl',\ßi'}(\ßx_{\ßl,\ßi}))_{(\ßl,\ßi),(\ßl',\ßi') \in K},\quad
  \ßu = (f_{\ßl,\ßi})_{(\ßl,\ßi) \in K},\quad
  \ßy = (\alpha_{\ßl',\ßi'})_{(\ßl',\ßi') \in K}.
\end{equation}
This means that we can determine the $\alpha_{\ßl',\ßi'}$ by solving
the $N \times N$ system of linear equations
\begin{equation}
  \ßy = L[\ßu]
  \quad\iff\quad
  A \cdot (\alpha_{\ßl',\ßi'})_{(\ßl',\ßi') \in K}
  = (f_{\ßl,\ßi})_{(\ßl,\ßi) \in K}.
\end{equation}

\paragraph{Notation}

We do not need the hierarchical level-index information $(\ßl, \ßi)$ in
$\Omega^\sparse$, $K$, $\ßu$, and $\ßy$
for most of the considerations in this chapter.
Therefore, we assume that in each dimension $t$, the level-index pairs
$(l_t, i_t)$ ($l_t \in \NN_0$, $i_t \in I_{l_t}$)
are continuously enumerated by a single index $k_t = k_t(l_t, i_t) \in \NN_0$.
We identify $(\ßl, \ßi)$ with a single index $\ßk$,
whose $t$-th component is given by $k_t(l_t, i_t)$.
Consequently,
we can regard $K$ as a subset $\{\ßk \mid \ßx_\ßk \in \Omega^\sparse\}$
of $\NN^d$.
We will switch between the notations whenever appropriate.
All statements that are formulated in the $\ßk$ notation are
valid for both the nodal and the hierarchical basis
(i.e., for all tensor product bases).

In the following, $k_t$ denotes the $t$-th component of a $d$-vector $\ßk$
as usual.
With $\ßk_{-t}$, we denote the $(d-1)$-vector that is obtained from $\ßk$
by omitting the $k$-th component,
i.e., $\ßk_{-t} := (k_1, \dotsc, k_{t-1}, k_{t+1}, \dotsc, k_d)$.
For a $j$-tuple $T = (t_1, \dotsc, t_j) \in \{1, \dotsc, d\}^j$,
we define $\ßk_T$ to be the $j$-vector $(k_{t_1}, \dotsc, k_{t_j})$
that only contains the entries of the dimensions listed in $T$.
Accordingly, $\ßk_{-T}$ is defined as the $(d-j)$-vector
that contains the entries of the remaining dimensions
(sorted by the dimension $t$).
We define $a:b := (a, a + 1, \dotsc, b)$ as an indexing shortcut ($a \le b$).
