\section{Nodal Spaces}

\subsection{Univariate Case}

\paragraph{Grid and basis functions}

Let us first consider univariate functions
that are defined on the unit interval $[0, 1]$.
\newgsymbol{l}{$l$}{Level $\in \NN_0$}%
We discretize this domain by splitting it into $2^l$ equally sized segments,
where $l \in \NN_0$ is the \term{level}.
\newgsymbol{xli}{$x_{l,i}$}{Grid point $:= i \cdot h_l$}%
\newgsymbol{i}{$i$}{Index $= 0, \dotsc, 2^l$}%
\newgsymbol{hl}{$h_l$}{Mesh size $:= 2^{-l}$}%
The resulting $2^l + 1$ \term{grid points} are given by
\begin{equation}
  x_{l,i} := i \cdot h_l,\quad
  i = 0, \dotsc, 2^l,
\end{equation}
where $i$ is the \term{index} and $h_l := 2^{-l}$ is the \term{mesh size}.
\newgsymbol{phili}{$\varphi_{l,i}$}{%
  Hierarchical basis function of level $l$, index $i$%
}%
Every grid point is associated with a \term{basis function}
\begin{equation}
  \varphi_{l,i}\colon [0, 1] \to \RR.
\end{equation}
In this thesis, we assume $\varphi_{l,i}$ to be arbitrary,
satisfying required assumptions when needed and stated.
However, it helps for both the theory and the intuition to have a
specific example of basis functions in mind.
\newgsymbol{.1}{$\cdot^1$}{%
  Superscript for ``Piecewise linear
  (basis function/function space/interpolant)%
}%
The so-called \term{hat functions} (linear B-splines), which are defined as
\begin{equation}
  \label{eq:hatFunctionUV}
  \varphi_{l,i}^1(x)
  := \max(1 - |x/h_l - i|, 0),
\end{equation}
are the most common choice for $\varphi_{l,i}$.
Here and in the following,
the superscript ``1'' is the degree of the linear B-spline and
is not to be read as an exponent.
We will generalize this notation to B-splines $\varphi_{l,i}^p$ of
arbitrary degrees $p$ (see \cref{chap:30BSplines}).

\paragraph{Nodal space}

\newgsymbol{Vl}{$V_l$}{Nodal space of level $l$}%
\newgsymbol{span}{$\spn$}{Linear span (set of all linear combinations)}%
The \emph{nodal space} $V_l$ of level $l$
is defined as the linear span of all basis functions
$\varphi_{l,i}$:
\begin{equation}
  V_l := \spn\{\varphi_{l,i} \mid i = 0, \dotsc, 2^l\}.
\end{equation}
We assume that the functions $\varphi_{l,i}$ form a basis of $V_l$, i.e.,
every linear combination of these functions is unique.
\newgsymbol{fl}{$f_l$}{Interpolant of $f$ in $V_l$}%
\newgsymbol{ci}{$c_i$}{Coefficients of a linear combination}%
This ensures that for every objective function $f\colon [0, 1] \to \RR$,
there is a unique function $f_l\colon [0, 1] \to \RR$ such that
\begin{equation}
  \label{eq:interpFullGridUV}
  f_l
  = \sum_{i=0}^{2^l} c_i \varphi_{l,i},\quad
  \fa{i = 0, \dotsc, 2^l}{f_l(x_{l,i}) = f(x_{l,i})},
\end{equation}
for some $c_i \in \RR$.
In this case, $f_l$ is called \term{interpolant} of $f$ in $V_l$.
The nodal space $V_l^1$ is defined analogously as the span of the
hat functions $\varphi_{l,i}^1$.
It is the space of all linear splines,
that is, the space of all continuous functions on $[0, 1]$ that are
piecewise linear polynomials on $[x_{l,i}, x_{l,i+1}]$ for
$i = 0, \dotsc, 2^l - 1$.
The nodal hat function basis of level $3$
and a linear combination are shown in \cref{fig:nodalHat}.

\begin{figure}
  \subcaptionbox{%
    Basis functions $\varphi_{l,i}^1$ and grid points $x_{l,i}$
    \emph{(dots)}.%
  }[75mm]{%
    \includegraphics{nodalHat_1}%
  }%
  \hfill%
  \subcaptionbox{%
    Piecewise linear interpolant $f_l$ as a weighted sum
    of the nodal hat functions.%
  }[75mm]{%
    \includegraphics{nodalHat_2}%
  }%
  \caption{Univariate nodal hat functions of level $l = 3$.}
  \label{fig:nodalHat}
\end{figure}

\subsection{Multivariate Case}

\paragraph{Cartesian and tensor products}

\newgsymbol{d}{$d$}{Dimensionality $\in \NN$}%
For the multivariate case with $d \in \NN$ dimensions,
we proceed with the usual tensor product approach,
for which we replace all indices, points, and functions with
multi-indices, Cartesian products, and tensor products, respectively.
\newgsymbol{0!}{$\ß0$}{$(0, \dotsc, 0) \in \NN_0^d$}%
\newgsymbol{1!}{$\ß1$}{$(1, \dotsc, 1) \in \NN^d$}%
\newgsymbol{01!c}{$[\ß0, \ß1]$}{%
  Unit hypercube $:= [0, 1]^d := \{x \in \RR \mid 0 \le x \le 1\}^d$%
}
\newgsymbol{l!}{$\ßl$}{Multivariate level $\in \NN_0^d$}%
\newgsymbol{||.||1}{$\norm{\cdot}_1$}{%
  $\ell_1$ norm $\norm{\ßx}_1 := \sum_{t=1}^d x_t$%
}%
Therefore, the domain is now $[\ß0, \ß1] := [0, 1]^d$,
which can be partitioned into
$\prod_{t=1}^d 2^{l_t} = 2^{\norm{\vec{l}}_1}$ equally sized hypercubes,
where $\ßl = (l_1, \dotsc, l_d) \in \NN_0^d$ is the $d$-dimensional level
and $\norm{\vec{l}}_1 := \sum_{t=1}^d l_t$ is the level sum.
\newgsymbol{xli!}{$\ßx_{\ßl,\ßi}$}{%
  Multivariate grid point $:= \ßi \cdot \ßh_\ßl$%
}%
\newgsymbol{i!}{$\ßi$}{Multivariate index $= \ß0, \dotsc, \ß2^l$}%
\newgsymbol{i02l!}{$\ßi = \ß0, \dotsc, \ß2^l$}{%
  For all $\ßi$ with $0 \le i_t \le 2^{l_t}$ for all $t = 1, \dotsc, d$%
}%
The corners of the hypercubes are given by the grid points
\begin{equation}
  \label{eq:gridPointMultivariate}
  \ßx_{\ßl,\ßi} := \ßi \cdot \ßh_\ßl,\quad
  \ßi = \ß0, \dotsc, \ß2^{\ßl}.
\end{equation}
To allow for a somewhat intuitive and suggestive notation,
relations and operations with vectors (in bold face)
are to be read coordinate-wise in this thesis, unless stated otherwise.
Bold-faced numbers like $\ß0$ are defined to be the vector $(0, \dotsc, 0)$
in which every entry is equal to that number.
\newgsymbol{hl!}{$\ßh_\ßl$}{Multivariate mesh size $:= \ß2^{-\ßl}$}%
For example, \eqref{eq:gridPointMultivariate} is equivalent to
\begin{equation}
  \ßx_{\ßl,\ßi}
  := (i_1 h_{l_1},\; \dotsc,\; i_d h_{l_d}),\quad
  i_t = 0, \dotsc, 2^{l_t},\quad
  t = 1, \dotsc, d,
\end{equation}
with the $d$-dimensional mesh size
$\ßh_\ßl := \ß2^{-\ßl} = (h_{l_1}, \dotsc, h_{l_d})$.
\newgsymbol{phili!}{$\varphi_{\ßl,\ßi}$}{%
  Multivariate hierarchical basis function of level $\ßl$, index $\ßi$%
}%
Again, every grid point is associated with a basis function that is defined
as the tensor product of the univariate functions:%
\footnote{%
  Note that one could employ basis functions of different types in
  each dimension, for example B-splines of different degrees.
  For simplicity, we first restrict ourselves to the case of a single type
  for all dimensions, but we will treat the more general case in
  \todo{insert reference}.%
}
\begin{equation}
  \varphi_{\ßl,\ßi}\colon [\ß0, \ß1] \to \RR,\quad
  \varphi_{\ßl,\ßi}(\ßx)
  := \prod_{t=1}^d \varphi_{l_t,i_t}(x_t).
\end{equation}
\cref{fig:nodalHat2D} shows an example of a bivariate nodal hat function
$\varphi_{\ßl,\ßi}^1$.

\begin{figure}
  \includegraphics{nodalHat2D_1}%
  \caption{%
    Bivariate nodal hat function of level $\ßl = (2, 1)$ and
    index $i = (1, 1)$ as the tensor product of two univariate
    nodal hat functions.%
  }
  \label{fig:nodalHat2D}
\end{figure}

\paragraph{Multivariate nodal space}

\newgsymbol{Vl!}{$V_\ßl$}{Multivariate nodal space of level $\ßl$}%
The multivariate nodal space $V_\ßl$ is defined analogously to
the univariate case:
\begin{equation}
  V_\ßl
  := \spn\{\varphi_{\ßl,\ßi} \mid \ßi = \ß0, \dotsc, \ß2^{\ßl}\}.
\end{equation}
\newgsymbol{ab!}{$[\ßa, \ßb]$}{%
  Hypercube $:= [a_1, b_1] \times \dotsb \times [a_d, b_d]$%
}%
In the case of hat functions $\varphi_{\ßl,\ßi}^1$,
the nodal space $V_\ßl^1$ is the $d$-linear spline space, i.e.,
the space of all continuous functions
on $[\ß0, \ß1]$ that are piecewise $d$-linear polynomials on
all hypercubes
\begin{equation}
  [\ßx_{\ßl,\ßi}, \ßx_{\ßl,\ßi+\ß1}]
  := [x_{l_1,i_1}, x_{l_1,i_1+1}] \times \dotsb \times
  [x_{l_d,i_d}, x_{l_d,i_d+1}],\quad
  \ßi = \ß0, \dotsc, \ß2^\ßl - \ß1.
\end{equation}
\newgsymbol{fl!}{$f_\ßl$}{Interpolant of $f$ in $V_\ßl$}%
\newgsymbol{ci!}{$c_\ßi$}{Coefficients of a linear combination}%
Analogously to \eqref{eq:interpFullGridUV},
we can interpolate objective functions $f\colon [\ß0, \ß1] \to \RR$
in the nodal space $V_\ßl$ with $f_\ßl\colon [\ß0, \ß1] \to \RR$ satisfying
\begin{equation}
  \label{eq:interpFullGridMV}
  f_\ßl
  = \sum_{\ßi=\ß0}^{\ß2^\ßl} c_\ßi \varphi_{\ßl,\ßi},\quad
  \fa{\ßi = \ß0, \dotsc, \ß2^\ßl}{f_\ßl(\ßx_{\ßl,\ßi}) = f(\ßx_{\ßl,\ßi})},
\end{equation}
where $c_\ßi \in \RR$ and
the sum is over all $\ßi = \ß0, \dotsc, \ß2^\ßl$.
\begin{lemma}[linear independence of tensor products]
  \label{lemma:tensorProductLinearIndependence}
  The functions $\varphi_{\ßl,\ßi}$ ($\ßi = \ß0, \dotsc, \ß2^\ßl$)
  form a basis of $V_\ßl$, if the univariate functions
  $\varphi_{l_t,i_t}$ ($i_t = 0, \dotsc, 2^{l_t}$)
  form a basis of the univariate nodal space $V_{l_t}$
  for $t = 1, \dotsc, d$.
\end{lemma}
\begin{proof}
  \newgsymbol{!equiv}{$\equiv$}{%
    Equality of functions everywhere on their domain
    (e.g., $f \equiv 0$ means $f(x) = 0$ for all feasible $x$)%
  }%
  Assume that $c_\ßi \in \RR$ are chosen in \eqref{eq:interpFullGridMV}
  such that $f_\ßl \equiv 0$.
  Then for all $\ßi' = \ß0, \dotsc, \ß2^\ßl$,
  we can evaluate \eqref{eq:interpFullGridMV} at $\ßx_{\ßl,\ßi'}$ to obtain
  \begin{equation}
    \sum_{i_1=0}^{2^{l_1}}
    \left(\sum_{i_2=0}^{2^{l_2}} \dotsb
    \left(\sum_{i_d=0}^{2^{l_d}} c_\ßi \varphi_{l_d,i_d}(x_{l_d,i_d'})\right) \dotsb
    \varphi_{l_2,i_2}(x_{l_2,i_2'})\right) \varphi_{l_1,i_1}(x_{l_1,i_1'})
    = 0.
  \end{equation}
  We apply the linear independence in 1D ($x_1$ direction) to conclude that
  the sum over $i_2$ must vanish.
  Repeating this argument for all dimensions, we infer that $c_\ßi = 0$
  for all $\ßi = \ß0, \dotsc, \ß2^\ßl$,
  implying the linear independence of the functions $\varphi_{\ßl,\ßi}$.
\end{proof}
\newgsymbol{Omegal!}{$\Omega_\ßl$}{Set of full grid points of level $\ßl$}%
The lemma is equivalent to the statement that the coefficients $c_\ßi \in \RR$
exist for every objective function $f$ and are uniquely determined by
the values at the grid points
\begin{equation}
  \Omega_\ßl
  := \{\ßx_{\ßl,\ßi} \mid \ßi = \ß0, \dotsc, \ß2^{\ßl}\}.
\end{equation}
\newgsymbol{n}{$n$}{Level $\in \NN_0$ of full or sparse grid}
A common choice for the level $\ßl$ is $n \cdot \ß1$ for some $n \in \NN_0$.
\newgsymbol{Vnd}{$V_{n,d}$}{%
  Multivariate nodal space
  $:= V_{n \cdot \ß1}$ of level $n$ with dimensionality $d$%
}%
In this case, we replace ``$\ßl$'' in the subscripts with ``$n,d$''
(for example, $V_{n,d} := V_{n \cdot \ß1}$).
\newgsymbol{||.||L2}{$\norm{\cdot}_{L^2}$}{%
  $L^2$ norm $\norm{f}_{L^2} := \sqrt{\int_\Omega f(x)^2 \dx}$
  of a function $f\colon \Omega \to \RR$%
}
\newgsymbol{O}{$\calO(f(x))$}{Big-$\calO$ Landau notation}%
For the hat function basis $\varphi_{l,i}^1$,
it can be shown that the $L^2$ interpolation error of the interpolant
$f_{n,d}^1$ is given by
\begin{equation}
  \norm{f - f_{n,d}^1}_{L^2} = \calO(h_n^2),
\end{equation}
i.e., the order of the interpolation error is quadratic in the mesh size
\cite{Hoellig13Approximation,Bungartz04Sparse}.
