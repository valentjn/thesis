\section{Example Application: Fuzzy Extension Principle}
\label{sec:54fuzzy}

To conclude this chapter, we consider the fuzzy extension principle
as an example application of the optimization of B-spline surrogates
on sparse grids.

\paragraph{Aleatoric and epistemic uncertainties}

Classical uncertainty quantification distinguishes between
aleatoric and epistemic uncertainties \cite{Walz16Fuzzy}.
Aleatoric uncertainties result from variability of inputs or
model components and from the ``intrinsic randomness''
of quantities and are best described by probability theory,
giving exact probabilities.
Epistemic uncertainties arise from subjectivity,
simplifying modeling assumptions, and incomplete knowledge.
These uncertainties are better captured by fuzzy theory,
which is more imprecise than the ``exact'' stochastic assumptions
of probabilities \cite{Walz16Fuzzy}.

\paragraph{Uncertainty quantification with fuzzy uncertainties}

In uncertainty quantification, a key question is as follows:
Given a model and uncertain input parameters for the model,
how is the model output distributed?
While there are many approaches available
for probabilistic uncertainties,
it is not straightforward to see how to solve this task
for fuzzy uncertainties.
Fortunately, Zadeh proposed in 1975
the \term{fuzzy extension principle} \cite{Zadeh75Concept},
which addresses this very question.

\paragraph{Sparse grids and B-splines for fuzzy uncertainties}

As we will explain in this section,
the fuzzy extension principle requires the solution of numerous
optimization problems that involve the original objective function
$\objfun$.
This predestines the use of sparse grid surrogates instead of
$\objfun$, as explained in the beginning of the chapter.
Previous work by Klimke \cite{Klimke06Uncertainty} already
studied this approach for piecewise linear functions on uniform sparse grids
and for global polynomials on sparse Clenshaw--Curtis grids.
We will assess the suitability of interpolation with (higher-order)
hierarchical B-splines on sparse grids for the fuzzy extension principle.
It should be mentioned that there is also work
directly incorporating (non-hierarchical) B-splines
into the framework of fuzzy theory for modeling uncertain surfaces
\multicite{Anile00Modeling,Zakaria14Fuzzy}.



\subsection{Fuzzy Sets and Fuzzy Intervals}
\label{sec:541fuzzySets}

In the following, we repeat very briefly the necessary
definitions of fuzzy theory.
A more in-depth introduction can be found in
\multicite{Klimke06Uncertainty,Walz16Fuzzy}.

\paragraph{Fuzzy sets}

Let $X \subset \real$ be a closed interval on the real line
and $\memfun{x}\colon X \to \clint{0, 1}$ be a function.
We call the graph $\fuzzy{x} := \{(x, \memfun{x}(x) \mid x \in X\}$
of $\memfun{x}$ a \term{fuzzy set} with
\term{membership function} $\memfun{x}$.
Fuzzy sets generalize ordinary subsets of $X$,
which can be obtained by requiring $\memfun{x}(X) \subset \{0, 1\}$.
In this case, the fuzzy set is called \term{crisp} and
$\fuzzy{x}$ can be identified with the ordinary set
$\{x \in X \mid \memfun{x}(x) = 1\}$.
A fuzzy set $\fuzzy{x}$ is \term{normalized}
if $\max_{x \in X} \memfun{x}(x) = 1$.
A \term{convex} fuzzy set $\fuzzy{x}$ satisfies
$\min(\memfun{x}(a), \memfun{x}(c)) \le \memfun{x}(b)$ for all $a, b, c \in X$
with $a \le b \le c$.

\paragraph{Fuzzy intervals and $\alpha$-cuts}

A convex and normalized fuzzy set $\fuzzy{x}$ with
piecewise continuous membership function $\memfun{x}$ is called
\term{fuzzy interval}.
If $\{x \in X \mid \memfun{x}(x) = 1\} = \{a\}$ for some $a \in X$,
then the fuzzy interval $\fuzzy{x}$ is called \term{fuzzy number}.

For $\alpha \in \clint{0, 1}$, the $\alpha$-cut of $\fuzzy{x}$ is
defined as $\acut{x}{\alpha} := \{x \in X \mid \memfun{x}(x) \ge \alpha\}$
for $\alpha > 0$ and $\acut{x}{0} := \supp \memfun{x}$ for $\alpha = 0$.
The $\alpha$-cuts of fuzzy intervals $\fuzzy{x}$ are always
nested closed intervals $\acut{x}{\alpha} = [a, b]$, i.e.,
$\acut{x}{\alpha_1} \supset \acut{x}{\alpha_2}$ for $\alpha_1 \le \alpha_2$.

\paragraph{Common types of fuzzy numbers and intervals}

There various types of fuzzy numbers and intervals \cite{Klimke06Uncertainty}.
Most common are
\term{triangular fuzzy numbers} (i.e., linear B-splines),
\term{trapezoidal fuzzy intervals}
(where a plateau of height one is inserted at the peak), and
\term{Gaussian fuzzy numbers} with membership function
$\memfun{x}(x) = \exp(-(x - \mu)^2/(2\sigma)^2)$.
As the support of Gaussian fuzzy numbers is unbounded,
\term{quasi-Gaussian fuzzy numbers} truncate the support
to a fixed multip1le of the standard deviation $\sigma$
\cite{Klimke06Uncertainty}.
However, it would be more natural to directly employ B-splines of
higher degree (normalized adequately), since they are a generalization
of triangular fuzzy numbers and their limit is a Gaussian fuzzy number.



\subsection{Fuzzy Extension Principle}
\label{sec:542fuzzyExtensionPrinciple}

\blindtext{}



\subsection{Using B-Splines on Sparse Grids to Propagate Fuzzy Uncertainties}
\label{sec:543fuzzyBSplines}

\blindtext{}
