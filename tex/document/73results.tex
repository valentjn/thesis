\section{Numerical Results}
\label{sec:73results}

\minitoc[-8mm]{90mm}{4}

\noindent
In the final section of this chapter,
we present and discuss numerical results for our
biomechanical model of the upper limb.



\subsection{Reference and Sparse Grid Solution}
\label{sec:731solutionTypes}

\paragraph{Reference solution}

Since the model is only two-dimensional, we can compute a reference solution
on a full grid.
Hence, we evaluate the exerted muscle forces $\forceT$ and $\forceB$ on
the full grid
\begin{equation}
  \{\ang{10}, \ang{11}, \dotsc, \ang{150}\} \times \{0, 0.1, \dotsc, 1\}
  \ni (\elbang, \actX),\quad
  X \in \{\mathrm{T}, \mathrm{B}\}.
\end{equation}
The resulting \num{1551} grid points
are interpolated with bicubic spline interpolation%
\footnote{%
  Computed with the Geometric Tools Engine, see
  \url{https://www.geometrictools.com/}
  \cite{Schneider03Geometric}.
}
to obtain \term{reference solutions}
$\forceTref, \forceBref\colon
\clint{\ang{10}, \ang{150}} \times \clint{0, 1} \to \real$,
which are shown in \cref{fig:biomech2ReferenceForce}.
Due to the high resolution of the full grid,
we may assume that the reference solutions are accurate enough
to ensure $\forceTref \approx \forceT$ and $\forceBref \approx \forceB$.
We refer to the resulting quantities with the superscript ``$\mathrm{ref}$'',
for instance, the corresponding equilibrium elbow angle
$\equielbangref{\forceL}$, which is displayed in
\cref{fig:biomech2ReferenceEquilibriumAngle}
for the exemplary loads of $\forceL = \SI{22}{\newton}$,
$\SI{-60}{\newton}$, and $\SI{180}{\newton}$.

\begin{figure}
  \includegraphics{biomech2ReferenceForce_1}%
  \;\;%
  \includegraphics{biomech2ReferenceForce_2}%
  \hfill%
  \rlap{\raisebox{53mm}{\;$\forceXref$ [\si{\kilo\newton}]}}%
  \includegraphics{biomech2ReferenceForce_3}%
  \caption[Reference triceps and biceps forces]{%
    Reference triceps and biceps forces $\forceXref$
    ($X \in \{\mathrm{T}, \mathrm{B}\}$).%
  }%
  \label{fig:biomech2ReferenceForce}%
\end{figure}

\begin{figure}
  \includegraphics{biomech2ReferenceEquilibriumAngle_4}%
  \\[2mm]%
  \subcaptionbox{%
    $\forceL = \SI{22}{\newton}$%
  }[49mm]{%
    \includegraphics{biomech2ReferenceEquilibriumAngle_1}%
  }%
  \hfill%
  \subcaptionbox{%
    $\forceL = \SI{-60}{\newton}$%
  }[49mm]{%
    \includegraphics{biomech2ReferenceEquilibriumAngle_2}%
  }%
  \hfill%
  \subcaptionbox{%
    $\forceL = \SI{180}{\newton}$%
  }[49mm]{%
    \includegraphics{biomech2ReferenceEquilibriumAngle_3}%
  }%
  \caption[Reference equilibrium elbow angle]{%
    Reference equilibrium elbow angle $\equielbangref{\forceL}$
    for different loads $\forceL$.
    The empty areas correspond to activation pairs
    at which $\equielbangref{\forceL}$ is not well-defined
    (see \cref{eq:equilibriumAngle}).%
  }%
  \label{fig:biomech2ReferenceEquilibriumAngle}%
\end{figure}

\paragraph{Sparse grid solution}

Additionally, we evaluate $\forceT$ and $\forceB$ at the $\ngp = 49$
grid points
\begin{equation}
  \{(\elbang^{(k)}, \actX^{(k)}) \mid k = 1, \dotsc, \ngp\}
  \subset \clint{\ang{10}, \ang{150}} \times \clint{0, 1},\quad
  X \in \{\mathrm{T}, \mathrm{B}\},
\end{equation}
of the regular sparse grid $\interiorregsgset{n}{d}$ of
level $n = 5$ in $d = 2$ dimensions.%
\footnote{%
  The domain $\clint{\ang{10}, \ang{150}} \times \clint{0, 1}$
  is assumed to be implicitly normalized to the unit square
  $\clint{\*0, \*1}$.%
}
These values are interpolated using three
different hierarchical B-spline bases of degree $p = 1$, $3$, and $5$:
modified hierarchical B-splines
$\bspl[\modified]{\*l,\*i}{p}$
(see \cref{sec:313modification}),
modified hierarchical Clenshaw--Curtis B-splines
$\bspl[\cc,\modified]{\*l,\*i}{p}$
(see \cref{sec:314nonUniform}), and
modified hierarchical not-a-knot B-splines
$\bspl[\nak,\modified]{\*l,\*i}{p}$
(see \cref{sec:323modifiedNAKBSplines}).
The implementation was done using the sparse grid toolbox
\sgpp{} \cite{Pflueger10Spatially}.%
\footnote{%
  \url{http://sgpp.sparsegrids.org/}%
}
The corresponding interpolants and resulting quantities
are denoted with the superscripts
``$\sparse,\!p$'', ``$\sparse,\!p,\!\cc$'', or ``$\sparse,\!p,\!\nak$'',
respectively.
A superscript of ``$\sparse$'' without any further specification
means one of the three sparse grid quantities in general.
Note that the equilibrium elbow angle is \emph{not} interpolated
(neither in the full grid nor in the sparse grid case),
but rather obtained by inserting the interpolated muscle forces
into \cref{eq:totalMomentSurrogate,eq:equilibriumAngleSurrogate}.



\subsection{Errors of Muscle Forces and Equilibrium Angle}
\label{sec:732errors}

\paragraph{Quality of the reference interpolants}

Before we turn to the sparse grid interpolants,
we assess the quality of the reference interpolant on the full grid.
For this purpose, we evaluate the full grid interpolants
$\forceTintp, \forceBintp$
at the sparse grid points $(\elbang^{(k)}, \actX^{(k)})$
(which are not a subset of the full grid points!)
and compare the resulting values with the known exact values
$\forceT(\elbang^{(k)}, \actT^{(k)})$ and
$\forceB(\elbang^{(k)}, \actB^{(k)})$
of the muscle forces $\forceT, \forceB$.
We also incorporate the known values at the sparse
Clenshaw--Curtis grid points.
In particular, to simplify the notation,
let $G$ be the union of
$\{(\elbang^{(k)}, \actX^{(k)}) \mid k = 1, \dotsc, \ngp\}$ and
$\{(\elbang^{(k,\cc)}, \actX^{(k,\cc)}) \mid k = 1, \dotsc, \ngp\}$.%
\footnote{%
  We have $\setsize{G} = 2\ngp - 1$, since sparse grids of
  uniform and Clenshaw--Curtis type only
  share the center point $(\elbang, \actX) = (\ang{80}, 0.5)$,
  if there are no boundary points.%
}
Then, we can approximate the relative $\Ltwo$ interpolation error
of the reference interpolants by
\begin{equation}
  \frac{\normLtwo{\forceX - \forceXref}}{\normLtwo{\forceX}}
  \approx
  \frac{
    \norm[l^2]{
      (\forceX(\elbang, \actX) - \forceXref(\elbang, \actX))_
      {(\elbang, \actX) \in G}
    }
  }{
    \norm[l^2]{(\forceX(\elbang, \actX))_{(\elbang, \actX) \in G}}
  },\quad
  X \in \{\mathrm{T}, \mathrm{B}\},
\end{equation}
where the $l^2$ norm of a vector $\*a \in \real^{\ngp}$ is given by
$\norm[l^2]{\*a} := \sqrt{\tfrac{1}{\ngp} \sum_{k=1}^{\ngp} (a_k)^2}$.
Inserting the known values $\forceX(\elbang, \actX)$ and
$\forceXref(\elbang, \actX)$ on the \rhs, we obtain
\begin{equation}
  \frac{\normLtwo{\forceT - \forceTref}}{\normLtwo{\forceT}}
  \approx \SI{2.19}{\permille},\qquad
  \frac{\normLtwo{\forceB - \forceBref}}{\normLtwo{\forceB}}
  \approx \SI{2.06}{\permille}.
\end{equation}
These errors are very small, which justifies our assumption of
$\forceTref \approx \forceT$ and $\forceBref \approx \forceB$.

\paragraph{Error of the muscle forces}

\Cref{tbl:biomech2ErrorL2_1} contains the relative $\Ltwo$
interpolation errors
$\normLtwo{\forceXref - \forceXintp}/\normLtwo{\forceXref}$
($X \in \{\mathrm{T}, \mathrm{B}\}$)
for all hierarchical bases and the degrees $p = 1, 3, 5$.
All reported errors are relatively small
due to the smoothness of the original functions
(cf.\ $\forceXref$ in \cref{fig:biomech2ReferenceForce}).
All in all, the modified Clenshaw--Curtis B-splines perform best,
achieving relative $\Ltwo$ errors of below \SI{3.6}{\permille}
in the cubic case.
Surprisingly, the not-a-knot B-splines are the worst choice in our
comparison.
Their corresponding errors exceed \SI{1}{\percent} for the triceps
and $p \in \{3, 5\}$.
Possible reasons are two-fold:
First, there might be slight noise in the given muscle force data,
which is visible in \cref{fig:biomech2ReferenceForce},
as there seems to be a kink in $\forceTref$ at $\elbang \approx \ang{25}$.
Second, the employed regular sparse grids might be too coarse
as the convergence order of not-a-knot B-splines is higher,
but we might still be in the pre-asymptotic range
(see \cref{sec:541interpolation}).
The same observations hold for the degree $p$,
for which $p = 3$ seems to be the best choice,
as the errors increase again for $p = 5$.

\begin{table}
  \newcommand*{\bi}{$\bspl[\modified]{l,i}{p}$}
  \newcommand*{\bii}{$\bspl[\cc,\modified]{l,i}{p}$}
  \newcommand*{\biii}{$\bspl[\nak,\modified]{l,i}{p}$}
  \subcaptionbox{%
    $\normLtwo{\forceXref - \forceXintp}/\normLtwo{\forceXref}$
    [\si{\permille}] given as triceps/biceps pairs
    ($X \in \{\mathrm{T}, \mathrm{B}\}$).%
    \label{tbl:biomech2ErrorL2_1}%
  }[85.2mm]{%
    \setnumberoftableheaderrows{1}%
    \begin{tabular}{%
      >{\kern\tabcolsep}=l<{\kern2mm}%
      +c<{\kern-1mm}+c<{\kern-1mm}+c<{\kern\tabcolsep}%
    }
      \toprulec
      \headerrow
      $p$&   $1$&                  $3$&                  $5$\\
      \midrulec
      \bi&   $3.60,7.12$&          $3.05,7.00$&          $\mathbf{2.98},7.90$\\
      \bii&  $\mathbf{3.28},4.35$& $3.31,\mathbf{3.56}$& $3.35,3.64$\\
      \biii& $3.60,7.12$&          $3.09,10.0$&          $7.13,24.6$\\
      \bottomrulec
    \end{tabular}%
  }%
  \hfill%
  \subcaptionbox{%
    $\normLtwo{\equielbangref{\forceL} - \equielbangintp{\forceL}}/
    \normLtwo{\equielbangref{\forceL}}$
    [\si{\permille}] for $\forceL = \SI{22}{\newton}$.%
    \label{tbl:biomech2ErrorL2_2}%
  }[59mm]{%
    \setnumberoftableheaderrows{1}%
    \begin{tabular}{%
      >{\kern\tabcolsep}=l<{\kern2mm}%
      +c<{\kern-1mm}+c<{\kern-1mm}+c<{\kern\tabcolsep}%
    }
      \toprulec
      \headerrow
      $p$&   $1$&    $3$&             $5$\\
      \midrulec
      \bi&   $4.15$& $3.74$&          $3.72$\\
      \bii&  $3.42$& $\mathbf{2.83}$& $2.86$\\
      \biii& $4.15$& $4.06$&          $8.28$\\
      \bottomrulec
    \end{tabular}%
  }%
  \caption[Relative $L^2$ errors of forces and equilibrium elbow angle]{%
    Relative $\Ltwo$ errors of triceps/biceps force \emph{(left)} and
    equilibrium elbow angle \emph{(right)}
    for different hierarchical bases $\basis{\*l,\*i}$ and
    B-spline degrees $p$.
    Highlighted entries are the best among those with
    the same hierarchical basis or the same degree
    (similar to Nash equilibria).%
  }%
  \label{tbl:biomech2ErrorL2}%
\end{table}

\Cref{fig:biomech2ErrorForce} shows the pointwise absolute error
$\abs{\forceXref(\elbang, \actX) - \forceXintp(\elbang, \actX)}$
for the modified B-splines on uniform and Clenshaw--Curtis grids
in the cubic case $p = 3$.
Note that in contrast to usual interpolation settings,
the absolute errors $\abs{\forceXref - \forceXintp}$
shown in \cref{fig:biomech2ErrorForce} do not vanish in the
sparse grid points $(\elbang^{(k)}, \actX^{(k)})$
($X \in \{\mathrm{T}, \mathrm{B}\}$, $k = 1, \dotsc, \ngp$),
since $\forceXintp$ does not $\forceXref$ interpolate
in these points.%
\footnote{%
  It would have been possible to construct $\forceXintp$
  as a sparse grid interpolant of $\forceXref$.
  However, building a spline surrogate ($\forceXintp$)
  of another spline surrogate ($\forceXref$) would skew the results.%
}
As it is typical for sparse grid interpolants,
the error is the largest near the boundary of the domain.
In the Clenshaw--Curtis case, the maximal errors are given by
\begin{equation}
  \norm[\Linfty]{\forceTref - \forceTintp[p,\cc]}
  \approx \SI{10.6}{\newton},\qquad
  \norm[\Linfty]{\forceBref - \forceBintp[p,\cc]}
  \approx \SI{9.51}{\newton},
\end{equation}
where $\norm[\Linfty]{\forceXref - \forceXintp[p,\cc]}
:= \max_{(\elbang, \actX)}
\abs{\forceXref(\elbang, \actX) - \forceXintp[p,\cc](\elbang, \actX)}$
(since the functions are continuous).
If we restrict the domain to
$\clint{\ang{31}, \ang{129}} \times \clint{0.15, 0.85}$
by omitting \SI{15}{\percent} on each side of the original domain,
then the maximal absolute errors drop to only
\SI{6.73}{\newton} (triceps) and \SI{0.967}{\newton} (biceps),
which is small compared to maximal possible forces of
around \SI{1}{\kilo\newton}.

\begin{figure}
  \includegraphics{biomech2ErrorForce_5}%
  \\[2mm]%
  \subcaptionbox{%
    $\abs{\forceXref - \forceXintp[p]}$ for
    $X = \mathrm{T}$ \emph{(left)} and
    $X = \mathrm{B}$ \emph{(right).}%
  }[73mm]{%
    \includegraphics{biomech2ErrorForce_1}%
    \hfill%
    \includegraphics{biomech2ErrorForce_2}%
  }%
  \hfill%
  \subcaptionbox{%
    $\abs{\forceXref - \forceXintp[p,\cc]}$ for
    $X = \mathrm{T}$ \emph{(left)} and
    $X = \mathrm{B}$ \emph{(right).}%
  }[73mm]{%
    \includegraphics{biomech2ErrorForce_3}%
    \hfill%
    \includegraphics{biomech2ErrorForce_4}%
  }%
  \caption[Absolute error of muscle forces]{%
    Absolute error of muscle forces $\forceT, \forceB$ for
    modified cubic B-splines ($p = 3$)
    on uniform sparse grids \emph{(left two plots)} and
    on Clenshaw--Curtis sparse grids \emph{(right two plots).}%
  }%
  \label{fig:biomech2ErrorForce}%
\end{figure}

\paragraph{Error of the elbow equilibrium angle}

The relative $\Ltwo$ errors
$\normLtwo{\equielbangref{\forceL} - \equielbangintp{\forceL}}/
\normLtwo{\equielbangref{\forceL}}$
of the elbow equilibrium angle function are shown in
\cref{tbl:biomech2ErrorL2_1} for the example load of
$\forceL = \SI{22}{\newton}$.
Modified cubic Clenshaw--Curtis B-splines achieve the best results.
Therefore, we use this type of hierarchical basis
for the remainder of this chapter.
Pointwise plots of the absolute error
$\abs{\equielbangref{\forceL} - \equielbangintp[p,\cc]{\forceL}}$
are presented in \cref{fig:biomech2ErrorEquilibriumAngle}.
Again, the maximal error is pretty small:
For $\forceL = \SI{22}{\newton}$, it is only \ang{0.886}.
If we restrict the domain to $\clint{0.15, 0.85}^2$,
then this maximal error drops to \ang{0.103} (or \ang{;6.18;}),
as the areas near the boundary of $\clint{\*0, \*1}$
contribute the most to the error.

\begin{figure}
  \includegraphics{biomech2ErrorEquilibriumAngle_4}%
  \\[2mm]%
  \subcaptionbox{%
    $\forceL = \SI{22}{\newton}$%
  }[49mm]{%
    \includegraphics{biomech2ErrorEquilibriumAngle_1}%
  }%
  \hfill%
  \subcaptionbox{%
    $\forceL = \SI{-60}{\newton}$%
  }[49mm]{%
    \includegraphics{biomech2ErrorEquilibriumAngle_2}%
  }%
  \hfill%
  \subcaptionbox{%
    $\forceL = \SI{180}{\newton}$%
  }[49mm]{%
    \includegraphics{biomech2ErrorEquilibriumAngle_3}%
  }%
  \caption[Absolute error of the equilibrium elbow angle]{%
    Absolute error
    $\abs{\equielbangref{\forceL} - \equielbangintp[p,\cc]{\forceL}}$
    of the equilibrium elbow angle for
    modified hierarchical cubic Clenshaw--Curtis B-splines ($p = 3$)
    for different loads $\forceL$.
    In the empty areas, at least one of
    $\equielbangref{\forceL}$ and $\equielbangintp[p,\cc]{\forceL}$
    is not well-defined (see \cref{eq:equilibriumAngle}).%
  }%
  \label{fig:biomech2ErrorEquilibriumAngle}%
\end{figure}



\subsection{Test Scenario}
\label{sec:733scenario}

\paragraph{Definition of the test scenario}

In the following, we want to assess the performance
of the sparse grid interpolants for the optimization problems O1 and O2.
For this goal, we use (as in \cite{Valentin18Gradient}) a test scenario
that simulates a pseudo-dynamic sequence of motions
by varying the load force and/or the target elbow angle
in discrete time steps $t$ as seen in \cref{fig:biomech2ScenarioA_1}.
The test scenario is as follows:
%
\begin{figure}
  \includegraphics{biomech2ScenarioA_5}%
  \\[2mm]%
  \subcaptionbox{%
    Load $\forceL$ and target elbow angle $\tarelbang$.%
    \label{fig:biomech2ScenarioA_1}%
  }[73.3mm]{%
    \hspace*{3.3mm}%
    \includegraphics{biomech2ScenarioA_1}%
    \hspace*{3.6mm}%
  }%
  \hfill%
  \subcaptionbox{%
    Optimal activation parameters $\actT$ and $\actB$.%
    \label{fig:biomech2ScenarioA_2}%
  }[73.3mm]{%
    \hspace*{2.0mm}%
    \includegraphics{biomech2ScenarioA_2}%
    \hspace*{9.5mm}%
  }%
  \\[1mm]%
  \subcaptionbox{%
    Deviation $\abs{\equielbang{\forceL} - \tarelbang}$
    of attained elbow angle to target and
    deviation \smash{$|\momentref|$} of the moment from equilibrium.%
    \label{fig:biomech2ScenarioA_3}%
  }[73.3mm]{%
    \includegraphics{biomech2ScenarioA_3}%
  }%
  \hfill%
  \subcaptionbox{%
    Number of evaluations of $\equielbang{\forceL}$
    and number of Newton iterations per evaluation
    of $\equielbang{\forceL}$.%
    \label{fig:biomech2ScenarioA_4}%
  }[73.3mm]{%
    \includegraphics{biomech2ScenarioA_4}%
  }%
  \caption[Settings and results of the test scenario]{%
    Setting (a) of the test scenario and corresponding results (b, c, d).%
  }%
  \label{fig:biomech2ScenarioA}%
\end{figure}
%
\begin{enumerate}
  \item
  Find a feasible initial solution for problem O1
  with $\forceL(t_0) := \SI{22}{\newton}$ and
  $\tarelbang(t_0) := \ang{75}$.
  
  \item
  Apply O1 with $\forceL(t_1) := \SI{22}{\newton}$ and
  $\tarelbang(t_1) := \ang{75}$.
  
  \item
  Apply O2 with $\forceL(t_2) := \SI{22}{\newton}$ and
  $\tarelbang(t_2) := \ang{60}$ (changed target angle).
  
  \item
  Apply O2 with $\forceL(t_3) := \SI{30}{\newton}$ and
  $\tarelbang(t_3) := \ang{60}$ (changed load).
  
  \item
  Apply O2 with $\forceL(t_4) := \SI{40}{\newton}$ and
  $\tarelbang(t_4) := \ang{50}$ (changed load and target angle).
\end{enumerate}
%
For each of the steps 2 to 5, the activation levels $\actT, \actB$ obtained
in the previous step (i.e., either the feasible initial solution
of step 1 or the optimal solution of steps 2 to 4) are used
as the inputs for the optimization problem O1 or O2.
The feasible initial solution in step 1 is determined as explained
in \cref{sec:513gradientBasedConstrained}.

\paragraph{Solutions of problem O1}

We note that independent of $\forceL$ and $\tarelbang$,
every solution $(\actT, \actB)$ of problem O1 will be
on the boundary part of the domain $\clint{\*0, \*1}$,
where at least one activation parameter vanishes, i.e.,
\begin{equation}
  \{(\actT, \actB) \in \clint{\*0, \*1} \mid
  (\actT = 0) \lor (\actB = 0)\}.
\end{equation}
The reason is that the two muscles triceps and biceps are antagonistic
(see \cref{sec:711models}), meaning that they work against each other.
If both $\actT > 0$ and $\actB > 0$, then the body will waste energy,
as the same target elbow angle can be attained by reducing both
$\actT$ and $\actB$ at the same time, thus requiring less energy.
A visual example for this is \cref{fig:biomech2ReferenceEquilibriumAngle},
where the contour lines generally go from the bottom left
(small $\actT, \actB$) to the top right (large $\actT, \actB$).
This issue may prevented by either
using more complicated musculoskeletal models with more
than two muscles or different optimization problems
such as problem O2, where the objective function differs.

\paragraph{Plots of numerical results}

\Cref{fig:biomech2ScenarioA_2,fig:biomech2ScenarioA_3,fig:biomech2ScenarioA_4}
show the results of the test scenario using the muscle forces
$\forceXintp[p,\cc]$ obtained by interpolating with
modified hierarchical cubic Clenshaw--Curtis B-splines (solid lines, $p = 3$).
For comparison, we repeat the solution process
with the forces obtained by interpolating with the
corresponding hierarchical piecewise linear basis (dashed lines, $p = 1$) and
with the reference forces $\forceXref$ (dotted lines).
For the piecewise linear basis,
we use exactly the same method as for the cubic case
(Newton method for $\equielbangintp{\forceL}$,
Augmented Lagrangian with adaptive gradient descent for the
solution of problems O1 and O2),
although the derivatives of the muscle forces have kinks.

\paragraph{Equilibrium elbow angle}

In \cref{fig:biomech2ScenarioA_2}, we see that the activation levels
of all three methods are more or less the same.
However, \cref{fig:biomech2ScenarioA_3} reveals that even these small
difference lead to deviations of the resulting equilibrium elbow angle
that differ from the target angle by up to two orders of magnitude.
The two green lines with filled markers at the bottom of
\cref{fig:biomech2ScenarioA_3} show the error of
the equilibrium elbow angle $\equielbangintp{\forceL}$
using sparse grid interpolation to the desired target angle $\tarelbang$.
Unsurprisingly, this error is very small as the optimizer minimizes it
as part of the constraint.
The true error, which is obtained by
using the reference equilibrium elbow angle $\equielbangref{\forceL}$,
is in general much larger
(top two green lines in \cref{fig:biomech2ScenarioA_3}
with hollow markers).
Here, we see that the cubic B-splines decrease the error
by to two orders of magnitude compared to the
standard piecewise linear basis.
The reasons are two-fold:
First, the error of $\equielbang{\forceL}$ is generally smaller
when using higher-order B-splines as we have seen above.
Second, higher-order B-splines are continuously differentiable,
which makes them suitable for gradient-based optimization.
In contrast, the surrogates obtained by piecewise linear interpolation
have kinks, which may complicate finding optimal points
in the augmented Lagrangian and Newton methods.

\paragraph{Number of evaluations and Newton iterations}

This is supported by \cref{fig:biomech2ScenarioA_4},
which shows the number of evaluations of $\equielbang{\forceL}$
during the optimization and the average number of Newton iterations
per evaluation.
While the number of total evaluations is similar for all three methods,
the number of required Newton iterations to achieve convergence
is in general around \SI{50}{\percent} larger for the piecewise linear
basis functions.



\subsection{Spatial Adaptivity}
\label{sec:734adaptivity}

\dummytext[6]{}
