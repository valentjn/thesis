\section{Numerical Results}
\label{sec:73results}

\minitoc[-5mm]{90mm}{4}

\noindent
In the final section of this chapter,
we present and discuss numerical results for our
biomechanical model of the upper limb.



\subsection{Reference and Sparse Grid Solution}
\label{sec:731solutionTypes}

\paragraph{Reference solution}

Since the model is only two-dimensional, we can compute a reference solution
on a full grid.
Hence, we evaluate the exerted muscle forces $\forceT$ and $\forceB$ on
the full grid
\begin{equation}
  \{\ang{10}, \ang{11}, \dotsc, \ang{150}\} \times \{0, 0.1, \dotsc, 1\}
  \ni (\elbang, \actX),\quad
  {\ast} \in \{\mathrm{T}, \mathrm{B}\}.
\end{equation}
The resulting grid data are interpolated with bicubic spline interpolation
to obtain \term{reference solutions}
$\forceTref, \forceBref\colon
\clint{\ang{10}, \ang{150}} \times \clint{0, 1} \to \real$.
Due to the high resolution of the full grid,
we may assume that the reference solutions are accurate enough
to ensure $\forceTref \approx \forceT$ and $\forceBref \approx \forceB$.

\paragraph{Sparse grid solution}



\todo{reference SG++}



\subsection{Interpolation Error of Muscle Forces}
\label{sec:732forceInterpolation}

\dummytext[3]{}



\subsection{Error of the Equilibrium Angle}
\label{sec:733equilbriumAngle}

\dummytext[3]{}



\subsection{Error in Test Scenarios}
\label{sec:734scenarios}

\dummytext[3]{}



\subsection{Outlook: Spatial Adaptivity}
\label{sec:735adaptivity}

\dummytext[3]{}
