\section{Numerical Results}
\label{sec:73results}

\minitoc[-5mm]{90mm}{4}

\noindent
In the final section of this chapter,
we present and discuss numerical results for our
biomechanical model of the upper limb.



\subsection{Reference and Sparse Grid Solution}
\label{sec:731solutionTypes}

\paragraph{Reference solution}

Since the model is only two-dimensional, we can compute a reference solution
on a full grid.
Hence, we evaluate the exerted muscle forces $\forceT$ and $\forceB$ on
the full grid
\begin{equation}
  \{\ang{10}, \ang{11}, \dotsc, \ang{150}\} \times \{0, 0.1, \dotsc, 1\}
  \ni (\elbang, \actX),\quad
  {\ast} \in \{\mathrm{T}, \mathrm{B}\}.
\end{equation}
The resulting \num{1551} grid points
are interpolated with bicubic spline interpolation%
\footnote{%
  Computed with the Geometric Tools Engine, see
  \url{https://www.geometrictools.com/}
  \cite{Schneider03Geometric}.
}
to obtain \term{reference solutions}
$\forceTref, \forceBref\colon
\clint{\ang{10}, \ang{150}} \times \clint{0, 1} \to \real$,
which are shown in \cref{fig:biomech2ReferenceForce}.
Due to the high resolution of the full grid,
we may assume that the reference solutions are accurate enough
to ensure $\forceTref \approx \forceT$ and $\forceBref \approx \forceB$.
We refer to the resulting quantities with the superscript ``$\mathrm{ref}$'',
for instance, the corresponding equilibrium elbow angle
$\equielbangref{\forceL}$, which is displayed in
\cref{fig:biomech2ReferenceEquilibriumAngle}
for the exemplary loads of $\forceL = \SI{22}{\newton}$,
$\SI{-60}{\newton}$, and $\SI{180}{\newton}$.

\begin{figure}
  \includegraphics{biomech2ReferenceForce_1}%
  \;\;%
  \includegraphics{biomech2ReferenceForce_2}%
  \hfill%
  \rlap{\raisebox{53mm}{\;$\forceXref$ [\si{\kilo\newton}]}}%
  \includegraphics{biomech2ReferenceForce_3}%
  \caption[Reference triceps and biceps forces]{%
    Reference triceps and biceps forces $\forceXref$
    ($X \in \{\mathrm{T}, \mathrm{B}\}$).%
  }%
  \label{fig:biomech2ReferenceForce}%
\end{figure}

\begin{figure}
  \includegraphics{biomech2ReferenceEquilibriumAngle_4}%
  \\[2mm]%
  \subcaptionbox{%
    $\forceL = \SI{22}{\newton}$%
  }[49mm]{%
    \includegraphics{biomech2ReferenceEquilibriumAngle_1}%
  }%
  \hfill%
  \subcaptionbox{%
    $\forceL = \SI{-60}{\newton}$%
  }[49mm]{%
    \includegraphics{biomech2ReferenceEquilibriumAngle_2}%
  }%
  \hfill%
  \subcaptionbox{%
    $\forceL = \SI{180}{\newton}$%
  }[49mm]{%
    \includegraphics{biomech2ReferenceEquilibriumAngle_3}%
  }%
  \caption[Reference equilibrium elbow angle]{%
    Reference equilibrium elbow angle $\equielbangref{\forceL}$
    for different loads $\forceL$.
    The empty areas correspond to activation pairs $(\actT, \actB)$
    at which $\equielbangref{\forceL}$ is not well-defined
    (see \cref{eq:equilibriumAngle}).%
  }%
  \label{fig:biomech2ReferenceEquilibriumAngle}%
\end{figure}

\paragraph{Sparse grid solution}

Additionally, we evaluate $\forceT$ and $\forceB$ at the $\ngp = 49$
grid points
\begin{equation}
  \{(\elbang^{(k)}, \actX^{(k)}) \mid k = 1, \dotsc, \ngp\}
  \subset \clint{\ang{10}, \ang{150}} \times \clint{0, 1},\quad
  X \in \{\mathrm{T}, \mathrm{B}\},
\end{equation}
of the regular sparse grid $\interiorregsgset{n}{d}$ of
level $n = 5$ in $d = 2$ dimensions.%
\footnote{%
  The domain $\clint{\ang{10}, \ang{150}} \times \clint{0, 1}$
  is assumed to be implicitly normalized to the unit square
  $\clint{\*0, \*1}$.%
}
These values are interpolated using three
different hierarchical B-spline bases of degree $p = 1$, $3$, and $5$:
modified hierarchical B-splines
$\bspl[\modified]{\*l,\*i}{p}$
(see \cref{sec:313modification}),
modified hierarchical Clenshaw--Curtis B-splines
$\bspl[\cc,\modified]{\*l,\*i}{p}$
(see \cref{sec:314nonUniform}), and
modified hierarchical not-a-knot B-splines
$\bspl[\nak,\modified]{\*l,\*i}{p}$
(see \cref{sec:323modifiedNAKBSplines}).
The implementation was done using the sparse grid toolbox
\sgpp{} \cite{Pflueger10Spatially}.%
\footnote{%
  \url{http://sgpp.sparsegrids.org/}%
}
The corresponding interpolants and resulting quantities
are denoted with the superscripts
``$\sparse,\!p$'', ``$\sparse,\!p,\!\cc$'', or ``$\sparse,\!p,\!\nak$'',
respectively.
Note that the equilibrium elbow angle is \emph{not} interpolated
(neither in the full grid nor in the sparse grid case),
but rather obtained by inserting the interpolated muscle forces
into \cref{eq:totalMomentSurrogate,eq:equilibriumAngleSurrogate}.



\subsection{Errors of Muscle Forces and Equilibrium Angle}
\label{sec:732errors}

\paragraph{Quality of reference interpolant}

\dummytext{}

\paragraph{Error of the muscle forces}

\dummytext{}

\begin{figure}
  \includegraphics{biomech2ErrorForce_5}%
  \\[2mm]%
  \subcaptionbox{%
    $\abs{\forceXref - \forceXintp[p]}$ for
    $X = \mathrm{T}$ \emph{(left)} and
    $X = \mathrm{B}$ \emph{(right).}%
  }[73mm]{%
    \includegraphics{biomech2ErrorForce_1}%
    \hfill%
    \includegraphics{biomech2ErrorForce_2}%
  }%
  \hfill%
  \subcaptionbox{%
    $\abs{\forceXref - \forceXintp[p,\cc]}$ for
    $X = \mathrm{T}$ \emph{(left)} and
    $X = \mathrm{B}$ \emph{(right).}%
  }[73mm]{%
    \includegraphics{biomech2ErrorForce_3}%
    \hfill%
    \includegraphics{biomech2ErrorForce_4}%
  }%
  \caption[TODO]{%
    TODO%
  }%
  \label{fig:TODO}%
\end{figure}

\begin{table}
  \subcaptionbox{%
    $\normLtwo{\forceXref - \forceXintp}/\normLtwo{\forceXref}$
    [\textperthousand] given as triceps/biceps pairs
    ($X \in \{\mathrm{T}, \mathrm{B}\}$).%
  }[85mm]{%
    \setnumberoftableheaderrows{1}%
    \begin{tabular}{%
      >{\kern\tabcolsep}=l<{\kern2mm}%
      +c<{\kern-1mm}+c<{\kern-1mm}+c<{\kern\tabcolsep}%
    }
      \toprulec
      \headerrow
      $p$&$1$&$3$&$5$\\
      \midrulec
      $\bspl[\modified]{l,i}{p}$&     $3.60,7.12$&$3.05,7.00$&$2.98,7.90$\\
      $\bspl[\cc,\modified]{l,i}{p}$& $3.28,4.35$&$3.30,3.56$&$3.35,3.64$\\
      $\bspl[\nak,\modified]{l,i}{p}$&$3.60,7.12$&$3.09,10.0$&$7.13,24.6$\\
      \bottomrulec
    \end{tabular}%
  }%
  \hfill%
  \subcaptionbox{%
    $\normLtwo{\equielbangref{\forceL} - \equielbangintp{\forceL}}/
    \normLtwo{\equielbangref{\forceL}}$
    [\textperthousand] for $\forceL = \SI{22}{\newton}$.%
  }[59mm]{%
    \setnumberoftableheaderrows{1}%
    \begin{tabular}{%
      >{\kern\tabcolsep}=l<{\kern2mm}%
      +c<{\kern-1mm}+c<{\kern-1mm}+c<{\kern\tabcolsep}%
    }
      \toprulec
      \headerrow
      $p$&$1$&$3$&$5$\\
      \midrulec
      $\bspl[\modified]{l,i}{p}$&     $4.15$&$3.74$&$3.72$\\
      $\bspl[\cc,\modified]{l,i}{p}$& $3.42$&$2.83$&$2.86$\\
      $\bspl[\nak,\modified]{l,i}{p}$&$4.15$&$4.06$&$8.28$\\
      \bottomrulec
    \end{tabular}%
  }%
  \caption[TODO]{%
    TODO%
  }%
  \label{tbl:TODO2}%
\end{table}

\paragraph{Error of the elbow equilibrium angle}

\dummytext{}




\subsection{Error in Test Scenarios}
\label{sec:733scenarios}

\dummytext[3]{}



\subsection{Outlook: Spatial Adaptivity}
\label{sec:734adaptivity}

\dummytext[3]{}
