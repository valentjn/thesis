\section{Proofs for Chapter 2}
\label{sec:a11chapter2}

\printornamentsfalse
\subsection{Proof of the Size of the Regular Sparse Grid with Coarse Boundaries}
\label{sec:a111proofGridSizeCoarseBoundary}
\printornamentstrue

\propGridSizeCoarseBoundary*

\begin{proof}
  Note that the outer union in the definition of $\coarselevelset{n}{d}{b}$ in
  \eqref{eq:coarseBoundary2} is indeed disjoint.
  Therefore,
  \begin{equation}
    \setsize{\coarseregsgset{n}{d}{b}}
    = \sum_{\substack{\*l \in \nat^d\\\normone{\*l} \le n}}
    \setsize{\hiset{\*l}} +
    \hspace*{-5mm}\sum_{
      \substack{
        \*l \in \natz^d \setminus \nat^d\\
        (\normone{\vecmax(\*l, \*1)} \le n - b + 1) \lor
        (\*l = \*0)
      }
    }\hspace*{-10mm} \setsize{\hiset{\*l}}.
  \end{equation}
  The first sum is the number $\setsize{\interiorregsgset{n}{d}}$
  of interior grid points in $\regsgset{n}{d}$.
  The second sum can be split into summands
  with the same number $q$ of zero entries,
  which we count with
  $N_\*l := \setsize{\{t \mid l_t = 0\}}
  = \normone{\vecmax(\*l, \*1)} - \normone{\*l}$,
  and the same level sum $m = \normone{\*l}$:
  \begin{equation}
    \setsize{\coarseregsgset{n}{d}{b}}
    = \setsize{\interiorregsgset{n}{d}} + 2^d +
    \sum_{q=1}^{d-1} \sum_{m=d-q}^{n-b-q+1}
    \sum_{
      \substack{
        \*l \in \natz^d\\
        N_\*l = q,\,\normone{\*l} = m}
    }\hspace*{-4mm} \setsize{\hiset{\*l}},
  \end{equation}
  where $2^d$ is the summand for $\*l = \*0$
  (number $\setsize{\hiset{\*0}}$ of corners of $\clint{\*0, \*1}$).
  The limits of the sum over $m$ are $d-q$,
  since there must be at least $d-q$ entries $\ge 1$ in a level vector
  with $q$ zero entries, and $n-b-q+1$,
  since $m = \normone{\*l}
  = \normone{\vecmax(\*l, \*1)} - N_\*l
  \le n-b+1 - N_\*l
  = n-b-q+1$.
  
  In general, the innermost summand $\setsize{\hiset{\*l}}$ can be calculated as
  $\setsize{\hiset{\*l}}
  = \prod_{l_t \ge 1} 2^{l_t-1} \cdot \prod_{l_t = 0} 2
  = 2^{\normone{\*l} - d + 2N_\*l}$.
  The number of innermost summands is given by
  \begin{equation}
    \setsize{\{\*l \in \natz^d \mid N_\*l = q,\; \normone{\*l} = m\}}
    = \binom{d}{q} \setsize{\{\*l \in \nat^{d-q} \mid \normone{\*l} = m\}}
    = \binom{d}{q} \binom{m-1}{d-q-1}
  \end{equation}
  by first putting $q$ zeros in $d$ places,
  for which there are $\binom{d}{q}$ possibilities, and then
  counting all positive vectors of length $d - q$ with level sum $m$,
  which can be done in $\binom{m-1}{d-q-1}$ ways.
  Thus,
  \begin{subequations}
    \begin{align}
      \setsize{\coarseregsgset{n}{d}{b}}
      &= \setsize{\interiorregsgset{n}{d}} + 2^d +
      \sum_{q=1}^{d-1} \binom{d}{q} \sum_{m=d-q}^{n-b-q+1}
      2^{m - d + 2q} \binom{m-1}{d-q-1}.\\
      \intertext{%
        Shifting the index $m \to (m + d - q)$ and rearranging the terms
        slightly, we obtain%
      }
      \cdots
      &= \setsize{\interiorregsgset{n}{d}} + 2^d +
      \sum_{q=1}^{d-1} 2^q \binom{d}{q} \sum_{m=0}^{n-d-b+1}
      2^m \binom{(d-q)+m-1}{(d-q)-1}.\\
      \intertext{%
        We can now use \thmref{lemma:numberOfGridPointsInterior} to
        conclude that%
      }
      \cdots
      &= \setsize{\interiorregsgset{n}{d}} +
      \sum_{q=1}^d 2^q \binom{d}{q}
      \setsize{\interiorregsgset{n-q-b+1}{d-q}}
    \end{align}
  \end{subequations}
  as desired.
\end{proof}



\subsection{%
  Correctness Proof of the Construction of the Regular Sparse Grid
  with Coarse Boundaries%
}
\label{sec:a112proofInvariantCoarseBoundary}

\propInvariantCoarseBoundary*

\begin{proof}
  First, we show that every inserted level $\*l' \in \natz^t$ in the inner loop
  can be found on the right-hand side of \eqref{eq:coarseInvariant}.
  If $\*l' := (\*l, 0)$
  is inserted for some $\*l \in \levelset^{(t-1)}$,
  then we have $\normone{\vecmax(\*l, \*1)} \le n+d+t-b$ or
  $\*l = \*0$ by \cref{line:algCoarseBoundary1} of
  \cref{alg:coarseBoundary}.
  In the first case, we have
  \begin{equation}
    \normone{\vecmax(\*l', \*1)}
    = \normone{\vecmax(\*l, \*1)} + 1
    \le n - d + t - b + 1,
  \end{equation}
  and in the second case $\*l' = \vec{0}$.
  In either case, $\*l'$ is contained in the \rhs of
  \eqref{eq:coarseInvariant}.
  
  If $\*l' := (\*l, l_t)$ is inserted
  for some $\*l \in \levelset^{(t-1)}$ and
  $l_t \in \{1, \dotsc, l^\ast\}$, then there are,
  depending on whether $\*l \in \nat^{t-1}$ or not, two cases:
  \begin{itemize}
    \item
    If $\*l \in \nat^{t-1}$, then $\*l' \in \nat^{t-1}$ as well and
    $\normone{\*l'} \le \normone{\*l} + l^\ast = n - d + t$
    due to \cref{line:algCoarseBoundary2},
    i.e., $\*l'$ is contained in the first set of the \rhs of
    \eqref{eq:coarseInvariant}.
    
    \item
    If $\*l \notin \nat^{t-1}$, then $\*l' \notin \nat^{t-1}$ as well and
    $\normone{\vecmax(\*l', \*1)}
    \le \normone{\vecmax(\*l, \*1)} + l^\ast
    = n - d + t - b + 1$
    due to \cref{line:algCoarseBoundary3},
    i.e., $\*l$ is contained in the second set of the \rhs of
    \eqref{eq:coarseInvariant}.
  \end{itemize}
  Thus, all levels that the algorithm inserts into $\levelset^{(t)}$
  can be found on the \rhs of \eqref{eq:coarseInvariant}.
  
  It remains to prove that all levels on the \rhs of
  \eqref{eq:coarseInvariant}
  are inserted by the algorithm into $\levelset^{(t)}$ eventually.
  We prove this by induction over $t = 1, \dotsc, d$.
  For $t = 1$, the \rhs of \eqref{eq:coarseInvariant} equals
  $\{l \in \natz \mid l \le n-d+1\}$, which is just $\levelset^{(1)}$.
  For the induction step $(t - 1) \to t$, we assume
  the validity of the induction hypothesis
  \begin{equation}
    \label{eq:proofCoarseInductionHypothesis}
    \begin{aligned}
      \levelset^{(t-1)}
      &= \{\*l \in \nat^{t-1} \mid
      \normone{\*l} \le n-d+t-1\} \dotcup {}\\
      &\hphantom{{}={}}
      \paren*{
        \{\*l \in \natz^{t-1} \setminus \nat^{t-1} \mid
        \normone{\vecmax(\*l, \*1)} \le n-d+t-b\} \cup
        \{\vec{0}\}
      }.
    \end{aligned}
  \end{equation}
  The \rhs of \eqref{eq:coarseInvariant} has three parts,
  so we check for elements $\*l' \in \natz^t$
  of each of the three sets that they are appended to $\levelset^{(t)}$
  eventually.
  
  First, let $\*l' = (\*l, l_t)$ be in the first set of the \rhs,
  i.e., $\*l' \in \nat^t$ (in particular $l_t \ge 1$) and
  $\normone{\*l'} \le n - d + t$.
  Note that $\*l$ will be encountered in the inner loop, as
  $\*l \in \nat^{t-1}$ and
  $\normone{\*l} = \normone{\*l'} - l_t \le n - d + t - 1$,
  which implies $\*l \in \levelset^{(t-1)}$ by the induction
  hypothesis \eqref{eq:proofCoarseInductionHypothesis}.
  Since $1 \le l_t \le l^\ast$
  (due to
  $l_t = \normone{\*l'} - \normone{\*l} \le n-d+t - \normone{\*l} = l^\ast$),
  the level $\*l'$ is inserted into $\levelset^{(t)}$ during the innermost loop
  in \cref{line:algCoarseBoundary4}.
  
  Second, let $\*l' = (\*l, l_t)$
  be in the second set of the \rhs, i.e., we have
  $\*l' \notin \nat^t$ and
  $\normone{\vecmax(\*l', \*1)} \le n-d+t-b+1$.
  Here, there are three cases:
  \begin{enumerate}
    \item
    $l_t \ge 1$:
    This implies $\*l \notin \nat^{t-1}$ and 
    $\normone{\vecmax(\*l, \*1)}
    = \normone{\vecmax(\*l', \*1)} - l_t
    \le n-d+t-b$.
    Consequently, $\*l \in \levelset^{(t-1)}$ by the induction hypothesis
    \eqref{eq:proofCoarseInductionHypothesis}.
    As $l_t \in \{1, \dotsc, l^\ast\}$
    (due to $l_t \le n-d+t-b -
    \normone{\vecmax(\*l, \*1)} = l^\ast$),
    $\*l$ is added to $\levelset^{(t)}$ in \cref{line:algCoarseBoundary4}.
    
    \item
    $l_t = 0$ and $\*l \in \nat^{t-1}$:
    This implies $\normone{\*l} = \normone{\*l'}
    = \normone{\vecmax(\*l', \*1)} - 1
    \le n - d + t - b
    \le n - d + t - 1$ since $b \ge 1$.
    Again, by the induction hypothesis
    \eqref{eq:proofCoarseInductionHypothesis},
    $\*l$ is added to $\levelset^{(t)}$ in \cref{line:algCoarseBoundary5}
    due to
    $\normone{\vecmax(\*l, \*1)}
    = \normone{\*l} \le n - d + t - b$.
    
    \item
    $l_t = 0$ and $\*l \notin \nat^{t-1}$:
    This implies $\normone{\vecmax(\*l, \*1)}
    = \normone{\vecmax(\*l', \*1)} - 1
    \le n - d + t - b$.
    Again, by the induction hypothesis
    \eqref{eq:proofCoarseInductionHypothesis},
    $\*l$ is added to $\levelset^{(t)}$ in \cref{line:algCoarseBoundary5}.
  \end{enumerate}
  
  \noindent
  Third, let $\*l = (\vec{0}, 0) \in \natz^t$
  be in the third set of the \rhs.
  This level is appended in \cref{line:algCoarseBoundary5}
  to $\levelset^{(t)}$, since $\*l' = \vec{0} \in \natz^{t-1}$
  is in $\levelset^{(t-1)}$ by the
  induction hypothesis~\eqref{eq:proofCoarseInductionHypothesis}.
\end{proof}
