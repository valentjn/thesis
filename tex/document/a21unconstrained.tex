\section{Unconstrained Problems}
\label{sec:a21unconstrained}

\disableornamentsfornextheadingtrue
\vspace{-5mm}
\subsection{Bivariate Unconstrained Problems}
\label{sec:a211bivariateUnconstrained}

\paragraph{Branin02}

The function originates from \cite{Munteanu98Global}.
Compared to \cite{Munteanu98Global},
we changed the domain from $\clint{-5, 10} \times \clint{0, 15}$
to $\clint{-5, 15}^2$,
which seems more common in recent literature \cite{Gavana13Global}.
In addition, \cite{Munteanu98Global} uses the reciprocal function value,
while searching for the maximum instead of the minimum.
\vspace{-1.6em}

\begin{subequations}
  \begin{gather}
    \centertestfunline{
      \vspace*{-10mm}
      \testobjfunscaled{Bra02}(\xscaled)
      := \paren*{
        -\,\frac{51\xse{1}^2}{40\pi^2} +
        \frac{5\xse{1}}{\pi} + \xse{2} - 6
      }^2 +
      \paren*{10 - \frac{5}{4\pi}} \cos(\xse{1}) \cos(\xse{2})
    }\\
    \centertestfunline{
      {} + \ln(\xse{1}^2 + \xse{2}^2 + 1) + 10,\hspace*{35mm}
    }\notag\\
    \centertestfunline{
      \xscaled \in \clint{-5, 15}^2,\quad
      \xoptscaled = (-3.196988424804, 12.52625788532),
    }\\
    \centertestfunline{
      \testobjfunscaled{Bra02}(\xoptscaled) = 5.558914403894
    }
  \end{gather}
\end{subequations}

\pagebreak

\paragraph{GoldsteinPrice}

This function originates from \cite{Goldstein71Descent},
where the function was stated without bounds for the optimization domain.
We took the domain $\clint{-2, 2}^2$ from \cite{Gavana13Global}.
In addition, we scaled the function values by the factor $10^{-4}$
for the sake of plotting.
\vspace{-1.6em}

\begin{subequations}
  \begin{gather}
    \centertestfunline{
      \testobjfunscaled{GoP}(\xscaled)
      := 10^{-4} \cdot \paren*{
        1 + (\xse{1} + \xse{2} + 1)^2
        (19 - 14\xse{1} + 3\xse{1}^2 - 14\xse{2} +
        6\xse{1}\xse{2} + 3\xse{2}^2)
      }
    }\\
    \centertestfunline{
      \hspace*{25mm}
      {} \cdot
      \paren*{
        30 + (2\xse{1} - 3\xse{2})^2
        (18 - 32\xse{1} + 12\xse{1}^2 + 48\xse{2} -
        36\xse{1}\xse{2} + 27\xse{2}^2)
      },
    }\notag\\
    \centertestfunline{
      \xscaled \in \clint{-2, 2}^2,\quad
      \xoptscaled = (0, -1),
    }\\
    \centertestfunline{
      \testobjfunscaled{GoP}(\xoptscaled) = 3 \cdot 10^{-4}
    }
  \end{gather}
\end{subequations}

\paragraph{Schwefel06}

This function originates from \cite{Schwefel77Numerische}.
We changed the domain from $\clint{-3, 5} \times \clint{-1, 7}$ to
$\clint{-6, 4}^2$, such that the optimum point is not located at the
center of the optimization domain.
\vspace{-1.6em}

\begin{subequations}
  \begin{gather}
    \centertestfunline{
      \testobjfunscaled{Sch06}(\xscaled)
      := \max(
        \abs{\xse{1} + 2\xse{2} - 7},
        \abs{2\xse{1} + \xse{2} - 5}
      ),
    }\\
    \centertestfunline{
      \xscaled \in \clint{-6, 4}^2,\quad
      \xoptscaled = (1, 3),\quad
      \testobjfunscaled{Sch06}(\xoptscaled) = 0
    }
  \end{gather}
\end{subequations}

\subsection{\texorpdfstring{$d$}{d}-Variate Unconstrained Problems}
\label{sec:a212dvariateUnconstrained}

\paragraph{Ackley}

The form of this function originates from \cite{Ackley87Connectionist},
where it was stated only for two variables.
We use the generalization to $d$ variables from \cite{Gavana13Global}.
The optimization domain $\clint{1.5, 6.5}^d$
was chosen such that it does not contain $\*0$,
where the gradient of the objective function becomes singular.
Otherwise, the function would not be continuously differentiable,
which would be a disadvantage for spline-based approaches
(see Schwefel06 and Schwefel22 for functions with discontinuous derivatives).
\vspace{-1.6em}

\begin{subequations}
  \begin{gather}
   \centertestfunline{
      \testobjfunscaled{Ack}(\xscaled)
      := -20 \exp\paren*{-\frac{\norm[2]{\xscaled}}{5\sqrt{d}}} -
      \exp\paren*{\frac{1}{d} \sum_{t=1}^d \cos(2\pi \xse{t})} +
      20 + \econst,
    }\\
    \centertestfunline{
      \xscaled \in \clint{1.5, 6.5}^d,\quad
      \xoptscaled = 1.974451986484 \cdot \*1,\quad
      \testobjfunscaled{Ack}(\xoptscaled) = 6.559645375628
    }
  \end{gather}
\end{subequations}

\paragraph{Alpine02}

This function originates from \cite{Clerc99Swarm}.
We changed the domain from $\clint{0, 10}^d$ to $\clint{2, 10}^d$
to exclude the singularities of the derivative of the objective function
at $\xse{t} = 0$.
In addition, the author of \cite{Clerc99Swarm} searched for maximal points.
For minimization, we changed the sign of the objective function.
\vspace{-1.6em}

\begin{subequations}
  \begin{gather}
    \centertestfunline{
      \testobjfunscaled{Alp02}(\xscaled)
      := -\prod_{t=1}^d \sqrt{\xse{t}} \sin(\xse{t}),\qquad
      \xscaled \in \clint{2, 10}^d,
    }\\
    \centertestfunline{
      \xoptscaled = 7.917052684666 \cdot \*1,\quad
      \testobjfunscaled{Alp02}(\xoptscaled) = -2.808131180070^d
    }
  \end{gather}
\end{subequations}

\paragraph{Schwefel22}

This function originates from \cite{Schwefel77Numerische}.
We changed the domain from $\clint{-10, 10}^d$ to
$\clint{-3, 7}^d$, such that the optimum point is not located at the
center of the optimization domain.
\vspace{-1.6em}

\begin{subequations}
  \begin{gather}
    \centertestfunline{
      \testobjfunscaled{Sch22}(\xscaled)
      := \sum_{t=1}^d \abs{\xse{t}} +
      \prod_{t=1}^d \abs{\xse{t}},\qquad
      \xscaled \in \clint{-3, 7}^d,
    }\\
    \centertestfunline{
      \xoptscaled = \*0,\quad
      \testobjfunscaled{Sch22}(\xoptscaled) = 0
    }
  \end{gather}
\end{subequations}
