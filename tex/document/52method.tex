\section{Optimization of Surrogates on Sparse Grids}
\label{sec:52method}

The methods presented in the last section can be combined to a
``meta-method'' for surrogate optimization.
The surrogates are constructed as interpolants on spatially adaptive
sparse grids, which we explain in the following.



\subsection{Novak--Ritter Adaptivity Criterion}
\label{sec:521novakRitter}

The classic surplus-based refinement strategy to generate
spatially adaptive sparse grids is not tailored to optimization,
as this strategy aims to minimize the overall $\Ltwo$ error.
However, in optimization, it is reasonable to generate more
points in regions where we suspect the global minimum
to increase the interpolant's accuracy in these regions.
Hence, we employ an adaptivity criterion proposed by
Novak and Ritter \cite{Novak96Global} for hyperbolic cross points
(it has also been applied to sparse grids
\multicite{Ferenczi05Globale,Valentin14Hierarchische,Valentin16Hierarchical}).

\paragraph{$m$-th order children}

As usual, the criterion works iteratively:
Starting with an initial regular sparse grid of a very coarse level,
the criterion selects a specific point $\gp{\*l,\*i}$ in each iteration
and inserts all its children into the grid.
This process is repeated until a given number $\ngpMax$ of grid points is
reached,
since we evaluate $\objfun$ at every grid point once and we assume that
function evaluations dominate the overall complexity.
The difference to common refinement criteria is that
a point may be selected multiple times, in which case
\term{higher-order children} are inserted.
The $m$-th order children $\gp{\*l',\*i'}$ of a grid point $\gp{\*l,\*i}$
satisfy
\begin{equation}
  \label{eq:indirectChild}
  \*l'_{-t} = \*l_{-t},\;\,
  \*i'_{-t} = \*i_{-t},\;\,
  l'_t = l_t + m,\;\,
  i'_t \in
  \begin{cases}
    \{1\},&(l_t = 0) \land (i_t = 0),\\
    \{2^m - 1\},&(l_t = 0) \land (i_t = 1),\\
    \{2^m i_t - 1, 2^m i_t + 1\},&\hphantom{(}l_t > 0,
  \end{cases}
\end{equation}
where $m \in \nat$ and $t \in \{1, \dotsc, d\}$
(cf.\ \eqref{eq:directAncestor} for $m = 1$).
The order is chosen individually for each child point to be inserted
as the lowest number $m$ such that $\gp{\*l',\*i'}$ does not exist
in the grid yet.

%\todo{add figure for m-th order children?}

\paragraph{Criterion}

The Novak--Ritter refinement criterion \cite{Novak96Global}
refines the grid point $\gp{\*l,\*i}$ that minimizes the product%
\footnote{%
  Compared to \cite{Novak96Global},
  we added one in the base of each factor to avoid ambiguities
  for $0^0$.
  In addition, we swapped $\gamma$ with $1-\gamma$
  to make $\gamma$ more consistent with its name as adaptivity.%
}
\begin{equation}
  (r_{\*l,\*i} + 1)^\gamma \cdot
  (\normone{\*l} + d_{\*l,\*i} + 1)^{1 - \gamma}.
\end{equation}
Here, $r_{\*l,\*i} := \setsize{
  \{(\*l',\*i') \in \liset \mid
  \objfun(\gp{\*l',\*i'}) \le \objfun(\gp{\*l,\*i})\}
} \in \{1, \dotsc, \setsize{\liset}\}$ is the \term{rank} of $\gp{\*l,\*i}$
(where $\liset$ is the current set of level-index pairs of the grid), i.e.,
the place of the function value at $\gp{\*l,\*i}$
in the ascending ordering of the function values at all points
of the current grid.
We denote the \term{degree} $d_{\*l,\*i} \in \natz$ of $\gp{\*l,\*i}$
as the number of previous refinements of this point.
Finally, $\gamma \in \clint{0, 1}$ is the \term{adaptivity parameter.}
%By choosing $\gamma = 0$, the function values become irrelevant and the
%criterion produces regular-like sparse grids.
%If we choose $\gamma = 1$, then the criterion always refines the point with
%the lowest function value, which means that the criterion is easily stuck
%in local minima.
We have to choose a suitable compromise between exploration ($\gamma = 0$)
and exploitation ($\gamma = 1$).
The best choice of course depends on the objective function $\objfun$ at hand,
but for our purposes, we choose a priori a value of $\gamma = 0.15$.
However, it may be an option to adapt the value of $\gamma$ automatically
during the grid generation phase.



\subsection{Global Optimization of Sparse Grid Surrogates}
\label{sec:522method}

\paragraph{Global, local, and globalized optimization methods}

In \cref{sec:51algorithms}, we presented various optimization methods
for the unconstrained case,
divided into global gradient-free methods such as differential evolution and
local gradient-based methods such as gradient descent.
A subset of these methods has been implemented in \sgpp
(see \cref{tbl:optimizationMethod}).
The gradient-based methods need an initial point and
they may get stuck in local minima.
Hence, we additionally implemented globalized versions
of the gradient-based methods
via a multi-start Monte Carlo approach with $m := \min(10d, 100)$
uniformly distributed pseudo-random initial points.%
\footnote{%
  We split the number of permitted function evaluations evenly
  among the $m$ parallel calls.%
  %of the gradient-based method%
}
This means there are three types of optimization methods:

\begin{enumerate}[label={(T\arabic*)},leftmargin=*]
  \item
  \label{item:globalMethods}
  Global gradient-free methods listed as implemented in
  \cref{tbl:optimizationMethod}
  
  \item
  \label{item:localMethods}
  Local gradient-based methods listed as implemented in
  \cref{tbl:optimizationMethod}
  %(need an initial point)%
  \footnote{%
     Excluding Levenberg--Marquardt, which is only applicable
     to least-squares problems.%
  }
  
  \item
  \label{item:globalizedMethods}
  Globalized versions of the methods of type \ref{item:localMethods}
  (do not need an initial point)
\end{enumerate}

\paragraph{Unconstrained optimization of sparse grid surrogates}

Given the objective function $\objfun\colon \clint{\*0, \*1} \to \real$,
the maximal number $\ngpMax \in \nat$ of evaluations of $f$, and
the adaptivity parameter $\gamma \in \clint{0, 1}$,
we determine an approximation $\xoptappr \in \clint{\*0, \*1}$
of the global minimum $\xopt$ of $\objfun$ as follows:

\begin{enumerate}
  \item
  Generate a spatially adaptive sparse grid $\sgset$
  with the Novak--Ritter refinement criterion
  for $\objfun$, $\ngpMax$, and $\gamma$.
  
  \item
  Determine the sparse grid interpolant $\sgintp$ of $\objfun$
  by solving the linear system \eqref{eq:hierarchizationProblem}.
  
  \item
  Optimize the interpolant:
  First, find the best grid point
  $\*x^{(0)} := \argmin_{\gp{\*l,\*i} \in \sgset} \objfun(\*x_{\*l,\*i})$.
  Second, apply the local methods of type \ref{item:localMethods}
  to the interpolant $\sgintp$ with $\*x^{(0)}$ as initial point.
  Let $\*x^{(1)}$ be the resulting point with minimal objective function value.
  Third, we apply the global and globalized methods
  of types \ref{item:globalMethods} and \ref{item:globalizedMethods}
  to the interpolant $\sgintp$.
  Again, let $\*x^{(2)}$ be the point with
  minimal $\objfun$ value.
  Finally, determine the point of $\{\*x^{(0)}, \*x^{(1)}, \*x^{(2)}\}$
  with minimal $\objfun$ value and return it as $\xoptappr$.
\end{enumerate}

Note that the third step requires a fixed number of additional
evaluations of the objective function,
which can be neglected compared to $\ngpMax$.
By default, we use the cubic modified hierarchical not-a-knot B-spline basis
$\bspl[\modified,\nak]{\*l,\*i}{p}$ ($p = 3$)
for the construction of the sparse grid surrogate.
However, we could apply any of the hierarchical (B-)spline bases presented in
\cref{chap:30BSplines,chap:40algorithms}.

\paragraph{Comparison methods}

We use two comparison methods.
First, we apply the gradient-free methods
(type \ref{item:globalMethods}) to the sparse grid interpolant
using modified piecewise linear hierarchical basis functions
(i.e., $p = 1$) on the same sparse grid as the cubic B-splines.
We cannot employ gradient-based optimization as the objective function
should be continuously differentiable and
discontinuous derivatives are usually numerically problematic
for gradient-based optimization methods
(see, e.g., \cite{Huebner14Mehrdimensionale}).
Second, we apply the gradient-free methods
(type \ref{item:globalMethods}) directly to the objective function.
We cannot use the gradient-based methods here as the gradient of the
objective function is assumed to be unknown.
For both of the comparison methods,
we make sure that the objective function is evaluated at most $\ngpMax$ times
by splitting the $\ngpMax$ evaluations
evenly among all employed optimization methods.

\paragraph{Constrained optimization}

For optimization problems with constraints,
we proceed exactly as for unconstrained optimization,
except that for optimizing the interpolant, we use the
constrained optimization algorithms implemented in \sgpp
as listed in \cref{tbl:optimizationMethod}.
We only replace the objective function $\objfun$ with a sparse grid
surrogate $\sgintp$ and we assume that the constraint function
$\ineqconfun$ can be evaluated fast.
However, it would also be possible to replace $\ineqconfun$
with a sparse grid surrogate.
In this case, it cannot be guaranteed that the resulting optimal point
$\xoptappr$ is feasible, i.e., we could have
$\lnot(\ineqconfun(\xoptappr) \le \*0)$.
