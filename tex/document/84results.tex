\section{Implementation and Numerical Results}
\label{sec:84results}

\minitoc[13mm]{65mm}{7}

\parbox{1em}{}
\vspace{-3em}



\printornamentsfalse
\subsection{Implementation}
\label{sec:841implementation}
\printornamentstrue

\paragraph{Parameter values and basis functions}

We use
a risk aversion factor of $\riskav = 3.5$,
a patience factor of $\patience = 0.97$,
a transaction cost rate of $\tac = \SI{1}{\percent}$, and
a minimum consumption of $\normconsume_{\min} = 0.001$.
The bond and stock return rates are taken from \cite{Cai10Stable}.
The models are solved for $T = 6$ time steps;
this number suffices to show all relevant numerical effects and results,
while keeping the computational effort at a reasonable level.
As higher-order B-spline basis functions,
we use the hierarchical weakly fundamental not-a-knot splines
$\bspl[\wfs,\nak]{\*l,\*i}{p}$ of cubic degree $p = 3$
to enable hierarchization with the unidirectional principle.

\vspace*{-0.5em}

\paragraph{Software}

The dynamic portfolio choice models were solved using a self-written
MATLAB framework.
The object-oriented framework was designed in such a way that
not only transaction costs problems,
but many other types of dynamic portfolio choice models can be handled.
For instance, the base class \texttt{LifecycleProblem} provides
an interface with abstract functions such as
\texttt{computeTerminalValueFunction} and
\texttt{computeStateTransition}.
The actual functionality implemented in the base class strongly resembles
the algorithms presented in \cref{sec:82algorithms}.
This is not only desirable from a modeling perspective,
but also facilitates future usage by other researchers.
For creating (hierarchization) and evaluating sparse grid interpolants,
the sparse grid toolbox \sgpp was used \cite{Pflueger10Spatially}.%
\footnote{%
  \url{http://sgpp.sparsegrids.org/}%
}
The emerging optimization problems were solved using
sequential quadratic programming methods supplied by the
NAG Toolbox for MATLAB.%
\footnote{%
  \url{https://www.nag.com/}%
}
To avoid getting stuck in local minima,
we repeat the optimization process for a varying number
of initial multi-start points (in the range of a few dozens).
All runtimes were measured on a shared memory computer
with 144 threads on 4x Intel Xeon E7-8880v3 (72 cores, 144 threads).



\subsection{Error Sources and Error Measure}
\label{sec:842errorSources}

\paragraph{Error sources}

In this application, there are the following error sources:

\begin{enumerate}[label=E\arabic*.,ref=E\arabic*,leftmargin=2.7em]
  \item
  \label{item:financeErrorInterpolationValue}
  Interpolation of the value function
  (i.e., $\normcetvalueintp_{t+1} \not= \normcetvaluefcn_{t+1}$)
  
  \item
  \label{item:financeErrorInterpolationPolicy}
  Interpolation of the policy functions
  (i.e., $\optnormpolicyintp_t \not= \optnormpolicyfcn_t$)
  
  \item
  \label{item:financeErrorExtrapolation}
  Extrapolation errors
  (i.e., $
  \normcetvalueintp_{t+1}(\state_{t+1})
  \not= \normcetvaluefcn_{t+1}(\state_{t+1})
  $)
  
  \item
  \label{item:financeErrorCropping}
  Errors from state space cropping
  (i.e., Euler errors do not vanish for exact solution)
  
  \item
  \label{item:financeErrorOptimization}
  Optimization error
  (i.e., the global minimum found by the optimizer is inaccurate)
  
  \item
  \label{item:financeErrorQuadrature}
  Quadrature error
  ($
  \expectation[t]{\cdots}
  \not= \sum_{j=1}^{m_{\quadweight}} \quadweight_t^{(j)}
  [\cdots](\stochastic_t^{(j)})
  $)
  
  \item
  \label{item:financeErrorRounding}
  Floating-point rounding errors
  (i.e., arithmetical operations are inaccurate)
\end{enumerate}
Due to the dynamic programming scheme,
the combination of all errors accumulates over $t$.
For instance, if the optimization does not find the global optimum
exactly or it only finds a local one for the time $t + 1$,
the error propagates from the interpolant $\normcetvalueintp_{t+1}$
on the right-hand side of the Bellman equation
\eqref{eq:normalizedTCPBellmanEquation} to $\normcetvalueintp_t$
on the left-hand side, and so on.
If the system does not damp these errors,
the error might become stronger and stronger backwards in time $t$.

\paragraph{Error measure}

We use the weighted Euler equation error
$\weightedeulererror_t(\normstate_t)$ ($\Ltwo$ norm or pointwise)
to assess the quality of the resulting policies.
As the errors generally grow backwards in time,
it suffices to consider $t = 0$.
However, since the Euler equation error can only be evaluated at points in
the simplex
$
  \Omega_\mathrm{simplex}
  := \{\normstate_t \in \clint{\*0, \*1} \mid \sumfcn(\normstate_t) \le 1\}
$, the $\Ltwo$ norm would
quickly converge to zero with growing dimensionality, even if the mean error
stayed constant.
Therefore, we normalize the $\Ltwo$ norm:
\begin{equation}
  \weightedeulererrorLtwo_t
  := \sqrt{d!} \cdot \normLtwo{\weightedeulererror_t}
  = \sqrt{
    \frac{1}{\vol{\Omega_\mathrm{simplex}}}
    \int_{\mathrlap{\Omega_\mathrm{simplex}}\hphantom{\Omega}}
    \weightedeulererror_t(\normstate_t)^2 \diff{}\normstate_t
  },
\end{equation}
where the expression under the root sign is approximated
via Monte Carlo quadrature as the mean of samples
of $\weightedeulererror_t(\normstate_t)^2$.



\subsection{Numerical Results}
\label{sec:843results}

\paragraph{Reference solution}

We use full grid solutions as reference solutions,
i.e., $\{\state_t^{(k)} \mid k = 1, \dotsc, \ngp_t\} = \fgset{n,d}$
for some fixed level $n \in \nat$ for all $t = 0, \dotsc, T$.
Obviously, this is only computationally feasible
for low dimensionalities $d$ due to the curse of dimensionality.
For each $d$,
\cref{tbl:financeSolutionFullRegularSparseGrids}
(see \cref{chap:a40financeDetails})
contains full grid solutions of different levels.
The reference solution for $d = 2$ is shown in
\cref{fig:financeSolution2DReference}.
Unfortunately, only full grid solutions up to $d = 3$ could be computed
due to excessive runtime for $d \ge 4$.
This underlines the need for sophisticated
discretization techniques such as sparse grids.

\begin{figure}
  \makebox[49mm][r]{%
    \includegraphics{financeSolution2D_1}%
  }%
  \hfill%
  \makebox[49mm][r]{%
    \includegraphics{financeSolution2D_2}%
  }%
  \hfill%
  \makebox[49mm][r]{%
    \includegraphics{financeSolution2D_4}%
  }%
  \\[0mm]%
  \makebox[49mm][r]{%
    \includegraphics{financeSolution2D_6}%
  }%
  \hfill%
  \makebox[49mm][r]{%
    \includegraphics{financeSolution2D_3}%
  }%
  \hfill%
  \makebox[49mm][r]{%
    \includegraphics{financeSolution2D_5}%
  }%
  \caption[Reference solution for the two-dimensional TCP]{%
    Full grid solution for the transaction costs problem
    with $d = 2$ stocks.
    Shown are the value function $\normcetvalueref_t$ \emph{(top left)} and the
    optimal policy $\optnormpolicyref_t$ for $t = 0$.%
  }%
  \label{fig:financeSolution2DReference}%
\end{figure}

\paragraph{Convergence of the weighted Euler equation error}

\Cref{fig:financeEulerError} shows the convergence of the
$\Ltwo$ norm $\normLtwo{\weightedeulererror_0}$
weighted Euler equation error for $t = 0$ for regular sparse grids
and spatially adaptive sparse grids
for the cases of $d = 1, \dotsc, 4$ stocks.
\begin{figure}
  \includegraphics{financeEulerError_5}%
  \\[2mm]%
  \subcaptionbox{%
    $d = 1$%
  }[37mm]{%
    \includegraphics{financeEulerError_1}%
  }%
  \hfill%
  \subcaptionbox{%
    $d = 2$%
  }[37mm]{%
    \includegraphics{financeEulerError_2}%
  }%
  \hfill%
  \subcaptionbox{%
    $d = 3$%
  }[37mm]{%
    \includegraphics{financeEulerError_3}%
  }%
  \hfill%
  \subcaptionbox{%
    $d = 4$%
  }[37mm]{%
    \includegraphics{financeEulerError_4}%
  }%
  \caption[Convergence of the weighted Euler equation error]{%
    Convergence of the $\Ltwo$ norm $\normLtwo{\weightedeulererror_t}$
    of the weighted Euler equation error for $t = 0$ for
    regular sparse grids \emph{\textcolor{C0}{(blue)}} and
    spatially adaptive sparse grids \emph{\textcolor{C1}{(red)}.}
    The number $\ngp_t$ is the mean number
    $\frac{1}{m_{\policy}} \sum_{j=1}^{m_{\policy}} \ngp_{t,j}$
    of grid points over all policy grids for $t = 0$,
    where $\ngp_{t,j}$ is the number of grid points
    of the $j$-th policy entry.%
  }%
  \label{fig:financeEulerError}%
\end{figure}%
For this and the following plots,
the value function grid is kept unchanged
(usually a slightly refined regular sparse grid of level three or four),
while the mean number $\ngp_t$ of policy grid points increases
with decreasing refinement threshold $\refinetol_t$,
since the value function grid does not seem to have a great influence
on the convergence of the Euler equation errors.
The spatial adaptivity decreases the error by
two orders of magnitude in one dimension.
The gain is smaller for higher dimensionalities $d$,
but spatial adaptive grids still outperform regular grids.
For $d = 2$, we observe that the error saturates
after $\ngp_t \approx \num{4000}$ points before dropping below $10^{-5}$.
This is most likely due to the parts
\ref{item:financeErrorExtrapolation} to
\ref{item:financeErrorRounding} of the error that are not influenced
by sparse grid interpolation.
In addition, convergence significantly decelerates starting with $d = 4$.
For $d = 4$, spatially adaptive sparse grids with
a mean number $\ngp_t = \num{4252}$ of policy grid points for $t = 0$
are able to achieve a weighted Euler equation error of
$\normLtwo{\weightedeulererror_t} \approx \num{2.0e-2}$.
For $d = 5$, we are still able to achieve an acceptable error of
$\normLtwo{\weightedeulererror_t} \approx \num{3.2e-2}$
with spatially adaptive sparse grids with
a mean number $\ngp_t = \num{7768}$ of policy grid points for $t = 0$.
While we do not see any convergence for this dimensionality,
this is still a major result as such high-dimensional models
could not be solved with state-of-the-art methods up to now.

\paragraph{Optimal policies in 2D and 5D}

The value function and optimal policies corresponding to
sparse grid solutions for
$d = 2$ stocks with $\ngp_0 = \num{879}$ policy grid points or
$d = 5$ stocks with $\ngp_0 = \num{7768}$ policy grid points
are shown in
\cref{fig:financeSolution2DSparseGrid} and
\cref{fig:financeSolution5DSparseGrid}, respectively.
Obviously, most grid points are placed along the various kinks in the
policies.
Interestingly, experiments show that the surplus-based refinement
criterion does not place more grid points along the perfect diagonal kink
caused by the cropping of the state space
(i.e., along $\sumfcn(\vstock_t) = 1$).
It is possible to circumvent this issue by either
transforming the domain (e.g., rotations as in \cite{Bohn18Optimally}) or
directly incorporating the distance to the diagonal into the
refinement criterion for the value function.
However, we refrain from doing so here as this does not seem to
drastically improve results.

\begin{figure}
  \subcaptionbox{%
    $\normcetvalueintp[1]_t$%
  }[48mm]{%
    \includegraphics{financeSolution2D_7}%
  }%
  \hfill%
  \subcaptionbox{%
    $\normbuy[\sparse,1]_{t,1}$%
  }[48mm]{%
    \includegraphics{financeSolution2D_8}%
  }%
  \hfill%
  \subcaptionbox{%
    $\normsell[\sparse,1]_{t,1}$%
  }[48mm]{%
    \includegraphics{financeSolution2D_10}%
  }%
  \\[2mm]%
  \subcaptionbox{%
    $\normbond_t^{\sparse,1}$%
  }[48mm]{%
    \includegraphics{financeSolution2D_12}%
  }%
  \hfill%
  \subcaptionbox{%
    $\normbuy[\sparse,1]_{t,2}$%
  }[48mm]{%
    \includegraphics{financeSolution2D_9}%
  }%
  \hfill%
  \subcaptionbox{%
    $\normsell[\sparse,1]_{t,2}$%
  }[48mm]{%
    \includegraphics{financeSolution2D_11}%
  }%
  \caption[Sparse grid solution for the two-dimensional TCP]{%
    Spatially adaptive sparse grid solution for the transaction costs problem
    with $d = 2$ stocks.
    \vspace{-0.15em}%
    Shown are the value function $\normcetvalueref_t$ \emph{(top left)} and the
    optimal policy $\optnormpolicyref_t$ for the initial time step $t = 0$,
    together with the corresponding grid points \emph{(dots).}
    The color coding is the same as in
    \cref{fig:financeSolution2DReference}.%
  }%
  \label{fig:financeSolution2DSparseGrid}%
\end{figure}

\begin{figure}
  \makebox[37mm][c]{%
    \hspace*{3.8mm}%
    \raisebox{-\height}{\includegraphics{financeSolution5D_13}}%
  }%
  \hfill%
  \makebox[37mm][c]{%
    \hspace*{2.9mm}%
    \raisebox{-\height}{\includegraphics{financeSolution5D_14}}%
  }%
  \hfill%
  \makebox[37mm][c]{%
    \hspace*{4.7mm}%
    \raisebox{-\height}{\includegraphics{financeSolution5D_16}}%
  }%
  \hfill%
  \makebox[37mm][c]{%
    \hspace*{4.5mm}%
    \raisebox{-\height}{\includegraphics{financeSolution5D_15}}%
  }%
  \\[1mm]%
  \makebox[37mm][c]{%
    \includegraphics{financeSolution5D_1}%
  }%
  \hfill%
  \makebox[37mm][c]{%
    \includegraphics{financeSolution5D_4}%
  }%
  \hfill%
  \makebox[37mm][c]{%
    \includegraphics{financeSolution5D_12}%
  }%
  \hfill%
  \makebox[37mm][c]{%
    \includegraphics{financeSolution5D_9}%
  }%
  \\[1mm]%
  \makebox[37mm][c]{%
    \includegraphics{financeSolution5D_2}%
  }%
  \hfill%
  \makebox[37mm][c]{%
    \includegraphics{financeSolution5D_5}%
  }%
  \hfill%
  \makebox[37mm][c]{%
    \includegraphics{financeSolution5D_7}%
  }%
  \hfill%
  \makebox[37mm][c]{%
    \includegraphics{financeSolution5D_10}%
  }%
  \\[1mm]%
  \makebox[37mm][c]{%
    \includegraphics{financeSolution5D_3}%
  }%
  \hfill%
  \makebox[37mm][c]{%
    \includegraphics{financeSolution5D_6}%
  }%
  \hfill%
  \makebox[37mm][c]{%
    \includegraphics{financeSolution5D_8}%
  }%
  \hfill%
  \makebox[37mm][c]{%
    \includegraphics{financeSolution5D_11}%
  }%
  \caption[Sparse grid solution for the five-dimensional TCP]{%
    Spatially adaptive sparse grid solution for the transaction costs problem
    with $d = 5$ stocks.
    \vspace{-0.15em}%
    Shown are slice plots of
    the value function $\normcetvalueref_t$ \emph{(top left)} and
    the optimal policy $\optnormpolicyref_t$ for the initial time step $t = 0$,
    where for each function, a pair $(o_1, o_2)$
    of dimensions to be plotted was chosen,
    and the stock fractions $\stock_{t,o}$ of the other dimensions $o$
    are set to $0.1$.
    In addition, the corresponding grid points \emph{(dots)}
    are shown as the projection onto the
    $\stock_{t,o_1}$--$\stock_{t,o_2}$ plane.%
  }%
  \label{fig:financeSolution5DSparseGrid}%
\end{figure}

\paragraph{Pointwise error}

\dummytext[2]{}

\begin{figure}
  \includegraphics{financePointwiseError_4}%
  \\[2mm]%
  \subcaptionbox{%
    TODO%
  }[49mm]{%
    \includegraphics{financePointwiseError_1}%
  }%
  \hfill%
  \subcaptionbox{%
    TODO%
  }[49mm]{%
    \includegraphics{financePointwiseError_2}%
  }%
  \hfill%
  \subcaptionbox{%
    TODO%
  }[49mm]{%
    \includegraphics{financePointwiseError_3}%
  }%
  \caption[TODO]{%
    TODO%
  }%
  \label{fig:financePointwiseError}%
\end{figure}

\paragraph{Monte Carlo simulation}

% Portfolio-Gewichte, die Sharp-Ratio maximieren (Einheit Überrendite pro Einheit Risiko)
%loadResult(22)
%problem.plotLifecycleProfile(simulation.state, simulation.discreteState, simulation.policy, simulation.shock)
%A = [0.0256, 0.00576, 0.00288, 0.00176; ...
%0.00576, 0.0324, 0.0090432, 0.010692; ...
%0.00288, 0.0090432, 0.04, 0.0132; ...
%0.00176, 0.010692, 0.0132, 0.0484];
%A
%b = diag(A)
%b = sqrt(diag(A))
%format long
%b = sqrt(diag(A))
%format short
%b = sqrt(diag(A))
%A
%cor(A)
%corr(A)
%doc corr
%corr(A(1:3,1:3))
%B = corr(A); B = B(1:3,1:3);
%B = corr(A); B = B(1:3,1:3)
%corrcov(A)
%corrcov(A(1:3,1:3))
%[B, sigma] = corrcov(A(1:3,1:3))
%b = [0.0572, 0.0638, 0.07, 0.0764]
%diff(b)
%rf = problem.Return.riskfreeRate
%b = b(1:3);
%b
%(b-rf)./sigma
%(b-rf)/sigma
%(b-rf)./sigma
%(b-rf)./sigma'
%ver
%x = 1/3*ones(1,3)
%(x*b'-rf)./(x*sigma'*x')
%(x*b'-rf)./(x*sigma*x')
%x*b'
%rf
%x*sigma
%sigma
%A
%(x*b'-rf)./(x*A(1:3,1:3)*x')
%f = @(x) (x*b'-rf)./(x*A(1:3,1:3)*x')
%help fminbnd
%help fmincon
%fmincon(f, [1/3 1/3 1/3], [], [], [], [], [0 0 0], [1 1 1])
%sum(ans)
%xopt = fmincon(f, [1/3 1/3 1/3], [], [], [], [], [0 0 0], [1 1 1])
%f(xopt)
%xopt = fmincon(-f, [1/3 1/3 1/3], [], [], [], [], [0 0 0], [1 1 1])
%f = @(x) -(x*b'-rf)./(x*A(1:3,1:3)*x')
%xopt = fmincon(f, [1/3 1/3 1/3], [], [], [], [], [0 0 0], [1 1 1])
%sum(xopt)
%xopt = fmincon(f, [1/3 1/3 1/3], [], [], [1 1 1], 1, [0 0 0], [1 1 1])
%sum(xopt)
%1.8/(1.45+1.45+1.8)
%1.45/(1.45+1.45+1.8)
%A
%A = diag(diag(A))
%xopt = fmincon(f, [1/3 1/3 1/3], [], [], [1 1 1], 1, [0 0 0], [1 1 1])

% certainty-equivalent consumption
% mit B-splines und linearen Funktionen

\begin{figure}
  \includegraphics{financeSimulation_5}%
  \\[2mm]%
  \subcaptionbox{%
    $d = 3$%
  }[48mm]{%
    \includegraphics{financeSimulation_1}%
  }%
  \hfill%
  \subcaptionbox{%
    $d = 4$%
  }[48mm]{%
    \includegraphics{financeSimulation_2}%
  }%
  \hfill%
  \subcaptionbox{%
    $d = 5$%
  }[48mm]{%
    \includegraphics{financeSimulation_3}%
  }%
  \\[2mm]%
  \subcaptionbox{%
    $d = 3$ (stacked)%
  }[48mm]{%
    \includegraphics{financeSimulation_4}%
  }%
  \hfill%
  \begin{minipage}[b]{92mm}%
    \caption[TODO]{%
      TODO TODO TODO TODO TODO TODO TODO TODO TODO TODO TODO TODO TODO
      TODO TODO TODO TODO TODO TODO TODO TODO TODO TODO TODO TODO TODO
      TODO TODO TODO TODO TODO TODO TODO TODO TODO TODO TODO TODO TODO
      TODO TODO TODO TODO TODO TODO TODO TODO TODO TODO TODO TODO TODO%
    }%
    \vspace*{-6.3mm}%
    \label{fig:financeSimulation}%
  \end{minipage}%
\end{figure}

\dummytext[2]{}

\paragraph{Complexity and runtime analysis}

% analyze runtime behavior over N (fixed d) and d

% (e-mail 2018-01-03)
%Runtime of one optimize() run
%- -----------------------------
%O(#New state grid points
%* #Optimization iterations
%* #Quadrature points
%* #Old state grid points
%* #State dimensions
%* Runtime of one 1D basis evaluation)
%
%where
%#(Optimization iterations) depends on #(Policy dimensions),
%#(Policy dimensions) = 2 * #Stocks + 1
%#(State dimensions) = #Stocks + 1
%
%(modulo extrapolations)

\begin{equation}
  \landauO{
    % for every time step
    T
    % for every grid point of new time step
    \cdot \ngp_t
    % for every optimization iteration
    \cdot \text{\#optimizer iterations}
    \cdot
    \underbrace{
      % for every gradient entry of the objective function
      m_{\policy}
      % for every quadrature point
      \cdot m_{\stochastic}
      \cdot
      \overbrace{
        % for every summand of the sparse grid function
        % (grid point of old time step)
        \ngp_{t+1}
        % for every tensor product factor
        \cdot m_{\state}
        % for every B-spline summand
        \cdot p
      }^{\mathclap{\text{evaluation of interpolant}}}
    }_{\text{evaluation of objective gradient}}\,
  }
\end{equation}

\#optimizer iterations contains multi-start points!

\begin{equation}
  \landauO{
    T
    \cdot \ngp
    \cdot \text{\#optim.\ it.}
    \cdot d
    \cdot d
    \cdot \ngp
    \cdot d
    \cdot p
  }
  = \landauO{T \ngp^2 d^3 p \cdot \text{\#optim.\ it.}}??
\end{equation}

\dummytext[3]{}

\paragraph{Impact of the B-spline degree}

\dummytext[3]{}

\paragraph{Comparing exact gradients to finite differences}

\dummytext[3]{}
