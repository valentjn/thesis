\section{Implementation and Numerical Results}
\label{sec:84results}

\minitoc[11.5mm]{65mm}{6}

\parbox{1em}{}
\vspace{-3em}



\printornamentsfalse
\subsection{Implementation}
\printornamentstrue

\paragraph{Parameter values and basis functions}

We use
a risk aversion factor of $\riskav = 3.5$,
a patience factor of $\patience = 0.97$,
a transaction cost rate of $\tac = \SI{1}{\percent}$, and
a minimum consumption of $\normconsume_{\min} = 0.001$.
The bond and stock return rates are taken from \cite{Cai10Stable}.
As higher-order B-spline basis functions,
we use the hierarchical weakly fundamental not-a-knot splines
$\bspl[\wfs,\nak]{\*l,\*i}{p}$ of cubic degree $p = 3$
to enable hierarchization with the unidirectional principle.

\vspace*{-0.5em}

\paragraph{Software}

The dynamic portfolio choice models were solved using a self-written
MATLAB framework.
The object-oriented framework was designed in such a way that
not only transaction costs problems,
but many other types of dynamic portfolio choice models can be handled.
For instance, the base class \texttt{LifecycleProblem} provides
an interface with abstract functions such as
\texttt{computeTerminalValueFunction} and
\texttt{computeStateTransition}.
The actual functionality implemented in the base class strongly resembles
the algorithms presented in \cref{sec:82algorithms}.
This is not only desirable from a modeling perspective,
but also facilitates the future usage by other researchers.
For the creation (hierarchization) and evaluation of sparse grid interpolants,
the sparse grid toolbox \sgpp was used \cite{Pflueger10Spatially}.%
\footnote{%
  \url{http://sgpp.sparsegrids.org/}%
}
The emerging optimization problems were solved using
sequential quadratic programming methods supplied by the
NAG Toolbox for MATLAB.%
\footnote{%
  \url{https://www.nag.com/}%
}



\subsection{Numerical Results}

\paragraph{Reference solution and error measures}

\dummytext[3]{}

\paragraph{Solution on regular sparse grids}

\dummytext[5]{}

\paragraph{Comparison with piecewise linear functions}

\dummytext[5]{}

\paragraph{Solution on spatially adaptive sparse grids}

\dummytext[5]{}

\paragraph{Monte Carlo simulation}

% certainty-equivalent consumption
% mit B-splines und linearen Funktionen

\dummytext[5]{}
