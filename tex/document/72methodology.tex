\section{Momentum Equilibrium and Elbow Angle Optimization}
\label{sec:72methodology}

\minitoc[-7mm]{70mm}{4}

\noindent
In this section, we give an overview of the methodology of our approach.
We closely follow the presentation of \cite{Valentin18Gradient}.



\subsection{From Muscle Forces to Equilibrium Angles}
\label{sec:721equilibrium}

\paragraph{Model inputs and outputs}

In the following, we regard simulations of the
human upper limb model described in \cref{sec:71model} as a black box,
which receives as its input
the elbow angle $\elbang \in \clint{\ang{10}, \ang{150}}$
and the activation parameters $\actT, \actB \in \clint{0, 1}$
of triceps and biceps.%
\footnote{%
  Here and in the following, the subscripts T, B, and L stand for
  triceps, biceps, and load, respectively.%
}
The outputs of the black box simulation are the forces
$\forceT(\elbang, \actT)$ and $\forceB(\elbang, \actB)$
that triceps and biceps exert.
These forces depend on the elbow angle as well as on the respective
activation parameter.
Gravitational forces due to the masses of bones or masses
are neglected in this context.
However, we allow the specification of an external load $\forceL$,
which is applied to the end of the forearm.
This load may the weight force of some object
that the arm is supposed to keep in position.

\paragraph{Moments and lever arms}

Each force exerts a \term{moment} (or \term{torque}) on the elbow joint.
The moments are the products of the forces $\forceX$
with the respective lever arms $\armX$
($X \in \{\mathrm{T}, \mathrm{B}, \mathrm{L}\}$).
The lever arms are approximated as in
\multicite{Roehrle16Two,Valentin18Gradient} by using
the tendon-displacement method of \cite{An84Determination}:
\begin{subequations}
  \begin{align}
    \armT(\elbang)
    &:= (-0.0009399 \{\elbang\}^2 + 0.1126 \{\elbang\} + 22.21)\;
    \si{\milli\meter},\\
    \armB(\elbang)
    &:= (-0.001482 \{\elbang\}^2 + 0.1776 \{\elbang\} + 35.02)\;
    \si{\milli\meter},\\
    \armL(\elbang)
    &:= \sin(\elbang) \cdot \SI{282.5}{\milli\meter},
  \end{align}
\end{subequations}
where $\{\elbang\}$ denotes the dimensionless value of $\elbang$
in degrees.
The lever arms are non-negative and the forces are signed, i.e.,
positive forces pull the forearm downwards,
negative forces pull it upwards.
In general, $\forceT, \forceL \ge \SI{0}{\newton}$ and
$\forceB \le \SI{0}{\newton}$.

\paragraph{Total moment and equilibrium elbow angle}

The \term{total moment} of the system is given by the function
\begin{subequations}
  \label{eq:totalMoment}
  \begin{gather}
    \moment_{\forceL,\actT,\actB}\colon
    \clint{\ang{10}, \ang{150}} \to \real,\\
    \moment_{\forceL,\actT,\actB}(\elbang)
    := \forceT(\elbang, \actT) \armT(\elbang) +
    \forceB(\elbang, \actB) \armB(\elbang) +
    \forceL \armL(\elbang),
  \end{gather}
\end{subequations}
cf.\ \cite{Valentin18Gradient}.
The system is in \term{equilibrium}
if the total moment vanishes, i.e.,
$\moment_{\forceL,\actT,\actB}(\elbang) = \SI{0}{\newton\meter}$.
We call the corresponding angle $\elbang$ the
\term{equilibrium elbow angle}
for the load $\forceL$ and the activation parameters $\actT, \actB$.
To find this angle for a given load $\forceL$ and activation parameters
$\actT$ and $\actB$, we first note that
$\moment_{\forceL,\actT,\actB}$ may have zero, exactly one,
or multiple zeros in $\clint{\ang{10}, \ang{150}}$.
Hence, the inverse function evaluated at $\SI{0}{\newton\meter}$
is partially defined depending on the load and the activation parameters:
\begin{equation}
  \label{eq:equilibriumAngle}
  \equielbang{\forceL}\colon \actdomain{\forceL} \to
  \clint{\ang{10}, \ang{150}},\quad
  \actdomain{\forceL} \subset \clint{0, 1}^2,\quad
  \equielbang{\forceL}(\actT,\actB)
  := (\moment_{\forceL,\actT,\actB})^{-1}(\SI{0}{\newton\meter}),
\end{equation}
which is well-defined whenever $\moment_{\forceL,\actT,\actB}$
has a unique root.
We approximate $\equielbang{\forceL}(\actT,\actB)$ with the Newton method
\multicite{Roehrle16Two,Valentin18Gradient}:
\begin{equation}
  \label{eq:newtonAngle}
  \elbang^{(j+1)}
  := \elbang^{(j)} -
  \frac{
    \moment_{\forceL,\actT,\actB}(\elbang^{(j)})
  }{
    \partialderiv{\partialdiff{} \elbang}{\moment_{\forceL,\actT,\actB}}
    (\elbang^{(j)})
  },\quad
  j \in \nat,
\end{equation}
with an initial value
$\elbang^{(0)} \in \clint{\ang{10}, \ang{150}}$
and the stopping criterion of
$\abs{\moment_{\forceL,\actT,\actB}(\elbang^{(j)})} <
\SI{e-9}{\newton\meter}$ or
$\abs{\partialderiv{\partialdiff{} \elbang}{\moment_{\forceL,\actT,\actB}}
(\elbang^{(j)})} < \SI{e-9}{\newton\meter\per\degree}$,
whichever is satisfied first.
We repeat the Newton method for the initial values
$\elbang^{(0)} = \ang{80}, \ang{40}, \ang{120}$
and use the first converged result
(i.e., we check if $\elbang^{(0)} = \ang{80}$ converges;
if not, proceed with $\elbang^{(0)} = \ang{40}$, and so on).
If all three initial values do not converge,
we conclude that $(\actT, \actB) \notin \actdomain{\forceL}$.



\subsection{Optimization Problems}
\label{sec:722optimization}

\paragraph{General task}

The general task in our setting is as follows:
For a given external load $\forceL$ and a target elbow angle $\tarelbang$,
find activation parameters $(\actT, \actB) \in \clint{\*0, \*1}$
such that the target elbow angle is attained in the equilibrium,
i.e., $\equielbang{\forceL}(\actT,\actB) = \elbang$.
Example applications of such a scenario are medicine and robotics,
when a specific movement should be carried out.

\paragraph{List of optimization problems}

However, as discussed in \cref{sec:711models},
musculoskeletal systems with an antagonistic muscle pair
such as our human upper limb model are usually overdetermined.
This means that there are multiple solutions to this problem.
As a remedy, one may solve one of the following
optimization problems \cite{Valentin18Gradient}:

\begin{enumerate}[label=O\arabic*.,leftmargin=2.7em]
  \item
  \label{item:biomech2MinSum}
  For a given external load $\forceL$ and a target angle
  $\tarelbang \in \clint{\ang{10}, \ang{150}}$,
  find the activation parameters $(\actT, \actB) \in \clint{\*0, \*1}$
  such that $\actT + \actB$ is minimized under the constraint
  $\equielbang{\forceL}(\actT, \actB) = \tarelbang$.
  
  \item
  \label{item:biomech2MinDist}
  For a given external load $\forceL(t_2)$ for a time $t_2 > t_1$,
  a target angle $\tarelbang(t_2) \in \clint{\ang{10}, \ang{150}}$,
  and initial activation parameters
  $(\actT(t_1), \actB(t_1)) \in \clint{\*0, \*1}$,
  find new activation parameters
  $(\actT(t_2), \actB(t_2)) \in \clint{\*0, \*1}$ such that
  $(\actT(t_2) - \actT(t_1))^2 + (\actB(t_2) - \actB(t_1))^2$
  is minimized under the constraint
  $\equielbang{\forceL(t_2)}(\actT(t_2), \actB(t_2)) = \tarelbang(t_2)$.
  %
  %\item
  %For a given external load $\forceL(t_2)$ for a time $t_2 > t_1$,
  %an initial angle $\tarelbang(t_1) \in \clint{\ang{10}, \ang{150}}$,
  %an initial activation parameters
  %$(\actT(t_1), \actB(t_1)) \in \clint{\*0, \*1}$,
  %find new activation parameters
  %$(\actT(t_2), \actB(t_2)) \in \clint{\*0, \*1}$ such that
  %$(\actT(t_2) - \actT(t_1))^2 + (\actB(t_2) - \actB(t_1))^2 +
  %c^2 \cdot (
  %  \equielbang{\forceL(t_2)}(\actT(t_2), \actB(t_2)) - \tarelbang(t_2)
  %)^2$ is minimized (with $c \in \nonnegreal$ arbitrary but fixed).
\end{enumerate}

\paragraph{Motivation of problem \ref{item:biomech2MinSum}}

The motivation of both problems is that the human body tries to
achieve a given movement with minimal energy effort.
For the first problem \ref{item:biomech2MinSum},
this effort is quantified by $\actT + \actB$,
i.e., the energy effort for each muscle is assumed to be proportional
to its activation parameter.

\paragraph{Motivation of problem \ref{item:biomech2MinDist}}

The second problem \ref{item:biomech2MinDist} is motivated as follows:
Before time $t = t_1$, the musculoskeletal system is in equilibrium for
an external load $\forceL(t_1)$,
activation parameters $\actT(t_1), \actB(t_1)$, and
elbow angle $\tarelbang(t_1)$, i.e.,
$\moment_{\forceL(t_1),\actT(t_1),\actB(t_1)}(\tarelbang(t_1))
= \SI{0}{\newton\meter}$.
Directly after $t = t_1$,
the external force and/or the target angle is suddenly changed
to $\forceL(t_2)$ and $\tarelbang(t_2)$, respectively.
Consequently, triceps and biceps adapt their activation parameters
such that the musculoskeletal system returns to equilibrium
at some time $t = t_2 > t_1$.
Hence, we have to determine the new activation parameters
$\actT(t_2), \actB(t_2)$ such that
$\moment_{\forceL(t_2),\actT(t_2),\actB(t_2)}(\tarelbang(t_2))
= \SI{0}{\newton\meter}$.
Again, these parameters
$\actT(t_2)$ and $\actB(t_2)$ are not uniquely determined.
Therefore, we want to find the pair of activation parameters
that is closest (in terms of the Euclidean norm) to the initial
activation parameters $\actT(t_1), \actB(t_1)$.

%\paragraph{Motivation of problem O3}
%
%The third problem O3 is derived from \ref{item:biomech2MinDist}.
%Instead of prescribing the target angle $\tarelbang(t_2)$,
%we try to stay ``near'' the initial angle $\tarelbang(t_1)$,
%ensuring that the desired target position is achieved.
%For this purpose, we introduce a weight parameter $c$ that
%allows for small deviations form $\tarelbang(t_1)$.

\paragraph{Optimization method}
%
%Problems \ref{item:biomech2MinSum} and \ref{item:biomech2MinDist} are
%both constrained optimization problems,
%while O3 is unconstrained.
%For problems \ref{item:biomech2MinSum} and \ref{item:biomech2MinDist},
%we employ the augmented Lagrangian method as
Problems \ref{item:biomech2MinSum} and \ref{item:biomech2MinDist}
are both constrained optimization problems.
For their solution, we employ the augmented Lagrangian method as
described in \cref{sec:513gradientBasedConstrained}
using an adaptive gradient descent algorithm
for the gradient-based optimization of the penalized objective function
(see \cref{sec:512gradientBasedUnconstrained}).
%Problem O3 is a non-linear least-squares problem,
%where the objective function $\objfun\colon \clint{\*0, \*1} \to \real$
%is of the form
%$\objfun(\*x) = \tr{\*\phi(\*x)} \*\phi(\*x) = \sum_{i=1}^m \phi_i(\*x)^2$
%for some function $\*\phi\colon \clint{\*0, \*1}^d \to \real^m$.
%We solve this problem with the Levenberg--Marquardt method
%(again, see \cref{sec:512gradientBasedUnconstrained} for details).



\subsection{B-Spline Surrogates on Sparse Grids}
\label{sec:723surrogates}

\paragraph{Complexity}

%To solve optimization problems \ref{item:biomech2MinSum} to O3,
To solve optimization problems
\ref{item:biomech2MinSum} and \ref{item:biomech2MinDist},
the optimization method needs to evaluate the objective
and constraint functions multiple times during the algorithm.
This requires the evaluation of $\equielbang{\forceL}$,
which in turn has to be approximated with the Newton method.
As we see in \cref{eq:newtonAngle},
each iteration of the Newton method needs not only the values of the
muscle forces $\forceT$ and $\forceB$, but also their
partial derivatives with respect to $\elbang$,
which have to be approximated with finite differences,
as the simulation does not provide derivatives.

Unfortunately, simulations of continuum-mechanical models are
computationally expensive.
One evaluation of the muscle force pair $\forceT, \forceB$
requires the solution of a solid mechanics model
with a complex constitutive law, pre-stretch, and contact between
bone and muscles \cite{Valentin18Gradient}.
On average, a single evaluation of $\forceT$ and $\forceB$ takes
about half an hour.
%
If we assume that we need four Newton iterations on average,
then a single iteration of the optimization algorithm to solve
%problems \ref{item:biomech2MinSum} to O3 will take four hours to complete
problems \ref{item:biomech2MinSum} and \ref{item:biomech2MinDist}
will take four hours to complete
(assuming one evaluation of objective and constraint functions
per optimizer iteration and
two evaluations per Newton iteration to approximate the missing derivative).
Consequently, the whole optimization process takes
two weeks to complete, if the optimizer converges after 100 iterations.

\paragraph{Sparse grid surrogates}

A popular way to reduce complexity is to employ surrogates.
In this case, the idea is to replace the muscle force functions
$\forceT, \forceB$ with surrogates $\forceTintp, \forceBintp$
\cite{Valentin18Gradient}, e.g., by interpolation.
We then automatically obtain a surrogate
\begin{subequations}
  \label{eq:totalMomentSurrogate}
  \begin{gather}
    \momentintp_{\forceL,\actT,\actB}\colon
    \clint{\ang{10}, \ang{150}} \to \real,\\
    \momentintp_{\forceL,\actT,\actB}(\elbang)
    := \forceTintp(\elbang, \actT) \armT(\elbang) +
    \forceBintp(\elbang, \actB) \armB(\elbang) +
    \forceL \armL(\elbang),
  \end{gather}
\end{subequations}
for the total moment (cf.\ \cref{eq:totalMoment}) and,
consequently, a surrogate
\begin{equation}
  \label{eq:equilibriumAngleSurrogate}
  \equielbangintp{\forceL}\colon \actdomainintp{\forceL} \to
  \clint{\ang{10}, \ang{150}},\quad
  \actdomainintp{\forceL} \subset \clint{0, 1}^2,\quad
  \equielbangintp{\forceL}(\actT,\actB)
  := (\momentintp_{\forceL,\actT,\actB})^{-1}(\SI{0}{\newton\meter}),
\end{equation}
for the equilibrium elbow angle function (cf.\ \cref{eq:equilibriumAngle}).
Since the surrogates are much cheaper to evaluate,
the runtime is decreased by up to seven orders of magnitude,
as experiments show.

The approach in \cite{Valentin18Gradient} and in this thesis is
to determine surrogates
$\forceXintp\colon \clint{\*0, \*1} \to \real$
($X \in \{\mathrm{T}, \mathrm{B}\}$) by sparse grid interpolation.
Compared to surrogate construction techniques based on full grids,
sparse grids help to reduce the number of samples that
are necessary to build a ``reasonably'' accurate surrogate,
especially if the number of dimensions is moderately large
($d \ge 4$, \term{curse of dimensionality}).

The present model has only $d = 2$ dimensions ($\actT$ and $\actB$),
since the model contains only two muscles.
However, as we will see,
already for this low-dimensional problem,
sparse grids outperform conventional full grid interpolation.
The results have to be seen as a proof of concept.
One will be able to handle higher dimensionalities
(i.e., models with a larger number of muscles) similarly with little
or even no adjustments at all.
The low dimensionality of the model in this thesis
enables us to compute and compare against reference solutions,
which would not be possible in a higher-dimensional setting.



\paragraph{Benefiting from B-splines}

As in \cite{Valentin18Gradient},
we use higher-order hierarchical B-splines as basis functions
for the sparse grid surrogates.
This has three advantages when compared with conventional
sparse grid bases such as piecewise linear functions.
%
First, the partial derivative
$\partialderiv{\partialdiff{} \elbang}{\momentintp}$ needed
for the Newton method in \cref{eq:newtonAngle} is continuous and
explicitly known.
There is no need to approximate the derivative with
finite differences, reducing both the error and the runtime.
%
Second, we can use gradient-based optimization methods
for the solution of the optimization problems \ref{item:biomech2MinSum} and
\ref{item:biomech2MinDist},
which involve the equilibrium elbow angle function
$\equielbangintp{\forceL}\colon \clint{\*0, \*1} \to \real$.
With the implicit function theorem \cite{Kudryavtsev95Implicit},
we obtain for the derivative of $\equielbangintp{\forceL}$
\begin{equation}
  \gradient{\actT,\actB}{\equielbangintp{\forceL}}
  = -\,(\gradient{\actT,\actB}{\momentintp}) \cdot
  (\gradient{\elbang}{\momentintp})^{-1}
  = -\,\frac{
    \gradient{\actT,\actB}{\momentintp}
  }{
    \partialderiv{\partialdiff{} \elbang}{\momentintp}
  },
\end{equation}
where $\gradient{\actT,\actB}{}$ is the transposed Jacobian%
\footnote{%
  For example, the first column is the gradient with respect to $\actT$
  and the second column is the gradient with respect to $\actB$.%
}.
Both the gradient $\gradient{\actT,\actB}{\momentintp}$ and
the partial derivative $\partialderiv{\partialdiff{} \elbang}{\momentintp}$
are continuous, explicitly known, and can be evaluated fast.
%
Third and finally,
the usage of higher-order B-splines as basis functions
increases the order of convergence of interpolation errors
as shown in \cref{sec:541interpolation}.
Thus, fewer interpolation points are necessary to build a surrogate
with the same error as for piecewise linear functions.
