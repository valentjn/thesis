\section{Momentum Equilibrium and Elbow Angle Optimization}
\label{sec:72methodology}

\minitoc{70mm}{4}

\noindent
In this section, we give an overview of the methodology of our approach.
We closely follow the presentation of \cite{Valentin18Gradient}.



\subsection{From Muscle Forces to Equilibrium Angles}
\label{sec:721equilibrium}

\paragraph{Model inputs and outputs}

In the following, we regard simulations of the
human upper limb model described in \cref{sec:71model} as a black box.
This black box receives as its input
the elbow angle $\elbang \in \clint{\ang{10}, \ang{150}}$
and the activation parameters $\actT, \actB \in \clint{0, 1}$
of triceps and biceps.%
\footnote{%
  Here and in the following, the subscripts T, B, and L stand for
  triceps, biceps, and load, respectively.%
}
The outputs of the black box simulation are the forces
$\forceT(\elbang, \actT)$ and $\forceB(\elbang, \actB)$
that triceps and biceps exert.
These forces depend on the elbow angle as well as on the respective
activation parameter.
Gravitational forces due to the masses of bones or masses
are neglected in this context.
However, we allow the specification of an external load $\forceL$
applied to the end of the forearm.
This load may the weight force of some object
that the arm is supposed to keep in position.

\paragraph{Moments and lever arms}

Every force exerts a \term{moment} (or \term{torque}) on the elbow joint.
The moments are the products of the forces $\forcevalue_\ast$
with the respective lever arms $\arm_\ast$
($\ast \in \{\mathrm{T}, \mathrm{B}, \mathrm{L}\}$).
The lever arms are approximated as in
\multicite{Roehrle16Two,Valentin18Gradient} by using
the tendon-displacement method of \cite{An84Determination},
resulting in
\begin{subequations}
  \begin{align}
    \armT(\elbang)
    &:= (-0.0009399 \{\elbang\}^2 + 0.1126 \{\elbang\} + 22.21)\;
    \si{\milli\meter},\\
    \armB(\elbang)
    &:= (-0.001482 \{\elbang\}^2 + 0.1776 \{\elbang\} + 35.02)\;
    \si{\milli\meter},\\
    \armL(\elbang)
    &:= \sin(\elbang) \cdot \SI{282.5}{\milli\meter},
  \end{align}
\end{subequations}
where $\{\elbang\}$ denotes the numerical value of $\elbang$
in degrees.
The lever arms are non-negative and the forces are signed, i.e.,
positive forces pull the forearm downwards,
negative forces pull it upwards.
In general, $\forceT, \forceL \ge \SI{0}{\newton}$ and
$\forceB \le \SI{0}{\newton}$.

\paragraph{Total moment and equilibrium elbow angle}

The \term{total moment} of the system is given by the function
\begin{subequations}
  \begin{gather}
    \moment_{\forceL,\actT,\actB}\colon
    \clint{\ang{10}, \ang{150}} \to \real,\\
    \moment_{\forceL,\actT,\actB}(\elbang)
    := \forceT(\elbang, \actT) \armT(\elbang) +
    \forceB(\elbang, \actB) \armB(\elbang) +
    \forceL \armL(\elbang).
  \end{gather}
\end{subequations}
The system is in \term{equilibrium,}
if the total moment vanishes, i.e.,
$\moment_{\forceL,\actT,\actB}(\elbang) = \SI{0}{\newton\meter}$.
The corresponding angle $\elbang$ is the
\term{equilibrium elbow angle.}
To find this angle for a given load $\forceL$ and activation parameters
$\actT$ and $\actB$, we first note that
$\moment_{\forceL,\actT,\actB}$ may have zero, exactly one,
or multiple zeros in $\clint{\ang{10}, \ang{150}}$.
Hence, the inverse function evaluated at $\SI{0}{\newton\meter}$
is partially defined depending on load and activation parameters:
\begin{subequations}
  \begin{gather}
    \elbang_{\forceL}\colon \actdomain{\forceL} \to
    \clint{\ang{10}, \ang{150}},\quad
    \actdomain{\forceL} \subset \clint{0, 1}^2,\\
    \elbang_{\forceL}(\actT,\actB)
    := (\moment_{\forceL,\actT,\actB})^{-1}(\SI{0}{\newton\meter}),
  \end{gather}
\end{subequations}
which is well-defined whenever $\moment_{\forceL,\actT,\actB}$
has a unique root.
We approximate $\elbang_{\forceL}(\actT,\actB)$ with the Newton method
\multicite{Roehrle16Two,Valentin18Gradient}:
\begin{equation}
  \elbang^{(j+1)}
  := \elbang^{(j)} -
  \frac{
    \moment_{\forceL,\actT,\actB}(\elbang^{(j)})
  }{
    \partialderiv{\partialdiff{} \elbang}{\moment_{\forceL,\actT,\actB}}
    (\elbang^{(j)})
  },\quad
  j \in \nat,
\end{equation}
with an initial value
$\elbang^{(0)} \in \clint{\ang{10}, \ang{150}}$
and the stopping criterion of
$\abs{\moment_{\forceL,\actT,\actB}(\elbang^{(j)})} <
\SI{e-9}{\newton\meter}$ or
$\abs{\partialderiv{\partialdiff{} \elbang}{\moment_{\forceL,\actT,\actB}}
(\elbang^{(j)})} < \SI{e-9}{\newton\meter\per\degree}$.
We repeat the Newton method for the initial values
$\elbang^{(0)} = \ang{80}, \ang{40}, \ang{120}$
and use the first converged result
(i.e., we check if $\elbang^{(0)} = \ang{80}$ converges;
if not, proceed with $\elbang^{(0)} = \ang{40}$ and so on).
If all three initial values do not converge,
we conclude that $(\actT, \actB) \notin \actdomain{\forceL}$.



\subsection{Optimization Problems}
\label{sec:722optimization}

\dummytext[3]{}



\subsection{B-Spline Surrogates on Sparse Grids}
\label{sec:723surrogates}

\todo{talk ``only'' about two dimensions}

\dummytext[3]{}
