\setdictum{%
  The goal is to buy as many iPads as possible during your lifetime.%
}{%
  In a talk at the 5th Workshop on\\Sparse Grids and Applications%
}

\longchapter{%
  Application 3: Dynamic Portfolio Choice Models%
}{%
  Application 3:\texorpdfstring{\\}{ }Dynamic Portfolio Choice Models%
}{%
  Application 3 -- Dynamic Portfolio Choice Models%
}
\label{chap:80finance}

\initial[lhang=0.06]{0.1em}{S}{urrogates based on B-splines on sparse grids}
can also be used for our third application,
which stems from mathematical finance.
In this application, we simulate the financial situation of an individual
over their lifetime in discrete time steps or iterations $t = 0, \dotsc, T$
(for example, years $t = 0, \dotsc, 80$, where $20+t$ is the age
of the individual).
We track three types of variables:

\begin{itemize}
  \item
  \term{State variables} $\state_t$
  such as the individual's wealth $\wealth_t$ or their income
  cannot be controlled directly by the individual.
  Instead, the individual's decisions influence the value of
  state variables of future iterations.%
  \footnote{%
    The time $t$ can also be regarded as a state variable.%
  }
  
  \item
  \term{Policy variables} $\policy_t$
  such as consumption $\consume_t$ or the amount of stocks to buy or sell
  represent the investment decisions the individual can make in
  each iteration.
  They are subject to specific constraints
  (for instance, you cannot spend more money than you have,
  if you do not allow debts).
  
  \item
  \term{Stochastic variables} $\stochastic_t$
  such as the return rates of stocks or the inflation
  cannot be controlled by individual at all.
  Therefore, statements about optimal investment conditions
  are usually made for expected values instead of exact values.
\end{itemize}

\noindent
We discretize the state space with a spatially adaptive sparse grid.
For each state grid point, an optimization problem over the policy
variables has to be solved, where the objective function depends
on the expected value over the stochastic variables.
By using B-splines as sparse grid basis functions,
the accuracy of the interpolants increases and
the explicitly known gradients enable the usage of
gradient-based optimization methods, thus accelerating convergence.
The process is repeated for each time step.
By using the Bellman principle, the objective functions that
occur at time $t$ depend on the interpolant of next iteration $t+1$.
Hence, the problem has to be solved backward in time
via a scheme that resembles dynamic programming.

The outline of this chapter is as follows:
In \cref{sec:81models}, we formalize the framework of
dynamic portfolio choice models and describe our approach.
Afterwards, we explain in \cref{sec:82algorithms} the algorithms
necessary to implement the solution of these models.
\Cref{sec:83problem} lists example problems as application
of the general framework introduced in \cref{sec:81models}.
Finally, in \cref{sec:84results}, we study numerical results.

This chapter is based on a collaboration with Prof.\ Dr.\ Raimond Maurer
and Peter Schober (both Goethe University Frankfurt, Germany).
\todo{mention paper if published}
In previous work, Peter Schober treated the solution of
dynamic portfolio choice models with piecewise linear basis functions
on sparse grids.
The original contribution of this thesis is the introduction
of higher-order B-splines for the solution of these problems.
The author of this thesis contributed the methodology of
hierarchical B-splines and large parts of the implementation.
The contributions of the collaborators at Goethe University Frankfurt
are the financial models, the literature review of related work,
and the assessment of the quality of results.

\section{Dynamic Portfolio Lifecycle Models}
\label{sec:81models}

\blindtext{}



\subsection{Bellman Equation}
\label{sec:811bellmanEquation}

\blindtext{}



\subsection{Solution with B-Spline Surrogates on Sparse Grids}
\label{sec:812surrogates}

\blindtext{}



\subsection{Related Work}
\label{sec:813relatedWork}

\blindtext{}

\section{Algorithms}
\label{sec:82algorithms}

This section gives an overview of the algorithms that
were used to implement the solution process of
the discretized Bellman equation \eqref{eq:gridBellman}.



\subsection{General Structure}
\label{sec:821generalStructure}

The general approach to solve dynamic portfolio choice models is as follows:
\begin{enumerate}
  \item
  Generation of value function interpolants $\valueintp_t$
  (\texttt{solveValueFunction}, \cref{alg:financeSolveValueFunction})
  
  \item
  Generation of optimal policy interpolants $\optpolicyintp_t$
  (\texttt{solvePolicy}, \cref{alg:financeSolvePolicy})
  
  \item
  Computation of Euler equation errors
  (\texttt{computeEulerErrors}, \cref{alg:computeEulerErrors})
\end{enumerate}
The separation of the solution processes for
the value function interpolants $\valueintp_t$
and the optimal policy interpolants $\optpolicyintp_t$
enables to generate different spatially adaptive sparse grids
for the value function and the optimal policies.
This is useful if the shapes of value function and optimal policies
have different characteristics.

In the following (\cref{%
  sec:822optimization,%
  sec:823interpolation,%
  sec:824quadrature,%
  sec:825gridGeneration%
}), we first describe the algorithmic details of
\texttt{solveValueFunction} (step 1).
The treatment of the other steps \texttt{solvePolicy} (step 2) and
\texttt{computeEulerErrors} (step 3) follows with
\cref{sec:826policyGeneration} and \cref{sec:827eulerEquationErrors},
respectively.

We track two interpolants $\valueintp[1]_t$ and $\valueintp[p]_t$
for each $t = 0, \dotsc, T$.
The former interpolates the value function data at the grid points
with the hierarchical piecewise linear basis
(used for the surplus-based grid generation),
while the latter interpolates the data with hierarchical B-splines
of higher order $p > 1$.
Each $\valueintp[\ast]_t$ ($\ast \in \{1, p\}$)
additionally stores the grid points $\state_t^{(k)}$
and the optimal policies $\optpolicyintp_t(\state_t^{(k)})$
at the grid points ($k = 1, \dotsc, \ngp_t$).
For simplicity, we do not pass them as separate data
to the algorithms.

\paragraph{\texttt{solveValueFunction}}

\blindtext{}

\begin{algorithm}
  \begin{algorithmic}[1]
    \Function{%
      $\text{%
        $((\valueintp[1]_t, \valueintp[p]_t))_{t=0,\dotsc,T}$%
      } = \texttt{solveValueFunction}$%
    }{%
      \hspace*{0mm}%
    }
      \State{$\valueintp[p]_{T+1} \gets \emptyset$}
      \Comment{dummy variable (is not used)}%
      \For{$t = T, T - 1, \dotsc, 0$}
        \State{%
          $\valueintp[1]_t \gets \text{%
            Initial regular sparse grid with no values%
          }$%
        }
        \State{%
          $\valueintp[1]_t \gets
          \texttt{optimize($t$, $\valueintp[1]_t$, $\valueintp[p]_{t+1}$)}$%
        }
        \State{%
          $\valueintp[1]_t \gets
          \texttt{refine($t$, $\valueintp[1]_t$, $\valueintp[p]_{t+1}$)}$%
        }
        \State{%
          $\valueintp[p]_t \gets
          \texttt{interpolate($\valueintp[1]_t$)}$%
        }
      \EndFor{}
    \EndFunction{}
  \end{algorithmic}
  \caption[TODO]{%
    TODO%
  }%
  \label{alg:financeSolveValueFunction}%
\end{algorithm}



\subsection{Optimization}
\label{sec:822optimization}

\todo{%
  certainty equivalence,
  parallelization,
  space transformation?
  value trafo?%
}

\blindtext{}

\begin{algorithm}
  \begin{algorithmic}[1]
    \Function{$\valueintp[1]_t = \texttt{optimize}$}{%
      $t$, $\valueintp[1]_t$, $\valueintp[p]_{t+1}$%
    }
      \State{%
        $(\state_t^{(k)})_{k=1,\dotsc,\ngp_t}
        \gets \text{grid of $\valueintp[1]_t$}$%
      }
      \For{$k = 1, \dotsc, \ngp_t$}
        \If{$\valueintp[1]_t(\state_t^{(k)})$ not already computed}
          \IfOneLine{$t = T$}{%
            $\valueintp[1]_T(\state_T^{(k)})
            \gets \texttt{computeKnownTerminalSolution($\state_T^{(k)}$)}$%
          }
          \ElseOneLine{%
            $\valueintp[1]_t(\state_t^{(k)})
            \gets \texttt{%
              optimizeSinglePoint(%
                $t$, $\state_t^{(k)}$, $\valueintp[p]_{t+1}$%
              )%
            }$%
          }
        \EndIf{}
      \EndFor{}
      \State{%
        Re-interpolate
        $(\valueintp[1]_t(\state_t^{(k)}))_{k=1,\dotsc,\ngp_t}$
        with piecewise linear functions%
      }
    \EndFunction{}
  \end{algorithmic}
  \caption[TODO]{%
    TODO%
  }%
  \label{alg:financeOptimize}%
\end{algorithm}



\subsection{Interpolation and Extrapolation}
\label{sec:823interpolation}

\todo{grid type, basis function type, why extrapolation necessary, extrapolation method}

\blindtext{}



\subsection{Quadrature}
\label{sec:824quadrature}

\blindtext{}



\subsection{Grid Generation}
\label{sec:825gridGeneration}

%The piecewise linear interpolant is used for the surplus-based
%grid generation,
%as the surpluses are easier to compute in the piecewise linear case,
%and they are more meaningful
%due to the integral representation formula \eqref{eq:surplusIntegral}.

\todo{gradient grids?}

\blindtext{}

\begin{algorithm}
  \begin{algorithmic}[1]
    \Function{$\valueintp[1]_t = \texttt{refine}$}{%
      $t$, $\valueintp[1]_t$, $\valueintp[p]_{t+1}$%
    }
      \For{$j = 1, \dotsc, \norefine_t$}
        \State{%
          Refine all grid points of $\valueintp[1]_t$ whose
          surplus is $< \refinetol_t$%
        }
        \IfOneLine{grid of $\valueintp[1]_t$ did not change}{\Break}
        \State{%
          $\valueintp[1]_t \gets
          \texttt{optimize($t$, $\valueintp[1]_t$, $\valueintp[p]_{t+1}$)}$%
        }
      \EndFor{}
    \EndFunction{}
  \end{algorithmic}
  \caption[TODO]{%
    TODO%
  }%
  \label{alg:financeRefine}%
\end{algorithm}



\subsection{Policy Generation}
\label{sec:826policyGeneration}

\blindtext{}



\subsection{Euler Equation Errors}
\label{sec:827eulerEquationErrors}

\blindtext{}

\section{Transaction Costs Model}
\label{sec:83problem}

\minitoc[-6mm]{70mm}{4}

\vspace{-1.5em}

\paragraph{Description}

In the \term{transaction costs model} \cite{Schober18Solving},
the individual can invest their money risk-free in bonds
(with a fixed interest rate similar to a bank account)
or in $m_{\stock} \in \nat$ different stocks, which is risk-affected.
Every stock transaction,
i.e., buy $\buy_{t,j}$ or sell $\sell_{t,j}$,
inflicts transaction costs $\tac \buysell_{t,j}$ ($\tac \in \nonnegreal$)
proportional to the amount $\buysell_{t,j}$ that was bought or sold
($j = 1, \dotsc, m_{\stock}$).
The individual only wants to invest a fixed
amount $\wealth_0$ in stocks, i.e., we omit the individual's income.



\subsection{Unnormalized Problem}
\label{sec:831unnormalized}

\paragraph{Consumption and state transition}

In the following,
$\stock_{t,j}$ denotes the fraction of the total wealth $\wealth_t$
that is invested in the $j$-th stock.
We combine these \term{stock fractions} $\stock_{t,j}$
in the vector $\vstock_t := (\stock_{t,1}, \dotsc, \stock_{t,m_{\stock}})$;
similarly, $\vbuysell_t := (\buysell_{t,1}, \dotsc, \buysell_{t,m_{\stock}})$.
Then, the consumption can be computed as a residual variable
(i.e., a variable that can be fully computed from $\state$ and $\policy$
and is thus omitted from $\policy$)
that is given by
\begin{equation}
  \consume_t
  := (1 - \sumfcn(\vstock_t)) \wealth_t - \bond_t -
  (1 + \tac) \sumfcn(\vbuy_t) + (1 - \tac) \sumfcn(\vsell_t),
\end{equation}
where $\sumfcn(\*a) := \tr{\*1} \*a$.
The state transition is computed by adding the returns of the stocks:%
\begin{subequations}%
  \begin{align}
    \wealth_{t+1}
    &:= \bond_t \bondreturn_t +
    %\sum_{j=1}^{m_{\stock}}
    %(\stock_{t,j} \wealth_t + \buy_{t,j} - \sell_{t,j}) \stockreturn_{t,j},\\
    \tr{(\vstock_t \wealth_t + \vbuy_t - \vsell_t)} \vstockreturn_t,\\
    \vstock_{t+1}
    &:= \frac{
      (\vstock_t \wealth_t + \vbuy_t - \vsell_t) \compmult \vstockreturn_t
    }{\wealth_{t+1}},
  \end{align}
\end{subequations}
where $\compmult$ is the component-wise multiplication,
$\bondreturn_t \in \real$ is the bond interest rate, and
$
  \vstockreturn_t
  = (\stockreturn_{t,1}, \dotsc, \stockreturn_{t,m_{\stock}})
  \in \real^{m_{\stock}}
$
is the vector of (stochastic) stock return rates.



\subsection{Normalization}
\label{sec:832normalized}

\paragraph{State transition}

The equations above can be normalized with respect to the wealth
$\wealth_t$:
By setting
$\normconsume_t  := \consume_t/\wealth_t$,
$\normbond_t     := \bond_t   /\wealth_t$, and
$\vnormbuysell_t := \vbuysell_t/\wealth_t$, we obtain
\begin{subequations}
  \label{eq:normalizedTCPStateTransition}
  \begin{align}
    \normconsume_t
    &:= (1 - \sumfcn(\vstock_t)) - \normbond_t -
    (1 + \tac) \sumfcn(\vnormbuy_t) + (1 - \tac) \sumfcn(\vnormsell_t),\\
    \wealthratio_{t+1}
    &:= \normbond_t \bondreturn_t +
    \tr{(\vstock_t + \vnormbuy_t - \vnormsell_t)} \vstockreturn_t,
    \qquad(= \wealth_{t+1}/\wealth_t)\\
    \vstock_{t+1}
    &:= \frac{
      (\vstock_t + \vnormbuy_t - \vnormsell_t) \compmult \vstockreturn_t
    }{\wealthratio_{t+1}},
  \end{align}
\end{subequations}
where both $\normconsume_t$ and $\wealthratio_{t+1}$ are residual
variables that specify \term{normalized consumption} and
\term{wealth ratio,} respectively.
The resulting dynamic portfolio choice model has
the following variables:
\begin{itemize}
  \item
  $\centerhphantom{d}{m_{\stochastic}} = m_{\stock}$
  state variables $\normstate_t$:
  Stock fractions $\stock_{t,1}, \dotsc, \stock_{t,m_{\stock}}$
  
  \item
  $\centerhphantom{m_{\policy}}{m_{\stochastic}} = 2m_{\stock} + 1$
  policy variables $\normpolicy_t$:
  Normalized bonds $\normbond_t$,
  normalized buy amounts $\normbuy_{t,1}, \dotsc, \normbuy_{t,m_{\stock}}$ and
  normalized sell amounts $\normsell_{t,1}, \dotsc, \normsell_{t,m_{\stock}}$
  
  \item
  $m_{\stochastic} = m_{\stock}$
  stochastic variables $\stochastic_t$:
  Stock return rates $\stockreturn_{t,1}, \dotsc, \stockreturn_{t,m_{\stock}}$
\end{itemize}
The state space and policy space constraints are given by
\begin{subequations}
  \label{eq:normalizedTCPConstraints}
  \newcommand*{\centereqline}[1]{%
    \mathclap{\hphantom{\mathrm{(8.99a)}}#1}%
  }%
  \begin{gather}
    \label{eq:normalizedTCPConstraintsShort}
    \centereqline{
      \vstock_t \ge \*0,\qquad
      \sumfcn(\vstock_t) \le 1,\qquad
      \normbond_t \ge 0,\qquad
      \vnormbuysell_t \ge \*0,\qquad
      \vnormsell_t \le \vstock_t,
    }\\
    \label{eq:normalizedTCPConstraintsLong}
    \centereqline{
      \normconsume_{\min} + \normbond_t +
      (1 + \tac) \sumfcn(\vnormbuy_t) - (1 - \tac) \sumfcn(\vnormsell_t)
      \le 1 - \sumfcn(\vstock_t),
    }
  \end{gather}
\end{subequations}
where $\normconsume_{\min} \in \nonnegreal$ is some minimal consumption
that must be maintained.

\paragraph{Bellman equation}

Consequently, the Bellman equation \eqref{eq:gridBellmanCET}
after the certainty-equiva\-lent transformation has to be
normalized as well.
By setting $\normcetvalueintp_t(\state_t^{(k)})
:= \cetvalueintp_t(\state_t^{(k)})/\wealth_t$, we obtain
\begin{subequations}
  \begin{align}
    %\normcetvalueintp_t(\state_t^{(k)})
    %&= \frac{\cetvalueintp_t(\state_t^{(k)})}{\wealth_t}\\
    %&= \max_{\policy_t} \left(
    %  \left(
    %    \left(
    %      \frac{\consume_t(\state_t^{(k)}, \policy_t)}{\wealth_t}
    %    \right)^{1-\riskav} +
    %    \patience \expectation[t]{
    %      \left(
    %        \frac{
    %          \cetvalueintp_{t+1}(
    %            \statefcn_t(\state_t^{(k)}, \policy_t, \stochastic_t)
    %          )
    %        }{
    %          \wealth_t
    %        }
    %      \right)^{1-\riskav}
    %    }
    %  \right)^{1/(1-\riskav)}
    %\right)\\
    &\hphantom{=}\hspace{0.6em} \normcetvalueintp_t(\state_t^{(k)})
    = \wealth_t^{-1} \cetvalueintp_t(\state_t^{(k)})\\
    &= \max_{\policy_t} \left(
      \left(
        \left(
          \wealth_t^{-1} \consume_t(\state_t^{(k)}, \policy_t)
        \right)^{1-\riskav} +
        \patience \expectation[t]{
          \left(
            \wealth_t^{-1} \cetvalueintp_{t+1}(
              \statefcn_t(\state_t^{(k)}, \policy_t, \stochastic_t)
            )
          \right)^{1-\riskav}
        }
      \right)^{1/(1-\riskav)}
    \right)\\
    &= \max_{\normpolicy_t} \left(
      \left(
        \normconsume_t(\state_t^{(k)}, \normpolicy_t)^{1-\riskav} +
        \patience \expectation[t]{
          \bigl(
            \wealthratio_{t+1} \normcetvalueintp_{t+1}(
              \normstatefcn_t(\state_t^{(k)}, \normpolicy_t, \stochastic_t)
            )
          \bigr)^{1-\riskav}
        }
      \right)^{1/(1-\riskav)}
    \right).
  \end{align}
\end{subequations}
This means that compared with the unnormalized Bellman equation
\eqref{eq:gridBellmanCET},
the value function in the expectation has to be multiplied by
the wealth ratio $\wealthratio_{t+1}$ introduced above in
\cref{eq:normalizedTCPStateTransition}.
Since there is no inheritance, the optimal terminal solution
is to sell all stocks and consume everything:
\begin{equation}
  \normcetvalueintp_t(\state_T^{(k)})
  = 1 - \tac \sumfcn(\vstock_t^{(k)}),\quad
  \normbond_T^{\opt}(\state_T^{(k)})
  = 0,\quad
  \vnormbuy[\opt]_T(\state_T^{(k)})
  = \*0,\quad
  \vnormsell[\opt]_T(\state_T^{(k)})
  = \vstock_t^{(k)}.
\end{equation}



\subsection{State Space Cropping}
\label{sec:833cropping}

\paragraph{Sparse grids on non-rectangular domains}

Unfortunately, the constraint $\sumfcn(\vstock_t) \le 1$
from \eqref{eq:normalizedTCPConstraints} limits the feasible state space
region to a proper subset (which is the unit simplex)
of the unit hypercube $\clint{\*0, \*1}$,
which impedes the direct application of sparse grids.
There are three possible remedies:
defining a suitable transformation from the unit hypercube to
the feasible state space,
applying extrapolation techniques as discussed in
\cref{sec:825interpolation}, or
choosing a model-tailored approach to obtain
function values outside the feasible state space.

\paragraph{Virtual selling of stocks}

We choose the third remedy and \term{virtually sell,}
if $\sumfcn(\vstock_t) > 1$,
as many stocks as needed to meet the constraint $\sumfcn(\vstock_t) \le 1$.
We already might need to sell stocks
even if $\sumfcn(\vstock_t)$ is smaller, but close to one
in order to satisfy the minimum consumption requirement
\eqref{eq:normalizedTCPConstraintsLong}.
In detail, we replace $\vstock_t$ by $\normcropfactor \vstock_t$
whenever $\normcropfactor < 1$,
where $\normcropfactor \in \posreal$ is a \term{cropping factor}
that is determined by
\begin{equation}
  \label{eq:virtualSelling}
  \Bigl[
    1 - \tac\, \bigl(
      \sumfcn(\vstock_t) - \sumfcn(\normcropfactor\vstock_t)
    \bigr)
  \Bigr]
  \cdot \bigl(1 - \sumfcn(\normcropfactor\vstock_t)\bigr)
  = \normconsume_{\min}.
\end{equation}
Here, $\bigl(\sumfcn(\vstock_t) - \sumfcn(\normcropfactor\vstock_t)\bigr)$
is the amount of virtually sold stocks.
Consequently, the term in square brackets is the fraction of wealth
that is still available after deducting the induced transaction costs.
The product of this term with
$\bigl(1 - \sumfcn(\normcropfactor\vstock_t)\bigr)$
is the fraction of wealth that can be consumed after the virtual selling,
which needs to be at least $\normconsume_{\min}$.
Solving \cref{eq:virtualSelling} for $\normcropfactor$ and
choosing the positive solution, we finally obtain
\begin{equation}
  \newcommand*{\sumX}{\sumfcn(\vstock_t)}
  \newcommand*{\cMin}{\normconsume_{\min}}
  \normcropfactor
  := \frac{
    \tac\, \bigl(1 + \sumX\bigr) - 1 +
    \sqrt{
      \tac^2\, \bigl(1 - \sumX\bigr)^2
      - 2 \tac\, \bigl(2 \cMin - 1 + \sumX\bigr) + 1
    }
  }{
    2 \tac \sumX
  }.
\end{equation}



\subsection{Euler Equation Errors}
\label{sec:834eulerErrors}

In the following, we abbreviate
the normalized and certainty-equivalent-transformed value function interpolant
$
  \normcetvalueintp_t
  = \normcetvalueintp_t(\normstate_t)
$,
its gradient
$
  \gradient{\normstate_t}{\normcetvalueintp_t}
  = \gradient{\normstate_t}{\normcetvalueintp_t}(\normstate_t)
$,
the interpolated optimal policy
$
  \optnormpolicyintp_t
  = \optnormpolicyintp_t(\normstate_t)
$,
the state transition function
$
  \statefcn_t
  = \statefcn_t(\normstate_t, \optnormpolicyintp_t, \stochastic_t)
$,
the wealth ratio
$
  \wealthratio_{t+1}
  = \wealthratio_{t+1}(
    \normstate_t, \optnormpolicyintp_t, \stochastic_t
  )
$, and
the consumption
$
  \normconsume_t
  = \normconsume_t(\normstate_t, \optnormpolicyintp_t)
$.

\begin{equation}
  \patience \bondreturn_t
  \cdot \expectationsign[t]\Bigl[
    \bigl(
      \normcetvalueintp_t
      - \tr{(\gradient{\normstate_t}{\normcetvalueintp_t})}
      \statefcn_t
    \bigr) \cdot
    \bigl(
      \wealthratio_{t+1}\; \normcetvalueintp_t
    \bigr)^{-\riskav}
  \Bigr]
  = \normconsume_t^{-\riskav}
\end{equation}

\begin{equation}
  \error_t(\normstate_t)
  := \Bigl|
    1 - \Bigl(
      \patience \bondreturn_t \normconsume_t^{\riskav}
      \cdot \expectationsign[t]\Bigl[
        &\bigl(
          \normcetvalueintp_t
          - \tr{(\gradient{\normstate_t}{\normcetvalueintp_t})}
          \statefcn_t
        \bigr) \cdot
        \bigl(
          \wealthratio_{t+1}\; \normcetvalueintp_t
        \bigr)^{-\riskav}
      \Bigr]
    \Bigr)^{-1/\riskav}
  \Bigr|,
\end{equation}

%\begin{equation}
%  \begin{split}
%    \patience \bondreturn_t
%    \cdot \expectationsign[t]\Bigl[
%      &\bigl(
%        \normcetvalueintp_t%(\normstate_t)
%        - \tr{
%          (
%            \gradient{\normstate_t}{\normcetvalueintp_t}%(\normstate_t)
%          )
%        }
%        \statefcn_t%(\normstate_t, \optnormpolicyfcn_t, \stochastic_t)
%      \bigr)\\[-1mm]
%      &{} \cdot
%      \bigl(
%        \wealthratio_{t+1}%(
%        %  \normstate_t, \optnormpolicyfcn_t, \stochastic_t
%        %)
%        \;
%        \normcetvalueintp_t%(\normstate_t)
%      \bigr)^{-\riskav}
%    \Bigr]
%    = \normconsume_t%(\normstate_t, \optnormpolicyfcn_t)
%    ^{-\riskav}
%  \end{split}
%\end{equation}
%
%\begin{equation}
%  \begin{split}
%    \error_t(\normstate_t)
%    := \Bigl|
%      1 - \Bigl(
%        \patience \bondreturn_t
%        \cdot \normconsume_t%(\normstate_t, \optnormpolicyfcn_t)
%        ^{\riskav}
%        \cdot \expectationsign[t]\Bigl[
%          &\bigl(
%            \normcetvalueintp_t%(\normstate_t)
%            - \tr{
%              (
%                \gradient{\normstate_t}{\normcetvalueintp_t}%(\normstate_t)
%              )
%            }
%            \statefcn_t%(\normstate_t, \optnormpolicyfcn_t, \stochastic_t)
%          \bigr)\\[-2mm]
%          &{} \cdot
%          \bigl(
%            \wealthratio_{t+1}%(
%            %  \normstate_t, \optnormpolicyfcn_t, \stochastic_t
%            %)
%            \;
%            \normcetvalueintp_t%(\normstate_t)
%          \bigr)^{-\riskav}
%        \Bigr]
%      \Bigr)^{-1/\riskav}
%    \Bigr|,
%  \end{split}
%\end{equation}

\dummytext[3]{}

\section{Implementation and Numerical Results}
\label{sec:84results}

\minitoc[13mm]{65mm}{6}

\parbox{1em}{}
\vspace{-3em}



\printornamentsfalse
\subsection{Implementation}
\label{sec:841implementation}
\printornamentstrue

\paragraph{Parameter values and basis functions}

We use
a risk aversion factor of $\riskav = 3.5$,
a patience factor of $\patience = 0.97$,
a transaction cost rate of $\tac = \SI{1}{\percent}$, and
a minimum consumption of $\normconsume_{\min} = 0.001$.
The bond and stock return rates are taken from \cite{Cai10Stable}.
The models are solved for $T = 6$ time steps;
this number suffices to show all relevant numerical effects and results,
while keeping the computational effort at a reasonable level.
As higher-order B-spline basis functions,
we use the hierarchical weakly fundamental not-a-knot splines
$\bspl[\wfs,\nak]{\*l,\*i}{p}$ of cubic degree $p = 3$
to enable hierarchization with the unidirectional principle.

\vspace*{-0.5em}

\paragraph{Software}

The dynamic portfolio choice models were solved using a self-written
MATLAB framework.
The object-oriented framework was designed in such a way that
not only transaction costs problems,
but many other types of dynamic portfolio choice models can be handled.
For instance, the base class \texttt{LifecycleProblem} provides
an interface with abstract functions such as
\texttt{computeTerminalValueFunction} and
\texttt{computeStateTransition}.
The actual functionality implemented in the base class strongly resembles
the algorithms presented in \cref{sec:82algorithms}.
This is not only desirable from a modeling perspective,
but also facilitates future usage by other researchers.
For creating (hierarchization) and evaluating sparse grid interpolants,
the sparse grid toolbox \sgpp was used \cite{Pflueger10Spatially}.%
\footnote{%
  \url{http://sgpp.sparsegrids.org/}%
}
The emerging optimization problems were solved using
sequential quadratic programming methods supplied by the
NAG Toolbox for MATLAB.%
\footnote{%
  \url{https://www.nag.com/}%
}
To avoid getting stuck in local minima,
we repeat the optimization process for a varying number
of initial multi-start points (in the range of a few dozens).
All runtimes were measured on a shared memory computer
with 144 threads on 4x Intel Xeon E7-8880v3 (72 cores, 144 threads).

\paragraph{Error measures}

To assess the quality of the resulting value and policy functions,
we use two error measures:
Euler equation error and policy error.
The weighted Euler equation error $\weightedeulererror_t(\normstate_t)$
measures the violation of the first-order optimality conditions,
derived as above.
The policy error is the difference
$\optnormpolicyref_t(\state_t) - \optnormpolicyintp_t(\state_t)$
of the optimal policy to some reference.
Both error measures can be analyzed pointwise or
via their $\Ltwo$ norms, where for the policy error,
we take the mean of the relative $\Ltwo$ error:
\begin{equation}
  \policyerror_t
  := \frac{1}{m_{\policy}} \sum_{j=1}^{m_{\policy}}
  \frac{
    \normLtwo{\optnormpolicyentryref_{t,j} - \optnormpolicyentryintp_{t,j}}
  }{
    \normLtwo{\optnormpolicyentryref_{t,j}}
  }.
\end{equation}
It suffices to consider $t = 0$, as both error measures
usually grow backwards in time due to the dynamic programming approach.



\subsection{Error Sources}
\label{sec:842errorSources}

% high-level explanation of accumulation of errors over time

In this application, there are the following error sources:

\begin{enumerate}[label=E\arabic*.,ref=E\arabic*,leftmargin=2.7em]
  \item
  Interpolation of the value function
  (i.e., $\normcetvalueintp_{t+1} \not= \normcetvaluefcn_{t+1}$)
  
  \item
  Interpolation of the policy functions
  (i.e., $\optnormpolicyintp_t \not= \optnormpolicyfcn_t$)
  
  \item
  Extrapolation errors
  (i.e., $
  \normcetvalueintp_{t+1}(\state_{t+1})
  \not= \normcetvaluefcn_{t+1}(\state_{t+1})
  $)
  
  \item
  Errors from state space cropping
  (i.e., Euler errors do not vanish for exact solution)
  
  \item
  Optimization error
  (i.e., the global minimum found by the optimizer is inaccurate)
  
  \item
  Quadrature error
  ($
  \expectation[t]{\cdots}
  \not= \sum_{j=1}^{m_{\quadweight}} \quadweight_t^{(j)}
  [\cdots](\stochastic_t^{(j)})
  $)
  
  \item
  Floating-point rounding errors
  (i.e., arithmetical operations are inaccurate)
\end{enumerate}
Due to the dynamic programming scheme,
the combination of all errors accumulates over $t$.
For instance, if the optimization does not find the global optimum
exactly or it only finds a local one for the time $t + 1$,
the error propagates from the interpolant $\normcetvalueintp_{t+1}$
on the right-hand side of the Bellman equation
\eqref{eq:normalizedTCPBellmanEquation} to $\normcetvalueintp_t$
on the left-hand side, and so on.
If the system does not damp these errors,
the error might become stronger and stronger backwards in time $t$.



\subsection{Numerical Results}
\label{sec:843results}

\paragraph{Reference solution}

We use full grid solutions as reference solutions,
i.e., $\{\state_t^{(k)} \mid k = 1, \dotsc, \ngp_t\} = \fgset{n,d}$
for some fixed level $n \in \nat$ for all $t = 0, \dotsc, T$.
Obviously, this is only computationally feasible
for low dimensionalities $d$ due to the curse of dimensionality.
For each $d$,
\cref{tbl:financeSolutionFullRegularSparseGrids}
(see \cref{chap:a40financeDetails})
contains full grid solutions of different levels,
where the solution with the highest level is used as reference solution.
The reference solution for $d = 2$ is shown in
\cref{fig:financeSolution2DReference}.
Unfortunately, only full grid solutions up to $d = 3$ could be computed
due to excessive runtime for $d \ge 4$,
which means that policy errors can only be given for $d = 2$ or $d = 3$.
This underlines the need for sophisticated
discretization techniques such as sparse grids.

\begin{figure}
  \makebox[49mm][r]{%
    \includegraphics{financeSolution2D_1}%
  }%
  \hfill%
  \makebox[49mm][r]{%
    \includegraphics{financeSolution2D_2}%
  }%
  \hfill%
  \makebox[49mm][r]{%
    \includegraphics{financeSolution2D_3}%
  }%
  \\[1mm]%
  \makebox[49mm][r]{%
    \includegraphics{financeSolution2D_4}%
  }%
  \hfill%
  \makebox[49mm][r]{%
    \includegraphics{financeSolution2D_5}%
  }%
  \hfill%
  \makebox[49mm][r]{%
    \includegraphics{financeSolution2D_6}%
  }%
  \caption[Reference solution for the two-dimensional TCP]{%
    Reference solution for the transaction costs problem
    with $d = 2$ stocks.
    Shown are the value function $\normcetvalueref_t$ \emph{(top left)} and the
    optimal policy $\optnormpolicyref_t$ for the initial time step $t = 0$.%
  }%
  \label{fig:financeSolution2DReference}%
\end{figure}

\paragraph{Convergence of 1D, 2D, and 3D solutions}

% FG vs. regular SG vs. adaptive SG
% Euler eq. errors, policy errors?

\Cref{fig:financeEulerError} shows the convergence of the
$\Ltwo$ norm $\normLtwo{\weightedeulererror_0}$
weighted Euler equation error for $t = 0$ for the reference full grids,
regular sparse grids, and spatially adaptive sparse grids.
For this and the following plots,
the value function grid is kept unchanged,
while the mean number $\ngp_t$ of policy grid points increases
with decreasing refinement threshold $\refinetol_t$.

\begin{figure}
  \includegraphics{financeEulerError_4}%
  \\[2mm]%
  \subcaptionbox{%
    $d = 1$%
  }[48mm]{%
    \includegraphics{financeEulerError_1}%
  }%
  \hfill%
  \subcaptionbox{%
    $d = 2$%
  }[48mm]{%
    \includegraphics{financeEulerError_2}%
  }%
  \hfill%
  \subcaptionbox{%
    $d = 3$%
  }[48mm]{%
    \includegraphics{financeEulerError_3}%
  }%
  \caption[Convergence of the weighted Euler equation error]{%
    Convergence of the $\Ltwo$ norm $\normLtwo{\weightedeulererror_t}$
    of the weighted Euler equation error for $t = 0$ for
    full grids \emph{\textcolor{C0}{(blue)},}
    regular sparse grids \emph{\textcolor{C1}{(red)},} and
    spatially adaptive sparse grids \emph{\textcolor{C2}{(yellow)}.}
    The number $\ngp_t$ is the mean number
    $\frac{1}{m_{\policy}} \sum_{j=1}^{m_{\policy}} \ngp_{t,j}$
    of grid points over all policy grids for $t = 0$,
    where $\ngp_{t,j}$ is the number of grid points
    of the $j$-th policy entry.%
  }%
  \label{fig:financeEulerError}%
\end{figure}



\begin{figure}
  \subcaptionbox{%
    $\normcetvalueintp_t$%
  }[48mm]{%
    \includegraphics{financeSolution2D_7}%
  }%
  \hfill%
  \subcaptionbox{%
    $\normbuy[\sparse]_{t,1}$%
  }[48mm]{%
    \includegraphics{financeSolution2D_8}%
  }%
  \hfill%
  \subcaptionbox{%
    $\normbuy[\sparse]_{t,2}$%
  }[48mm]{%
    \includegraphics{financeSolution2D_9}%
  }%
  \\[2mm]%
  \subcaptionbox{%
    $\normsell[\sparse]_{t,1}$%
  }[48mm]{%
    \includegraphics{financeSolution2D_10}%
  }%
  \hfill%
  \subcaptionbox{%
    $\normsell[\sparse]_{t,2}$%
  }[48mm]{%
    \includegraphics{financeSolution2D_11}%
  }%
  \hfill%
  \subcaptionbox{%
    $\normbond_t^{\sparse}$%
  }[48mm]{%
    \includegraphics{financeSolution2D_12}%
  }%
  \caption[Sparse grid solution for the two-dimensional TCP]{%
    Spatially adaptive sparse grid solution for the transaction costs problem
    with $d = 2$ stocks.
    \vspace{-0.15em}%
    Shown are the value function $\normcetvalueref_t$ \emph{(top left)} and the
    optimal policy $\optnormpolicyref_t$ for the initial time step $t = 0$,
    together with the corresponding grid points \emph{(dots).}
    The color coding is the same as in
    \cref{fig:financeSolution2DReference}.%
  }%
  \label{fig:financeSolution2DSparseGrid}%
\end{figure}

\dummytext[2]{}

\paragraph{Pointwise error}

\begin{figure}
  \includegraphics{financePointwiseError_4}%
  \\[2mm]%
  \subcaptionbox{%
    TODO%
  }[49mm]{%
    \includegraphics{financePointwiseError_1}%
  }%
  \hfill%
  \subcaptionbox{%
    TODO%
  }[49mm]{%
    \includegraphics{financePointwiseError_2}%
  }%
  \hfill%
  \subcaptionbox{%
    TODO%
  }[49mm]{%
    \includegraphics{financePointwiseError_3}%
  }%
  \caption[TODO]{%
    TODO%
  }%
  \label{fig:financePointwiseError}%
\end{figure}

\dummytext[3]{}

\paragraph{4D and 5D solutions}

\dummytext[2]{}

\paragraph{Monte Carlo simulation}

% Portfolio-Gewichte, die Sharp-Ratio maximieren (Einheit Überrendite pro Einheit Risiko)
%loadResult(22)
%problem.plotLifecycleProfile(simulation.state, simulation.discreteState, simulation.policy, simulation.shock)
%A = [0.0256, 0.00576, 0.00288, 0.00176; ...
%0.00576, 0.0324, 0.0090432, 0.010692; ...
%0.00288, 0.0090432, 0.04, 0.0132; ...
%0.00176, 0.010692, 0.0132, 0.0484];
%A
%b = diag(A)
%b = sqrt(diag(A))
%format long
%b = sqrt(diag(A))
%format short
%b = sqrt(diag(A))
%A
%cor(A)
%corr(A)
%doc corr
%corr(A(1:3,1:3))
%B = corr(A); B = B(1:3,1:3);
%B = corr(A); B = B(1:3,1:3)
%corrcov(A)
%corrcov(A(1:3,1:3))
%[B, sigma] = corrcov(A(1:3,1:3))
%b = [0.0572, 0.0638, 0.07, 0.0764]
%diff(b)
%rf = problem.Return.riskfreeRate
%b = b(1:3);
%b
%(b-rf)./sigma
%(b-rf)/sigma
%(b-rf)./sigma
%(b-rf)./sigma'
%ver
%x = 1/3*ones(1,3)
%(x*b'-rf)./(x*sigma'*x')
%(x*b'-rf)./(x*sigma*x')
%x*b'
%rf
%x*sigma
%sigma
%A
%(x*b'-rf)./(x*A(1:3,1:3)*x')
%f = @(x) (x*b'-rf)./(x*A(1:3,1:3)*x')
%help fminbnd
%help fmincon
%fmincon(f, [1/3 1/3 1/3], [], [], [], [], [0 0 0], [1 1 1])
%sum(ans)
%xopt = fmincon(f, [1/3 1/3 1/3], [], [], [], [], [0 0 0], [1 1 1])
%f(xopt)
%xopt = fmincon(-f, [1/3 1/3 1/3], [], [], [], [], [0 0 0], [1 1 1])
%f = @(x) -(x*b'-rf)./(x*A(1:3,1:3)*x')
%xopt = fmincon(f, [1/3 1/3 1/3], [], [], [], [], [0 0 0], [1 1 1])
%sum(xopt)
%xopt = fmincon(f, [1/3 1/3 1/3], [], [], [1 1 1], 1, [0 0 0], [1 1 1])
%sum(xopt)
%1.8/(1.45+1.45+1.8)
%1.45/(1.45+1.45+1.8)
%A
%A = diag(diag(A))
%xopt = fmincon(f, [1/3 1/3 1/3], [], [], [1 1 1], 1, [0 0 0], [1 1 1])

% certainty-equivalent consumption
% mit B-splines und linearen Funktionen

\begin{figure}
  \includegraphics{financeSimulation_5}%
  \\[2mm]%
  \subcaptionbox{%
    TODO%
  }[48mm]{%
    \includegraphics{financeSimulation_1}%
  }%
  \hfill%
  \subcaptionbox{%
    TODO%
  }[48mm]{%
    \includegraphics{financeSimulation_2}%
  }%
  \hfill%
  \subcaptionbox{%
    TODO%
  }[48mm]{%
    \includegraphics{financeSimulation_3}%
  }%
  \\[2mm]%
  \subcaptionbox{%
    TODO%
  }[48mm]{%
    \includegraphics{financeSimulation_4}%
  }%
  \hfill%
  \begin{minipage}[b]{92mm}%
    \caption[TODO]{%
      TODO TODO TODO TODO TODO TODO TODO TODO TODO TODO TODO TODO TODO
      TODO TODO TODO TODO TODO TODO TODO TODO TODO TODO TODO TODO TODO
      TODO TODO TODO TODO TODO TODO TODO TODO TODO TODO TODO TODO TODO
      TODO TODO TODO TODO TODO TODO TODO TODO TODO TODO TODO TODO TODO%
    }%
    \vspace*{-6.3mm}%
    \label{fig:financeSimulation}%
  \end{minipage}%
\end{figure}

\dummytext[2]{}

\paragraph{Complexity and runtime analysis}

% analyze runtime behavior over N (fixed d) and d

% (e-mail 2018-01-03)
%Runtime of one optimize() run
%- -----------------------------
%O(#New state grid points
%* #Optimization iterations
%* #Quadrature points
%* #Old state grid points
%* #State dimensions
%* Runtime of one 1D basis evaluation)
%
%where
%#(Optimization iterations) depends on #(Policy dimensions),
%#(Policy dimensions) = 2 * #Stocks + 1
%#(State dimensions) = #Stocks + 1
%
%(modulo extrapolations)

\begin{equation}
  \landauO{
    % for every time step
    T
    % for every grid point of new time step
    \cdot \ngp_t
    % for every optimization iteration
    \cdot \text{\#optimizer iterations}
    \cdot
    \underbrace{
      % for every gradient entry of the objective function
      m_{\policy}
      % for every quadrature point
      \cdot m_{\stochastic}
      \cdot
      \overbrace{
        % for every summand of the sparse grid function
        % (grid point of old time step)
        \ngp_{t+1}
        % for every tensor product factor
        \cdot m_{\state}
        % for every B-spline summand
        \cdot p
      }^{\mathclap{\text{evaluation of interpolant}}}
    }_{\text{evaluation of objective gradient}}\,
  }
\end{equation}

\#optimizer iterations contains multi-start points!

\begin{equation}
  \landauO{
    T
    \cdot
    2^n n^{d-1}
    \cdot
    2^n n^{d-1}
    \cdot
    (2d + 1)
    \cdot
    2^n n^{d-1}
    \cdot
    2^n n^{d-1}
    \cdot
    d
    \cdot
    p
  }
  = \landauO{Tdp 2^{4n} n^{4d-4}}??
\end{equation}

\dummytext[3]{}

\paragraph{Impact of the B-spline degree}

\dummytext[3]{}

\paragraph{Comparing exact gradients to finite differences}

\dummytext[3]{}


\cleardoublepage
