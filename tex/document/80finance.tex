\setdictum{%
  The goal is to buy as many iPads as possible during your lifetime.%
}{%
  In a talk at the 5th Workshop on\\Sparse Grids and Applications%
}

\longchapter{%
  Application 3: Dynamic Portfolio Lifecycle Models%
}{%
  Application 3:\texorpdfstring{\\}{ }Dynamic Portfolio Lifecycle Models%
}{%
  Application 3: Dynamic Portfolio Lifecycle Models%
}
\label{chap:80finance}

\initial[lhang=0.06]{0.1em}{S}{urrogates based on B-splines on sparse grids}
can also be used for our third application,
which stems from mathematical finance.
In this application, we simulate the financial situation of an individual
over their lifetime in discrete time steps or iterations $t = 0, \dotsc, T$
(for example, years $t = 0, \dotsc, 80$, where $20+t$ is the age
of the individual).
We track three types of variables:

\begin{itemize}
  \item
  \term{State variables} $\state_t$
  such as the individual's wealth or their income
  cannot be controlled directly by the individual.
  Instead, the individual's decisions influence the value of
  state variables of future iterations.%
  \footnote{%
    The time $t$ can also be regarded as a state variable.%
  }
  
  \item
  \term{Policy variables} $\policy_t$
  such as consumption or the amount of stocks to buy or sell
  represent the investment decisions the individual can make in
  each iteration.
  They are subject to specific constraints
  (for instance, you cannot spend more money than you have,
  if you do not allow debts).
  
  \item
  \term{Stochastic variables} $\stochastic_t$
  such as the return rates of stocks or the inflation
  cannot be controlled by individual at all.
  Therefore, statements about optimal investment conditions
  are usually made for expected values instead of exact values.
\end{itemize}

\blindtext{}

\section{Dynamic Portfolio Choice Models}
\label{sec:81models}

In this section, we give a mathematical framework for
dynamic portfolio choice models,
briefly mention related literature briefly, and
explain where B-splines on sparse grids come into play.



\subsection{Bellman Equation}
\label{sec:811bellmanEquation}

\paragraph{Utility maximization}

Dynamic portfolio choice models aim to maximize the expected
\term{discounted utility} over the lifetime of the individual.
If we neglect stochastic factors, then these models solve
\begin{equation}
  \label{eq:utilityMaximization}
  (\optpolicyfcn_0, \dotsc, \optpolicyfcn_T)
  = \argmax_{\policy_0, \dotsc, \policy_T}
  \sum_{t=0}^T \patience^t \utilityfcn(\consume_t(\state_t, \policy_t))
  \quad\text{s.t. specific constraints.}
\end{equation}
Here, $\state_t \in \clint{\*0, \*1} \subset \real^d$ and
$\policy_t \in \real^{m_{\policy}}$
are the state and policy of time $t = 0, \dotsc, T$, respectively.
The constraints ensure that for instance, we do not spend more money
than we actually have.
Starting from a given initial state $\state_0$,
the state $\state_t$ of time $t > 0$ can be computed from
$\state_0$ and $\policy_0, \dotsc, \policy_{t-1}$ with a
\term{state transition function} $(\state_t, \policy_t) \mapsto \state_{t+1}$.
As shown in \cref{fig:dynamicPortfolioChoice},
in each time step, a fraction of the available wealth
is invested in \term{consumption} $\consume_t$,
which can be computed from the state $\state_t$ and the policy $\policy_t$.
We rate the consumption with a \term{utility function}
$\utilityfcn(\consume_t)$.
A common choice for $\utilityfcn$ is the
\term{constant relative risk aversion (CRRA) utility}
$\utilityfcn(\consume_t) := (c_t^{1-\riskav})/(1-\riskav)$
with the \term{risk aversion} $\riskav \in \real \setminus \{1\}$.
Positive and negative values correspond to risk-averse and risk-affine
individuals, respectively.
The factor $\patience \in \pohint{0, 1}$ is the \term{patience}
or \term{time discount factor.}

\begin{SCfigure}
  \includegraphics{dynamicPortfolioChoice_1}%
  \caption[Dynamic portfolio choice models]{%
    Dynamic portfolio choice models.
    The available wealth $\wealth_t$ is invested
    into bonds ($\bond_t$) and two stocks ($\stock_{t,1}, \stock_{t,2}$).
    The rest is consumed ($\consume_t$), resulting in the utility
    $\utilityfcn(\consume_t)$.
    In the last time step $T$ \emph{(far right),}
    the whole wealth is consumed, if we do not take inheritance
    into account.%
  }%
  \label{fig:dynamicPortfolioChoice}%
\end{SCfigure}

\paragraph{Limitations of naive utility maximization}

When solving the utility maximization problem \eqref{eq:utilityMaximization},
there are two issues.
First, solving \eqref{eq:utilityMaximization} for all times $t$ at once
implies solving a $(T+1) m_{\policy}$-dimensional optimization problem,
which is usually computationally infeasible.
Second, \eqref{eq:utilityMaximization} does not take stochastic variables
$\stochastic_t$ such as stock return rates into account.
These variables influence the state transition, i.e.,
$(\state_t, \policy_t, \stochastic_t) \mapsto \state_{t+1}$.
Consequently, $\state_t$ cannot be computed from $\state_0$ and
$\policy_0, \dotsc, \policy_{t-1}$ alone,
which complicates the solution of \eqref{eq:utilityMaximization}
even in the expected value.

\paragraph{Bellman principle}

To solve the first issue,
Bellman's principle of optimality \cite{Bellman57Dynamic}
can be applied to problems like
\eqref{eq:utilityMaximization} that are said to have
\term{optimal substructure.}
The principle states that the optimal policy for all times $t = 0, \dotsc, T$
is also optimal with respect to $t = 1, \dotsc, T$, i.e.,
\begin{equation}
  \max_{\policy_0, \dotsc, \policy_T}
  \sum_{t=0}^T \patience^t \utilityfcn(\consume_t(\state_t, \policy_t))
  = \max_{\policy_0} \left(
    \utilityfcn(\consume_0(\state_0, \policy_0))
    + \patience \max_{\policy_1, \dotsc, \policy_T}
    \sum_{t=1}^T \patience^{t-1} \utilityfcn(\consume_t(\state_t, \policy_t))
  \right),
\end{equation}
where we omitted the constraints for brevity.
The inner maximum problem over $\policy_1, \dotsc, \policy_T$
has the same structure as the original problem on the \lhs.
With the \term{value function}
$\valuefcn_t\colon \clint{\*0, \*1} \to \real$,
$\valuefcn_t(\state_t) :=
\max_{\policy_0, \dotsc, \policy_T}
\sum_{t'=t}^T \patience^{t'}
\utilityfcn(\consume_{t'}(\state_{t'}, \policy_{t'}))$, this can be rewritten as
\begin{equation}
  \label{eq:simpleBellman}
  \valuefcn_0(\state_0)
  = \max_{\policy_0} \left(
    \utilityfcn(\consume_0(\state_0, \policy_0)) +
    \patience \valuefcn_1(\state_1)
  \right)
  \quad\text{s.t. specific constraints,}
\end{equation}
where $\state_1$ is the result of the state transition
from $(\state_0, \policy_0)$.

\paragraph{General Bellman equation}

If we formulate \eqref{eq:simpleBellman} for arbitrary times $t$ and
consider constraints, state transition, and stochastic variables,
we obtain the \term{Bellman equation:}
\begin{subequations}
  \label{eq:generalBellman}
  \begin{gather}
    \valuefcn_t(\state_t)
    = \max_{\policy_t} \left(
      \utilityfcn(\consume_t(\state_t, \policy_t)) +
      \patience \expectation[t]{
        \valuefcn_{t+1}(\statefcn_t(\state_t, \policy_t, \stochastic_t))
      }
    \right),\quad
    t < T,\\
    \policy_t \in \real^{m_{\policy}}\;\;\text{s.t.}\;\;
    \ineqconfun_t(\state_t, \policy_t) \le \*0,
  \end{gather}
\end{subequations}
where
$\statefcn_t\colon \clint{\*0, \*1} \times \real^{m_{\policy}} \times
\stochdomain_t \to \clint{\*0, \*1}$,
$(\state_t, \policy_t, \stochastic_t) \mapsto \state_{t+1}$,
is the \term{state transition function,}
$\ineqconfun_t\colon \clint{\*0, \*1} \times \real^{m_{\policy}} \to
\real^{m_{\ineqconfun}}$ is the \term{constraint function,}
and the expected value is
\begin{equation}
  \expectation[t]{
    \valuefcn_{t+1}(\statefcn_t(\state_t, \policy_t, \stochastic_t))
  }
  = \int_{\stochdomain_t}
  \valuefcn_{t+1}(\statefcn_t(\state_t, \policy_t, \stochastic_t))
  P_t(\stochastic_t) \diff{}\stochastic_t
\end{equation}
with the probability density function
$P_t\colon \stochdomain_t \to \nonnegreal$ of $\stochastic_t$.
We denote the location of the maximum of \eqref{eq:generalBellman}
as the optimal policy $\optpolicyfcn_t$,
which may be regarded as a function
$\optpolicyfcn_t\colon \clint{\*0, \*1} \to \real^{m_{\policy}}$,
$\state_t \mapsto \optpolicyfcn_t(\state_t)$.



\subsection{Solution with B-Spline Surrogates on Sparse Grids}
\label{sec:812surrogates}

\blindtext{}



\subsection{Related Work}
\label{sec:813relatedWork}

\blindtext{}

\section{Algorithms}
\label{sec:82algorithms}

This section gives an overview of the algorithms that
were used to implement the solution process of
the discretized Bellman equation \eqref{eq:gridBellman}.



\subsection{General Structure}
\label{sec:821generalStructure}

The general approach to solve dynamic portfolio choice models is as follows:
\begin{enumerate}
  \item
  Generation of value function interpolants $\valueintp_t$
  (\texttt{solveValueFunction}, \cref{alg:financeSolveValueFunction})
  
  \item
  Generation of optimal policy interpolants $\optpolicyintp_t$
  (\texttt{solvePolicy}, \cref{alg:financeSolvePolicy})
  
  \item
  Computation of Euler equation errors
  (\texttt{computeEulerErrors}, \cref{alg:computeEulerErrors})
\end{enumerate}
The separation of the solution processes for
the value function interpolants $\valueintp_t$
and the optimal policy interpolants $\optpolicyintp_t$
enables to generate different spatially adaptive sparse grids
for the value function and the optimal policies.
This is useful if the shapes of value function and optimal policies
have different characteristics.

In the following \cref{%
  sec:822solveValueFunction,%
  sec:823optimization,%
  sec:824quadrature,%
  sec:825interpolation,%
  sec:826gridGeneration%
}, we describe the algorithmic details of
\texttt{solveValueFunction} (step 1).
The treatment of the other steps \texttt{solvePolicy} (step 2) and
\texttt{computeEulerErrors} (step 3) follows with
\cref{sec:827solvePolicy} and \cref{sec:828eulerEquationErrors},
respectively.

\paragraph{Interpolants}

We track two interpolants $\valueintp[1]_t$ and $\valueintp[p]_t$
for each $t = 0, \dotsc, T$.
The former interpolates the value function data at the grid points
with the hierarchical piecewise linear basis
(used for the surplus-based grid generation),
while the latter interpolates the data with hierarchical B-splines
of higher order $p > 1$.
Each $\valueintp[\ast]_t$ ($\ast \in \{1, p\}$)
additionally stores the grid points $\state_t^{(k)}$
and the optimal policies $\optpolicyintp_t(\state_t^{(k)})$
at the grid points ($k = 1, \dotsc, \ngp_t$).
For simplicity, we do not pass them as separate data
to the algorithms.



\subsection{Solution for Value Function}
\label{sec:822solveValueFunction}

\Cref{alg:financeSolveValueFunction} shows \texttt{solveValueFunction},
which generates the value function interpolants
$\valueintp[1]_t$ and $\valueintp[p]_t$ ($t = 0, \dotsc, T$).
The algorithm follows a simple optimize--refine--interpolate scheme:
First, the Bellman equation \eqref{eq:gridBellman} is solved
on some initial sparse grid (\texttt{optimize}).
Then, we \texttt{refine} the grid spatially adaptively.
Finally, the data on the resulting grid are \texttt{interpolate}d
with hierarchical higher-order B-splines.

\begin{algorithm}
  \begin{algorithmic}[1]
    \Function{%
      $\text{%
        $((\valueintp[1]_t, \valueintp[p]_t))_{t=0,\dotsc,T}$%
      } = \texttt{solveValueFunction}$%
    }{%
      \hspace*{0mm}%
    }
      \State{$\valueintp[p]_{T+1} \gets \emptyset$}
      \Comment{dummy variable (is not used)}%
      \For{$t = T, T - 1, \dotsc, 0$}
        \State{%
          $\valueintp[1]_t \gets \text{%
            Initial regular sparse grid with no values%
          }$%
        }
        \State{%
          $\valueintp[1]_t \gets
          \texttt{optimize($t$, $\valueintp[1]_t$, $\valueintp[p]_{t+1}$)}$%
        }
        \State{%
          $\valueintp[1]_t \gets
          \texttt{refine($t$, $\valueintp[1]_t$, $\valueintp[p]_{t+1}$)}$%
        }
        \State{%
          $\valueintp[p]_t \gets
          \texttt{interpolate($\valueintp[1]_t$)}$%
        }
      \EndFor{}
    \EndFunction{}
  \end{algorithmic}
  \caption[%
    Generation of value function interpolants (\texttt{solveValueFunction})%
  ]{%
    Generation of value function interpolants.
    Outputs are the piecewise linear interpolants $\valueintp[1]_t$
    and the higher-order B-spline interpolants $\valueintp[p]_t$
    for all $t = 0, \dotsc, T$.%
  }%
  \label{alg:financeSolveValueFunction}%
\end{algorithm}

At the beginning of every iteration $t$,
the grid of the piecewise linear interpolant is reset
to an initial, possibly regular sparse grid.
It is also possible to reuse the grid from the
previous iteration $t + 1$.
Nevertheless, the results we then obtain become worse,
likely due to the different characteristics of $\valueintp[1]_t$
for different $t$ (e.g., kinks).

The higher-order B-spline interpolant
$\valueintp[p]_{t+1}$ of the previous iteration $t+1$ is used
for the \rhs of the Bellman equation \eqref{eq:gridBellman},
if $t < T$.
In the first iteration $t = T$,
there is no such interpolant.
However, it is not needed anyway, as the exact terminal solution
$\valuefcn_T$ is assumed to be explicitly known.



\subsection{Optimization}
\label{sec:823optimization}

\todo{%
  certainty equivalence,
  parallelization,
  space transformation?
  value trafo?%
}

\blindtext{}

\begin{algorithm}
  \begin{algorithmic}[1]
    \Function{$\valueintp[1]_t = \texttt{optimize}$}{%
      $t$, $\valueintp[1]_t$, $\valueintp[p]_{t+1}$%
    }
      \State{%
        $(\state_t^{(k)})_{k=1,\dotsc,\ngp_t}
        \gets \text{grid of $\valueintp[1]_t$}$%
      }
      \For{$k = 1, \dotsc, \ngp_t$}
        \If{$\valueintp[1]_t(\state_t^{(k)})$ not previously computed}
          \IfOneLine{$t = T$}{%
            $\valueintp[1]_T(\state_T^{(k)})
            \gets \texttt{computeKnownTerminalSolution($\state_T^{(k)}$)}$%
          }
          \ElseOneLine{%
            $\valueintp[1]_t(\state_t^{(k)})
            \gets \texttt{%
              optimizeSinglePoint(%
                $t$, $\state_t^{(k)}$, $\valueintp[p]_{t+1}$%
              )%
            }$%
          }
        \EndIf{}
      \EndFor{}
      \State{%
        Re-interpolate
        $(\valueintp[1]_t(\state_t^{(k)}))_{k=1,\dotsc,\ngp_t}$
        with piecewise linear functions%
      }
    \EndFunction{}
  \end{algorithmic}
  \caption[Evaluation of the value function (\texttt{optimize})]{%
    Evaluation of the value function at all
    grid points $\state_t^{(k)}$ of $\valueintp[1]_t$
    at which the value function has not been evaluated yet.
    Inputs are
    the time $t$,
    the piecewise linear interpolant $\valueintp[1]_t$
    of the current iteration $t$ (with the underlying sparse grid and
    corresponding function values, possibly unset), and
    the higher-order B-spline interpolant $\valueintp[p]_{t+1}$
    of the previous iteration $t + 1$
    (not used if $t = T$).
    Output is the updated piecewise linear interpolant $\valueintp[1]_t$,
    where all missing function values at grid points have been computed.%
  }%
  \label{alg:financeOptimize}%
\end{algorithm}




\subsection{Quadrature}
\label{sec:824quadrature}

\blindtext{}



\subsection{Interpolation and Extrapolation}
\label{sec:825interpolation}

\todo{%
  grid type, basis function type,
  why extrapolation necessary, extrapolation method%
}

\blindtext{}



\subsection{Grid Generation}
\label{sec:826gridGeneration}

%The piecewise linear interpolant is used for the surplus-based
%grid generation,
%as the surpluses are easier to compute in the piecewise linear case,
%and they are more meaningful
%due to the integral representation formula \eqref{eq:surplusIntegral}.

\todo{gradient grids?}

\blindtext{}

\begin{algorithm}
  \begin{algorithmic}[1]
    \Function{$\valueintp[1]_t = \texttt{refine}$}{%
      $t$, $\valueintp[1]_t$, $\valueintp[p]_{t+1}$%
    }
      \For{$j = 1, \dotsc, \norefine_t$}
        \State{%
          Refine all grid points of $\valueintp[1]_t$ whose
          surplus is $< \refinetol_t$%
        }
        \IfOneLine{grid of $\valueintp[1]_t$ did not change}{\Break}
        \State{%
          $\valueintp[1]_t \gets
          \texttt{optimize($t$, $\valueintp[1]_t$, $\valueintp[p]_{t+1}$)}$%
        }
      \EndFor{}
    \EndFunction{}
  \end{algorithmic}
  \caption[TODO]{%
    TODO%
  }%
  \label{alg:financeRefine}%
\end{algorithm}



\subsection{Solution for Optimal Policies}
\label{sec:827solvePolicy}

\blindtext{}



\subsection{Euler Equation Errors}
\label{sec:828eulerEquationErrors}

\blindtext{}

\section{Example Problems}
\label{sec:83problems}

\blindtext{}

\section{Implementation and Numerical Results}
\label{sec:84results}

\minitoc[13mm]{65mm}{7}

\parbox{1em}{}
\vspace{-3em}



\printornamentsfalse
\subsection{Implementation}
\label{sec:841implementation}
\printornamentstrue

\paragraph{Parameter values and basis functions}

We use
a risk aversion factor of $\riskav = 3.5$,
a patience factor of $\patience = 0.97$,
a transaction cost rate of $\tac = \SI{1}{\percent}$, and
a minimum consumption of $\normconsume_{\min} = 0.001$.
The bond and stock return rates are taken from \cite{Cai10Stable}.
The models are solved for $T = 6$ time steps;
this number suffices to show all relevant numerical effects and results,
while keeping the computational effort at a reasonable level.
As higher-order B-spline basis functions,
we use the hierarchical weakly fundamental not-a-knot splines
$\bspl[\wfs,\nak]{\*l,\*i}{p}$ of cubic degree $p = 3$
to enable hierarchization with the unidirectional principle.

\vspace*{-0.5em}

\paragraph{Software}

The dynamic portfolio choice models were solved using a self-written
MATLAB framework.
The object-oriented framework was designed in such a way that
not only transaction costs problems,
but many other types of dynamic portfolio choice models can be handled.
For instance, the base class \texttt{LifecycleProblem} provides
an interface with abstract functions such as
\texttt{computeTerminalValueFunction} and
\texttt{computeStateTransition}.
The actual functionality implemented in the base class strongly resembles
the algorithms presented in \cref{sec:82algorithms}.
This is not only desirable from a modeling perspective,
but also facilitates future usage by other researchers.
For creating (hierarchization) and evaluating sparse grid interpolants,
the sparse grid toolbox \sgpp was used \cite{Pflueger10Spatially}.%
\footnote{%
  \url{http://sgpp.sparsegrids.org/}%
}
The emerging optimization problems were solved using
sequential quadratic programming methods supplied by the
NAG Toolbox for MATLAB.%
\footnote{%
  \url{https://www.nag.com/}%
}
To avoid getting stuck in local minima,
we repeat the optimization process for a varying number
of initial multi-start points (in the range of a few dozens).
All runtimes were measured on a shared memory computer
with 144 threads on 4x Intel Xeon E7-8880v3 (72 cores, 144 threads).



\subsection{Error Sources and Error Measure}
\label{sec:842errorSources}

\paragraph{Error sources}

In this application, there are the following error sources:

\begin{enumerate}[label=E\arabic*.,ref=E\arabic*,leftmargin=2.7em]
  \item
  \label{item:financeErrorInterpolationValue}
  Interpolation of the value function
  (i.e., $\normcetvalueintp_{t+1} \not= \normcetvaluefcn_{t+1}$)
  
  \item
  \label{item:financeErrorInterpolationPolicy}
  Interpolation of the policy functions
  (i.e., $\optnormpolicyintp_t \not= \optnormpolicyfcn_t$)
  
  \item
  \label{item:financeErrorExtrapolation}
  Extrapolation errors
  (i.e., $
  \normcetvalueintp_{t+1}(\state_{t+1})
  \not= \normcetvaluefcn_{t+1}(\state_{t+1})
  $)
  
  \item
  \label{item:financeErrorCropping}
  Errors from state space cropping
  (i.e., Euler errors do not vanish for exact solution)
  
  \item
  \label{item:financeErrorOptimization}
  Optimization error
  (i.e., the global minimum found by the optimizer is inaccurate)
  
  \item
  \label{item:financeErrorQuadrature}
  Quadrature error
  ($
  \expectation[t]{\cdots}
  \not= \sum_{j=1}^{m_{\quadweight}} \quadweight_t^{(j)}
  [\cdots](\stochastic_t^{(j)})
  $)
  
  \item
  \label{item:financeErrorRounding}
  Floating-point rounding errors
  (i.e., arithmetical operations are inaccurate)
\end{enumerate}
Due to the dynamic programming scheme,
the combination of all errors accumulates over $t$.
For instance, if the optimization does not find the global optimum
exactly or it only finds a local one for the time $t + 1$,
the error propagates from the interpolant $\normcetvalueintp_{t+1}$
on the right-hand side of the Bellman equation
\eqref{eq:normalizedTCPBellmanEquation} to $\normcetvalueintp_t$
on the left-hand side, and so on.
If the system does not damp these errors,
the error might become stronger and stronger backwards in time $t$.

\paragraph{Error measure}

We use the weighted Euler equation error
$\weightedeulererror_t(\normstate_t)$
to assess the quality of the resulting optimal policies.
This error measure can be analyzed pointwise or
via its $\Ltwo$ norm.
As the errors generally grow backwards in time,
it suffices to consider $t = 0$.



\subsection{Numerical Results}
\label{sec:843results}

\paragraph{Reference solution}

We use full grid solutions as reference solutions,
i.e., $\{\state_t^{(k)} \mid k = 1, \dotsc, \ngp_t\} = \fgset{n,d}$
for some fixed level $n \in \nat$ for all $t = 0, \dotsc, T$.
Obviously, this is only computationally feasible
for low dimensionalities $d$ due to the curse of dimensionality.
For each $d$,
\cref{tbl:financeSolutionFullRegularSparseGrids}
(see \cref{chap:a40financeDetails})
contains full grid solutions of different levels.
The reference solution for $d = 2$ is shown in
\cref{fig:financeSolution2DReference}.
Unfortunately, only full grid solutions up to $d = 3$ could be computed
due to excessive runtime for $d \ge 4$.
This underlines the need for sophisticated
discretization techniques such as sparse grids.

\begin{figure}
  \makebox[49mm][r]{%
    \includegraphics{financeSolution2D_1}%
  }%
  \hfill%
  \makebox[49mm][r]{%
    \includegraphics{financeSolution2D_2}%
  }%
  \hfill%
  \makebox[49mm][r]{%
    \includegraphics{financeSolution2D_4}%
  }%
  \\[1mm]%
  \makebox[49mm][r]{%
    \includegraphics{financeSolution2D_6}%
  }%
  \hfill%
  \makebox[49mm][r]{%
    \includegraphics{financeSolution2D_3}%
  }%
  \hfill%
  \makebox[49mm][r]{%
    \includegraphics{financeSolution2D_5}%
  }%
  \caption[Reference solution for the two-dimensional TCP]{%
    Full grid solution for the transaction costs problem
    with $d = 2$ stocks.
    Shown are the value function $\normcetvalueref_t$ \emph{(top left)} and the
    optimal policy $\optnormpolicyref_t$ for the initial time step $t = 0$.%
  }%
  \label{fig:financeSolution2DReference}%
\end{figure}

\paragraph{Convergence of the weighted Euler equation error}

\Cref{fig:financeEulerError} shows the convergence of the
$\Ltwo$ norm $\normLtwo{\weightedeulererror_0}$
weighted Euler equation error for $t = 0$ for regular sparse grids
and spatially adaptive sparse grids
for the cases of $d = 1, \dotsc, 4$ stocks.
\begin{figure}
  \includegraphics{financeEulerError_5}%
  \\[2mm]%
  \subcaptionbox{%
    $d = 1$%
  }[37mm]{%
    \includegraphics{financeEulerError_1}%
  }%
  \hfill%
  \subcaptionbox{%
    $d = 2$%
  }[37mm]{%
    \includegraphics{financeEulerError_2}%
  }%
  \hfill%
  \subcaptionbox{%
    $d = 3$%
  }[37mm]{%
    \includegraphics{financeEulerError_3}%
  }%
  \hfill%
  \subcaptionbox{%
    $d = 4$%
  }[37mm]{%
    \includegraphics{financeEulerError_4}%
  }%
  \caption[Convergence of the weighted Euler equation error]{%
    Convergence of the $\Ltwo$ norm $\normLtwo{\weightedeulererror_t}$
    of the weighted Euler equation error for $t = 0$ for
    regular sparse grids \emph{\textcolor{C0}{(blue)}} and
    spatially adaptive sparse grids \emph{\textcolor{C1}{(red)}.}
    The number $\ngp_t$ is the mean number
    $\frac{1}{m_{\policy}} \sum_{j=1}^{m_{\policy}} \ngp_{t,j}$
    of grid points over all policy grids for $t = 0$,
    where $\ngp_{t,j}$ is the number of grid points
    of the $j$-th policy entry.%
  }%
  \label{fig:financeEulerError}%
\end{figure}%
For this and the following plots,
the value function grid is kept unchanged
(usually a slightly refined regular sparse grid of level three or four),
while the mean number $\ngp_t$ of policy grid points increases
with decreasing refinement threshold $\refinetol_t$,
since the value function grid does not seem to have a great influence
on the convergence of the Euler equation errors.
The spatial adaptivity decreases the error by
two orders of magnitude in one dimension.
The gain is smaller for higher dimensionalities $d$,
but spatial adaptive grids still outperform regular grids.
For $d = 2$, we observe that the error saturates
after $\ngp_t \approx \num{4000}$ points before dropping below $10^{-5}$.
This is most likely due to the parts
\ref{item:financeErrorExtrapolation} to
\ref{item:financeErrorRounding} of the error that are not influenced
by sparse grid interpolation.
In addition, convergence significantly decelerates starting with $d = 4$.
For $d = 4$, spatially adaptive sparse grids with
a mean number $\ngp_t \approx \num{4252}$ of policy grid points for $t = 0$
are able to achieve a weighted Euler equation error of
$\normLtwo{\weightedeulererror_t} \approx \num{2.0e-2}$.
For $d = 5$, we are still able to achieve an acceptable error of
$\normLtwo{\weightedeulererror_t} \approx \num{3.2e-2}$
with spatially adaptive sparse grids with
a mean number $\ngp_t \approx \num{7768}$ of policy grid points for $t = 0$.
While we do not see any convergence for this dimensionality,
this is still a major result as such high-dimensional models
could not be solved with state-of-the-art methods up to now.

\paragraph{Optimal policies in 2D and 5D}

The value function and optimal policies corresponding to
sparse grid solutions for $d = 2$ or $d = 5$ stocks are shown in
\cref{fig:financeSolution2DSparseGrid} and
\cref{fig:financeSolution5DSparseGrid}, respectively.


\begin{figure}
  \subcaptionbox{%
    $\normcetvalueintp[1]_t$%
  }[48mm]{%
    \includegraphics{financeSolution2D_7}%
  }%
  \hfill%
  \subcaptionbox{%
    $\normbuy[\sparse,1]_{t,1}$%
  }[48mm]{%
    \includegraphics{financeSolution2D_8}%
  }%
  \hfill%
  \subcaptionbox{%
    $\normsell[\sparse,1]_{t,1}$%
  }[48mm]{%
    \includegraphics{financeSolution2D_10}%
  }%
  \\[2mm]%
  \subcaptionbox{%
    $\normbond_t^{\sparse,1}$%
  }[48mm]{%
    \includegraphics{financeSolution2D_12}%
  }%
  \hfill%
  \subcaptionbox{%
    $\normbuy[\sparse,1]_{t,2}$%
  }[48mm]{%
    \includegraphics{financeSolution2D_9}%
  }%
  \hfill%
  \subcaptionbox{%
    $\normsell[\sparse,1]_{t,2}$%
  }[48mm]{%
    \includegraphics{financeSolution2D_11}%
  }%
  \caption[Sparse grid solution for the two-dimensional TCP]{%
    Spatially adaptive sparse grid solution for the transaction costs problem
    with $d = 2$ stocks.
    \vspace{-0.15em}%
    Shown are the value function $\normcetvalueref_t$ \emph{(top left)} and the
    optimal policy $\optnormpolicyref_t$ for the initial time step $t = 0$,
    together with the corresponding grid points \emph{(dots).}
    The color coding is the same as in
    \cref{fig:financeSolution2DReference}.%
  }%
  \label{fig:financeSolution2DSparseGrid}%
\end{figure}

\begin{figure}
  \makebox[37mm][c]{%
    \hspace*{3.8mm}%
    \raisebox{-\height}{\includegraphics{financeSolution5D_13}}%
  }%
  \hfill%
  \makebox[37mm][c]{%
    \hspace*{2.9mm}%
    \raisebox{-\height}{\includegraphics{financeSolution5D_14}}%
  }%
  \hfill%
  \makebox[37mm][c]{%
    \hspace*{4.7mm}%
    \raisebox{-\height}{\includegraphics{financeSolution5D_16}}%
  }%
  \hfill%
  \makebox[37mm][c]{%
    \hspace*{4.5mm}%
    \raisebox{-\height}{\includegraphics{financeSolution5D_15}}%
  }%
  \\[1mm]%
  \makebox[37mm][c]{%
    \includegraphics{financeSolution5D_1}%
  }%
  \hfill%
  \makebox[37mm][c]{%
    \includegraphics{financeSolution5D_4}%
  }%
  \hfill%
  \makebox[37mm][c]{%
    \includegraphics{financeSolution5D_12}%
  }%
  \hfill%
  \makebox[37mm][c]{%
    \includegraphics{financeSolution5D_9}%
  }%
  \\[1mm]%
  \makebox[37mm][c]{%
    \includegraphics{financeSolution5D_2}%
  }%
  \hfill%
  \makebox[37mm][c]{%
    \includegraphics{financeSolution5D_5}%
  }%
  \hfill%
  \makebox[37mm][c]{%
    \includegraphics{financeSolution5D_7}%
  }%
  \hfill%
  \makebox[37mm][c]{%
    \includegraphics{financeSolution5D_10}%
  }%
  \\[1mm]%
  \makebox[37mm][c]{%
    \includegraphics{financeSolution5D_3}%
  }%
  \hfill%
  \makebox[37mm][c]{%
    \includegraphics{financeSolution5D_6}%
  }%
  \hfill%
  \makebox[37mm][c]{%
    \includegraphics{financeSolution5D_8}%
  }%
  \hfill%
  \makebox[37mm][c]{%
    \includegraphics{financeSolution5D_11}%
  }%
  \caption[Sparse grid solution for the five-dimensional TCP]{%
    Spatially adaptive sparse grid solution for the transaction costs problem
    with $d = 5$ stocks.
    \vspace{-0.15em}%
    Shown are slice plots of
    the value function $\normcetvalueref_t$ \emph{(top left)} and
    the optimal policy $\optnormpolicyref_t$ for the initial time step $t = 0$,
    where for each function, a pair $(o_1, o_2)$
    of dimensions to be plotted was chosen,
    and the stock fractions $\stock_{t,o}$ of the other dimensions $o$
    are set to $0.1$.
    In addition, the corresponding grid points \emph{(dots)}
    are shown as the projection onto the
    $\stock_{t,o_1}$--$\stock_{t,o_2}$ plane.%
  }%
  \label{fig:financeSolution5DSparseGrid}%
\end{figure}

\paragraph{Pointwise error}

\dummytext[2]{}

\begin{figure}
  \includegraphics{financePointwiseError_4}%
  \\[2mm]%
  \subcaptionbox{%
    TODO%
  }[49mm]{%
    \includegraphics{financePointwiseError_1}%
  }%
  \hfill%
  \subcaptionbox{%
    TODO%
  }[49mm]{%
    \includegraphics{financePointwiseError_2}%
  }%
  \hfill%
  \subcaptionbox{%
    TODO%
  }[49mm]{%
    \includegraphics{financePointwiseError_3}%
  }%
  \caption[TODO]{%
    TODO%
  }%
  \label{fig:financePointwiseError}%
\end{figure}

\paragraph{Monte Carlo simulation}

% Portfolio-Gewichte, die Sharp-Ratio maximieren (Einheit Überrendite pro Einheit Risiko)
%loadResult(22)
%problem.plotLifecycleProfile(simulation.state, simulation.discreteState, simulation.policy, simulation.shock)
%A = [0.0256, 0.00576, 0.00288, 0.00176; ...
%0.00576, 0.0324, 0.0090432, 0.010692; ...
%0.00288, 0.0090432, 0.04, 0.0132; ...
%0.00176, 0.010692, 0.0132, 0.0484];
%A
%b = diag(A)
%b = sqrt(diag(A))
%format long
%b = sqrt(diag(A))
%format short
%b = sqrt(diag(A))
%A
%cor(A)
%corr(A)
%doc corr
%corr(A(1:3,1:3))
%B = corr(A); B = B(1:3,1:3);
%B = corr(A); B = B(1:3,1:3)
%corrcov(A)
%corrcov(A(1:3,1:3))
%[B, sigma] = corrcov(A(1:3,1:3))
%b = [0.0572, 0.0638, 0.07, 0.0764]
%diff(b)
%rf = problem.Return.riskfreeRate
%b = b(1:3);
%b
%(b-rf)./sigma
%(b-rf)/sigma
%(b-rf)./sigma
%(b-rf)./sigma'
%ver
%x = 1/3*ones(1,3)
%(x*b'-rf)./(x*sigma'*x')
%(x*b'-rf)./(x*sigma*x')
%x*b'
%rf
%x*sigma
%sigma
%A
%(x*b'-rf)./(x*A(1:3,1:3)*x')
%f = @(x) (x*b'-rf)./(x*A(1:3,1:3)*x')
%help fminbnd
%help fmincon
%fmincon(f, [1/3 1/3 1/3], [], [], [], [], [0 0 0], [1 1 1])
%sum(ans)
%xopt = fmincon(f, [1/3 1/3 1/3], [], [], [], [], [0 0 0], [1 1 1])
%f(xopt)
%xopt = fmincon(-f, [1/3 1/3 1/3], [], [], [], [], [0 0 0], [1 1 1])
%f = @(x) -(x*b'-rf)./(x*A(1:3,1:3)*x')
%xopt = fmincon(f, [1/3 1/3 1/3], [], [], [], [], [0 0 0], [1 1 1])
%sum(xopt)
%xopt = fmincon(f, [1/3 1/3 1/3], [], [], [1 1 1], 1, [0 0 0], [1 1 1])
%sum(xopt)
%1.8/(1.45+1.45+1.8)
%1.45/(1.45+1.45+1.8)
%A
%A = diag(diag(A))
%xopt = fmincon(f, [1/3 1/3 1/3], [], [], [1 1 1], 1, [0 0 0], [1 1 1])

% certainty-equivalent consumption
% mit B-splines und linearen Funktionen

\begin{figure}
  \includegraphics{financeSimulation_5}%
  \\[2mm]%
  \subcaptionbox{%
    $d = 3$%
  }[48mm]{%
    \includegraphics{financeSimulation_1}%
  }%
  \hfill%
  \subcaptionbox{%
    $d = 4$%
  }[48mm]{%
    \includegraphics{financeSimulation_2}%
  }%
  \hfill%
  \subcaptionbox{%
    $d = 5$%
  }[48mm]{%
    \includegraphics{financeSimulation_3}%
  }%
  \\[2mm]%
  \subcaptionbox{%
    $d = 3$ (stacked)%
  }[48mm]{%
    \includegraphics{financeSimulation_4}%
  }%
  \hfill%
  \begin{minipage}[b]{92mm}%
    \caption[TODO]{%
      TODO TODO TODO TODO TODO TODO TODO TODO TODO TODO TODO TODO TODO
      TODO TODO TODO TODO TODO TODO TODO TODO TODO TODO TODO TODO TODO
      TODO TODO TODO TODO TODO TODO TODO TODO TODO TODO TODO TODO TODO
      TODO TODO TODO TODO TODO TODO TODO TODO TODO TODO TODO TODO TODO%
    }%
    \vspace*{-6.3mm}%
    \label{fig:financeSimulation}%
  \end{minipage}%
\end{figure}

\dummytext[2]{}

\paragraph{Complexity and runtime analysis}

% analyze runtime behavior over N (fixed d) and d

% (e-mail 2018-01-03)
%Runtime of one optimize() run
%- -----------------------------
%O(#New state grid points
%* #Optimization iterations
%* #Quadrature points
%* #Old state grid points
%* #State dimensions
%* Runtime of one 1D basis evaluation)
%
%where
%#(Optimization iterations) depends on #(Policy dimensions),
%#(Policy dimensions) = 2 * #Stocks + 1
%#(State dimensions) = #Stocks + 1
%
%(modulo extrapolations)

\begin{equation}
  \landauO{
    % for every time step
    T
    % for every grid point of new time step
    \cdot \ngp_t
    % for every optimization iteration
    \cdot \text{\#optimizer iterations}
    \cdot
    \underbrace{
      % for every gradient entry of the objective function
      m_{\policy}
      % for every quadrature point
      \cdot m_{\stochastic}
      \cdot
      \overbrace{
        % for every summand of the sparse grid function
        % (grid point of old time step)
        \ngp_{t+1}
        % for every tensor product factor
        \cdot m_{\state}
        % for every B-spline summand
        \cdot p
      }^{\mathclap{\text{evaluation of interpolant}}}
    }_{\text{evaluation of objective gradient}}\,
  }
\end{equation}

\#optimizer iterations contains multi-start points!

\begin{equation}
  \landauO{
    T
    \cdot \ngp
    \cdot \text{\#optim.\ it.}
    \cdot d
    \cdot d
    \cdot \ngp
    \cdot d
    \cdot p
  }
  = \landauO{T \ngp^2 d^3 p \cdot \text{\#optim.\ it.}}??
\end{equation}

\dummytext[3]{}

\paragraph{Impact of the B-spline degree}

\dummytext[3]{}

\paragraph{Comparing exact gradients to finite differences}

\dummytext[3]{}


\cleardoublepage
