\chapter{Conclusion}
\label{chap:90conclusion}

\noindent
Finally, we conclude the thesis by summarizing its results
and giving an outlook on possible future work.
In particular, we highlight key contributions of this thesis to research,
give recommendations for future applications of the presented method, and
state possible downsides and limitations.

\paragraph{Summary of the thesis}

The contribution of this thesis consists of two major parts.
In the first part of the thesis,
hierarchical B-splines on sparse grids were comprehensively presented
and embedded in a sparse grid framework with general
tensor product basis functions.
The advantage of this approach was that the framework could be reused
for different hierarchical bases
(as for the various spline bases derived in this thesis)
and that it clarified which properties only held for the
classical piecewise linear bases and not for other tensor product bases.
We saw that standard hierarchical B-splines suffer from
approximation issues near the boundary, and
we solved this issue by incorporating not-a-knot boundary conditions
into the hierarchical B-spline basis.
The focus was on the algorithmic implications
of the novel bases, where we focused on the hierarchization problem as
an example problem.
We looked at requirements that grids and bases had to satisfy
to enable efficient hierarchization algorithms,
for which we gave clear formulations and formal correctness proofs.
As a result, a whole ``zoo'' of hierarchical (B-)spline functions
has been derived in this thesis.
The main types were
standard hierarchical B-splines,
modified hierarchical B-splines,
hierarchical not-a-knot B-splines,
hierarchical fundamental splines, and
hierarchical weakly fundamental splines
(where the former two are not novel).
Modified, not-a-knot, and (weakly) fundamental splines could be combined
almost arbitrarily to tailor the ansatz functions to suit one's specific needs.

The second, more practical part of the thesis was dedicated to transferring
the newly gained theoretical knowledge to
academic and real-life application test cases.
We verified that only with the new hierarchical not-a-knot conditions,
one is able to retrieve the best possible order of convergence
$\landauO{\ms{h}^{p+1}}$ for interpolation
(B-spline degree $p$, fixed dimensionality $d$).
Using the Novak--Ritter criterion,
which was specifically designed for optimization,
we were able to achieve optimization gaps
that were for some test functions up to six orders of magnitude smaller
for cubic B-splines than for standard piecewise linear functions.
We transfered the Novak--Ritter criterion to uncertainty quantification
and obtained similarly strong results for the
propagation of fuzzy uncertainties with the fuzzy extension principle.
Furthermore, we successfully showed the suitability of hierarchical B-splines
for three real-world applications
(see the summary in \cref{tbl:applicationSummary}).
\begin{table}
  \setnumberoftableheaderrows{1}%
  \begin{tabular}{%
    >{\kern\tabcolsep}=l<{\kern5mm}+l+l+l<{\kern\tabcolsep}%
  }
    \toprulec
    \headerrow
    Category&                   Biomechanics&      Topology opt.&      Finance\\
    \midrulec
    Interpolated quantities&    Muscle forces&     Elasticity tensors& Value functions\\
    SG dimensionality&          2&                 5&                  5\\
    \#Optimization variables&   2&                 \num{40000}&        11\\
    Time per evaluation&        \SI{30}{\minute}&  \SI{30}{\second}&   ---\\
    \#Eval. per opt. iteration& 4&                 \num{8000}&         150\\
    Objective function type&    Linear/non-linear& Non-linear&         Non-linear\\
    Constraint function type&   Non-linear/---&    Non-linear&         Linear\\
    Optimization method&        Augm. Lagrangian&  SQP&                SQP\\
    \bottomrulec
  \end{tabular}
  \caption[TODO]{%
    TODO%
  }%
  \label{tbl:TODO}%
\end{table}%
First, by interpolating Cholesky factors of elasticity tensors,
we accomplished to solve three-dimensional topology optimization problems
with complex micro-cell structures.
Second, in the biomechanical application,
we dramatically reduced the computational time to solve
test scenarios by up to \SI{99}{\percent} by using
sparse grid surrogates with B-splines instead of the
actual continuum-mechanical model.
Third, we were able to solve dynamic portfolio choice problems
with five state variables and eleven policy variables
with unprecedented precision, as one could only speculate
how the solution looks like with state-of-the-art methods.
In all of these applications, the advantages of B-splines were made clear
by comparing the results to the classical piecewise linear basis.
The implementation of hierarchical B-splines of sparse grids is
publicly available under a free and open-source license.%
\footnote{%
  \url{http://sgpp.sparsegrids.org/}%
}

\paragraph{Recommendations (advantages and disadvantages)}

% Chapter 8: model limitations, no inheritance,
% no saving money for housing, \dots

\dummytext[1]{}

\paragraph{Outlook and future work}

\todo{mention the following future work (not in that order)}

\begin{itemize}
  \item
  steal from future work of Milestone Report,
  steal from project description (JP program)
  
  \item
  dimension-wise degree adaptivity
  
  \item
  h-p-adaptivity
  
  \item
  other adaptivity criteria than Novak--Ritter
  (e.g., for constrained optimization)
  
  \item
  other real-world optimization applications
  
  \item
  biomechanical application: more complicated musculoskeletal model
  with more parameters, have to exploit spatial adaptivity
  (was proof of concept)
  
  \item
  topology optimization: more complicated micro-cell models,
  more complicated constraints (bar widths like in \cite{Allaire16Towards})
  
  \item
  finance application: other models such as immediate annuity,
  eliminate limitations
  
  \item
  data mining with the new bases
  
  \item
  uncertainty quantification with B-splines on sparse grids
  
  \item
  dimensional adaptivity with the combination technique
  
  \item
  gradient-based approximation/interpolation methods
  \cite{Baar15Gradient}
\end{itemize}

\dummytext[2]{}

\cleardoublepage
