\section{Numerical Results}
\label{sec:64results}

\minitoc[-5mm]{72mm}{3}

\noindent
In this final section of the chapter,
we study optimal results of the test scenarios and
analyze interpolation and optimization errors
for topology optimization with B-spline surrogates on sparse grids.



\subsection{Methodology}
\label{sec:641methodology}

For simplicity, in the following,
we combine the functions to be interpolated,
i.e., the Cholesky factor
$\cholfactor\colon \clint{\*0, \*1} \to \real^{6 \times 6}$ and
the micro-cell density $\denscell\colon \clint{\*0, \*1} \to \real$,
to one single objective function
$\*\objfun\colon \clint{\*0, \*1} \to \real^{m+1}$.

\paragraph{Overview over offline and online phase}

Our method is divided into an offline phase and an online phase,
both of which are sketched in \cref{fig:topoOptPhases}.
The offline phase comprises
generating the spatially adaptive sparse grid
$\sgset = \{\gp{\*l_k,\*i_k} \mid k = 1, \dotsc, \ngp\}$,
solving corresponding micro-problems,
computing the Cholesky factors, and
hierarchizing the Cholesky factor entries and micro-cell densities
to obtain the sparse grid interpolant $\*\sgintp$.
Each optimization iteration of the online phase consists of
evaluating the interpolant $\*\sgintp$
for each micro-cell parameter $\mcp{j}$ ($j = 1, \dotsc, M$),
reconstructing the elasticity tensor $\etensorcholintp$ from
the Cholesky factors $\cholfactorintp$, and
solving the macro-problem to retrieve the approximated compliance value
$\complianceintp(\mcp{1}, \dotsc, \mcp{M})$.%
\footnote{%
  In addition, the partial derivatives
  $\partialdiff{} \etensorcholintp/\partialdiff{} x_t$
  ($t = 1, \dotsc, d$)
  are evaluated using \cref{eq:choleskyFactorDerivative}.
  This is required as we employ gradient-based optimization.%
}
The superscript in $\complianceintp$ indicates that
we do not use the exact elasticity tensors to compute the compliance value.

\begin{figure}
  \tikzset{
    myCircle/.style={
      circle,
      fill=mittelblau!30,
      draw=mittelblau,
      inner sep=0.5mm,
    }
  }%
  \subcaptionbox{%
    Offline phase (without the actual grid generation).%
  }[149mm]{%
    \begin{tikzpicture}
      \node[myCircle] (points) at (0mm,0mm) {%
        $
          \begin{matrix}
            \gp{\*l_1,\*i_1},\\
            \dots,\\
            \gp{\*l_{\ngp},\*i_{\ngp}}
          \end{matrix}
        $%
      };
      \node[myCircle] (elasticityTensors) at (43mm,0mm) {%
        $
          \begin{matrix}
            \etensor(\gp{\*l_1,\*i_1}),\\
            \dots,\\
            \etensor(\gp{\*l_{\ngp},\*i_{\ngp}})
          \end{matrix}
        $%
      };
      \node[myCircle] (choleskyFactors) at (80mm,0mm) {%
        $
          \begin{matrix}
            \cholfactor(\gp{\*l_1,\*i_1}),\\
            \dots,\\
            \cholfactor(\gp{\*l_{\ngp},\*i_{\ngp}})
          \end{matrix}
        $%
      };
      \node[myCircle] (choleskyInterpolant) at (118mm,0mm) {%
        $
          \begin{matrix}
            \cholfactorintp\colon \clint{\*0, \*1}\\
            {} \to \real^{6 \times 6}
          \end{matrix}
        $%
      };
      \draw[->,draw=C0] (points) -- node[above] {%
        \footnotesize{}micro-problem%
      } (elasticityTensors);
      \draw[->,draw=C0] (elasticityTensors) -- node[above] {%
        \footnotesize{}%
        $
          \tr{\cholfactor} \cholfactor = \etensor
        $\vphantom{p}%
      } (choleskyFactors);
      \draw[->,draw=C0] (choleskyFactors) -- node[above] {%
        \footnotesize{}interpolate%
      } (choleskyInterpolant);
    \end{tikzpicture}%
  }%
  \\[2mm]%
  \subcaptionbox{%
    Online phase (one iteration of the optimizer).%
  }[149mm]{%
    \begin{tikzpicture}
      \node[myCircle] (points) at (0mm,0mm) {%
        $
          \begin{matrix}
            \mcp{1},\\
            \dots,\\
            \mcp{M}
          \end{matrix}
        $%
      };
      \node[myCircle] (choleskyFactors) at (34mm,0mm) {%
        $
          \begin{matrix}
            \cholfactorintp(\mcp{1}),\\
            \dots,\\
            \cholfactorintp(\mcp{M})
          \end{matrix}
        $%
      };
      \node[myCircle] (elasticityTensors) at (83mm,0mm) {%
        $
          \begin{matrix}
            \etensorcholintp(\mcp{1}),\\
            \dots,\\
            \etensorcholintp(\mcp{M})
          \end{matrix}
        $%
      };
      \node[myCircle] (complianceValue) at (129.5mm,0mm) {%
        $
          \begin{matrix}
            \complianceintp(\mcp{1},\\
            \dotsc,\\
            \mcp{M})
          \end{matrix}
        $%
      };
      \draw[->,draw=C0] (points) -- node[above] {%
        \footnotesize{}evaluate\vphantom{p}%
      } (choleskyFactors);
      \draw[->,draw=C0] (choleskyFactors) -- node[above] {%
        \footnotesize{}%
        $
          \etensorcholintp
          = \tr{(\cholfactorintp)} \cholfactorintp
        $\vphantom{p}%
      } (elasticityTensors);
      \draw[->,draw=C0] (elasticityTensors) -- node[above] {%
        \footnotesize{}macro-problem%
      } (complianceValue);
    \end{tikzpicture}%
  }%
  \caption[Offline and online phase for topology optimization]{%
    Offline and online phase for topology optimization.
    The interpolation of the micro-cell density with $denscellintp$
    (see \cref{sec:622BSplines}) has been omitted for brevity.%
  }%
  \label{fig:topoOptPhases}%
\end{figure}

\paragraph{Generation of spatially adaptive sparse grids}

We use the classical surplus-based refinement criterion
(see, e.g., \cite{Pflueger10Spatially})
to generate the spatially adaptive sparse grids
as show in \cref{alg:topoOptGridGeneration}.
The difference to common surrogate settings is that the objective function
$\*f\colon \clint{\*0, \*1} \to \real^{m+1}$
is not scalar-valued, but matrix-valued.
As the components of $\cholfactor$ cannot be evaluated individually,
the adaptivity criterion has to consider all entries at once
to avoid performing unnecessary evaluations.
We use the surpluses in the piecewise linear hierarchical basis,
as their absolute values correlate with the second mixed derivative
of the objective function due to \eqref{eq:surplusIntegral}.

\begin{algorithm}
  \begin{algorithmic}[1]
    \Function{$\sgset = \texttt{offlinePhase}$}{%
      $\*\objfun$, $n$, $b$, $\*c$, $l_{\max}$, $\varepsilon$,
      $\ngp_{\mathrm{refine}}$%
    }
      \State{$\sgset \gets \coarseregsgset{n}{d}{b}$}
      \Comment{initial regular sparse grid}%
      \While{\True}
        \State{$\ngp \gets \setsize{\sgset}$}
        \Comment{number of grid points}%
        \State{%
          Let $(\surplus{\*l_{k'},\*i_{k'}})_{k' = 1, \dotsc, \ngp}$
          satisfy $
            \fa{k = 1, \dotsc, \ngp}{
              \sum_{k'=1}^{\ngp} \vsurplus_{\*l_{k'},\*i_{k'}}
              \bspl{\*l_{k'},\*i_{k'}}{1}(\gp{\*l_k,\*i_k})
              = \*\objfun(\gp{\*l_k,\*i_k})
            }
          $%
        }
        \ForOneLine{$k = 1, \dotsc, \ngp$}{%
          $\beta_k \gets \tr{\*c} \abs{\vsurplus_{\*l,\*i}}$%
        }
        \Comment{combine surpluses to a scalar value}%
        \State{%
          $
            \liset^\ast \gets \{
              k = 1, \dotsc, \ngp \mid
              \ex{\gp{\*l,\*i} \notin \sgset}{
                \gp{\*l_k,\*i_k} \to \gp{\*l,\*i}
              },\,
              \norm[\infty]{\*l_k} < l_{\max},\,
              \abs{\surplus{\*l_k,\*i_k}} > \varepsilon
            \}
          $%
        }
        \IfOneLine{$\liset^\ast = \emptyset$}{\Break{}}
        \Comment{stop when there are not refinable grid points left}%
        \State{%
          Refine $\le \ngp_{\mathrm{refine}}$ of the points
          $\{\gp{\*l_k,\*i_k} \in \sgset \mid k \in \liset^\ast\}$
          with largest $\abs{\beta_k}$%
        }
      \EndWhile{}
    \EndFunction{}
  \end{algorithmic}
  \caption[%
    Generation of spatially adaptive sparse grids for topology optimization%
  ]{%
    Generation of spatially adaptive sparse grids for topology optimization.
    Inputs are
    the objective function $\*f\colon \clint{\*0, \*1} \to \real^{m+1}$
    (combination of Cholesky factors of elasticity tensors and
    micro-cell densities),
    the level $n \ge d$ and boundary parameter $b \in \natz$ of the
    initial regular sparse grid,
    the vector $\*c \in \real^d$ of coefficients with which the
    absolute values of the entries of the surpluses are combined,
    the maximal level $l_{\max} \in \nat$,
    the refinement threshold $\varepsilon \in \posreal$, and
    the number $\ngp_{\mathrm{refine}} \in \nat$ of points to refine
    in each iteration.
    Output is the spatially adaptive sparse grid $\sgset$.%
  }%
  \label{alg:topoOptGridGeneration}%
\end{algorithm}

\paragraph{Parameter bounds}

In the micro-cell models presented in \cref{sec:631models},
extreme micro-cell parameters near zero or one may cause problems
with the resulting elasticity tensors.
For instance, the elasticity tensor entries corresponding to
the 2D cross model are discontinuous near the lines $x_1 = 1$ or $x_2 = 1$
\multicite{Huebner14Mehrdimensionale,Valentin14Hierarchische}.
This is due to the fact the micro-cell is completely filled with material
on these lines,
independent of the other micro-cell parameter that is not one.
Similar issues occur for the other models and the shearing angles.
Hence, we have to restrict the range of the feasible micro-cell parameters,
i.e., the sparse grid points $\*x$ are still defined on the unit hyper-cube
$\clint{\*0, \*1}$,
but the actual micro-cell parameters $\xscaled$ are retrieved by an
affine transformation $\xscaled := \*a + (\*b - \*a) \*x$.
For the models in \cref{sec:631models},
we restrict the bar widths to $\clint{0.01, 0.99}$ and
the shearing angles to $\clint{-0.35\pi, 0.35\pi}$.

\paragraph{Software, algorithms, and domain discretization}

The micro-problems and macro-problems are solved with the \fem
implemented in the \fem software package CFS++ \cite{Kaltenbacher10Advanced}.%
\footnote{%
  \url{http://www.lse.uni-erlangen.de/cfs/}%
}
The micro-problems were discretized by dividing the micro-cells into
$128 \times 128$ elements (models in two dimensions) or
$16 \times 16 \times 16$ elements (models in three dimensions).
Constructing the sparse grids (offline phase) was done via a MATLAB code,
while the evaluations of the interpolants (online phase) were performed
by the sparse grid toolbox \sgpp \cite{Pflueger10Spatially}.%
\footnote{%
  \url{http://sgpp.sparsegrids.org/}%
}
For the solution of the emerging optimization problems,
a sequential quadratic programming method was employed
(see \cref{sec:513gradientBasedConstrained}).



\subsection{Error Sources}
\label{sec:642errorSources}

There are multiple sources that contribute to the numerical error
of our method:

\begin{enumerate}[label=E\arabic*.,ref=E\arabic*,leftmargin=2.7em]
  \item
  \label{item:topoOptErrorMicro}
  Discretization of the micro-problem
  (i.e., the elasticity tensors $\etensor$ are inaccurate)
  
  \item
  \label{item:topoOptErrorInterpolation}
  Sparse grid interpolation
  (i.e., $\etensorintp \not= \etensor$)
  
  \item
  \label{item:topoOptErrorCholesky}
  Reconstruction of elasticity tensors with Cholesky factors
  (i.e., $\etensorcholintp \not= \etensorintp$)
  
  \item
  \label{item:topoOptErrorMacro}
  Discretization of the macro-problem
  (i.e., the compliance $\compliance$ is inaccurate)
  
  \item
  \label{item:topoOptErrorOptimization}
  Optimization
  (i.e., the global minimum found by the optimizer is inaccurate)
  
  \item
  \label{item:topoOptErrorRounding}
  Floating-point rounding errors
  (i.e., arithmetical operations are inaccurate)
\end{enumerate}

\noindent
\ref{item:topoOptErrorRounding}-type errors are always present and
will not be analyzed in this chapter.
Errors of type \ref{item:topoOptErrorMicro} and \ref{item:topoOptErrorMacro}
are intrinsic to the homogenization approach
and will not be discussed here either.
In the remainder of this chapter,
we will focus on the analysis of the errors of types
\ref{item:topoOptErrorInterpolation} and \ref{item:topoOptErrorCholesky}
(\cref{sec:643interpolation})
and of the error of type \ref{item:topoOptErrorOptimization}
(\cref{sec:644optimization}).



\subsection{Interpolation Error}
\label{sec:643interpolation}

\paragraph{Spectral interpolation error measure}

For the interpolation error \ref{item:topoOptErrorInterpolation} and
the Cholesky factorization error \ref{item:topoOptErrorCholesky},
we cannot simply take the absolute value of the difference
of the objective function $\*\objfun\colon \clint{\*0, \*1} \to \real^{m+1}$
and its surrogate $\*\sgintp$, since both are vector-valued.
As the micro-cell density (i.e., $\objfun_{m+1}$)
is not affected by the Cholesky factorization,
we consider in the following only the elasticity tensor
$\etensor\colon \clint{\*0, \*1} \to \real^{6 \times 6}$ and
its surrogate
$\etensorcholintp\colon \clint{\*0, \*1} \to \real^{6 \times 6}$
obtained by Cholesky factorization.
To retrieve a scalar error measure,
we consider the spectral norm
\begin{equation}
  \norm[2]{\etensor(\*x) - \etensorcholintp(\*x)},\quad
  \*x \in \clint{\*0, \*1},
\end{equation}
i.e., the largest absolute eigenvalue of
$\etensor(\*x) - \etensorcholintp(\*x)$.
However, the choice of the norm is arbitrary,
as all matrix norms on $\real^{6 \times 6}$ are equivalent to each other.

\paragraph{Pointwise spectral interpolation error}

\Cref{fig:topoOptInterpolationErrorPointwise}
shows the pointwise spectral interpolation error for the 2D cross model
and the corresponding spatially adaptive sparse grid
generated with the refinement algorithm as explained in
\cref{sec:641methodology}.
The above-mentioned discontinuity of elasticity tensor entries
near $x_1 = 0$ or $x_2 = 0$
is most severe near the corners $\*x \in \{(0, 1), (1, 0)\}$
(cf.\ \cref{fig:cholesky}),
as some entries vanish if one of the micro-cell bars has zero width.
Hence, most points are placed near the singularity corners.

\begin{figure}
  \subcaptionbox{%
    $\norm[2]{\etensor(\*x) - \etensorintp(\*x)}$%
    \label{fig:topoOptInterpolationErrorPointwise_1}%
  }[63mm]{%
    \includegraphics{topoOptInterpolationPointwise_1}%
  }%
  \hspace{3mm}%
  \subcaptionbox{%
    $\norm[2]{\etensor(\*x) - \etensorcholintp(\*x)}$%
    \label{fig:topoOptInterpolationErrorPointwise_2}%
  }[63mm]{%
    \includegraphics{topoOptInterpolationPointwise_2}%
  }%
  \hfill%
  \includegraphics{topoOptInterpolationPointwise_3}%
  \caption[Pointwise spectral interpolation error for the 2D cross model]{%
    Pointwise spectral interpolation error for the 2D cross model and
    cubic B-splines on
    $\ngp = 1320$ spatially adaptive sparse grid points \emph{(dots)} for
    the direct elasticity tensor interpolation \emph{(left)} and
    the Cholesky factor interpolation \emph{(right).}%
  }%
  \label{fig:topoOptInterpolationErrorPointwise}%
\end{figure}

The left plot (\cref{fig:topoOptInterpolationErrorPointwise_1})
shows the spectral interpolation error
$\norm[2]{\etensor(\*x) - \etensorintp(\*x)}$
of the direct elasticity tensor interpolant without Cholesky factorization
(i.e., error \ref{item:topoOptErrorInterpolation}).
The maximum error is \num{1.2e-3},
which is attained near the critical lines $x_1 = 0$ or $x_2 = 0$.
Note that the mean error over the whole domain $\clint{\*0, \*1}$
is only \num{4.5e-5}.
In the right plot (\cref{fig:topoOptInterpolationErrorPointwise_2}),
the picture changes slightly when looking at the spectral interpolation error
$\norm[2]{\etensor(\*x) - \etensorcholintp(\*x)}$
of the elasticity tensor resulting by Cholesky factorization
(i.e., errors \ref{item:topoOptErrorInterpolation} and
\ref{item:topoOptErrorCholesky} combined).
The maximum error becomes \num{3.4e-3},
while the mean error increases to \num{1.1e-4}.
We conclude that the Cholesky factorization leads to an increase
of interpolation errors by only less than half an order of magnitude.

\paragraph{Convergence of spectral interpolation error}

\Cref{fig:topoOptInterpolationErrorBasisFunctions_1} shows
the convergence of the relative $\Ltwo$ spectral interpolation error
\begin{equation}
  \frac{
    \normLtwoscaled{
      \vphantom{\big(}
      \norm[2]{\etensor({\cdot}) - \etensorcholintp({\cdot})}
    }
  }{
    \normLtwoscaled{
      \vphantom{\big(}
      \norm[2]{\etensor({\cdot})}
    }
  }
\end{equation}
and the corresponding error for $\etensorintp({\cdot})$
for the 2D cross model, i.e.,
the $\Ltwo$ norm of the function depicted in
\cref{fig:topoOptInterpolationErrorPointwise}.
Relative errors of \SI{1}{\permille} are already obtained
for $\ngp = 200$ grid points.
Even for higher B-spline degrees $p > 1$,
the order of convergence is only quadratic
due to the singularities of the elasticity tensor.
This slow convergence does not improve for the other
micro-cell models as shown in
\Cref{fig:topoOptInterpolationErrorBasisFunctions_2}.
In the contrary, the convergence decelerates even more
as the number of micro-cell parameters increases.
For the 2D sheared cross and 3D cross models with three parameters,
the spatially adaptive sparse grid with $\ngp = \num{10000}$ grid points
is able to achieve a relative error of around \SIrange{2}{3}{\permille}.
However, for the 2D sheared framed cross and 3D sheared cross models
with five parameters, only errors of about \SI{5}{\percent} are reached
for the same grid size.

\begin{figure}
  \hspace*{5mm}%
  \includegraphics{topoOptInterpolation_3}%
  \hfill%
  \raisebox{0.5mm}{\includegraphics{topoOptInterpolation_4}}%
  \\[2mm]%
  \subcaptionbox{%
    2D cross (different degrees)%
    \label{fig:topoOptInterpolationErrorBasisFunctions_1}%
  }[72mm]{%
    \includegraphics{topoOptInterpolation_1}%
  }%
  \hfill%
  \subcaptionbox{%
    Other micro-cell models ($p = 3$)%
    \label{fig:topoOptInterpolationErrorBasisFunctions_2}%
  }[72mm]{%
    \includegraphics{topoOptInterpolation_2}%
  }%
  \caption[Convergence of relative $L^2$ spectral interpolation errors]{%
    Convergence of relative $\Ltwo$ spectral interpolation errors
    over the increasing number $\ngp$ of spatially adaptive grid points
    (i.e., decreasing threshold $\varepsilon$)
    for the 2D cross model without or with Cholesky factor interpolation
    and different degrees $p$ \emph{(left)} and
    for the other models and cubic degree \emph{(right)}.%
  }%
  \label{fig:topoOptInterpolationErrorBasisFunctions}%
\end{figure}



\subsection{Optimized Structures and Optimization Error}
\label{sec:644optimization}

\dummytext[1]{}

\begin{table}
  \setnumberoftableheaderrows{1}%
  \begin{tabular}{%
    >{\kern\tabcolsep}=l<{\kern5mm}*{6}{+c}<{\kern\tabcolsep}%
  }
    \toprulec
    \headerrow
    Scenario&       2D-C&   2D-FC&  2D-SC&  2D-SFC& 3D-C&   3D-SC\\
    \midrulec
    2D cantilever&  XX.XXX& XX.XXX& XX.XXX& XX.XXX& ---&    ---\\
    2D L shape&     XX.XXX& XX.XXX& XX.XXX& XX.XXX& ---&    ---\\
    \midrulec
    3D cantilever&  ---&    ---&    ---&    ---&    XX.XXX& XX.XXX\\
    3D center load& ---&    ---&    ---&    ---&    XX.XXX& XX.XXX\\
    \bottomrulec
  \end{tabular}
  \caption[Optimal compliance values for different micro-cell models]{%
    Optimal compliance values for the different scenarios
    and micro-cell models (maximum number $\ngpMax = \num{10000}$
    of sparse grid points).
    The columns correspond to the micro-cell models as presented
    in \cref{fig:microCell}:
    2D cross,
    2D framed cross,
    2D shared cross,
    2D shared framed cross,
    3D cross, and
    3D sheared cross.
    The highlighted entries indicate the best choice
    of micro-cell models for a given scenario.%
  }%
  \label{tbl:TODO1}%
\end{table}

\begin{table}
  \setnumberoftableheaderrows{1}%
  \begin{tabular}{%
      >{\kern\tabcolsep}=l<{\kern5mm}*{3}{+c}<{\kern\tabcolsep}%
    }
    \toprulec
    \headerrow
    Scenario&       $p = 1$& $p = 3$& $p = 5$\\
    \midrulec
    2D cantilever&  XX.XXX&  XX.XXX&  XX.XXX\\
    2D L shape&     XX.XXX&  XX.XXX&  XX.XXX\\
    \midrulec
    3D cantilever&  XX.XXX&  XX.XXX&  XX.XXX\\
    3D center load& XX.XXX&  XX.XXX&  XX.XXX\\
    \bottomrulec
  \end{tabular}
  \caption[Optimal compliance values for different B-spline degrees]{%
    Optimal compliance values for the different scenarios
    and B-spline degrees (maximum number $\ngpMax = \num{10000}$
    of sparse grid points).
    The highlighted entries indicate the best choice
    of B-spline degree for a given scenario.%
  }%
  \label{tbl:TODO2}%
\end{table}

\dummytext[12]{}
