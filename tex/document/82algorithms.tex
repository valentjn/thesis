\section{Algorithms}
\label{sec:82algorithms}

\minitoc{83mm}{7}

\noindent
This section gives an overview of the algorithms that
were used to implement the solution process after discretization of
the Bellman equation \eqref{eq:gridBellman}.
In the following, we assume that the probability density functions of the
stochastic variables are known.



\subsection{General Structure}
\label{sec:821generalStructure}

The general approach to solve dynamic portfolio choice models is as follows:
\begin{enumerate}
  \item
  Generation of value function interpolants $\valueintp_t$
  %(\texttt{solveValueFunction}, \cref{alg:financeSolveValueFunction})
  
  \item
  Generation of optimal policy interpolants $\optpolicyintp_t$
  %(\texttt{solvePolicies}, \cref{alg:financeSolvePolicies})
  
  \item
  Post-processing, e.g., Monte Carlo simulation
  %or computation of Euler equation errors
  %(\cref{sec:828postProecessing})
\end{enumerate}
The separation of the solution processes for
the value function interpolants $\valueintp_t$
and the optimal policy interpolants $\optpolicyintp_t$
enables to generate different spatially adaptive sparse grids
for the value function and the optimal policies.
This is useful if the shapes of value function and optimal policies
have different characteristics.

In the following \cref{%
  sec:822solveValueFunction,%
  sec:823optimization,%
  sec:824quadrature,%
  sec:825interpolation,%
  sec:826gridGeneration%
}, we describe the algorithmic details of
\texttt{solveValueFunction} (step 1).
The treatment of the other steps \texttt{solvePolicy} (step 2) and
post-processing (step 3) follows with
\cref{sec:827solvePolicies} and \cref{sec:828postProecessing},
respectively.

%\paragraph{Interpolants}

We track two interpolants $\valueintp[1]_t$ and $\valueintp[p]_t$
for each $t = 0, \dotsc, T$.
The former interpolates the value function data at the grid points
with the hierarchical piecewise linear basis
(used for the surplus-based grid generation),
while the latter interpolates the data with hierarchical B-splines
of higher order $p > 1$.
Each $\valueintp[\ast]_t$ ($\ast \in \{1, p\}$)
additionally stores the grid points $\state_t^{(k)}$
and the optimal policies $\optpolicyintp_t(\state_t^{(k)})$
at the grid points ($k = 1, \dotsc, \ngp_t$).
For simplicity, we do not pass them as separate data
to the algorithms.



\subsection{Solution for the Value Function}
\label{sec:822solveValueFunction}

\paragraph{\texttt{solveValueFunction} algorithm}

\Cref{alg:financeSolveValueFunction} shows \texttt{solveValueFunction},
generating the value function interpolants
$\valueintp[1]_t$ and $\valueintp[p]_t$ ($t = 0, \dotsc, T$).
The algorithm follows a simple optimize--refine--interpolate scheme,
which is presented in \cref{fig:structureSolveValueFunction}:
First, the Bellman equation \eqref{eq:gridBellman} is solved
on an initial sparse grid (\texttt{optimize}).
Then, we \texttt{refine} the grid spatially adaptively.
Finally, the resulting grid data are \texttt{interpolate}d
with hierarchical higher-order B-splines.

\begin{algorithm}
  \begin{algorithmic}[1]
    \Function{%
      $\text{%
        $(\valueintp[p]_t)_{t=0,\dotsc,T}$%
      } = \texttt{solveValueFunction}$%
    }{%
      \hspace*{0mm}%
    }
      \State{$\valueintp[p]_{T+1} \gets \emptyset$}
      \Comment{dummy variable (is not used)}%
      \For{$t = T, T - 1, \dotsc, 0$}
        \State{%
          $\valueintp[1]_t \gets \text{%
            Initial regular sparse grid with no values%
          }$%
        }
        \State{%
          $\valueintp[1]_t \gets
          \texttt{optimize($t$, $\valueintp[1]_t$, $\valueintp[p]_{t+1}$)}$%
        }
        \State{%
          $\valueintp[1]_t \gets
          \texttt{refine($t$, $\valueintp[1]_t$, $\valueintp[p]_{t+1}$)}$%
        }
        \State{%
          $\valueintp[p]_t \gets
          \texttt{interpolate($\valueintp[1]_t$)}$%
        }
      \EndFor{}
    \EndFunction{}
  \end{algorithmic}
  \caption[%
    Generation of value function interpolants (\texttt{solveValueFunction})%
  ]{%
    Generation of value function interpolants.
    The output is the higher-order B-spline interpolant $\valueintp[p]_t$
    for all $t = 0, \dotsc, T$.%
  }%
  \label{alg:financeSolveValueFunction}%
\end{algorithm}

\begin{figure}
  \newcommand*{\getscale}{
    \pgfgettransformentries{%
      \myxscaletmp%
    }{\@tempa}{\@tempa}{\myyscaletmp}{\@tempa}{\@tempa}
    \gdef\myxscale{\myxscaletmp}
  }%
  \newcommand*{\drawcircle}[2][]{
    \node[
      circle,minimum size=60mm,fill=mittelblau!#2,draw=mittelblau,
    ] at (0mm,0mm) (myCircle#2) {};
    \IfStrEq{#1}{noArrow}{}{
      \getscale
      \pgfmathparse{150+20/\myxscale^0.8}
      \let\myin\pgfmathresult
      \draw[->] (myCircle#2) to[
        out=150,in=\myin,looseness=2.3,
      ] node[below,scale=1/\myxscale] {\mytext} (myCircle#2);
    }
  }%
  \newcommand*{\drawtext}[1]{
    \getscale
    \centerarc[
      draw=none,postaction={decorate},decoration={
        text along path,
        text={#1},
        text align=center
      },
    ](0mm,0mm)(180:0:30mm-1em/\myxscale);
  }%
  \newenvironment*{circlescope}{
    \getscale
    %\begin{scope}[scale=0.8,shift={(5mm,-5mm)},transform shape]
    \begin{scope}[
      scale=1-0.15/\myxscale,
      shift={(3mm/\myxscale,-3mm/\myxscale)},
      transform shape,
    ]
  }{
    \end{scope}
  }%
  \begin{tikzpicture}
    \draw[dashed] (0mm,9mm) -- (84mm,9mm);
    \draw[dashed] (-4.3mm,-8mm) -- (71.2mm,-48.2mm);
    \begin{scope}[rotate=-130]
      \node[
        circle,
        minimum size=18mm,
        fill=mittelblau!30,
        draw=mittelblau,
        inner sep=0mm,
      ] at (0mm,0mm) (optimize) {\texttt{optimize}};
      \begin{scope}[
        shift={(40mm/2,40mm*sin(60))},
        transform shape,
      ]
        \begin{scope}[
          rotate=130,
          scale=18mm/60mm+0.15,
          transform shape,
        ]
          \getscale
          \drawcircle[noArrow]{30}
          \drawtext{|\ttfamily|refine}
          \begin{scope}[
            scale=1-0.15/\myxscale,
            shift={(3mm/\myxscale,-3mm/\myxscale)},
            transform shape,
          ]
            \def\mytext{}
            \drawcircle{50}
            \getscale
            \node[scale=1/\myxscale] at (0mm,0mm) {\texttt{optimize}};
          \end{scope}
        \end{scope}
      \end{scope}
      \node[
        circle,
        minimum size=18mm,
        fill=mittelblau!30,
        draw=mittelblau,
        inner sep=0mm,
        text width=15mm,
        align=center,
      ] at (40mm,0mm) (interpolate) {\ttfamily{}inter-\\polate};
      \draw[->] (optimize) to[
        bend left=30,
      ] (myCircle30);
      \draw[->] (myCircle30) to[
        bend left=30,
      ] (interpolate);
      \draw[->] (interpolate) to[
        bend left=30,
      ] node[above,rotate=50] {$t \gets (t - 1)$} (optimize);
    \end{scope}
    \begin{scope}[shift={(84mm,-21mm)}]
      \drawcircle[noArrow]{20}
      \drawtext{|\ttfamily|optimize}
      \begin{circlescope}
        \def\mytext{\hspace*{5mm}$\state_t$}
        \drawcircle{30}
        \drawtext{|\ttfamily|optimizeSinglePoint}
        \begin{circlescope}
          \def\mytext{\hspace*{5mm}$\policy_t$}
          \drawcircle{40}
          \drawtext{|\ttfamily|evalObjFcnGrad}
          \begin{circlescope}
          \def\mytext{\hspace*{5mm}$\stochastic_t$}
            \drawcircle{50}
            \drawtext{|\ttfamily|evalQuadPoint}
            \begin{circlescope}
              \def\mytext{\hspace*{4mm}$s$}
              \drawcircle{60}
              \getscale
              \node[
                text width=30mm,align=center,scale=1/\myxscale,
              ] at (0mm,0mm) {%
                \ttfamily{}evalInterp-\\PartDeriv%
              };
            \end{circlescope}
          \end{circlescope}
        \end{circlescope}
      \end{circlescope}
    \end{scope}
  \end{tikzpicture}%
  \caption[Scheme of the generation of value function interpolants]{%
    Scheme of the generation of value function interpolants
    with \texttt{solveValueFunction}
    (\cref{alg:financeSolveValueFunction}, \emph{left}),
    which repeatedly calls the \texttt{optimize} routine
    (\cref{alg:financeOptimize}, \emph{right}),
    which in turn consists of various sub-functions.
    The function \texttt{optimize} iterates over all state grid points
    $\state_t = \state_t^{(k)}$ ($k = 1, \dotsc, \ngp_t$)
    and calls \texttt{optimizeSinglePoint} for each point.
    The optimization method evaluates the objective function and
    its gradient at a sequence of different policy points $\policy_t$
    to find $\optpolicyintp_t(\state_t^{(k)})$.
    This evaluation (denoted by \texttt{evalObjFcnGrad})
    has to compute the expectation in \cref{eq:gridBellman},
    which is done using a quadrature rule.
    For every quadrature point $\stochastic_t = \stochastic_t^{(j)}$
    ($j = 1, \dotsc, m_{\quadweight}$),
    \texttt{evalQuadPoint} computes the corresponding value of
    the expression in the expectation.
    Finally, \texttt{evalInterpPartDeriv} evaluates the interpolant
    $\valueintp[p]_{t+1}$ and its partial derivatives,
    for which we have to loop over the state dimensions $s = 1, \dotsc, d$.%
  }%
  \label{fig:structureSolveValueFunction}%
\end{figure}

At the beginning of every iteration $t$,
the grid of the piecewise linear interpolant is reset
to an initial, possibly regular sparse grid.
It is also possible to reuse the grid from the
previous iteration $t + 1$.
Nevertheless, the results we then obtain become worse,
likely due to the different characteristics of $\valueintp[1]_t$
for different $t$ (e.g., kinks).

The higher-order B-spline interpolant
$\valueintp[p]_{t+1}$ of the previous iteration $t+1$ is used
for the \rhs of the Bellman equation \eqref{eq:gridBellman},
if $t < T$.
In the first iteration $t = T$,
there is no such interpolant.
However,
%it is not needed anyway, as
the terminal solution $\valuefcn_T$ is usually a known function.



\subsection{Optimization}
\label{sec:823optimization}

\paragraph{\texttt{optimize} algorithm}

The \texttt{optimize} step can be seen in \cref{alg:financeOptimize}.
This algorithm accepts in $\valueintp[1]_t$
a spatially adaptive sparse grid
$\gridset{t}{\sparse}
= \{\state_t^{(k)} \mid k = 1, \dotsc, \ngp_t\}$,
where the function values $\valueintp[1]_t(\state_t^{(k)})$
may already be known for some grid points $\state_t^{(k)}$,
if \texttt{optimize} is called from within \texttt{refine}.
The function \texttt{optimize} computes the missing value function values.
For $t = T$, we assume that the terminal solution
$\valuefcn_T$ can be computed by some function
\texttt{computeKnownTerminalSolution}.%
\footnote{%
  In any case, the terminal solution may be computed as the
  solution of the corresponding single-time optimization problem,
  e.g., $\valuefcn_T(\state_T^{(k)})
  = \max_{\policy_T} \utilityfcn(\consume_T(\state_T^{(k)}, \policy_T))$.%
}
Otherwise, for $t < T$, we solve the Bellman equation
\eqref{eq:gridBellman} by using the higher-order B-spline interpolant
$\valueintp[p]_{t+1}$ of the previous iteration $t + 1$
(\texttt{optimizeSinglePoint}).
The computations for the different $\state_t^{(k)}$ are
independent of each other,
which means that they can be computed in parallel \cite{Horneff16Efficient}.%
\footnote{%
  Such a problem is usually referred to as \term{embarrassingly parallel.}%
}
After generating all missing data,
we update the hierarchical surpluses of the
piecewise linear interpolant $\valueintp[1]_t$
to interpolate the new data at all grid points of $\gridset{t}{\sparse}$.

\begin{algorithm}
  \begin{algorithmic}[1]
    \Function{$\valueintp[1]_t = \texttt{optimize}$}{%
      $t$, $\valueintp[1]_t$, $\valueintp[p]_{t+1}$%
    }
      \State{%
        $(\state_t^{(k)})_{k=1,\dotsc,\ngp_t}
        \gets \text{grid of $\valueintp[1]_t$}$%
      }
      \For{$k = 1, \dotsc, \ngp_t$}
        \If{$\valueintp[1]_t(\state_t^{(k)})$ not previously computed}
          \IfOneLine{$t = T$}{%
            $\valueintp[1]_T(\state_T^{(k)})
            \gets \texttt{computeKnownTerminalSolution($\state_T^{(k)}$)}$%
          }
          \ElseOneLine{%
            $\valueintp[1]_t(\state_t^{(k)})
            \gets \texttt{%
              optimizeSinglePoint(%
                $t$, $\state_t^{(k)}$, $\valueintp[p]_{t+1}$%
              )%
            }$%
          }
        \EndIf{}
      \EndFor{}
      \State{%
        Re-interpolate
        $(\valueintp[1]_t(\state_t^{(k)}))_{k=1,\dotsc,\ngp_t}$
        with piecewise linear functions%
      }
    \EndFunction{}
  \end{algorithmic}
  \caption[Evaluation of the value function (\texttt{optimize})]{%
    Evaluation of the value function at all
    grid points $\state_t^{(k)}$ of $\valueintp[1]_t$
    at which the value function has not been evaluated yet.
    Inputs are
    the time $t$,
    the piecewise linear interpolant $\valueintp[1]_t$
    of the current iteration $t$ (with the underlying sparse grid and
    corresponding function values, possibly unset), and
    the higher-order B-spline interpolant $\valueintp[p]_{t+1}$
    of the previous iteration $t + 1$
    (not used if $t = T$).
    The output is the updated piecewise linear interpolant $\valueintp[1]_t$,
    where all missing function values at grid points have been computed.%
  }%
  \label{alg:financeOptimize}%
\end{algorithm}

\paragraph{Certainty-equivalent transformation}

For utility functions of \crra-type, i.e., of the form
$\utilityfcn(\consume_t) = \consume_t^{1-\riskav}/(1-\riskav)$,
the curvature of the objective function in the Bellman equation
\eqref{eq:gridBellman} can be very high
(depending on the risk aversion parameter $\riskav$),
which may impede convergence of the optimizer.
As a remedy, we transform the value function $\valueintp_t$ with the
\term{certainty-equivalent transformation}
$\valueintp_t \mapsto \cetvalueintp_t
:= ((1 - \riskav) \valueintp_t)^{1/(1-\riskav)}$ if $\riskav > 0$.
\Cref{eq:gridBellman} then becomes
$\cetvalueintp_T(\state_t^{(k)}) = \max_{\policy_T}
\consume_T(\state_T^{(k)}, \policy_T)$ for $t = T$ and
\begin{equation}
  \label{eq:gridBellmanCET}
  \cetvalueintp_t(\state_t^{(k)})
  = \max_{\policy_t} \left(
    \left(
      \consume_t(\state_t^{(k)}, \policy_t)^{1-\riskav} +
      \patience \expectation[t]{
        (
          \cetvalueintp_{t+1}(
            \statefcn_t(\state_t^{(k)}, \policy_t, \stochastic_t)
          )
        )^{1-\riskav}
      }
    \right)^{1/(1-\riskav)}
  \right)
\end{equation}
for $t < T$, since for $\riskav > 0$,
$(\cdot)^{1/(1-\riskav)}$ is strictly monotonously decreasing and
$(1-\riskav) < 0$.
The notation in the remainder of this section
does not distinguish between $\valueintp[\ast]_t$ and $\cetvalueintp[\ast]_t$
and uses $\valueintp[\ast]_t$ for both, if it is not relevant
whether the value function is transformed.
%and hence,
%$(\max ((1-\riskav) (\cdot)))^{1/(1-\riskav)}
%= ((1-\riskav) \min (\cdot))^{1/(1-\riskav)}
%= (1-\riskav)^{1/(1-\riskav)} \max ((\cdot)^{1/(1-\riskav)})$.

%\paragraph{State space transformation}
%
%Depending on the dynamic portfolio lifecycle model at hand,
%it might be beneficial to transform the state space as well
%with a \term{state space transformation}
%$\statetrafofcn_t\colon \clint{\*0, \*1} \to \real^d$,
%for instance, to increase the density of grid points for low
%wealth values $\wealth_t$,
%where the value function might have a high curvature.
%The untransformed states $\state_t^{(k)}$ are then still contained in
%the unit cube $\clint{\*0, \*1}$,
%and the occurrences of $\state_t^{(k)}$ in \cref{eq:gridBellmanCET}
%are replaced with $\statetrafofcn_t(\state_t^{(k)})$.



\subsection{Quadrature}
\label{sec:824quadrature}

We need to approximate the expectation in \cref{eq:gridBellmanCET}
by quadrature,
\begin{equation}
  \label{eq:bellmanQuadrature}
  \expectation[t]{
    (
      \cetvalueintp[p]_{t+1}(
        \statefcn_t(\state_t^{(k)}, \policy_t, \stochastic_t)
      )
    )^{1-\riskav}
  }
  \approx \sum_{j=1}^{m_{\quadweight}} \quadweight_t^{(j)}
  (
    \cetvalueintp[p]_{t+1}(
      \statefcn_t(\state_t^{(k)}, \policy_t, \stochastic_t^{(j)})
    )
  )^{1-\riskav},
\end{equation}
for some weights $\quadweight_t^{(j)} \in \real$ and
nodes $\stochastic_t^{(j)} \in \stochdomain$
($j = 1, \dotsc, m_{\quadweight}$).
Since the stochastic domain $\stochdomain \subset \real^{m_{\stochastic}}$
might be high-dimensional as well,
full grid quadrature rules suffer from the curse of dimensionality.
Therefore, we use sparse grid quadrature rules based
on Gauss--Hermite quadrature
\multicite{Gerstner98Numerical,Horneff16Efficient}.
Note that this sparse grid in the stochastic space $\stochdomain$
is independent of the sparse grid in the state space $\clint{\*0, \*1}$.
However, it would also be feasible to employ Monte-Carlo quadrature,
albeit usually far more expensive.



\subsection{Interpolation and Extrapolation}
\label{sec:825interpolation}

\paragraph{Sparse grid interpolation}

As already mentioned,
$\valueintp[1]_t$ is constructed as the sparse grid interpolant
of the grid data $\state_t^{(k)}$ ($k = 1, \dotsc, \ngp_t$)
using the hierarchical piecewise linear basis.
For $\valueintp[p]_t$,
we use cubic hierarchical weakly fundamental not-a-knot splines
(see \cref{sec:454wfs}).
The not-a-knot boundary conditions help to decrease the
interpolation error (see \cref{sec:541interpolation}),
while the weakly fundamental property eases the hierarchization
complexity by enabling us to use the unidirectional principle
(see \cref{sec:45spatAdaptiveUP,sec:543complexity}).

\paragraph{Extrapolation}

Unfortunately, for many dynamic portfolio choice models,
the state transition is not a function
$\statefcn_t\colon \clint{\*0, \*1} \times \real^{m_{\policy}} \times
\stochdomain \to \clint{\*0, \*1}$,
especially if the state space is actually unbounded.
It may then happen that
$\statefcn_t(\state_t^{(k)}, \policy_t, \stochastic_t^{(j)})
\notin \clint{\*0, \*1}$
for some quadrature nodes
$\stochastic_t^{(j)} \in \stochdomain$ in \eqref{eq:bellmanQuadrature}.
%i.e., $\statefcn_t(\state_t^{(k)}, \policy_t, \stochastic_t^{(j)})$
%is not a feasible state.
Hence, we might not be able to evaluate the value function interpolant
$\valueintp[p]_{t+1}(
  \statefcn_t(\state_t^{(k)}, \policy_t, \stochastic_t^{(j)})
)$, as it is only defined on $\clint{\*0, \*1}$.
Scaling of the domain is not an option due to the dynamic nature of
the problem.
%Modifying infeasible states by cropping them to $\clint{\*0, \*1}$
%is not straightforward, as this introduces new errors and
%the modified state should satisfy the model constraints.

Instead, we extend the interpolant $\valueintp[p]_{t+1}$
to $\real^d$ by an extrapolation method based on Taylor approximation.
First, we crop the evaluation point
$\state_{t+1} \in \real^d \setminus \clint{\*0, \*1}$ to
a point $\state_{t+1}^\mathrm{in} \in \clint{\*0, \*1}$.
The extrapolation type,
which may be \texttt{constant}, \texttt{linear}, and \texttt{quadratic},
determines the degree of the Taylor approximation:
\begin{equation}
  \begin{split}
    \valueintp[p]_{t+1}(\state_{t+1})
    &\approx \valueintp[p]_{t+1}(\state_{t+1}^\mathrm{in}) +
    \tr{(\gradient{\state_{t+1}}{\valueintp[p]_{t+1}}(\state_{t+1}^\mathrm{in}))}
    (\state_{t+1} - \state_{t+1}^\mathrm{in})\\
    &\qquad + \tr{(\state_{t+1} - \state_{t+1}^\mathrm{in})}
    \hessian{\state_{t+1}}{\valueintp[p]_{t+1}}(\state_{t+1}^\mathrm{in})
    (\state_{t+1} - \state_{t+1}^\mathrm{in}),
  \end{split}
\end{equation}
where \texttt{constant} and \texttt{linear}
only use the first summand and first two summands, respectively.
Note that since the hierarchical B-splines enable us to
exactly and efficiently compute
the gradient $\gradient{\state}{\valueintp[p]_{t+1}}$ and
the Hessian $\hessian{\state}{\valueintp[p]_{t+1}}$;
we do not have to approximate the derivatives with finite differences.



\subsection{Grid Generation}
\label{sec:826gridGeneration}

\paragraph{\texttt{refine} algorithm}

\Cref{alg:financeRefine} shows how to generate the spatially adaptive
sparse grid in \texttt{solveValueFunction}
(\cref{alg:financeSolveValueFunction}).
The criterion is the common surplus-based refinement criterion
\cite{Pflueger10Spatially}.
As for the application in topology optimization (see \cref{chap:60topoOpt}),
we use the piecewise linear interpolant for the surplus-based
grid generation,
as the surpluses are easier to compute in the piecewise linear case,
and they are more meaningful
due to the integral representation formula \eqref{eq:surplusIntegral}.
Parameters for \cref{alg:financeRefine} are
the tolerance $\refinetol_t \in \nonnegreal$,
by which the set of grid points to be refined is determined, and
the number $\norefine_t \in \natz$ of refinement iterations.
These parameters depend on the time $t$,
since it might be beneficial to change the adaptivity of the
grid over time.

\begin{algorithm}
  \begin{algorithmic}[1]
    \Function{$\valueintp[1]_t = \texttt{refine}$}{%
      $t$, $\valueintp[1]_t$, $\valueintp[p]_{t+1}$%
    }
      \For{$j = 1, \dotsc, \norefine_t$}
        \State{%
          $\ngp_t
          \gets \text{number of grid points of $\valueintp[1]_t$}$%
        }
        \ForOneLine{$k = 1, \dotsc, \ngp_t$}{%
          $\surplus[(k)]{t}
          \gets \text{%
            surplus of $\state_t^{(k)}$ in $\valueintp[1]_t$%
          }$%
        }
        \State{%
          $K_\mathrm{refine}
          \gets \{k = 1, \dotsc, \ngp_t \mid
          \abs{\surplus[(k)]{t}} \ge \refinetol_t\}$%
        }
        \IfOneLine{$K_\mathrm{refine} = \emptyset$}{\Break}
        \State{%
          Refine all grid points in
          $\{\state_t^{(k)} \mid k \in K_\mathrm{refine}\}$%
        }
        \State{%
          $\valueintp[1]_t \gets
          \texttt{optimize($t$, $\valueintp[1]_t$, $\valueintp[p]_{t+1}$)}$%
        }
      \EndFor{}
    \EndFunction{}
  \end{algorithmic}
  \caption[Refinement of the value function (\texttt{refine})]{%
    In-place refinement of the value function $\valueintp[1]_t$.
    Inputs are
    the time $t$,
    the piecewise linear interpolant $\valueintp[1]_t$
    of the current iteration $t$, and
    the higher-order B-spline interpolant $\valueintp[p]_{t+1}$
    of the previous iteration $t + 1$
    (not used if $t = T$).
    The output is the updated piecewise linear interpolant $\valueintp[1]_t$
    with the refined sparse grid.%
  }%
  \label{alg:financeRefine}%
\end{algorithm}

\paragraph{Gradient grids}

The classical surplus-refinement criterion puts its emphasis
in regions where the mixed second derivative
$\partialderiv[2d]{
  \partialdiff \stateentry_{t,1}^2 \dotsm \partialdiff \stateentry_{t,d}^2
}{
  \valueintp[1]_t
}$
of $\valueintp[1]_t$ has large absolute values, i.e.,
where $\valueintp[1]_t$ has large high-frequency oscillations.
In gradient-based optimization,
it might be advisable to apply this criterion also
to the partial derivatives
$\partialderiv{\partialdiff \stateentry_{t,s}}{\valueintp[1]_t}$
of $\valueintp[1]_t$ ($s = 1, \dotsc, d$),
since the optimizer depends on the accuracy of the gradient.
In this case, we have to track in \cref{alg:financeSolveValueFunction}
additional sparse grid interpolants for every partial derivative
$\partialderiv{\partialdiff \stateentry_{t,s}}{\valueintp[1]_t}$
that is affected by a policy variable.
This possibility is omitted of the algorithms in this section,
as this would complicate their presentation.



\subsection{Solution for Optimal Policies}
\label{sec:827solvePolicies}

\paragraph{\texttt{solvePolicies} algorithm}

After explaining the generation of the value function interpolants
$\valueintp[p]_t$ ($t = 0, \dotsc, T$),
we can move on with step 2 of the general structure of our method
(see \cref{sec:821generalStructure}),
which is the generation of optimal policy interpolants.
The corresponding \cref{alg:financeSolvePolicies} is similar to
\texttt{solveValueFunction} (\cref{alg:financeSolveValueFunction}),
except that it operates on the policy instead of
the value function interpolants.
The functions \texttt{optimize}, \texttt{refine}, and \texttt{interpolate}
have been replaced by corresponding policy versions
\texttt{optimizePolicy}, \texttt{refinePolicy}, and \texttt{interpolatePolicy},
that work very much like their value function counterpart.
\texttt{optimizePolicy} only has to generate new values,
if the initial regular sparse grid for the policies
is not contained in the grid of $\valueintp[p]_t$.
The policy grid is then refined and interpolated independently
of the value function grid.
The iterations are independent of each other,
which means that they can be parallelized.

\begin{algorithm}
  \begin{algorithmic}[1]
    \Function{%
      $\text{%
        $(\optpolicyintp_t)_{t=0,\dotsc,T}$%
      } = \texttt{solvePolicies}$%
    }{%
      $(\valueintp[p]_t)_{t=0,\dotsc,T}$%
    }
      \State{$\valueintp[p]_{T+1} \gets \emptyset$}
      \Comment{dummy variable (is not used)}%
      \For{$t = 0, \dotsc, T$}
        \State{%
          $\optpolicyintp[1]_t \gets \text{%
            Initial regular sparse grid,
            retrieve values from $\valueintp[p]_t$%
          }$%
        }
        \State{%
          $\optpolicyintp[1]_t \gets
          \texttt{%
            optimizePolicy($t$, $\optpolicyintp[1]_t$, $\valueintp[p]_{t+1}$)%
          }$%
        }
        \State{%
          $\optpolicyintp[1]_t \gets
          \texttt{%
            refinePolicy($t$, $\optpolicyintp[1]_t$, $\valueintp[p]_{t+1}$)%
          }$%
        }
        \State{%
          $\optpolicyintp[p]_t \gets
          \texttt{interpolatePolicy($\optpolicyintp[1]_t$)}$%
        }
      \EndFor{}
    \EndFunction{}
  \end{algorithmic}
  \caption[%
    Generation of interpolants for optimal policies (\texttt{solvePolicies})%
  ]{%
    Generation of interpolants for optimal policies.
    The input is the higher-order B-spline interpolant $\valueintp[p]_t$
    of the value function for all $t = 0, \dotsc, T$.
    The output is the higher-order B-spline interpolant $\optpolicyintp[p]_t$
    of the optimal policies for all $t = 0, \dotsc, T$.%
  }%
  \label{alg:financeSolvePolicies}%
\end{algorithm}



\vspace{-0.5em}
\subsection{Post-Processing}
\label{sec:828postProecessing}
\vspace{-0.5em}

\paragraph{Monte Carlo simulation}

There are various ways to
assess if the resulting optimal policy B-spline interpolants
$(\optpolicyintp[p]_t)_{t=0,\dotsc,T}$
are reasonable.
One possibility is a Monte Carlo simulation,
where we calculate the mean optimal policy
{%
  \setlength{\abovedisplayskip}{9pt}%
  \setlength{\belowdisplayskip}{9pt}%
  \begin{equation}
    \optpolicymean_t
    = \frac{1}{m_\mathrm{MC}} \sum_{j=1}^{m_\mathrm{MC}} \policy_{t,(j)}
  \end{equation}%
}
for $m_\mathrm{MC} \in \nat$ individuals.
The optimal policies $\policy_{t,(j)}$ of the individuals
($t = 0, \dotsc, T$ and $j = 1, \dotsc, m_\mathrm{MC}$)
are determined by
\begin{subequations}
  \setlength{\abovedisplayskip}{9pt}%
  \setlength{\belowdisplayskip}{9pt}%
  \begin{align}
    \policy_{t,(j)}
    &:= \optpolicyintp[p]_t(\state_{t,(j)}),\\
    \state_{t,(j)}
    &:= \statefcn_{t-1}(
      \state_{t-1,(j)}, \policy_{t-1,(j)}, \stochastic_{t-1,(j)}
    ),\quad
    t > 0,\qquad
    \state_{t,(0)}
    \sim P_{0,\state},\\
    \stochastic_{t,(j)}
    &\centerhphantom{\sim}{\hspace*{1.6em}} P_{t,\stochastic},
  \end{align}
\end{subequations}
i.e., the initial state $\state_{t,(0)}$ and the
stochastic variables $\stochastic_{t,(j)}$ are samples of random variables.
The Monte Carlo simulation enables us to draw macro-economic conclusions,
e.g., the evolution of the amount of consumption of the average individual
over time.

%\paragraph{Euler equation errors}
%
%\dummytext[1]{}
