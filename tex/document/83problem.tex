\section{Transaction Costs Model}
\label{sec:83problem}

\minitoc[-6mm]{70mm}{4}

\vspace{-1.5em}

\paragraph{Description}

In the \term{transaction costs model} \cite{Schober18Solving},
the individual can invest their money risk-free in bonds
(with a fixed interest rate similar to a bank account)
or in $m_{\stock} \in \nat$ different stocks, which is risk-affected.
Every stock transaction,
i.e., buy $\buy_{t,j}$ or sell $\sell_{t,j}$,
inflicts transaction costs $\tac \buysell_{t,j}$ ($\tac \in \nonnegreal$)
proportional to the amount $\buysell_{t,j}$ that was bought or sold
($j = 1, \dotsc, m_{\stock}$).
The individual only wants to invest a fixed
amount $\wealth_0$ in stocks, i.e., we omit the individual's income.



\subsection{Unnormalized Problem}
\label{sec:831unnormalized}

\paragraph{Consumption and state transition}

In the following,
$\stock_{t,j}$ denotes the fraction of the total wealth $\wealth_t$
that is invested in the $j$-th stock.
We combine these \term{stock fractions} $\stock_{t,j}$
in the vector $\vstock_t := (\stock_{t,1}, \dotsc, \stock_{t,m_{\stock}})$;
similarly, $\vbuysell_t := (\buysell_{t,1}, \dotsc, \buysell_{t,m_{\stock}})$.
Then, the consumption can be computed as a residual variable
(i.e., a variable that can be fully computed from $\state$ and $\policy$
and is thus omitted from $\policy$)
that is given by
\begin{equation}
  \consume_t
  := (1 - \sumfcn(\vstock_t)) \wealth_t - \bond_t -
  (1 + \tac) \sumfcn(\vbuy_t) + (1 - \tac) \sumfcn(\vsell_t),
\end{equation}
where $\sumfcn(\*a) := \tr{\*1} \*a$.
The state transition is computed by adding the returns of the stocks:%
\begin{subequations}%
  \begin{align}
    \wealth_{t+1}
    &:= \bond_t \bondreturn_t +
    %\sum_{j=1}^{m_{\stock}}
    %(\stock_{t,j} \wealth_t + \buy_{t,j} - \sell_{t,j}) \stockreturn_{t,j},\\
    \tr{(\vstock_t \wealth_t + \vbuy_t - \vsell_t)} \vstockreturn_t,\\
    \vstock_{t+1}
    &:= \frac{
      (\vstock_t \wealth_t + \vbuy_t - \vsell_t) \compmult \vstockreturn_t
    }{\wealth_{t+1}},
  \end{align}
\end{subequations}
where $\compmult$ is the component-wise multiplication,
$\bondreturn_t \in \real$ is the bond interest rate, and
$
  \vstockreturn_t
  = (\stockreturn_{t,1}, \dotsc, \stockreturn_{t,m_{\stock}})
  \in \real^{m_{\stock}}
$
is the vector of (stochastic) stock return rates.



\subsection{Normalization}
\label{sec:832normalized}

\paragraph{State transition}

The equations above can be normalized with respect to the wealth
$\wealth_t$:
By setting
$\normconsume_t  := \consume_t/\wealth_t$,
$\normbond_t     := \bond_t   /\wealth_t$, and
$\vnormbuysell_t := \vbuysell_t/\wealth_t$, we obtain
\begin{subequations}
  \label{eq:normalizedTCPStateTransition}
  \begin{align}
    \normconsume_t
    &:= (1 - \sumfcn(\vstock_t)) - \normbond_t -
    (1 + \tac) \sumfcn(\vnormbuy_t) + (1 - \tac) \sumfcn(\vnormsell_t),\\
    \wealthratio_{t+1}
    &:= \normbond_t \bondreturn_t +
    \tr{(\vstock_t + \vnormbuy_t - \vnormsell_t)} \vstockreturn_t,
    \qquad(= \wealth_{t+1}/\wealth_t)\\
    \vstock_{t+1}
    &:= \frac{
      (\vstock_t + \vnormbuy_t - \vnormsell_t) \compmult \vstockreturn_t
    }{\wealthratio_{t+1}},
  \end{align}
\end{subequations}
where both $\normconsume_t$ and $\wealthratio_{t+1}$ are residual
variables that specify \term{normalized consumption} and
\term{wealth ratio,} respectively.
The resulting dynamic portfolio choice model has
the following variables:
\begin{itemize}
  \item
  $\centerhphantom{d}{m_{\stochastic}} = m_{\stock}$
  state variables $\normstate_t$:
  Stock fractions $\stock_{t,1}, \dotsc, \stock_{t,m_{\stock}}$
  
  \item
  $\centerhphantom{m_{\policy}}{m_{\stochastic}} = 2m_{\stock} + 1$
  policy variables $\normpolicy_t$:
  Normalized bonds $\normbond_t$,
  normalized buy amounts $\normbuy_{t,1}, \dotsc, \normbuy_{t,m_{\stock}}$ and
  normalized sell amounts $\normsell_{t,1}, \dotsc, \normsell_{t,m_{\stock}}$
  
  \item
  $m_{\stochastic} = m_{\stock}$
  stochastic variables $\stochastic_t$:
  Stock return rates $\stockreturn_{t,1}, \dotsc, \stockreturn_{t,m_{\stock}}$
\end{itemize}
The state space and policy space constraints are given by
\begin{subequations}
  \label{eq:normalizedTCPConstraints}
  \newcommand*{\centereqline}[1]{%
    \mathclap{\hphantom{\mathrm{(8.99a)}}#1}%
  }%
  \begin{gather}
    \label{eq:normalizedTCPConstraintsShort}
    \centereqline{
      \vstock_t \ge \*0,\qquad
      \sumfcn(\vstock_t) \le 1,\qquad
      \normbond_t \ge 0,\qquad
      \vnormbuysell_t \ge \*0,\qquad
      \vnormsell_t \le \vstock_t,
    }\\
    \label{eq:normalizedTCPConstraintsLong}
    \centereqline{
      \normconsume_{\min} + \normbond_t +
      (1 + \tac) \sumfcn(\vnormbuy_t) - (1 - \tac) \sumfcn(\vnormsell_t)
      \le 1 - \sumfcn(\vstock_t),
    }
  \end{gather}
\end{subequations}
where $\normconsume_{\min} \in \nonnegreal$ is some minimal consumption
that must be maintained.

\paragraph{Bellman equation}

Consequently, the Bellman equation \eqref{eq:gridBellmanCET}
after the certainty-equiva\-lent transformation has to be
normalized as well.
By setting $\normcetvalueintp_t(\state_t^{(k)})
:= \cetvalueintp_t(\state_t^{(k)})/\wealth_t$, we obtain
\begin{subequations}
  \begin{align}
    %\normcetvalueintp_t(\state_t^{(k)})
    %&= \frac{\cetvalueintp_t(\state_t^{(k)})}{\wealth_t}\\
    %&= \max_{\policy_t} \left(
    %  \left(
    %    \left(
    %      \frac{\consume_t(\state_t^{(k)}, \policy_t)}{\wealth_t}
    %    \right)^{1-\riskav} +
    %    \patience \expectation[t]{
    %      \left(
    %        \frac{
    %          \cetvalueintp_{t+1}(
    %            \statefcn_t(\state_t^{(k)}, \policy_t, \stochastic_t)
    %          )
    %        }{
    %          \wealth_t
    %        }
    %      \right)^{1-\riskav}
    %    }
    %  \right)^{1/(1-\riskav)}
    %\right)\\
    &\hphantom{=}\hspace{0.6em} \normcetvalueintp_t(\state_t^{(k)})
    = \wealth_t^{-1} \cetvalueintp_t(\state_t^{(k)})\\
    &= \max_{\policy_t} \left(
      \left(
        \left(
          \wealth_t^{-1} \consume_t(\state_t^{(k)}, \policy_t)
        \right)^{1-\riskav} +
        \patience \expectation[t]{
          \left(
            \wealth_t^{-1} \cetvalueintp_{t+1}(
              \statefcn_t(\state_t^{(k)}, \policy_t, \stochastic_t)
            )
          \right)^{1-\riskav}
        }
      \right)^{1/(1-\riskav)}
    \right)\\
    &= \max_{\normpolicy_t} \left(
      \left(
        \normconsume_t(\state_t^{(k)}, \normpolicy_t)^{1-\riskav} +
        \patience \expectation[t]{
          \bigl(
            \wealthratio_{t+1} \normcetvalueintp_{t+1}(
              \normstatefcn_t(\state_t^{(k)}, \normpolicy_t, \stochastic_t)
            )
          \bigr)^{1-\riskav}
        }
      \right)^{1/(1-\riskav)}
    \right).
  \end{align}
\end{subequations}
This means that compared with the unnormalized Bellman equation
\eqref{eq:gridBellmanCET},
the value function in the expectation has to be multiplied by
the wealth ratio $\wealthratio_{t+1}$ introduced above in
\cref{eq:normalizedTCPStateTransition}.
Since there is no inheritance, the optimal terminal solution
is to sell all stocks and consume everything:
\begin{equation}
  \normcetvalueintp_t(\state_T^{(k)})
  = 1 - \tac \sumfcn(\vstock_t^{(k)}),\quad
  \normbond_T^{\opt}(\state_T^{(k)})
  = 0,\quad
  \vnormbuy[\opt]_T(\state_T^{(k)})
  = \*0,\quad
  \vnormsell[\opt]_T(\state_T^{(k)})
  = \vstock_t^{(k)}.
\end{equation}



\subsection{State Space Cropping}
\label{sec:833cropping}

\paragraph{Sparse grids on non-rectangular domains}

Unfortunately, the constraint $\sumfcn(\vstock_t) \le 1$
from \eqref{eq:normalizedTCPConstraints} limits the feasible state space
region to a proper subset (which is the unit simplex)
of the unit hypercube $\clint{\*0, \*1}$,
which impedes the direct application of sparse grids.
There are three possible remedies:
defining a suitable transformation from the unit hypercube to
the feasible state space,
applying extrapolation techniques as discussed in
\cref{sec:825interpolation}, or
choosing a model-tailored approach to obtain
function values outside the feasible state space.

\paragraph{Virtual selling of stocks}

We choose the third remedy and \term{virtually sell,}
if $\sumfcn(\vstock_t) > 1$,
as many stocks as needed to meet the constraint $\sumfcn(\vstock_t) \le 1$.
We already might need to sell stocks
even if $\sumfcn(\vstock_t)$ is smaller, but close to one
in order to satisfy the minimum consumption requirement
\eqref{eq:normalizedTCPConstraintsLong}.
In detail, we replace $\vstock_t$ by $\normcropfactor \vstock_t$
whenever $\normcropfactor < 1$,
where $\normcropfactor \in \posreal$ is a \term{cropping factor}
that is determined by
\begin{equation}
  \label{eq:virtualSelling}
  \Bigl[
    1 - \tac\, \bigl(
      \sumfcn(\vstock_t) - \sumfcn(\normcropfactor\vstock_t)
    \bigr)
  \Bigr]
  \cdot \bigl(1 - \sumfcn(\normcropfactor\vstock_t)\bigr)
  = \normconsume_{\min}.
\end{equation}
Here, $\bigl(\sumfcn(\vstock_t) - \sumfcn(\normcropfactor\vstock_t)\bigr)$
is the amount of virtually sold stocks.
Consequently, the term in square brackets is the fraction of wealth
that is still available after deducting the induced transaction costs.
The product of this term with
$\bigl(1 - \sumfcn(\normcropfactor\vstock_t)\bigr)$
is the fraction of wealth that can be consumed after the virtual selling,
which needs to be at least $\normconsume_{\min}$.
Solving \cref{eq:virtualSelling} for $\normcropfactor$ and
choosing the positive solution, we finally obtain
\begin{equation}
  \newcommand*{\sumX}{\sumfcn(\vstock_t)}
  \newcommand*{\cMin}{\normconsume_{\min}}
  \normcropfactor
  := \frac{
    \tac\, \bigl(1 + \sumX\bigr) - 1 +
    \sqrt{
      \tac^2\, \bigl(1 - \sumX\bigr)^2
      - 2 \tac\, \bigl(2 \cMin - 1 + \sumX\bigr) + 1
    }
  }{
    2 \tac \sumX
  }.
\end{equation}



\subsection{Euler Equation Errors}
\label{sec:834eulerErrors}

In the following, we abbreviate
the normalized and certainty-equivalent-transformed value function interpolant
$
  \normcetvalueintp_t
  = \normcetvalueintp_t(\normstate_t)
$,
its gradient
$
  \gradient{\normstate_t}{\normcetvalueintp_t}
  = \gradient{\normstate_t}{\normcetvalueintp_t}(\normstate_t)
$,
the interpolated optimal policy
$
  \optnormpolicyintp_t
  = \optnormpolicyintp_t(\normstate_t)
$,
the state transition function
$
  \statefcn_t
  = \statefcn_t(\normstate_t, \optnormpolicyintp_t, \stochastic_t)
$,
the wealth ratio
$
  \wealthratio_{t+1}
  = \wealthratio_{t+1}(
    \normstate_t, \optnormpolicyintp_t, \stochastic_t
  )
$, and
the consumption
$
  \normconsume_t
  = \normconsume_t(\normstate_t, \optnormpolicyintp_t)
$.

\begin{equation}
  \patience \bondreturn_t
  \cdot \expectationsign[t]\Bigl[
    \bigl(
      \normcetvalueintp_t
      - \tr{(\gradient{\normstate_t}{\normcetvalueintp_t})}
      \statefcn_t
    \bigr) \cdot
    \bigl(
      \wealthratio_{t+1}\; \normcetvalueintp_t
    \bigr)^{-\riskav}
  \Bigr]
  = \normconsume_t^{-\riskav}
\end{equation}

\begin{equation}
  \error_t(\normstate_t)
  := \Bigl|
    1 - \Bigl(
      \patience \bondreturn_t \normconsume_t^{\riskav}
      \cdot \expectationsign[t]\Bigl[
        \bigl(
          \normcetvalueintp_t
          - \tr{(\gradient{\normstate_t}{\normcetvalueintp_t})}
          \statefcn_t
        \bigr) \cdot
        \bigl(
          \wealthratio_{t+1}\; \normcetvalueintp_t
        \bigr)^{-\riskav}
      \Bigr]
    \Bigr)^{-1/\riskav}
  \Bigr|,
\end{equation}

%\begin{equation}
%  \begin{split}
%    \patience \bondreturn_t
%    \cdot \expectationsign[t]\Bigl[
%      &\bigl(
%        \normcetvalueintp_t%(\normstate_t)
%        - \tr{
%          (
%            \gradient{\normstate_t}{\normcetvalueintp_t}%(\normstate_t)
%          )
%        }
%        \statefcn_t%(\normstate_t, \optnormpolicyfcn_t, \stochastic_t)
%      \bigr)\\[-1mm]
%      &{} \cdot
%      \bigl(
%        \wealthratio_{t+1}%(
%        %  \normstate_t, \optnormpolicyfcn_t, \stochastic_t
%        %)
%        \;
%        \normcetvalueintp_t%(\normstate_t)
%      \bigr)^{-\riskav}
%    \Bigr]
%    = \normconsume_t%(\normstate_t, \optnormpolicyfcn_t)
%    ^{-\riskav}
%  \end{split}
%\end{equation}
%
%\begin{equation}
%  \begin{split}
%    \error_t(\normstate_t)
%    := \Bigl|
%      1 - \Bigl(
%        \patience \bondreturn_t
%        \cdot \normconsume_t%(\normstate_t, \optnormpolicyfcn_t)
%        ^{\riskav}
%        \cdot \expectationsign[t]\Bigl[
%          &\bigl(
%            \normcetvalueintp_t%(\normstate_t)
%            - \tr{
%              (
%                \gradient{\normstate_t}{\normcetvalueintp_t}%(\normstate_t)
%              )
%            }
%            \statefcn_t%(\normstate_t, \optnormpolicyfcn_t, \stochastic_t)
%          \bigr)\\[-2mm]
%          &{} \cdot
%          \bigl(
%            \wealthratio_{t+1}%(
%            %  \normstate_t, \optnormpolicyfcn_t, \stochastic_t
%            %)
%            \;
%            \normcetvalueintp_t%(\normstate_t)
%          \bigr)^{-\riskav}
%        \Bigr]
%      \Bigr)^{-1/\riskav}
%    \Bigr|,
%  \end{split}
%\end{equation}

\dummytext[3]{}
