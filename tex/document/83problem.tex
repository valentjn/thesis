\section{Transaction Costs Model}
\label{sec:83problem}

\minitoc{70mm}{3}

\noindent
This section defines the transaction costs model as an example problem for
dynamic portfolio choice models.



\subsection{Unnormalized Problem}

\paragraph{Description}

In the \term{transaction costs model} \cite{Schober18Solving},
the individual can invest their money risk-free in bonds
(with a fixed interest rate similar to a bank account)
or in $m_{\stock} \in \nat$ different stocks, which is risk-affected.
Every stock transaction,
i.e., buy $\buy_{t,m_j}$ or sell $\sell_{t,m_j}$,
inflicts transaction costs $\tac \buysell_{t,m_j}$ ($\tac \in \nonnegreal$)
proportional to the amount $\buysell_{t,m_j}$ that was bought or sold.
The individual only wants to invest a fixed
amount $\wealth_0$ in stocks, i.e., we omit the individual's income.

\paragraph{Consumption and state transition}

If $\stock_{t,j}$ denotes the fraction of the total wealth $\wealth_t$
that is invested in the $j$-th stock,
then the consumption can be computed as a residual variable
(i.e., a variable that can be fully computed from $\state$ and $\policy$
and is thus omitted from $\policy$)
that is given by
\begin{equation}
  \consume_t
  := \paren*{1 - \sum_{j=1}^{m_{\stock}} \stock_{t,j}} \wealth_t - \bond_t -
  (1 + \tac) \sum_{j=1}^{m_{\stock}} \buy_{t,j} +
  (1 - \tac) \sum_{j=1}^{m_{\stock}} \sell_{t,j}.
\end{equation}
The state transition is computed by adding the returns of the stocks:
\begin{subequations}
  \begin{align}
    \wealth_{t+1}
    &:= \bond_t \bondreturn_t + \sum_{j=1}^{m_{\stock}}
    (\stock_{t,j} \wealth_t + \buy_{t,j} - \sell_{t,j}) \stockreturn_{t,j},\\
    \stock_{t+1,j}
    &:= \frac{
      (\stock_{t,j} \wealth_t + \buy_{t,j} - \sell_{t,j}) \stockreturn_{t,j},
    }{\wealth_{t+1}},\quad
    j = 1, \dotsc, m_{\stock},
  \end{align}
\end{subequations}
where $\bondreturn_t \in \real$ is the bond interest rate and
$\stockreturn_{t,j} \in \real$ is the return of the $j$-th stock
($j = 1, \dotsc, m_{\stock}$).



\subsection{Normalization}

\paragraph{State transition}

The equations above can be normalized with respect to the wealth
$\wealth_t$:
By setting
$\normconsume_t := \consume_t/\wealth_t$,
$\normbond_t    := \bond_t   /\wealth_t$, and
$\normbuysell_t := \buysell_t/\wealth_t$, we obtain
\begin{subequations}
  \label{eq:normalizedTCPStateTransition}
  \begin{align}
    \normconsume_t
    &:= \paren*{1 - \sum_{j=1}^{m_{\stock}} \stock_{t,j}} - \normbond_t -
    (1 + \tac) \sum_{j=1}^{m_{\stock}} \normbuy_{t,j} +
    (1 - \tac) \sum_{j=1}^{m_{\stock}} \normsell_{t,j},\\
    \wealthratio_{t+1}
    &:= \normbond_t \bondreturn_t + \sum_{j=1}^{m_{\stock}}
    (\stock_{t,j} + \normbuy_{t,j} - \normsell_{t,j}) \stockreturn_{t,j},
    \qquad\left(= \frac{\wealth_{t+1}}{\wealth_t}\right)\\
    \stock_{t+1,j}
    &:= \frac{
      (\stock_{t,j} + \normbuy_{t,j} - \normsell_{t,j}) \stockreturn_{t,j},
    }{\wealthratio_{t+1}},\quad
    j = 1, \dotsc, m_{\stock},
  \end{align}
\end{subequations}
where both $\normconsume_t$ and $\wealthratio_{t+1}$ are residual
variables that specify \term{normalized consumption} and
\term{wealth ratio,} respectively.
%\newcommand*{\normconst}{\kappa}
%\begin{gather}
%  \normconst
%  := \frac{
%    2\tac\Sigma
%  }{
%    \tac + \tac \Sigma - 1 + \sqrt{
%      1 -
%      4 \consume_{\min} \tac +
%      2 \tac +
%      \tac^2 -
%      2 \tac^2 \Sigma +
%      \Sigma^2 \tac^2 -
%      2 \Sigma \tac
%    }
%  },\\
%  \normconst
%  := \frac{
%    2 (\tac\Sigma)
%  }{
%    -(1 - \tac - \tac \Sigma) + \sqrt{
%      (1 - \tac - \tac \Sigma)^2 -
%      4 (\tac\Sigma) (\tac + (\consume_{\min} - 1)/\Sigma)
%    }
%  },\\
%  (\tac\Sigma) \frac{1}{\normconst^2} +
%  (1 - \tac - \tac \Sigma) \frac{1}{\normconst} +
%  (\tac + (\consume_{\min} - 1)/\Sigma) \ge 0,\\
%  \hat{\stock}_{t,j}
%  \to \frac{\hat{\stock}_{t,j}}{\max(\normconst, 1)}
%\end{gather}
The resulting dynamic portfolio choice model has
the following variables:
\begin{itemize}
  \item
  $\centerhphantom{d}{m_{\stochastic}} = m_{\stock}$
  state variables $\normstate_t$:
  Stock fractions $\stock_{t,1}, \dotsc, \stock_{t,m_{\stock}}$
  
  \item
  $\centerhphantom{m_{\policy}}{m_{\stochastic}} = 2m_{\stock} + 1$
  policy variables $\normpolicy_t$:
  Normalized bonds $\normbond_t$,
  normalized buy amounts $\normbuy_{t,1}, \dotsc, \normbuy_{t,m_{\stock}}$ and
  normalized sell amounts $\normsell_{t,1}, \dotsc, \normsell_{t,m_{\stock}}$
  
  \item
  $m_{\stochastic} = m_{\stock}$
  stochastic variables $\stochastic_t$:
  Stock return rates $\stockreturn_{t,1}, \dotsc, \stockreturn_{t,m_{\stock}}$
\end{itemize}
The state space and policy space constraints are given by
\begin{subequations}
  \begin{gather}
    \stock_{t,j} \ge 0,\qquad
    \sum_{j=1}^{m_{\stock}} \stock_{t,j} \le 1,\\[-0.4em]
    \normconsume_{\min} + \normbond_t +
    (1 + \tac) \sum_{j=1}^{m_{\stock}} \normbuy_{t,j} -
    (1 - \tac) \sum_{j=1}^{m_{\stock}} \normsell_{t,j}
    \le 1 - \sum_{j=1}^{m_{\stock}} \stock_{t,j},\\
    \normbond_t \ge 0,\qquad
    \normbuysell_{t,j} \ge 0,\qquad
    \normsell_{t,j} \le \stock_{t,j},\qquad
    j = 1, \dotsc, m_{\stock},
  \end{gather}
\end{subequations}
where $\normconsume_{\min} \in \nonnegreal$ is some minimal consumption
that must be maintained.

\paragraph{Bellman equation}

Consequently, the Bellman equation \eqref{eq:gridBellmanCET}
after the certainty-equiva\-lent transformation has to be
normalized as well.
By setting $\normcetvalueintp_t(\state_t^{(k)})
:= \cetvalueintp_t(\state_t^{(k)})/\wealth_t$, we obtain
\begin{subequations}
  \begin{align}
    %\normcetvalueintp_t(\state_t^{(k)})
    %&= \frac{\cetvalueintp_t(\state_t^{(k)})}{\wealth_t}\\
    %&= \max_{\policy_t} \left(
    %  \left(
    %    \left(
    %      \frac{\consume_t(\state_t^{(k)}, \policy_t)}{\wealth_t}
    %    \right)^{1-\riskav} +
    %    \patience \expectation[t]{
    %      \left(
    %        \frac{
    %          \cetvalueintp_{t+1}(
    %            \statefcn_t(\state_t^{(k)}, \policy_t, \stochastic_t)
    %          )
    %        }{
    %          \wealth_t
    %        }
    %      \right)^{1-\riskav}
    %    }
    %  \right)^{1/(1-\riskav)}
    %\right)\\
    &\hphantom{=}\hspace{0.6em} \normcetvalueintp_t(\state_t^{(k)})\\
    &= \wealth_t^{-1} \cetvalueintp_t(\state_t^{(k)})\\
    &= \max_{\policy_t} \left(
      \left(
        \left(
          \wealth_t^{-1} \consume_t(\state_t^{(k)}, \policy_t)
        \right)^{1-\riskav} +
        \patience \expectation[t]{
          \left(
            \wealth_t^{-1} \cetvalueintp_{t+1}(
              \statefcn_t(\state_t^{(k)}, \policy_t, \stochastic_t)
            )
          \right)^{1-\riskav}
        }
      \right)^{1/(1-\riskav)}
    \right)\\
    &= \max_{\policy_t} \left(
      \left(
        \normconsume_t(\state_t^{(k)}, \policy_t)^{1-\riskav} +
        \patience \expectation[t]{
          (
            \wealthratio_{t+1} \normcetvalueintp_{t+1}(
              \statefcn_t(\state_t^{(k)}, \policy_t, \stochastic_t)
            )
          )^{1-\riskav}
        }
      \right)^{1/(1-\riskav)}
    \right).
  \end{align}
\end{subequations}
This means that compared with the unnormalized Bellman equation
\eqref{eq:gridBellmanCET},
the value function in the expectation has to be multiplied by
the wealth ratio $\wealthratio_{t+1}$ introduced above in
\cref{eq:normalizedTCPStateTransition}.
Since there is no inheritance, the optimal terminal solution
is to sell all stocks and consume everything:
\begin{subequations}
  \begin{gather}
    \normcetvalueintp_t(\state_T^{(k)})
    = 1 - \tac \sum_{j=1}^{m_{\stock}} \stock_{t,j}^{(k)},\\
    \normbond_T^{\opt}(\state_T^{(k)})
    = 0,\quad
    \normbuy[\opt]_{T,j}(\state_T^{(k)})
    = 0,\quad
    \normsell[\opt]_{T,j}(\state_T^{(k)})
    = \stock_{t,j}^{(k)},\quad
    j = 1, \dotsc, m_{\stock}.
  \end{gather}
\end{subequations}
