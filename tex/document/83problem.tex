\section{Transaction Costs Model}
\label{sec:83problem}

\minitoc[-6mm]{70mm}{4}

\vspace{-1.5em}

\paragraph{Description}

In the \term{transaction costs model} \cite{Schober18Solving},
the individual can invest their money risk-free in bonds
(with a fixed interest rate similar to a bank account)
or in $m_{\vstock} \in \nat$ different stocks, which is risk-affected.
Every stock transaction,
i.e., buy $\buy_{t,j}$ or sell $\sell_{t,j}$,
inflicts transaction costs $\tac \buysell_{t,j}$ ($\tac \in \nonnegreal$)
proportional to the amount $\buysell_{t,j}$ that was bought or sold
($j = 1, \dotsc, m_{\vstock}$).
The individual only wants to invest a fixed
amount $\wealth_0$ in stocks, i.e., we omit the individual's income.



\subsection{Unnormalized Problem}
\label{sec:831unnormalized}

\paragraph{Consumption and state transition}

In the following,
$\stock_{t,j}$ denotes the fraction of the total wealth $\wealth_t$
that is invested in the $j$-th stock.
We combine these \term{stock fractions} $\stock_{t,j}$
in the vector $\vstock_t := (\stock_{t,1}, \dotsc, \stock_{t,m_{\vstock}})$;
similarly, $\vbuysell_t := (\buysell_{t,1}, \dotsc, \buysell_{t,m_{\vstock}})$.
Then, the consumption can be computed as a residual variable
(i.e., a variable that can be fully computed from $\state$ and $\policy$
and is thus omitted from $\policy$)
that is given by
\begin{equation}
  \consume_t
  := (1 - \sumfcn(\vstock_t)) \wealth_t - \bond_t -
  (1 + \tac) \sumfcn(\vbuy_t) + (1 - \tac) \sumfcn(\vsell_t),
\end{equation}
where $\sumfcn(\*a) := \tr{\*1} \*a$.
The state transition is computed by adding the returns of the stocks:%
\begin{equation}
  \wealth_{t+1}
  := \bond_t \bondreturn_t +
  %\sum_{j=1}^{m_{\vstock}}
  %(\stock_{t,j} \wealth_t + \buy_{t,j} - \sell_{t,j}) \stockreturn_{t,j},\\
  \tr{(\vstock_t \wealth_t + \vbuy_t - \vsell_t)} \vstockreturn_t,\qquad
  \vstock_{t+1}
  := \frac{
    (\vstock_t \wealth_t + \vbuy_t - \vsell_t) \compmult \vstockreturn_t
  }{\wealth_{t+1}},
\end{equation}
where $\bondreturn_t \in \real$ is the bond interest rate,
$
  \vstockreturn_t
  = (\stockreturn_{t,1}, \dotsc, \stockreturn_{t,m_{\vstock}})
  \in \real^{m_{\vstock}}
$
is the vector of (stochastic) stock return rates, and
$\compmult$ is the component-wise multiplication.



\subsection{Normalization}
\label{sec:832normalized}

\paragraph{State transition}

The equations above can be normalized with respect to the wealth
$\wealth_t$:
By setting
$\normconsume_t  := \consume_t/\wealth_t$,
$\normbond_t     := \bond_t   /\wealth_t$, and
$\vnormbuysell_t := \vbuysell_t/\wealth_t$, we obtain
\begin{subequations}
  \label{eq:normalizedTCPStateTransition}
  \begin{align}
    \normconsume_t
    &:= (1 - \sumfcn(\vstock_t)) - \normbond_t -
    (1 + \tac) \sumfcn(\vnormbuy_t) + (1 - \tac) \sumfcn(\vnormsell_t),\\
    \wealthratio_{t+1}
    &:= \normbond_t \bondreturn_t +
    \tr{(\vstock_t + \vnormbuy_t - \vnormsell_t)} \vstockreturn_t,
    \qquad(= \wealth_{t+1}/\wealth_t)\\
    \vstock_{t+1}
    &:= \frac{
      (\vstock_t + \vnormbuy_t - \vnormsell_t) \compmult \vstockreturn_t
    }{\wealthratio_{t+1}},
  \end{align}
\end{subequations}
where both $\normconsume_t$ and $\wealthratio_{t+1}$ are residual
variables that specify \term{normalized consumption} and
\term{wealth ratio,} respectively.
The resulting dynamic portfolio choice model has
the following variables:
\begin{itemize}
  \item
  $\centerhphantom{d}{m_{\stochastic}} = m_{\vstock}$
  state variables $\normstate_t$:
  Stock fractions $\stock_{t,1}, \dotsc, \stock_{t,m_{\vstock}}$
  
  \item
  $\centerhphantom{m_{\policy}}{m_{\stochastic}} = 2m_{\vstock} + 1$
  policy variables $\normpolicy_t$:
  Normalized bonds $\normbond_t$,
  normalized buy amounts $\normbuy_{t,1}, \dotsc, \normbuy_{t,m_{\vstock}}$ and
  normalized sell amounts $\normsell_{t,1}, \dotsc, \normsell_{t,m_{\vstock}}$
  
  \item
  $m_{\stochastic} = m_{\vstock}$
  stochastic variables $\stochastic_t$:
  Stock return rates $\stockreturn_{t,1}, \dotsc, \stockreturn_{t,m_{\vstock}}$
\end{itemize}
The state space and policy space constraints are given by
\begin{subequations}
  \label{eq:normalizedTCPConstraints}
  \newcommand*{\centereqline}[1]{%
    \mathclap{\hphantom{\mathrm{(8.99a)}}#1}%
  }%
  \begin{gather}
    \label{eq:normalizedTCPConstraintsShort}
    \centereqline{
      \vstock_t \ge \*0,\qquad
      \sumfcn(\vstock_t) \le 1,\qquad
      \normbond_t \ge 0,\qquad
      \vnormbuysell_t \ge \*0,\qquad
      \vnormsell_t \le \vstock_t,
    }\\
    \label{eq:normalizedTCPConstraintsLong}
    \centereqline{
      \normconsume_{\min} + \normbond_t +
      (1 + \tac) \sumfcn(\vnormbuy_t) - (1 - \tac) \sumfcn(\vnormsell_t)
      \le 1 - \sumfcn(\vstock_t),
    }
  \end{gather}
\end{subequations}
where $\normconsume_{\min} \in \nonnegreal$ is some minimal consumption
that must be maintained.

\paragraph{Bellman equation}

Consequently, the Bellman equation \eqref{eq:gridBellmanCET}
after the certainty-equiva\-lent transformation has to be
normalized as well.
By setting $\normcetvalueintp_t(\state_t^{(k)})
:= \cetvalueintp_t(\state_t^{(k)})/\wealth_t$, we obtain
\begin{subequations}
  \begin{align}
    %\normcetvalueintp_t(\state_t^{(k)})
    %&= \frac{\cetvalueintp_t(\state_t^{(k)})}{\wealth_t}\\
    %&= \max_{\policy_t} \left(
    %  \left(
    %    \left(
    %      \frac{\consume_t(\state_t^{(k)}, \policy_t)}{\wealth_t}
    %    \right)^{1-\riskav} +
    %    \patience \expectation[t]{
    %      \left(
    %        \frac{
    %          \cetvalueintp_{t+1}(
    %            \statefcn_t(\state_t^{(k)}, \policy_t, \stochastic_t)
    %          )
    %        }{
    %          \wealth_t
    %        }
    %      \right)^{1-\riskav}
    %    }
    %  \right)^{1/(1-\riskav)}
    %\right)\\
    &\hphantom{=}\hspace{0.6em} \normcetvalueintp_t(\state_t^{(k)})
    = \wealth_t^{-1} \cetvalueintp_t(\state_t^{(k)})\\
    &= \max_{\policy_t} \left(
      \left(
        \left(
          \wealth_t^{-1} \consume_t(\state_t^{(k)}, \policy_t)
        \right)^{1-\riskav} +
        \patience \expectation[t]{
          \left(
            \wealth_t^{-1} \cetvalueintp_{t+1}(
              \statefcn_t(\state_t^{(k)}, \policy_t, \stochastic_t)
            )
          \right)^{1-\riskav}
        }
      \right)^{1/(1-\riskav)}
    \right)\\
    \label{eq:normalizedTCPBellmanEquation}
    &= \max_{\normpolicy_t} \left(
      \left(
        \normconsume_t(\state_t^{(k)}, \normpolicy_t)^{1-\riskav} +
        \patience \expectation[t]{
          \bigl(
            \wealthratio_{t+1} \normcetvalueintp_{t+1}(
              \normstatefcn_t(\state_t^{(k)}, \normpolicy_t, \stochastic_t)
            )
          \bigr)^{1-\riskav}
        }
      \right)^{1/(1-\riskav)}
    \right).
  \end{align}
\end{subequations}
This means that compared with %the unnormalized Bellman equation
\eqref{eq:gridBellmanCET},
the value function in the expectation has to be multiplied by
the wealth ratio $\wealthratio_{t+1}$ introduced above in
\eqref{eq:normalizedTCPStateTransition}.
Since there is no inheritance, the optimal terminal solution
is to sell all stocks and consume everything:
\begin{equation}
  \normcetvalueintp_t(\state_T^{(k)})
  = 1 - \tac \sumfcn(\vstock_t^{(k)}),\quad
  \normbond_T^{\opt}(\state_T^{(k)})
  = 0,\quad
  \vnormbuy[\opt]_T(\state_T^{(k)})
  = \*0,\quad
  \vnormsell[\opt]_T(\state_T^{(k)})
  = \vstock_t^{(k)}.
\end{equation}



\subsection{State Space Cropping}
\label{sec:833cropping}

\paragraph{Sparse grids on non-rectangular domains}

Unfortunately, the constraint $\sumfcn(\vstock_t) \le 1$
from \eqref{eq:normalizedTCPConstraints} limits the feasible state space
region to a proper subset (which is the unit simplex)
of the unit hypercube $\clint{\*0, \*1}$,
which impedes the direct application of sparse grids.
There are three possible remedies:
transforming the unit hypercube to the feasible state space,
applying extrapolation techniques as discussed in
\cref{sec:825interpolation}, or
choosing a model-tailored approach to obtain
function values outside the feasible state space.

\paragraph{Virtual selling of stocks}

We choose the third remedy and \term{virtually sell,}
if $\sumfcn(\vstock_t) > 1$,
as many stocks as needed to meet the constraint $\sumfcn(\vstock_t) \le 1$.
We already might need to sell stocks
even if $\sumfcn(\vstock_t)$ is smaller, but close to one
in order to satisfy the minimum consumption requirement
\eqref{eq:normalizedTCPConstraintsLong}.
In detail, we replace $\vstock_t$ by $\normcropfactor \vstock_t$
whenever $\normcropfactor < 1$,
where $\normcropfactor \in \posreal$ is a \term{cropping factor}
that is determined by
\begin{equation}
  \label{eq:virtualSelling}
  \Bigl[
    1 - \tac\, \bigl(
      \sumfcn(\vstock_t) - \sumfcn(\normcropfactor\vstock_t)
    \bigr)
  \Bigr]
  \cdot \bigl(1 - \sumfcn(\normcropfactor\vstock_t)\bigr)
  = \normconsume_{\min}.
\end{equation}
Here, $\bigl(\sumfcn(\vstock_t) - \sumfcn(\normcropfactor\vstock_t)\bigr)$
is the amount of virtually sold stocks.
Consequently, the term in square brackets is the fraction of wealth
that is still available after deducting the induced transaction costs.
The product of this term with
$\bigl(1 - \sumfcn(\normcropfactor\vstock_t)\bigr)$
is the fraction of wealth that can be consumed after the virtual selling,
which needs to be at least $\normconsume_{\min}$.
Solving \cref{eq:virtualSelling} for $\normcropfactor$ and
choosing the positive solution, we finally obtain
\begin{equation}
  \newcommand*{\sumX}{\sumfcn(\vstock_t)}
  \newcommand*{\cMin}{\normconsume_{\min}}
  \normcropfactor
  := \frac{
    \tac\, \bigl(1 + \sumX\bigr) - 1 +
    \sqrt{
      \tac^2\, \bigl(1 - \sumX\bigr)^2
      - 2 \tac\, \bigl(2 \cMin - 1 + \sumX\bigr) + 1
    }
  }{
    2 \tac \sumX
  }.
\end{equation}



\subsection{Euler Equation Errors}
\label{sec:834eulerErrors}

\paragraph{Motivation}

Due to the curse of dimensionality,
reasonably accurate full grid reference solutions
of the transaction costs problem can only be computed
if the number $m_{\vstock}$ of stocks is small.
Mainly (but not only) in higher-dimensional settings,
a different means of assessing the
quality of sparse grid solutions is desirable.
We use Euler equation errors to measure the deviation in
the first-order optimality conditions.

\paragraph{Derivation}

In the following, we fix the state $\normstate_t \in \clint{\*0, \*1}$
for which we want to compute the Euler equation error.
We abbreviate
the
%normalized and certainty-equivalent-transformed
value function interpolant
$
  \normcetvalueintp_t
  := \normcetvalueintp_t(\normstate_t)
$,
the state transition function
$
  \normstatefcn_t
  := \normstatefcn_t(\normstate_t, \normpolicy_t, \stochastic_t)
$,
the wealth ratio
$
  \wealthratio_{t+1}
  := \wealthratio_{t+1}(
    \normstate_t, \normpolicy_t, \stochastic_t
  )
$, and
the consumption
$
  \normconsume_t
  := \normconsume_t(\normstate_t, \normpolicy_t)
$.
The Lagrangian of the optimization problem corresponding
to the Bellman equation \eqref{eq:normalizedTCPBellmanEquation}
of the normalized transaction costs problem
vanishes with respect to the problem's constraints
\eqref{eq:normalizedTCPConstraints} is given by
{%
  \setlength{\abovedisplayskip}{9pt}%
  \setlength{\belowdisplayskip}{9pt}%
  \begin{equation}
    \begin{split}
      \lagrangian_t(\normstate_t, \normpolicy_t, \*\multiplier)
      &:= \left(
        (\normconsume_t)^{1-\riskav} +
        \patience \expectation[t]{
          \bigl(
            \wealthratio_{t+1}\;
            \normcetvalueintp_{t+1}(\normstatefcn_t)
          \bigr)^{1-\riskav}
        }
      \right)^{1/(1-\riskav)}\\
      &\hspace*{6mm} {}
      - \multiplier_1 \normbond_t
      - \tr{\*\multiplier_2} \vnormbuy_t
      - \tr{\*\multiplier_3} \vnormsell_t
      + \tr{\*\multiplier_4}\; (\vnormsell_t - \vstock_t)
      + \multiplier_5\; (\normconsume_{\min} - \normconsume_t)
    \end{split}
  \end{equation}%
}%
with $
\*\multiplier := (
  \multiplier_1,
  \*\multiplier_2,
  \*\multiplier_3,
  \*\multiplier_4,
  \multiplier_5
)$,
$\multiplier_1, \multiplier_5 \in \real$, and
$\*\multiplier_2, \*\multiplier_3, \*\multiplier_4 \in \real^{m_{\vstock}}$.
According to the first-order conditions
\term{(Karush--Kuhn--Tucker (KKT) conditions),}
the partial derivative
$
  \partialderiv{\partialdiff \normbond_t}{\lagrangian_t}(
    \normstate_t, \normpolicy_t, \*\multiplier
  )
$
with respect to $\normbond_t$
vanishes in the exact optimum
$\normpolicy_t = \optnormpolicyfcn_t := \optnormpolicyfcn_t(\normstate_t)$,
i.e.,
{%
  \setlength{\abovedisplayskip}{9pt}%
  \setlength{\belowdisplayskip}{9pt}%
  \begin{equation}
    \label{eq:eulerErrorFirstOrderCondition}
    \partialderiv{\partialdiff \normbond_t}{
      \left(
        (\normconsume_t^{\opt})^{1-\riskav} +
        \patience \expectation[t]{
          \bigl(
            \wealthratio_{t+1}^{\opt}\;
            \normcetvalueintp_{t+1}(\normstatefcn_t^{\opt})
          \bigr)^{1-\riskav}
        }
      \right)^{1/(1-\riskav)}
    }
    - \multiplier_1
    - \multiplier_5
    \partialderiv{\partialdiff \normbond_t}{\normconsume_t^{\opt}}
    = 0,
  \end{equation}%
}%
where
$
  \normstatefcn_t^{\opt}
  := \normstatefcn_t(\normstate_t, \optnormpolicyfcn_t, \stochastic_t)
$,
$
  \wealthratio_{t+1}^{\opt}
  := \wealthratio_{t+1}(
    \normstate_t, \optnormpolicyfcn_t, \stochastic_t
  )
$, and
$
  \normconsume_t^{\opt}
  := \normconsume_t(\normstate_t, \optnormpolicyfcn_t)
$.
We now neglect binding constraints,
i.e., we assume that $\multiplier_1 = \multiplier_5 = 0$,
otherwise we cannot compute the error.
After calculating the derivatives,
\cref{eq:eulerErrorFirstOrderCondition} becomes
{%
  \setlength{\abovedisplayskip}{9pt}%
  \setlength{\belowdisplayskip}{9pt}%
  \begin{equation}
    \patience \bondreturn_t
    \cdot \expectationsign[t]\Bigl[
      \bigl(
        \normcetvalueintp_t
        - \tr{(\gradient{\normstate_t}{\normcetvalueintp_t})}
        \statefcn_t^{\opt}
      \bigr) \cdot
      \bigl(
        \wealthratio_{t+1}^{\opt}\; \normcetvalueintp_t
      \bigr)^{-\riskav}
    \Bigr]
    = (\normconsume_t^{\opt})^{-\riskav}.
  \end{equation}
}%
This equation can be used as an error measure by substituting
$\optnormpolicyfcn_t$ for the interpolated optimum
$\optnormpolicyintp_t = \optnormpolicyintp_t(\normstate_t)$.
By multiplying the resulting equation by
$
  (\normconsume_t^{\sparse,\opt})^{\riskav}
  := (\normconsume_t(\normstate_t, \optnormpolicyintp_t))^{\riskav}
$, we obtain the
\term{unit-free Euler equation errors $\error_t(\normstate_t)$}
with respect to $\normbond_t$:
{%
  \setlength{\abovedisplayskip}{9pt}%
  \setlength{\belowdisplayskip}{9pt}%
  \begin{equation}
    \error_t(\normstate_t)
    := \Bigl|
      1 - \Bigl(
        \patience \bondreturn_t (\normconsume_t^{\sparse,\opt})^{\riskav}
        \cdot \expectationsign[t]\Bigl[
          \bigl(
            \normcetvalueintp_t
            - \tr{(\gradient{\normstate_t}{\normcetvalueintp_t})}
            \statefcn_t^{\sparse,\opt}
          \bigr) \cdot
          \bigl(
            \wealthratio_{t+1}^{\sparse,\opt}\;
            \normcetvalueintp_t
          \bigr)^{-\riskav}
        \Bigr]
      \Bigr)^{-1/\riskav}
    \Bigr|
  \end{equation}
}%
with
$
  \normstatefcn_t^{\sparse, \opt}
  := \normstatefcn_t(\normstate_t, \optnormpolicyintp_t, \stochastic_t)
$ and
$
  \wealthratio_{t+1}^{\sparse,\opt}
  := \wealthratio_{t+1}(
    \normstate_t, \optnormpolicyintp_t, \stochastic_t
  )
$.

%\begin{equation}
%  \begin{split}
%    \patience \bondreturn_t
%    \cdot \expectationsign[t]\Bigl[
%      &\bigl(
%        \normcetvalueintp_t%(\normstate_t)
%        - \tr{
%          (
%            \gradient{\normstate_t}{\normcetvalueintp_t}%(\normstate_t)
%          )
%        }
%        \statefcn_t%(\normstate_t, \optnormpolicyfcn_t, \stochastic_t)
%      \bigr)\\[-1mm]
%      &{} \cdot
%      \bigl(
%        \wealthratio_{t+1}%(
%        %  \normstate_t, \optnormpolicyfcn_t, \stochastic_t
%        %)
%        \;
%        \normcetvalueintp_t%(\normstate_t)
%      \bigr)^{-\riskav}
%    \Bigr]
%    = \normconsume_t%(\normstate_t, \optnormpolicyfcn_t)
%    ^{-\riskav}
%  \end{split}
%\end{equation}
%
%\begin{equation}
%  \begin{split}
%    \error_t(\normstate_t)
%    := \Bigl|
%      1 - \Bigl(
%        \patience \bondreturn_t
%        \cdot \normconsume_t%(\normstate_t, \optnormpolicyfcn_t)
%        ^{\riskav}
%        \cdot \expectationsign[t]\Bigl[
%          &\bigl(
%            \normcetvalueintp_t%(\normstate_t)
%            - \tr{
%              (
%                \gradient{\normstate_t}{\normcetvalueintp_t}%(\normstate_t)
%              )
%            }
%            \statefcn_t%(\normstate_t, \optnormpolicyfcn_t, \stochastic_t)
%          \bigr)\\[-2mm]
%          &{} \cdot
%          \bigl(
%            \wealthratio_{t+1}%(
%            %  \normstate_t, \optnormpolicyfcn_t, \stochastic_t
%            %)
%            \;
%            \normcetvalueintp_t%(\normstate_t)
%          \bigr)^{-\riskav}
%        \Bigr]
%      \Bigr)^{-1/\riskav}
%    \Bigr|,
%  \end{split}
%\end{equation}

\paragraph{Weighted Euler equation errors}

However, the state space cropping as introduced above
distorts Euler equation errors:
The error $\eulererror_t(\normstate_t)$ does not vanish
even for the exact solution and even inside the feasible state space.
This is because the cropping already occurs for large stock holdings
$\sumfcn(\vstock_t)$ that are close to, but lower than one,
as stocks have to be sold to maintain
minimum consumption $\normconsume_{\min}$.
Numerical experiments showed that due to this issue,
the error attains large values near the hyperplane
$\sumfcn(\vstock_t) = 1$.
Economically, this region is not significant
as such high stock fractions are unusual
(which is confirmed by Monte Carlo simulations).
We therefore use the \term{weighted Euler equation error}
\begin{equation}
  \weightedeulererror_t(\normstate_t)
  := \bigl(1 - \sumfcn(\normstate_t)\bigr) \cdot
  \eulererror_t(\normstate_t)
\end{equation}
instead of $\eulererror_t$,
although other strategies exist
such as restricting the state domain where the error is computed or
weighting the error with the probability that a given state
occurs in Monte Carlo simulations.
