\setdictum[0.57\textwidth]{%
  There is a fine line between wrong and visionary.
  Unfortunately, you have to be a visionary to see it\dots%
}{%
  Sheldon Cooper (The Big Bang Theory)%
}

\chapter{Introduction}
\label{chap:10introduction}

% https://visca.com/regexdict/
\initial{0.1em}{B}{efore simulation became available}
as a widespread tool,
knowledge in science and engineering could only advance through
theoretical or experimental considerations.
Nowadays, processes can be simulated
that would be too complicated or even impossible
to be studied theoretically or experimentally,
justifying that simulation is widely viewed as the
``third pillar in knowledge acquisition''
besides theory and experiments \cite{Bungartz14Modeling}.

However, simulations cannot be performed without constructing
a suitable model beforehand (e.g., based on first principles).
Such a model often depends on a number of uncertain or unknown parameters.
Simulations can only represent the real-world circumstances well
if the parameters are well-chosen.
The problem of determining appropriate values for these parameters,
given experimental data, is known as the \term{inverse problem.}
Unfortunately, the solution process of such an inverse problem is non-trivial:
Inverse problems are equivalent to optimization problems of the form
\begin{equation}
  \label{eq:generalOptimizationProblem}
  \min \objfun(\*x),\quad
  \*x \in \real^d\;\;\text{s.t.}\;\;
  \ineqconfun(\*x) \le \*0,
\end{equation}
where $\objfun(\*x)$ gives, for instance,
a measure for the error between the simulation for the model with
the parameter $\*x$ and experimental real-world data and
$\ineqconfun$ constrains the set of feasible parameters.
Since simulations are often time-consuming,
exhaustive search fails
if the dimensionality $d$ is already moderately large ($d > 4$):
Full grid approaches sample each dimension of the domain
independently and construct the Cartesian product of the univariate
samples.
The number of resulting full grid points
grows exponentially in the dimensionality $d$,
which is known as the \term{curse of dimensionality} \cite{Bellman61Adaptive}.

Inverse problems are a motivating example for optimization problems
of the form \eqref{eq:generalOptimizationProblem}, which we will consider in
this thesis as our main application.
The curse affects not only optimization,
but also various tasks such as interpolation, quadrature, and regression,
which play a vital role in numerics and computational science.

\paragraph{Sparse grid surrogates}

The surrogate-based approach we pursue in this thesis is simple and
yet powerful:
Instead of directly optimizing the expensive objective function $\objfun$,
we replace it with a surrogate $\sgintp$ that can be evaluated cheaply.
We choose interpolation as our method to construct the surrogates,
although other methods such as quasi-interpolation
\cite{Hoellig13Approximation} or regression \cite{Pflueger10Spatially} exist.
Again, conventional full grid interpolation schemes are afflicted by the
curse of dimensionality, which rules them out for our purposes.

This is where \term{sparse grids} come into play.
In their simplest form, sparse grids give an \term{a priori} selection of
full grid points and corresponding basis functions such that
the exponential dependency of the grid size on the dimensionality
is removed, while not deteriorating the $\Ltwo$ interpolation error too much
\cite{Bungartz04Sparse}.
However, sparse grids can also be employed spatially adaptively,
where grid points are refined \term{a posteriori} according to suitable
refinement criteria.
This is of particular interest for the scope of this thesis,
as spatial adaptivity enables us to increase the
accuracy in regions of interest,
simultaneously keeping the number of grid points at an acceptable level.

\paragraph{B-splines}

Conventional basis functions for sparse grids
(most common are piecewise linear functions)
are not continuously differentiable.
This poses problems for gradient-based optimization algorithms,
which use the gradient $\gradient{\*x}{f}$ of the
objective function $\objfun$ to update the search direction.
Employing finite differences as a remedy
is time-consuming and introduces new error sources.

Previous studies
\multicite{Pflueger10Spatially,Sickel11Spline,Valentin14Hierarchische}
suggest that \term{hierarchical B-splines}
as sparse grid basis functions may significantly improve results.
B-splines of degree~$p$ are $(p-1)$ times continuously differentiable
piecewise polynomials of degree $p$ that form a basis of the space of splines.
As the derivatives of B-splines can be evaluated fast and explicitly,
the convergence of gradient-based optimization techniques is
greatly accelerated.
Additionally, the higher order of B-splines increases the accuracy
of surrogates obtained by interpolation
when compared to piecewise linear bases.

\paragraph{Main goals}

So far, there is no work that brings sparse grids and B-splines together,
thoroughly examining the theoretical implications on algorithms and
assessing the practical performance in real-world applications.
This thesis addresses this very intersection of theory and practice of
\term{B-splines for sparse grids.}
The main goals of the thesis are
\begin{itemize}
  \item
  to establish a consistent notational and theoretical
  framework for sparse grids with general basis functions,
  
  \item
  to construct new algorithmically efficient
  B-spline-based basis function types for sparse grids,
  
  \item
  to study the algorithmic properties of the new bases and
  to formulate suitable new algorithms,
  while proving their formal correctness, and
  
  \item
  to apply the new bases and algorithms to different real-world
  optimization scenarios.
\end{itemize}
These goals set the agenda for the outline of the rest of the thesis,
which is described in the following.

\paragraph{Outline}

We start in \cref{chap:20sparseGrids} by defining sparse grids
for arbitrary tensor product basis functions.
The advantage of introducing the notation independently
of the type of basis functions is that
different hierarchical B-spline bases can be substituted easily
for the following chapters.
%Due to the tensor product approach,
%we only have to treat the univariate case.
%As we will see (and prove), the tensor product allows to maintain all
%univariate properties also in the multivariate case.
We first define sparse grids with points on the boundary of the domain
and then study options for the boundary treatment
(as opposed to some literature %, which does it in the reverse order
\multicite{Bungartz04Sparse,Pflueger10Spatially}).

In \cref{chap:30BSplines}, we define
the standard hierarchical B-spline basis for sparse grids.
In addition, we construct various new hierarchical
B-spline basis types, such as
non-uniform B-splines (e.g., Clenshaw--Curtis B-splines) and
modified B-splines.
%which give reasonable boundary values for
%sparse grids without points on the boundary.
A mismatch of dimensions between the uniform spline space and the
hierarchical B-spline space implies that, surprisingly,
polynomials cannot be replicated by the
standard hierarchical B-spline basis.
Hence, we have to incorporate specific boundary conditions
\term{(not-a-knot conditions)} into the hierarchical B-splines,
which we explain in the second half of \cref{chap:30BSplines}.

The new hierarchical B-spline bases call for novel algorithmic approaches
to solve numerical tasks such as
hierarchization (interpolation) on sparse grids.
In \cref{chap:40algorithms}, we show new algorithms for
spatially adaptive sparse grids with the example problem of hierarchization
based on existing algorithms,
which work for B-splines only in specific special cases.
In the course of \cref{chap:40algorithms}, we construct several new
hierarchical B-spline basis types to enable the
applicability of the new algorithms.
As mentioned above, we prove the formal correctness of every algorithm
that we repeat from the literature or develop from scratch.

\Cref{chap:50optimization} shows how to apply B-splines on sparse grids
to gradient-based optimization problems.
We briefly discuss different optimization scenarios and how to solve them
with various gradient-free and gradient-based optimization techniques.
Numerical results are given for a number of test scenarios as well
as for an example application from fuzzy arithmetic.

Three real-world applications follow in
\cref{chap:60topoOpt,chap:70muscle,chap:80finance}.
In these chapters,
the theoretical knowledge gained in the first half of the thesis
is applied to the solution of the three real-world optimization problems:

First, in \cref{chap:60topoOpt},
we study topology optimization via a homogenized two-scale approach.
For this application, the key ingredient is an interpolation scheme
that preserves both the positive definiteness of the interpolated tensors and
their explicit differentiability.

Second, in \cref{chap:70muscle},
we consider a biomechanical application in which the interpolated data values
are the result of very expensive continuum-mechanical calculations.
The optimization problems posed in this chapter ask for
muscle activation levels such that a specific joint angle is attained,
which is a recurring problem in medicine and robotics.
B-spline surrogates on sparse grids
decrease the necessary computing time significantly.

Third, in \cref{chap:80finance},
we examine dynamic portfolio choice models.
This financial application features some peculiarities that have
to be considered when solving the corresponding optimization problems.
For instance, it is necessary to evaluate interpolants outside their domain
and calculate integrals due to random factors such as stock returns.

\Cref{chap:90conclusion} concludes the thesis by
summarizing its results and giving an overview of possible future work.
In the appendix, one can find supplementary information such as
technical proofs that are too verbose to be included in the main text.

\paragraph{Original contribution}

This thesis is written to be largely self-contained.
Therefore, it is necessary that some introductory definitions and
results are repeated from the literature,
which is properly attributed in the respective chapters.
In addition, some new results have already been published.
Whenever a publication is co-authored by collaborators,
the original contribution of the author of this thesis is highlighted
at the beginning of the respective chapters or sections.

\paragraph{Notation}

The notation of this thesis should be intuitive and suggestive.
It is designed to be as natural as possible (i.e., not distracting)
and as unambiguous as necessary.
One example is that vectors are written in bold face, which leads to
very similar formulas for the univariate and the multivariate cases.
For instance, $\clint{0, 1}$ and $\sum_{l=0}^n$
%and $a \le b$
become
$\clint{\*0, \*1} = \clint{0, 1}^d$ and
$\sum_{\*l=\*0}^{\*n} = \sum_{l_1=0}^{n_1} \dotsb \sum_{l_d=0}^{n_d}$,
%$\*a \le \*b \iff (a_1 \le b_1) \land \dotsb \land (a_d \le b_d)$,
respectively.
This and other necessary notation is introduced in the text when needed.
If a symbol or an abbreviation is unclear,
it is likely explained in the glossary at the beginning of the thesis.

\cleardoublepage
