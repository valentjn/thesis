\setdictum{%
  Now, these points of data make a beautiful line.
  And we're out of beta. We're releasing on time!%
}{%
  GLaDOS (Portal)%
}

\longchapter{%
  Application 2: Biomechanical Application%
}{%
  Application 2:\texorpdfstring{\\}{ }Biomechanical Application%
}{%
  Application 2 -- Biomechanical Application%
}
\label{chap:70muscle}

\initial{0em}{E}{xisting biomechanical models of muscle-tendon complexes},
e.g., of the human upper limb, can mainly be divided into two different types.
The most common type are \term{lumped-parameter musculoskeletal models,}
for instance Hill-type models based on multi-body simulations
\multicite{Roehrle16Two,Valentin18Gradient}.
These models assume the components of the musculoskeletal system
to be rigid.
The mechanical behavior is reduced to point masses
associated with their moment of inertia;
thus, these models can be described by few (lumped) parameters.

\term{Continuum-mechanical musculoskeletal models} form the other common type.
Their advantage is that they are more detailed and, hence, more realistic.
However, their increased complexity leads to higher computational costs.
An example is an inverse problem (see \cref{chap:10introduction})
that involves a continuum-mechanical simulation of such a
musculoskeletal model,
i.e., we search values of model parameters
such that a specific movement is attained.
Each iteration of the solution process for such an inverse problem
may take hours or days, depending on the model at hand.

Surrogate methods based on sparse grid help to decrease the complexity
in two ways:
First, the evaluation of surrogates is obviously drastically cheaper
than objective function evaluations.
Second, the usage of sparse grids in particular decreases the number
of necessary samples to construct the surrogates.
As for the previous application,
the choice of B-splines as hierarchical basis functions enables
the evaluation of continuously differentiable surrogate gradients.
For the example of inverse problems, this means that
gradient-based optimization methods may be employed,
which significantly accelerates convergence.

This chapter is split into three sections.
First, in \cref{sec:71model}, we introduce a continuum-mechanical
model of the human upper limb.
Second, in \cref{sec:72problems}, we list the types of inverse problems
of interest.
Finally, in \cref{sec:73results}, we present numerical results
regarding the solution of these inverse problems.

The results of this chapter are based on a collaboration with
Prof.\ Oliver Röhrle, PhD, and Dr.\ Michael Sprenger
(SimTech/University of Stuttgart, Germany).%
\footnote{%
  Michael Sprenger left the University of Stuttgart in 2015.%
}
The collaborators contributed the biomechanical model
with its theory, its geometry, and its implementation,
while the author of this thesis contributed the
sparse grid/B-spline methodology and
computed the numerical results.
Note that the results have already been
published in \cite{Valentin18Gradient}.

\section{Continuum-Mechanical Model of the Human Upper Limb}

\blindtext{}

\section{Optimization Problems}

\blindtext{}

\section{Numerical Results}
\label{sec:73results}

\minitoc[-5mm]{90mm}{4}

\noindent
In the final section of this chapter,
we present and discuss numerical results for our
biomechanical model of the upper limb.



\subsection{Reference and Sparse Grid Solution}
\label{sec:731solutionTypes}

\paragraph{Reference solution}

Since the model is only two-dimensional, we can compute a reference solution
on a full grid.
Hence, we evaluate the exerted muscle forces $\forceT$ and $\forceB$ on
the full grid
\begin{equation}
  \{\ang{10}, \ang{11}, \dotsc, \ang{150}\} \times \{0, 0.1, \dotsc, 1\}
  \ni (\elbang, \actX),\quad
  {\ast} \in \{\mathrm{T}, \mathrm{B}\}.
\end{equation}
The resulting \num{1551} grid points
are interpolated with bicubic spline interpolation
to obtain \term{reference solutions}
$\forceTref, \forceBref\colon
\clint{\ang{10}, \ang{150}} \times \clint{0, 1} \to \real$,
which are shown in \cref{fig:biomech2ReferenceForce}.
Due to the high resolution of the full grid,
we may assume that the reference solutions are accurate enough
to ensure $\forceTref \approx \forceT$ and $\forceBref \approx \forceB$.
We refer to the resulting quantities with the superscript ``$\mathrm{ref}$'',
for instance, the equilibrium elbow angle $\equielbangref{\forceL}$.

\begin{figure}
  \includegraphics{biomech2ReferenceForce_1}%
  \;\;%
  \includegraphics{biomech2ReferenceForce_2}%
  \hfill%
  \rlap{\raisebox{53mm}{\;$\forceXref$ [\si{\kilo\newton}]}}%
  \includegraphics{biomech2ReferenceForce_3}%
  \caption[Reference triceps and biceps forces]{%
    Reference triceps and biceps forces $\forceXref$
    ($X \in \{\mathrm{T}, \mathrm{B}\}$).%
  }%
  \label{fig:biomech2ReferenceForce}%
\end{figure}

\paragraph{Sparse grid solution}

Additionally, we evaluate $\forceT$ and $\forceB$ at the $\ngp = 49$
grid points of the regular sparse grid $\interiorregsgset{n}{d}$ of
level $n = 5$ in $d = 2$ dimensions.
We interpolate these values using
modified hierarchical not-a-knot B-splines
$\bspl[\nak,\modified]{\*l,\*i}{p}$ of degree $p = 1$, $3$, and $5$
(see \cref{sec:323modifiedNAKBSplines}).
The implementation was done using the sparse grid toolbox
\sgpp{} \cite{Pflueger10Spatially}.
The corresponding interpolants and resulting quantities
are denoted with the superscript ``$\sparse,p$''.
Note that the equilibrium elbow angle is \emph{not} interpolated
(neither in the full grid nor in the sparse grid case),
but rather obtained by inserting the interpolated muscle forces
into \cref{eq:totalMomentSurrogate,eq:equilibriumAngleSurrogate}.



\subsection{Interpolation Error of Muscle Forces}
\label{sec:732forceInterpolation}

\dummytext[3]{}



\subsection{Error of the Equilibrium Angle}
\label{sec:733equilbriumAngle}

\dummytext[3]{}



\subsection{Error in Test Scenarios}
\label{sec:734scenarios}

\dummytext[3]{}



\subsection{Outlook: Spatial Adaptivity}
\label{sec:735adaptivity}

\dummytext[3]{}


\cleardoublepage
