\setdictum{%
  Now, these points of data make a beautiful line.
  And we're out of beta. We're releasing on time!%
}{%
  GLaDOS (Portal)%
}

\longchapter{%
  Application 2: Biomechanical Application%
}{%
  Application 2:\texorpdfstring{\\}{ }Biomechanical Application%
}{%
  Application 2 -- Biomechanical Application%
}
\label{chap:70muscle}

\initial{0em}{E}{xisting biomechanical models of muscle-tendon complexes,}
e.g., of the human upper limb, can mainly be divided into two different types.
The most common type are \term{lumped-parameter musculoskeletal models,}
e.g., Hill-type models based on multi-body simulations
\multicite{Roehrle16Two,Valentin18Gradient}.
These models assume that the components of the musculoskeletal system
are rigid.
The mechanics is reduced to point masses
associated with their moment of inertia;
thus, these models can be described by few parameters.

\term{Continuum-mechanical musculoskeletal models} form the other common type.
Their advantage is that they are more detailed and, hence, more realistic.
However, their increased complexity leads to higher computational costs.
An example is an inverse problem (see \cref{chap:10introduction})
that involves a continuum-mechanical simulation of such a
musculoskeletal model,
where we search values of model parameters
such that a specific movement is attained.
Each iteration of the solution process for such an inverse problem
may take hours or days, depending on the model at hand.

Surrogate methods based on sparse grids help to decrease the complexity
in two ways:
First, the evaluation of surrogates is obviously drastically cheaper
than objective function evaluations.
Second, the particular choice of sparse grids decreases the number
of necessary samples to construct the surrogates.
As for the previous application,
the choice of B-splines as hierarchical basis functions enables
the evaluation of continuously differentiable surrogate gradients.
For the example of inverse problems, this means that
gradient-based optimization methods may be employed,
which significantly accelerates convergence.

This chapter is split into three sections.
First, in \cref{sec:71model}, we introduce a continuum-mechanical
model of the human upper limb.
Second, in \cref{sec:72methodology}, we list the types of inverse problems
of interest.
Finally, in \cref{sec:73results}, we present numerical results
regarding the solution of these inverse problems.

The results of this chapter are based on a collaboration with
Prof.\ Oliver Röhrle, PhD, and Dr.\ Michael Sprenger
(SimTech/University of Stuttgart, Germany).%
\footnote{%
  Michael Sprenger left the University of Stuttgart in 2015.%
}
The collaborators contributed the biomechanical model
with its theory, its geometry, and its implementation,
while the author of this thesis contributed the
sparse grid/B-spline methodology and
computed the numerical results.
Note that the results have already been
published in \cite{Valentin18Gradient},
which we will follow closely in this chapter.

\section{Continuum-Mechanical Model of the Human Upper Limb}

\todo{write}

\section{Momentum Equilibrium and Elbow Angle Optimization}
\label{sec:72methodology}

\minitoc{70mm}{4}

\noindent
In this section, we give an overview of the methodology of our approach.
We closely follow the presentation of \cite{Valentin18Gradient}.



\subsection{From Muscle Forces to Equilibrium Angles}
\label{sec:721equilibrium}

\paragraph{Model inputs and outputs}

In the following, we regard simulations of the
human upper limb model described in \cref{sec:71model} as a black box.
This black box receives as its input
the elbow angle $\elbang \in \clint{\ang{10}, \ang{150}}$
and the activation parameters $\actT, \actB \in \clint{0, 1}$
of triceps and biceps.%
\footnote{%
  Here and in the following, the subscripts T, B, and L stand for
  triceps, biceps, and load, respectively.%
}
The outputs of the black box simulation are the forces
$\forceT(\elbang, \actT)$ and $\forceB(\elbang, \actB)$
that triceps and biceps exert.
These forces depend on the elbow angle as well as on the respective
activation parameter.
Gravitational forces due to the masses of bones or masses
are neglected in this context.
However, we allow the specification of an external load $\forceL$
applied to the end of the forearm.
This load may the weight force of some object
that the arm is supposed to keep in position.

\paragraph{Moments and lever arms}

Every force exerts a \term{moment} (or \term{torque}) on the elbow joint.
The moments are the products of the forces $\forcevalue_\ast$
with the respective lever arms $\arm_\ast$
($\ast \in \{\mathrm{T}, \mathrm{B}, \mathrm{L}\}$).
The lever arms are approximated as in
\multicite{Roehrle16Two,Valentin18Gradient} by using
the tendon-displacement method of \cite{An84Determination},
resulting in
\begin{subequations}
  \begin{align}
    \armT(\elbang)
    &:= (-0.0009399 \{\elbang\}^2 + 0.1126 \{\elbang\} + 22.21)\;
    \si{\milli\meter},\\
    \armB(\elbang)
    &:= (-0.001482 \{\elbang\}^2 + 0.1776 \{\elbang\} + 35.02)\;
    \si{\milli\meter},\\
    \armL(\elbang)
    &:= \sin(\elbang) \cdot \SI{282.5}{\milli\meter},
  \end{align}
\end{subequations}
where $\{\elbang\}$ denotes the numerical value of $\elbang$
in degrees.
The lever arms are non-negative and the forces are signed, i.e.,
positive forces pull the forearm downwards,
negative forces pull it upwards.
In general, $\forceT, \forceL \ge \SI{0}{\newton}$ and
$\forceB \le \SI{0}{\newton}$.

\paragraph{Total moment and equilibrium elbow angle}

The \term{total moment} of the system is given by the function
\begin{subequations}
  \begin{gather}
    \moment_{\forceL,\actT,\actB}\colon
    \clint{\ang{10}, \ang{150}} \to \real,\\
    \moment_{\forceL,\actT,\actB}(\elbang)
    := \forceT(\elbang, \actT) \armT(\elbang) +
    \forceB(\elbang, \actB) \armB(\elbang) +
    \forceL \armL(\elbang),
  \end{gather}
\end{subequations}
cf.\ \cite{Valentin18Gradient}.
The system is in \term{equilibrium,}
if the total moment vanishes, i.e.,
$\moment_{\forceL,\actT,\actB}(\elbang) = \SI{0}{\newton\meter}$.
We call the corresponding angle $\elbang$ the
\term{equilibrium elbow angle.}
To find this angle for a given load $\forceL$ and activation parameters
$\actT$ and $\actB$, we first note that
$\moment_{\forceL,\actT,\actB}$ may have zero, exactly one,
or multiple zeros in $\clint{\ang{10}, \ang{150}}$.
Hence, the inverse function evaluated at $\SI{0}{\newton\meter}$
is partially defined depending on load and activation parameters:
\begin{subequations}
  \begin{gather}
    \elbang_{\forceL}\colon \actdomain{\forceL} \to
    \clint{\ang{10}, \ang{150}},\quad
    \actdomain{\forceL} \subset \clint{0, 1}^2,\\
    \elbang_{\forceL}(\actT,\actB)
    := (\moment_{\forceL,\actT,\actB})^{-1}(\SI{0}{\newton\meter}),
  \end{gather}
\end{subequations}
which is well-defined whenever $\moment_{\forceL,\actT,\actB}$
has a unique root.
We approximate $\elbang_{\forceL}(\actT,\actB)$ with the Newton method
\multicite{Roehrle16Two,Valentin18Gradient}:
\begin{equation}
  \elbang^{(j+1)}
  := \elbang^{(j)} -
  \frac{
    \moment_{\forceL,\actT,\actB}(\elbang^{(j)})
  }{
    \partialderiv{\partialdiff{} \elbang}{\moment_{\forceL,\actT,\actB}}
    (\elbang^{(j)})
  },\quad
  j \in \nat,
\end{equation}
with an initial value
$\elbang^{(0)} \in \clint{\ang{10}, \ang{150}}$
and the stopping criterion of
$\abs{\moment_{\forceL,\actT,\actB}(\elbang^{(j)})} <
\SI{e-9}{\newton\meter}$ or
$\abs{\partialderiv{\partialdiff{} \elbang}{\moment_{\forceL,\actT,\actB}}
(\elbang^{(j)})} < \SI{e-9}{\newton\meter\per\degree}$.
We repeat the Newton method for the initial values
$\elbang^{(0)} = \ang{80}, \ang{40}, \ang{120}$
and use the first converged result
(i.e., we check if $\elbang^{(0)} = \ang{80}$ converges;
if not, proceed with $\elbang^{(0)} = \ang{40}$, and so on).
If all three initial values do not converge,
we conclude that $(\actT, \actB) \notin \actdomain{\forceL}$.



\subsection{Optimization Problems}
\label{sec:722optimization}

\paragraph{General task}

The general task in our setting is as follows:
For a given external load $\forceL$ and a target elbow angle $\elbang^\ast$,
find activation parameters $(\actT, \actB) \in \clint{\*0, \*1}$
such that the target elbow angle is attained in the equilibrium,
i.e., $\elbang_{\forceL}(\actT,\actB) = \elbang$.
Example applications of such a scenario are medicine and robotics,
when a specific movement should be carried out.

\paragraph{List of optimization problems}

However, as discussed in \cref{sec:711models},
musculoskeletal systems with an antagonistic muscle pair,
such as our human upper limb model, are usually overdetermined.
This means that there are multiple solutions to this problem.
To solve this issue, one may solve one of the three
optimization problems that the following list proposes
\cite{Valentin18Gradient}:

\begin{enumerate}[label=O\arabic*.,leftmargin=2.7em]
  \item
  For a given external load $\forceL$ and a target angle
  $\elbang^\ast \in \clint{\ang{10}, \ang{150}}$,
  find the activation parameters $(\actT, \actB) \in \clint{\*0, \*1}$
  such that $\actT + \actB$ is minimized under the constraint
  $\elbang_{\forceL}(\actT, \actB) = \elbang^\ast$.
  
  \item
  For a given external load $\forceL(t_2)$ for a time $t_2 > t_1$,
  a target angle $\elbang^\ast(t_2) \in \clint{\ang{10}, \ang{150}}$,
  and initial activation parameters
  $(\actT(t_1), \actB(t_1)) \in \clint{\*0, \*1}$,
  find new activation parameters
  $(\actT(t_2), \actB(t_2)) \in \clint{\*0, \*1}$ such that
  $(\actT(t_2) - \actT(t_1))^2 + (\actB(t_2) - \actB(t_1))^2$
  is minimized under the constraint
  $\elbang_{\forceL(t_2)}(\actT(t_2), \actB(t_2)) = \elbang^\ast(t_2)$.
  
  \item
  For a given external load $\forceL(t_2)$ for a time $t_2 > t_1$,
  an initial angle $\elbang^\ast(t_1) \in \clint{\ang{10}, \ang{150}}$,
  an initial activation parameters
  $(\actT(t_1), \actB(t_1)) \in \clint{\*0, \*1}$,
  find new activation parameters
  $(\actT(t_2), \actB(t_2)) \in \clint{\*0, \*1}$ such that
  $(\actT(t_2) - \actT(t_1))^2 + (\actB(t_2) - \actB(t_1))^2 +
  c^2 \cdot (
    \elbang_{\forceL(t_2)}(\actT(t_2), \actB(t_2)) - \elbang^\ast(t_2)
  )^2$ is minimized (with $c \in \nonnegreal$ arbitrary but fixed).
\end{enumerate}

\paragraph{Motivation of problem O1}

The motivation of all problems is that the human body tries to
achieve a given movement with minimal energy effort.
For the first problem O1, this effort is quantified by $\actT + \actB$,
i.e., the energy effort for each muscle is assumed to be proportional
to its activation parameter.

\paragraph{Motivation of problem O2}

The second problem O2 is motivated as follows:
Before time $t = t_1$, the musculoskeletal system is in equilibrium for
an external load $\forceL(t_1)$,
activation parameters $\actT(t_1), \actB(t_1)$, and
elbow angle $\elbang^\ast(t_1)$, i.e.,
$\moment_{\forceL(t_1),\actT(t_1),\actB(t_1)}(\elbang^\ast(t_1))
= \SI{0}{\newton\meter}$.
Directly after $t = t_1$,
the external force and/or the target angle is suddenly changed
to $\forceL(t_2)$ and $\elbang^\ast(t_2)$, respectively.
Consequently, triceps and biceps adapt their activation parameters
such that the musculoskeletal system returns to equilibrium
at some time $t = t_2 > t_1$.
Hence, we have to determine the new activation parameters
$\actT(t_2), \actB(t_2)$ such that
$\moment_{\forceL(t_2),\actT(t_2),\actB(t_2)}(\elbang^\ast(t_2))
= \SI{0}{\newton\meter}$.

TODO

\paragraph{Motivation of problem O3}

TODO



\subsection{B-Spline Surrogates on Sparse Grids}
\label{sec:723surrogates}

\todo{talk about sparse grids}
\todo{talk ``only'' about two dimensions}
\todo{talk about derivatives (14)}

\dummytext[3]{}

\section{Numerical Results}
\label{sec:73results}

\minitoc[-5mm]{90mm}{4}

\noindent
In the final section of this chapter,
we present and discuss numerical results for our
biomechanical model of the upper limb.



\subsection{Reference and Sparse Grid Solution}
\label{sec:731solutionTypes}

\paragraph{Reference solution}

Since the model is only two-dimensional, we can compute a reference solution
on a full grid.
Hence, we evaluate the exerted muscle forces $\forceT$ and $\forceB$ on
the full grid
\begin{equation}
  \{\ang{10}, \ang{11}, \dotsc, \ang{150}\} \times \{0, 0.1, \dotsc, 1\}
  \ni (\elbang, \actX),\quad
  X \in \{\mathrm{T}, \mathrm{B}\}.
\end{equation}
The resulting \num{1551} grid points
are interpolated with bicubic spline interpolation%
\footnote{%
  Computed with the Geometric Tools Engine, see
  \url{https://www.geometrictools.com/}
  \cite{Schneider03Geometric}.
}
to obtain \term{reference solutions}
$\forceTref, \forceBref\colon
\clint{\ang{10}, \ang{150}} \times \clint{0, 1} \to \real$,
which are shown in \cref{fig:biomech2ReferenceForce}.
Due to the high resolution of the full grid,
we may assume that the reference solutions are accurate enough
to ensure $\forceTref \approx \forceT$ and $\forceBref \approx \forceB$.
We refer to the resulting quantities with the superscript ``$\mathrm{ref}$'',
for instance, the corresponding equilibrium elbow angle
$\equielbangref{\forceL}$, which is displayed in
\cref{fig:biomech2ReferenceEquilibriumAngle}
for the exemplary loads of $\forceL = \SI{22}{\newton}$,
$\SI{-60}{\newton}$, and $\SI{180}{\newton}$.

\begin{figure}
  \includegraphics{biomech2ReferenceForce_1}%
  \;\;%
  \includegraphics{biomech2ReferenceForce_2}%
  \hfill%
  \rlap{\raisebox{53mm}{\;$\forceXref$ [\si{\kilo\newton}]}}%
  \includegraphics{biomech2ReferenceForce_3}%
  \caption[Reference triceps and biceps forces]{%
    Reference triceps and biceps forces $\forceXref$
    ($X \in \{\mathrm{T}, \mathrm{B}\}$).%
  }%
  \label{fig:biomech2ReferenceForce}%
\end{figure}

\begin{figure}
  \includegraphics{biomech2ReferenceEquilibriumAngle_4}%
  \\[2mm]%
  \subcaptionbox{%
    $\forceL = \SI{22}{\newton}$%
  }[49mm]{%
    \includegraphics{biomech2ReferenceEquilibriumAngle_1}%
  }%
  \hfill%
  \subcaptionbox{%
    $\forceL = \SI{-60}{\newton}$%
  }[49mm]{%
    \includegraphics{biomech2ReferenceEquilibriumAngle_2}%
  }%
  \hfill%
  \subcaptionbox{%
    $\forceL = \SI{180}{\newton}$%
  }[49mm]{%
    \includegraphics{biomech2ReferenceEquilibriumAngle_3}%
  }%
  \caption[Reference equilibrium elbow angle]{%
    Reference equilibrium elbow angle $\equielbangref{\forceL}$
    for different loads $\forceL$.
    The empty areas correspond to activation pairs $(\actT, \actB)$
    at which $\equielbangref{\forceL}$ is not well-defined
    (see \cref{eq:equilibriumAngle}).%
  }%
  \label{fig:biomech2ReferenceEquilibriumAngle}%
\end{figure}

\paragraph{Sparse grid solution}

Additionally, we evaluate $\forceT$ and $\forceB$ at the $\ngp = 49$
grid points
\begin{equation}
  \{(\elbang^{(k)}, \actX^{(k)}) \mid k = 1, \dotsc, \ngp\}
  \subset \clint{\ang{10}, \ang{150}} \times \clint{0, 1},\quad
  X \in \{\mathrm{T}, \mathrm{B}\},
\end{equation}
of the regular sparse grid $\interiorregsgset{n}{d}$ of
level $n = 5$ in $d = 2$ dimensions.%
\footnote{%
  The domain $\clint{\ang{10}, \ang{150}} \times \clint{0, 1}$
  is assumed to be implicitly normalized to the unit square
  $\clint{\*0, \*1}$.%
}
These values are interpolated using three
different hierarchical B-spline bases of degree $p = 1$, $3$, and $5$:
modified hierarchical B-splines
$\bspl[\modified]{\*l,\*i}{p}$
(see \cref{sec:313modification}),
modified hierarchical Clenshaw--Curtis B-splines
$\bspl[\cc,\modified]{\*l,\*i}{p}$
(see \cref{sec:314nonUniform}), and
modified hierarchical not-a-knot B-splines
$\bspl[\nak,\modified]{\*l,\*i}{p}$
(see \cref{sec:323modifiedNAKBSplines}).
The implementation was done using the sparse grid toolbox
\sgpp{} \cite{Pflueger10Spatially}.%
\footnote{%
  \url{http://sgpp.sparsegrids.org/}%
}
The corresponding interpolants and resulting quantities
are denoted with the superscripts
``$\sparse,\!p$'', ``$\sparse,\!p,\!\cc$'', or ``$\sparse,\!p,\!\nak$'',
respectively.
A superscript of ``$\sparse$'' without any further specification
means one of the three sparse grid quantities in general.
Note that the equilibrium elbow angle is \emph{not} interpolated
(neither in the full grid nor in the sparse grid case),
but rather obtained by inserting the interpolated muscle forces
into \cref{eq:totalMomentSurrogate,eq:equilibriumAngleSurrogate}.



\subsection{Errors of Muscle Forces and Equilibrium Angle}
\label{sec:732errors}

\paragraph{Quality of the reference interpolants}

Before we turn to the sparse grid interpolants,
we assess the quality of the reference interpolant on the full grid.
For this purpose, we evaluate the full grid interpolants
$\forceTintp, \forceBintp$
at the sparse grid points $(\elbang^{(k)}, \actX^{(k)})$
(which are not a subset of the full grid points!)
and compare the resulting values with the known exact values
$\forceT(\elbang^{(k)}, \actT^{(k)})$ and
$\forceB(\elbang^{(k)}, \actB^{(k)})$
of the muscle forces $\forceT, \forceB$.
We also incorporate the known values at the sparse
Clenshaw--Curtis grid points.
In particular, to simplify the notation,
let $G$ be the union of
$\{(\elbang^{(k)}, \actX^{(k)}) \mid k = 1, \dotsc, \ngp\}$ and
$\{(\elbang^{(k,\cc)}, \actX^{(k,\cc)}) \mid k = 1, \dotsc, \ngp\}$.%
\footnote{%
  We have $\setsize{G} = 2\ngp - 1$, since sparse grids of
  uniform and Clenshaw--Curtis type only
  share the center point $(\elbang, \actX) = (\ang{80}, 0.5)$,
  if there are no boundary points.%
}
Then, we can approximate the relative $\Ltwo$ interpolation error
of the reference interpolants by
\begin{equation}
  \frac{\normLtwo{\forceX - \forceXref}}{\normLtwo{\forceX}}
  \approx
  \frac{
    \norm[l^2]{
      (\forceX(\elbang, \actX) - \forceXref(\elbang, \actX))_
      {(\elbang, \actX) \in G}
    }
  }{
    \norm[l^2]{(\forceX(\elbang, \actX))_{(\elbang, \actX) \in G}}
  },\quad
  X \in \{\mathrm{T}, \mathrm{B}\},
\end{equation}
where the $l^2$ norm of a vector $\*a \in \real^{\ngp}$ is given by
$\norm[l^2]{\*a} := \sqrt{\tfrac{1}{\ngp} \sum_{k=1}^{\ngp} (a_k)^2}$.
Inserting the known values $\forceX(\elbang, \actX)$ and
$\forceXref(\elbang, \actX)$ on the \rhs, we obtain
\begin{equation}
  \frac{\normLtwo{\forceT - \forceTref}}{\normLtwo{\forceT}}
  \approx \SI{2.19}{\permille},\qquad
  \frac{\normLtwo{\forceB - \forceBref}}{\normLtwo{\forceB}}
  \approx \SI{2.06}{\permille}.
\end{equation}
These errors are very small, which justifies our assumption of
$\forceTref \approx \forceT$ and $\forceBref \approx \forceB$.

\paragraph{Error of the muscle forces}

\dummytext{}

\begin{table}
  \newcommand*{\bi}{$\bspl[\modified]{l,i}{p}$}
  \newcommand*{\bii}{$\bspl[\cc,\modified]{l,i}{p}$}
  \newcommand*{\biii}{$\bspl[\nak,\modified]{l,i}{p}$}
  \subcaptionbox{%
    $\normLtwo{\forceXref - \forceXintp}/\normLtwo{\forceXref}$
    [\si{\permille}] given as triceps/biceps pairs
    ($X \in \{\mathrm{T}, \mathrm{B}\}$).%
    \label{tbl:biomech2ErrorL2_1}%
  }[85.2mm]{%
    \setnumberoftableheaderrows{1}%
    \begin{tabular}{%
      >{\kern\tabcolsep}=l<{\kern2mm}%
      +c<{\kern-1mm}+c<{\kern-1mm}+c<{\kern\tabcolsep}%
    }
      \toprulec
      \headerrow
      $p$&   $1$&                  $3$&                  $5$\\
      \midrulec
      \bi&   $3.60,7.12$&          $3.05,7.00$&          $\mathbf{2.98},7.90$\\
      \bii&  $\mathbf{3.28},4.35$& $3.30,\mathbf{3.56}$& $3.35,3.64$\\
      \biii& $3.60,7.12$&          $3.09,10.0$&          $7.13,24.6$\\
      \bottomrulec
    \end{tabular}%
  }%
  \hfill%
  \subcaptionbox{%
    $\normLtwo{\equielbangref{\forceL} - \equielbangintp{\forceL}}/
    \normLtwo{\equielbangref{\forceL}}$
    [\si{\permille}] for $\forceL = \SI{22}{\newton}$.%
    \label{tbl:biomech2ErrorL2_2}%
  }[59mm]{%
    \setnumberoftableheaderrows{1}%
    \begin{tabular}{%
      >{\kern\tabcolsep}=l<{\kern2mm}%
      +c<{\kern-1mm}+c<{\kern-1mm}+c<{\kern\tabcolsep}%
    }
      \toprulec
      \headerrow
      $p$&   $1$&    $3$&             $5$\\
      \midrulec
      \bi&   $4.15$& $3.74$&          $3.72$\\
      \bii&  $3.42$& $\mathbf{2.83}$& $2.86$\\
      \biii& $4.15$& $4.06$&          $8.28$\\
      \bottomrulec
    \end{tabular}%
  }%
  \caption[Relative $L^2$ errors of forces and equilibrium elbow angle]{%
    Relative $\Ltwo$ errors of triceps/biceps force \emph{(left)} and
    equilibrium elbow angle \emph{(right)}
    for different hierarchical bases $\basis{\*l,\*i}$ and
    B-spline degrees $p$.
    Highlighted entries are the best among those with
    the same hierarchical basis or the same degree
    (similar to Nash equilibria).%
  }%
  \label{tbl:biomech2ErrorL2}%
\end{table}

\begin{figure}
  \includegraphics{biomech2ErrorForce_5}%
  \\[2mm]%
  \subcaptionbox{%
    $\abs{\forceXref - \forceXintp[p]}$ for
    $X = \mathrm{T}$ \emph{(left)} and
    $X = \mathrm{B}$ \emph{(right).}%
  }[73mm]{%
    \includegraphics{biomech2ErrorForce_1}%
    \hfill%
    \includegraphics{biomech2ErrorForce_2}%
  }%
  \hfill%
  \subcaptionbox{%
    $\abs{\forceXref - \forceXintp[p,\cc]}$ for
    $X = \mathrm{T}$ \emph{(left)} and
    $X = \mathrm{B}$ \emph{(right).}%
  }[73mm]{%
    \includegraphics{biomech2ErrorForce_3}%
    \hfill%
    \includegraphics{biomech2ErrorForce_4}%
  }%
  \caption[Absolute error of muscle forces]{%
    Absolute error of muscle forces $\forceT, \forceB$ for
    modified cubic B-splines ($p = 3$)
    on uniform sparse grids \emph{(left two plots)} and
    on Clenshaw--Curtis sparse grids \emph{(right two plots).}%
  }%
  \label{fig:biomech2ErrorForce}%
\end{figure}

\paragraph{Error of the elbow equilibrium angle}

\dummytext{}

\begin{figure}
  \includegraphics{biomech2ErrorEquilibriumAngle_4}%
  \\[2mm]%
  \subcaptionbox{%
    $\forceL = \SI{22}{\newton}$%
  }[49mm]{%
    \includegraphics{biomech2ErrorEquilibriumAngle_1}%
  }%
  \hfill%
  \subcaptionbox{%
    $\forceL = \SI{-60}{\newton}$%
  }[49mm]{%
    \includegraphics{biomech2ErrorEquilibriumAngle_2}%
  }%
  \hfill%
  \subcaptionbox{%
    $\forceL = \SI{180}{\newton}$%
  }[49mm]{%
    \includegraphics{biomech2ErrorEquilibriumAngle_3}%
  }%
  \caption[Absolute error of the equilibrium elbow angle]{%
    Absolute error
    $\abs{\equielbangref{\forceL} - \equielbangintp[p,\cc]{\forceL}}$
    of the equilibrium elbow angle for
    modified hierarchical cubic Clenshaw--Curtis B-splines ($p = 3$)
    for different loads $\forceL$.%
  }%
  \label{fig:biomech2ErrorEquilibriumAngle}%
\end{figure}



\subsection{Error in Test Scenarios}
\label{sec:733scenarios}

\dummytext[3]{}



\subsection{Spatial Adaptivity}
\label{sec:734adaptivity}

\dummytext[3]{}


\cleardoublepage
