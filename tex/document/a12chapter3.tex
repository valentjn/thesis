\section{Proofs for Chapter 3}
\label{sec:a12chapter3}

\printornamentsfalse
\subsection{Proof of the Linear Independence of Hierarchical B-Splines}
\label{sec:a121proofHierBSplineLinearlyIndependent}
\printornamentstrue

\propHierBSplineLinearlyIndependent*

\begin{proof}
  The proof is rigorous for the common B-spline degrees of
  $p \in \{1, 3, 5, 7\}$.
  For higher degrees, the proof has to be viewed as a sketch.
  
  We follow the presentation in \cite{Valentin16Hierarchical} and
  prove the assertion by induction over $l \in \natz$.
  For $l = 0$, the B-splines $\bspl{0,i'}{p}$ with $i' \in \{0, 1\}$
  are linearly independent.
  For the induction step $(l-1) \to l$, let
  \begin{equation}
    \label{eq:proofPropHierBSplineLinearlyIndependent1}
    \sum_{l'=0}^l \sum_{i' \in \hiset{l'}} \surplus{l',i'} \bspl{l',i'}{p}
    \equiv 0
  \end{equation}
  be a linear combination of the zero function.
  We separate the summands of level $l$
  from the summands of coarser levels $l' < l$:
  \begin{equation}
    \sum_{i \in \hiset{l}} \surplus{l,i} \bspl{l,i}{p}
    =: g_1 \equiv g_2 :=
    -\sum_{l'=0}^{l-1} \sum_{i' \in \hiset{l'}} \surplus{l',i'} \bspl{l',i'}{p}.
  \end{equation}
  The right-hand side $g_2$ is smooth in every grid point
  $\gp{l,i}$ of level $l$ ($i \in \hiset{l}$),
  since these grid points are not knots of the hierarchical B-splines
  $\bspl{l',i'}{p}$ of level $l' < l$ ($i' \in \hiset{l'}$).
  This implies that the left-hand side $g_1$ must be smooth there as well:
  \begin{equation}
    \label{eq:proofPropHierBSplineLinearlyIndependent2}
    \underbrace{
      \sum_{i \in \hiset{l}} \surplus{l,i}
      \partialdiff_-^p \bspl{l,i}{p}(\gp{l,i'})
    }_{{} = \partialdiff_-^p g_1(\gp{l,i'})}
    =
    \underbrace{
      \sum_{i \in \hiset{l}} \surplus{l,i}
      \partialdiff_+^p \bspl{l,i}{p}(\gp{l,i'})
    }_{{} = \partialdiff_+^p g_1(\gp{l,i'})},\quad
    i' \in \hiset{l},
  \end{equation}
  where $\partialdiff_-^p$ and $\partialdiff_+^p$ denote the left and right
  derivative of order $p$, respectively.
  By repeated application of \eqref{eq:cardinalBSplineDerivative},
  one can show that
  \begin{equation}
    \partialdiff_-^p \cardbspl{p}(k + 1)
    = (-1)^k \binom{p}{k}
    = \partialdiff_+^p \cardbspl{p}(k),\quad
    k \in \integer,
  \end{equation}
  where $\binom{p}{k} = 0$ for $k < 0$ or $k > p$
  \cite{Hoellig13Approximation}.
  We can insert this relation into
  \eqref{eq:proofPropHierBSplineLinearlyIndependent2}
  and use \eqref{eq:uniformHierarchicalBSplineUV} to obtain
  \begin{alignat}{2}
    \sum_{i \in \hiset{l}} \surplus{l,i} (-1)^{k-1} \binom{p}{k-1}
    &= \sum_{i \in \hiset{l}} \surplus{l,i} (-1)^k \binom{p}{k},\quad
    &&i' \in \hiset{l},\quad
    k := \frac{p+1}{2} + i' - i.
  \end{alignat}
  As $\binom{p}{k-1} + \binom{p}{k} = \binom{p+1}{k}$
  and $(-1)^k$ is constant for $i \in \hiset{l}$ when $i'$ is fixed,
  this is equivalent to
  \begin{equation}
    \label{eq:proofPropHierBSplineLinearlyIndependent3}
    \sum_{i \in \hiset{l}} \surplus{l,i}
    \binom{p+1}{\frac{p+1}{2} + i' - i} = 0,\quad
    i' \in \hiset{l}.
  \end{equation}
  
  This is a quadratic linear system of equations
  whose system matrix
  $\mat{A}(p)$ is a banded symmetric Toeplitz matrix%
  \footnote{%
    The entries $A_{k,j}$ of a Toeplitz matrix $\mat{A}$
    solely depend on $k - j$, i.e.,
    $A_{k,j} = c_{k-j}$ for some vector $\*c$.%
  }
  of size
  $2^{l-1} \times 2^{l-1}$ with bandwidth $\ceil{\tfrac{p-1}{4}}$.
  %For $p = 1, 3, 5, 7$, the top rows $a(p)$ of $\mat{A}(p)$,
  %which fully determines a symmetric Toeplitz matrix, is given as follows:
  %\begin{align}
  %  a(1) &= (\hphantom{7}2, \hphantom{2}0, 0, 0, \dotsc, 0),\\
  %  a(3) &= (\hphantom{7}6, \hphantom{2}1, 0, 0, \dotsc, 0),\\
  %  a(5) &= (20, \hphantom{2}6, 0, 0, \dotsc, 0),\\
  %  a(7) &= (70, 28, 1, 0, \dotsc, 0).
  %\end{align}
  %For $p = 1, 3, 5, 7$, $\mat{A}(p)$ is as follows:
  %\begin{equation}
  %  \begin{gathered}
  %    \mat{A}(1) =
  %    \begin{pmatrix}
  %      2&&\\
  %      &\ddots&\\
  %      &&2
  %    \end{pmatrix},\quad
  %    \mat{A}(3) =
  %    \begin{pmatrix}
  %      6&1&&\\
  %      1&\ddots&\ddots\\
  %      &\ddots&\ddots&1\\
  %      &&1&6
  %    \end{pmatrix},\\
  %    \mat{A}(5) =
  %    \begin{pmatrix}
  %      20&6&&\\
  %      6&\ddots&\ddots\\
  %      &\ddots&\ddots&6\\
  %      &&6&20
  %    \end{pmatrix},\quad
  %    \mat{A}(7) =
  %    \begin{pmatrix}
  %      70&28&1&&\\
  %      28&\ddots&\ddots&\ddots&\\
  %      1&\ddots&\ddots&\ddots&1\\
  %      &\ddots&\ddots&\ddots&28\\
  %      &&1&28&70
  %    \end{pmatrix},
  %  \end{gathered}
  %\end{equation}
  The non-zero values of $\mat{A}(p)$ are tabulated for some degrees $p$
  in \cref{tbl:proofHierBSplineLinearlyIndependent}.
  For $p = 1, 3, 5, 7$,
  the corresponding matrices are diagonally dominant and therefore regular.
  For higher B-spline degrees $p$, the regularity of $\mat{A}(p)$ has
  to be shown differently.
  
  \begin{table}
    \setnumberoftableheaderrows{1}%
    \begin{tabular}{>{\kern\tabcolsep}=l<{\kern7mm}+c+c+c+c<{\kern\tabcolsep}}
      \toprulec
      \headerrow
      &$k = 0$&$k = 1$&$k = 2$&$k = 3$\\
      \midrulec
      $p = 1$&2&&&\\
      $p = 3$&6&1&&\\
      $p = 5$&20&6&&\\
      $p = 7$&70&28&1&\\
      $p = 9$&252&120&10&\\
      $p = 11$&924&495&66&1\\
      \bottomrulec
    \end{tabular}%
    \caption[%
      Non-zero matrix values in the proof of linear independence%
    ]{%
      Non-zero values $A_{j,j+k}(p)$ of the diagonals of $\mat{A}(p)$
      obtained in \eqref{eq:proofPropHierBSplineLinearlyIndependent3}.%
    }%
    \label{tbl:proofHierBSplineLinearlyIndependent}%
  \end{table}
  
  Due to the regularity of $\mat{A}(p)$, we infer from
  \eqref{eq:proofPropHierBSplineLinearlyIndependent3} that
  $\surplus{l,i} = 0$ for $i \in \hiset{l}$.
  According to \eqref{eq:proofPropHierBSplineLinearlyIndependent1},
  we obtain
  a linear combination of the zero function with the hierarchical
  B-splines of level $< l$, i.e.,
  \begin{equation}
    \sum_{l'=0}^{l-1} \sum_{i' \in \hiset{l'}} \surplus{l',i'} \bspl{l',i'}{p}
    = 0,
  \end{equation}
  which implies $\surplus{l',i'} = 0$ for all $l' < l$ and $i' \in \hiset{l'}$
  by the induction hypothesis.
  Thus, the hierarchical B-splines $\bspl{l',i'}{p}$
  ($l' \le l$, $i' \in \hiset{l'}$) are linearly independent.
\end{proof}
