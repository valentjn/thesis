\section{Test Problems}
\label{sec:52testProblems}

It is impossible to assess the capability of optimization methods
for every possible optimization problem.
The most widespread approach in literature
is the selection of a subset of specific problems
with different characteristics \term{(test problems)} and
the application of the methods to only these problems,
in the hope that the methods perform similarly in
actual application settings.
In the following, we select five unconstrained problems
and three constrained problems.
For the unconstrained case and the standard hierarchical
B-spline basis, a more exhaustive list of test functions has been
studied previously \cite{Valentin14Hierarchische}.

\paragraph{Trivial test functions}

When testing methods that involve sparse grid interpolation,
one has to consider that the function to be interpolated
does not satisfy a specific \term{trivial property.}
A test function $\objfun\colon \clint{\*0, \*1} \to \real$ is trivial, if
$f$ is of the form
\begin{equation}
  \objfun \equiv \sum_{q=1}^m \prod_{t=1}^d \objfun_{q,t},\quad
  m \in \natz,\;\;
  \objfun_{q,t}\colon \clint{\*0, \*1} \to \real,\;\;
  \faex{q = 1, \dotsc, m}{t \in \{1, \dotsc, d\}}{
    \objfun_{q,t} \in \polyspace{1},
  }
\end{equation}
where $\polyspace{1}$ is the space of polynomials up to linear degree,
i.e., sums of tensor products of which at least one factor is a
linear polynomial.
Note that this is already fulfilled if the summands of $\objfun(\*x)$
do not depend on all coordinates $x_t$ of $\*x$.
One can show that for hat functions on sparse grids,
the hierarchical surpluses $\surplus{\*l,\*i}$ for trivial functions
vanish if $\*l \ge \*1$.
This means that trivial functions can well be approximated by hat functions
on sparse grids with boundary points, without placing any points
in the interior.
As this would distort our results,
we avoid trivial test functions in the following,
which include popular function such as the
Branin, Rosenbrock, and Schwefel functions.

\subsection{Unconstrained Problems}
\label{sec:511unconstrainedProblems}

\blindtext{}

\subsection{Constrained Problems}
\label{sec:512constrainedProblems}

\blindtext{}
