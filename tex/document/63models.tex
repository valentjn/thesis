\section{Micro-Cell Models and Optimization Scenarios}
\label{sec:63models}

\minitoc{72mm}{3}

\noindent
In the following, we present the different micro-cell models
and optimization scenarios for which we perform numerical
experiments in the following section.



\subsection{Micro-Cell Models}
\label{sec:631models}

We use the various micro-cell models that depicted in \cref{fig:microCell}.
\todo{mention paper if published}
The models differ in the spatial dimensionality $\dimdomain$ (two or three)
and the number $d$ of micro-cell parameters
$\*x \in \clint{\*0, \*1} = \clint{0, 1}^d$.

\begin{figure}
  \subcaptionbox{%
    \lefthphantom{Cross}{($d = 2$)}\\($d = 2$)%
    \label{fig:microCell_1}%
  }[31mm]{%
    \includegraphics{microCell_1}%
  }%
  \hfill%
  \subcaptionbox{%
    Framed cross\\($d = 4$)%
    \label{fig:microCell_2}%
  }[31mm]{%
    \includegraphics{microCell_3}%
  }%
  \hfill%
  \subcaptionbox{%
    Sheared cross\\\rlap{\hspace*{13mm}{($d = 3$)}}%
    \label{fig:microCell_3}%
  }[41mm]{%
    \includegraphics{microCell_2}%
  }%
  \hfill%
  \subcaptionbox{%
    Sheared framed cross\\($d = 5$)%
    \label{fig:microCell_4}%
  }[37mm]{%
    \hspace*{-45mm}%
    \rlap{\includegraphics{microCell_4}}%
  }\\[2mm]%
  \subcaptionbox{%
    3D cross ($d = 3$)%
    \label{fig:microCell_5}%
  }[44mm]{%
    \includegraphics{microCell_5}%
  }%
  \qquad%
  \subcaptionbox{%
    3D sheared cross ($d = 5$)%
    \label{fig:microCell_6}%
  }[50mm]{%
    \includegraphics{microCell_6}%
  }%
  \caption[Types of micro-cell models]{%
    Types of micro-cell models in two dimensions \emph{(top row)}
    and three dimensions \emph{(bottom row)}.%
  }%
  \label{fig:microCell}%
\end{figure}

\paragraph{Orthogonal (non-sheared) models in two dimensions}

The basic component of the four two-dimensional models
is a square with a \term{cross} (\cref{fig:microCell_1})
of two axis-aligned orthogonal bars,
whose widths are determined by two micro-cell parameters $x_1$ and $x_2$.
For the sake of micro-cell parameters,
we have to assume the micro-cell to have edge lengths of one,
since the micro-cells are assumed to be infinitesimally small,
i.e., they do not have a positive size.
To complicate the model, we add a diagonal cross with orthogonal bars
of widths $x_3$ and $x_4$ (horizontally measured), rotated by \ang{45}.
To simplify the boundary treatment,
we shift the contents of the micro-cell by $0.5$ in both directions,
such that a previous corner of the micro-cell corresponds to the new center.
This is possible because the micro-cells are periodic.
The resulting micro-model is a \term{framed cross} (\cref{fig:microCell_2}).

\paragraph{Sheared models in two dimensions}

Both models (cross and framed cross) can be extended by shearing.
The idea is to increase the stability of the resulting structure
with respect to forces that act at an angle other than
\ang{45} for the framed cross model.
If we just rotated the crosses in the micro-cells,
then the micro-structure would not be periodic.
Instead, we shear the whole micro-cell in the horizontal direction,
where the angle $\theta$ is a micro-cell parameter,
which gives us another degree of freedom.
This results in the \term{sheared cross} (\cref{fig:microCell_3})
and \term{sheared framed cross} (\cref{fig:microCell_4})
with three and five micro-parameters each.%
\footnote{%
  To be more precise, the angle $\theta$ corresponds to an
  additional micro-cell parameter $x_3$ (sheared cross) or
  $x_5$ (sheared framed cross) that is determined by normalization
  from $\clint{-\pi/2, \pi/2}$, i.e., $\theta/\pi + 1/2$.%
}

\paragraph{Models in three dimensions}

The two-dimensional cross model can be transferred to three
spatial dimensions by adding another bar in the new dimension.
Each of the three bars has square cross-section with given edge lengths
$x_1, x_2, x_3$,
resulting in the \term{3D cross} model with three micro-cell parameters
(\cref{fig:microCell_5}).
By shearing in the two horizontal directions,
we obtain two new degrees of freedom $\theta_1$ and $\theta_2$
(shearing angles).
The emerging \term{3D sheared cross} model has five micro-cell parameters
(\cref{fig:microCell_6}).



\subsection{Test Scenarios}
\label{sec:632scenarios}

\cite{Valdez17Topology}

\dummytext{}

\begin{figure}
  \subcaptionbox{%
    2D cantilever%
    \label{fig:topoOptScenario_1}%
  }[72mm]{%
    \includegraphics{topoOptScenario2D_1}%
  }%
  \hfill%
  \smash{\raisebox{-22mm}{\subcaptionbox{%
    2D L shape%
    \label{fig:topoOptScenario_2}%
  }[72mm]{%
    \includegraphics{topoOptScenario2D_2}%
  }}}%
  \\[7mm]%
  \subcaptionbox{%
    3D cantilever%
    \label{fig:topoOptScenario_3}%
  }[72mm]{%
    \includegraphics{topoOptScenario3D_1}%
  }%
  \hfill%
  \subcaptionbox{%
    3D center load%
    \label{fig:topoOptScenario_4}%
  }[72mm]{%
    \includegraphics{topoOptScenario3D_2}%
  }%
  \caption[Test scenarios in topology optimization]{%
    Test scenarios in topology optimization in
    two \emph{(top row)} and three spatial dimensions \emph{(bottom row)}.
    Shown are
    the domains $\domain$,
    the load points $\force$,
    the location of homogeneous Dirichlet boundary conditions
    \emph{\textcolor{C1}{(red)},} and
    examples for optimal structures \emph{\textcolor{C0}{(blue)}.}%
  }%
  \label{fig:topoOptScenario}%
\end{figure}
