\setdictum{%
  \todo{dictum text}%
}{%
  \todo{dictum authors}%
}

\longchapter{%
  Gradient-Based Optimization with B-Splines on Sparse Grids%
}{%
  Gradient-Based Optimization with B-Splines on Sparse Grids%
}{%
  Gradient-Based Optimization%
}
\label{chap:50optimization}

\initial{0em}{I}{n this chapter,}
we apply the hierarchical B-spline bases derived in
\cref{chap:30BSplines,chap:40algorithms} to optimization,
which is a major task in simulation technology,
for instance in inverse problems (see \cref{chap:10introduction}).
We pursue a surrogate-based optimization approach:
First, we sample the objective function at specific sparse grid points
to retrieve objective function values.
Second, by interpolating these values with hierarchical bases,
we obtain a surrogate for the objective function.
Finally, we discard the original objective function and apply
already existing optimization methods to the surrogate.
One of the key advantages of hierarchical B-splines
over common hierarchical bases for
sparse grids is their continuous differentiability.
The gradients of B-spline surrogates on sparse grids are not only continuous,
but also explicitly known and can be evaluated fast.
This gives the opportunity to employ gradient-based optimization techniques,
which usually converge significantly faster than gradient-free alternatives.

The outline of this chapter is as follows:
We start in \cref{sec:51algorithms}
with a compact overview of textbook optimization algorithms,
which comprises gradient-free and gradient-based optimization methods
for unconstrained problems as well as algorithms for constrained optimization.
In \cref{sec:52method}, we present the main method that
conflates the various textbook optimization algorithms
to a single method for the optimization of surrogates on sparse grids.
\Cref{sec:53testProblems} continues with a small array of test problems
for unconstrained and constrained optimization.
In \cref{sec:54results}, we apply the presented method
to the test problems, studying the effects of the different
hierarchical B-splines to optimization errors and conducting various
numerical experiments.
Finally, in \cref{sec:55fuzzy}, we examine with the fuzzy extension principle
an example application of optimization of hierarchical B-spline surrogates on
sparse grids.

Parts of this chapter have already been published in previous work
of the author of this thesis \cite{Valentin14Hierarchische}, especially
the overview of optimization algorithms (\cref{sec:51algorithms})
and partly the library of test problems (\cref{sec:53testProblems}).
However, as the previous work included other basis functions as well,
this thesis represents the first comprehensive study
that focuses on the application of hierarchical B-splines to optimization.

\section{Overview of Optimization Algorithms}
\label{sec:51algorithms}

\blindtext{}

\subsection{Gradient-Free Optimization}
\label{sec:511gradientFree}

\blindtext{}

\paragraph{Nelder--Mead}

\blindtext{}

\paragraph{Differential Evolution}

\blindtext{}

\paragraph{CMA-ES}

\blindtext{}

\subsection{Gradient-Based Unconstrained Optimization}
\label{sec:512gradientBasedUnconstrained}

\blindtext{}

\paragraph{Gradient Descent}

\blindtext{}

\paragraph{Newton}

\blindtext{}

\paragraph{NLCG}

\blindtext{}

\subsection{Gradient-Based Constrained Optimization}
\label{sec:513gradientBasedConstrained}

\blindtext{}

\section{Optimization of Surrogates on Sparse Grids}
\label{sec:52method}

\blindtext{}

\input{document/53testProblems}
\section{Numerical Results}
\label{sec:54results}

\blindtext{}



\subsection{Interpolation Error and Decay of Surpluses}
\label{sec:541interpolation}

% * different test functions for d=2 (p = 1, 3, 5)
% * different dimensionalities for sch22 (p = 1, 3, 5)
% * different grid types for sch22, d=2
% * NAK effect?

% alp02 instead of sch22 would be better...

\paragraph{Interpolation error for different test functions}

\blindtext{}

\paragraph{Interpolation error for different dimensionalities}

\blindtext{}

\paragraph{Interpolation error for different basis functions}

\blindtext{}

\paragraph{Pointwise interpolation error}

\blindtext{}

\paragraph{Decay of surpluses}

\blindtext{}



\subsection{Optimization Error}
\label{sec:542optimization}

\paragraph{Unconstrained optimization}

\blindtext{}

\paragraph{Constrained optimization}

\blindtext{}



\subsection{Complexity of Hierarchization}
\label{sec:543complexity}

\blindtext{}

\input{document/55fuzzy}

\cleardoublepage
