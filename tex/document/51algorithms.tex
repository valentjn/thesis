\section{Overview of Optimization Algorithms}
\label{sec:51algorithms}

\paragraph{Problem setting}

Generally, \term{unconstrained optimization problems} have the form
\begin{equation}
  \label{eq:unconstrainedOptimization}
  \xopt = \argmin \objfun(\*x),\quad
  \*x \in \real^d,
\end{equation}
where $\objfun\colon \real^d \to \real$ is the \term{objective function.}
In contrast, \term{constrained optimization problems} are given by
\begin{equation}
  \label{eq:constrainedOptimization}
  \xopt = \argmin \objfun(\*x),\quad
  \*x \in \real^d\;\;\text{s.t.}\;\;
  \ineqconfun(\*x) \le \*0,\;
  \eqconfun(\*x) = \*0,
\end{equation}
where
$\ineqconfun\colon \clint{\*0, \*1} \to \real^{m_{\ineqconfun}}$ and
$\eqconfun\colon \clint{\*0, \*1} \to \real^{m_{\eqconfun}}$
($m_{\ineqconfun}, m_{\eqconfun} \in \natz$)
are the \term{inequality and equality constraint functions,} respectively.
As sparse grid surrogates $f$ are only defined on the
unit hyper-cube $\clint{\*0, \*1}$,
the choice of $\*x$ has to be restricted to $\clint{\*0, \*1}$.
In the case of \eqref{eq:unconstrainedOptimization},
this results in a \term{box-constrained optimization problem.}
A simple method of applying algorithms for unconstrained problems
to box-constraints is extending $\objfun$ to $\real^d$ by
$\objfun(\*x) = +\infty$ for all $\*x \in \real^d \setminus \clint{\*0, \*1}$.
However, more sophisticated
%, algorithm-tailored
approaches are also available \cite{More87Optimization}.

\paragraph{Black-box optimization methods}

Problems of the form \eqref{eq:unconstrainedOptimization} or
\eqref{eq:constrainedOptimization} are \term{black-box optimization problems},
where we do not have any insight into the structure or algebraic
properties of $\objfun$.
Black-box optimization methods perform a series of evaluations
$\objfun(\*x_k)$ of $\objfun$.
Based on the function values $\*x_0, \dotsc, \*x_k$,
the methods choose the next evaluation point $\*x_{k+1}$.
Gradient-based methods differ from gradient-free approaches
in the way that they also take values of the gradient
$\gradient{\*x}{\objfun}(\*x_k)$ or of the Hessian
$\hessian{\*x}{\objfun}(\*x_k)$ into account.
A vast array of optimization method exists in literature.
Some methods are better suited for specific optimization problems
than others.
However, according to the No Free Lunch theorem \cite{Wolpert97No},
all methods perform equally well in the mean of all possible
optimization problems under some assumptions.

\paragraph{Local and global optima}

Most optimization methods depend on an initial point $\*x_0$ and
only find local optima,
where \eqref{eq:unconstrainedOptimization} or
\eqref{eq:constrainedOptimization} only holds for $\*x$
in a neighborhood of $\xopt$.
One can globalize local methods to increase the probability
of finding a global minimum with a Monte-Carlo multi-start approach:
The method is repeated with different pseudo-random initial points
and the best local minimum is chosen.

In the following,
we give a brief survey of a small selection of optimization methods
(see \cref{tbl:optimizationMethods}),
highlighting the key ingredients for each method.

\begin{table}
  \setnumberoftableheaderrows{1}%
  \begin{tabular}{%
    >{\kern\tabcolsep}=l+l<{\kern5mm}*{5}{+c}<{\kern\tabcolsep}%
  }
    \toprulec
    \headerrow
    Method&                 Type&                    C&    D&  S&    I\\
    \midrulec
    Nelder--Mead&           Simplex heuristic&       \no&  0&  \no&  \yes\\
    Differential evolution& Evolutionary&            \no&  0&  \yes& \yes\\
    CMA-ES&                 Evolutionary&            \no&  0&  \yes& \yes\\
    Simulated annealing&    Temperature heuristic&   \no&  0&  \yes& \no\\
    PSO&                    Swarm heuristic&         \no&  0&  \yes& \no\\
    GP-LCB&                 Bayesian&                \no&  0&  \yes& \no\\
    \midrulec
    Gradient descent&       Descent&                 \no&  1&  \no&  \yes\\
    NLCG&                   Descent&                 \no&  1&  \no&  \yes\\
    Newton&                 Newton&                  \no&  2&  \no&  \yes\\
    BFGS&                   Quasi-Newton&            \no&  1&  \no&  \yes\\
    Rprop&                  Heuristic&               \no&  1&  \no&  \yes\\
    Levenberg--Marquardt&   Least sq., trust-region& \no&  1&  \no&  \yes\\
    \midrulec
    Log-barrier&            Interior-point&          \yes& 0+& --&   \yes\\
    Squared penalty&        Penalty&                 \yes& 0+& --&   \yes\\
    Augmented Lagrangian&   Penalty&                 \yes& 0+& --&   \yes\\
    SQP&                    Quadratic subproblems&   \yes& 2&  --&   \no\\
    \bottomrulec
  \end{tabular}
  \caption[Selection of optimization methods]{%
    Selection of optimization methods.
    The columns show
    (C) if constrained problems are supported,
    (D) the order of required derivatives,
    (S) if the algorithm is stochastic, and
    (I) if the algorithm has been implemented in \sgpp.%
  }%
  \label{tbl:optimizationMethods}%
\end{table}

\begin{figure}
  \subcaptionbox{%
    $p = 1$%
  }[72mm]{%
    \includegraphics{fundamentalSpline_1}%
  }%
  \hfill%
  \subcaptionbox{%
    $p = 3$%
  }[72mm]{%
    \includegraphics{fundamentalSpline_2}%
  }\\[4mm]%
  \subcaptionbox{%
    $p = 5$%
  }[72mm]{%
    \includegraphics{fundamentalSpline_3}%
  }%
  \hfill%
  \subcaptionbox{%
    $p = 7$%
  }[72mm]{%
    \includegraphics{fundamentalSpline_4}%
  }%
  \caption[%
  Fundamental splines and their B-spline coefficients%
  ]{%
    The fundamental spline $\parentfundspl{p}$ \emph{\textcolor{C0}{(blue)}}
    is a linear combination of cardinal B-splines $\cardbspl{p}({\cdot} - k)$,
    $k \in \integer$ \emph{\textcolor{C0!50}{(light blue)}},
    with coefficients $\fundsplcoeff{k}{p}$ \emph{\textcolor{C1}{(red points)}}.
    The absolute values of the fundamental spline $\parentfundspl{p}$ and
    its coefficients are bounded by a multiple of $(\gamma_p)^{-\abs{k}}$
    \emph{(\textcolor{C0}{blue}-\textcolor{C1}{red}-dashed line)}.
    The axis for $\fundsplcoeff{k}{p}$ (on the right side) is scaled such that
    both bounding functions are on top of each other.%
  }%
  \label{fig:fundamentalSpline}%
\end{figure}


\subsection{Gradient-Free Unconstrained Optimization Methods}
\label{sec:511gradientFree}

\paragraph{Nelder--Mead}

The Nelder--Mead method
\multicite{Nelder65Simplex,Gao12Implementing,Valentin14Hierarchische}
maintains a list of $d + 1$ vertices of a $d$-dimensional simplex,
where the points are sorted by ascending function value.
In each iteration,
the method performs one of the operations
reflection, expansion, outer contraction, inner contraction, and shrinking
on the vertices.
Typically, convergence can be detected by
checking the size of the simplex,
as the simplex tends to contract around local minima.
However, there are counterexamples where the method converges to
a non-critical point even for an only bivariate, strictly convex, and
twice continuously differentiable objective function
\cite{McKinnon98Convergence}.
%For higher dimensionalities, the method quickly suffers from
%the curse of dimensionality \cite{Gao12Implementing}.

\paragraph{Differential evolution}

The method of differential evolution
\multicite{%
  Storn97Differential,%
  Zielinski09Optimizing,%
  Valentin14Hierarchische%
}
is an evolutionary meta-heuristic algorithm.
Being similar to genetic algorithms,
the method maintains a population of $m$ points
that is iteratively updated according to pseudo-random mutations,
which are weighted sums of the points of the previous generation.
The mutated vector is crossed over with the original vector
entry by entry.
The resulting offsprings are only accepted if they lead to an improvement in
terms of objective function value.
%There are various stopping criteria
%that are based on the function value or on the location of the population
%\cite{Zielinski09Optimizing}.

\paragraph{CMA-ES}

CMA-ES (covariance matrix adaption, evolution strategy)
\cite{Hansen03Reducing}
is also an evolutionary algorithm,
which addresses the issue
that simple evolution strategies do not prefer a search direction
due to the lack of gradients \cite{Toussaint15Introduction}.
The name of the algorithm stems from the fact
that it keeps track of the covariance matrix of the
Gaussian search distribution.
After $m$ points have been sampled from the current distribution and
the mean of the distribution for the next iteration
has been calculated as the weighted mean of the $k$ best samples,
the covariance matrix is updated accordingly.
One advantage of the method is that if the population is large enough,
local minima are smoothed out \cite{Toussaint15Introduction}.

\paragraph{Simulated annealing}

Simulated annealing
\multicite{Laarhoven87Simulated,Press07Numerical,Kiranyaz14Multidimensional}
imitates the cooling of a solid by randomly drawing samples
from a proposal distribution and calculating an acceptance
probability that depends on the function value improvement as well as
on a temperature parameter $T$.
The temperature is slowly decreased in the course of the algorithm.
Simulated annealing is closely connected to the
Metropolis--Hastings algorithm for drawing random samples of arbitrary
probability distributions.
For arbitrary runtime,
%(e.g., if the decrease rate of the temperature is small enough),
simulated annealing is guaranteed to find the global
optimum \cite{Toussaint15Introduction}.

\paragraph{Particle swarm optimization (PSO)}

The method of particle swarm optimization (PSO)
\multicite{Kennedy95Particle,Zielinski09Optimizing,Kiranyaz14Multidimensional}
can be seen as another evolutionary algorithm
that stems from swarm intelligence.
For each particle of the population,
not only the position $\*x_k$ is stored,
but also a current velocity $\*v_k$,
the best known position in a neighborhood of $\*x_k$
(which may be the whole swarm), and
the best known position of the $k$-th particle.
The next velocity $\*v_{k+1}$ is calculated as
the pseudo-randomly weighted sum of $\*v_k$,
the vector from $\*x_k$ to the best neighborhood position, and
the vector from $\*x_k$ to the best own position.

\paragraph{GP-LCB}

GP-LCB (Gaussian process, lower confidence bound)
\multicite{Srinivas10Gaussian,Toussaint15Introduction} is an example
for a Bayesian optimization strategy.
The objective function is treated as a stochastic process.
A prior distribution is updated according to the previous function
evaluations to calculate the posterior distribution.
The posterior distribution is used to form the acquisition function,
which is turn determines the point at which the objective
function is evaluated next.
The GP-LCB is obtained by choosing
Gaussian processes for the family of stochastic processes and
lower confidence bounds (which are the sum of the mean
and a multiple of the standard deviation) for the acquisition function.



\subsection{Gradient-Based Unconstrained Optimization Methods}
\label{sec:512gradientBasedUnconstrained}

Most gradient-based optimization algorithms determine in
each iteration $k$ a normed search direction $\*d_k$
($\norm[2]{\*d_k} = 1$) to update the current iterate $\*x_k$:
\begin{equation}
  \*x_k
  \to \*x_{k+1}
  := \*x_k + \alpha_k \*d_k,\qquad
  \alpha_k
  := \argmin_{\alpha \in \posreal} \objfun(\*x_k + \alpha \*d_k),
\end{equation}
where $\alpha_k \in \posreal$ is the step size.
The various algorithms differ in the choice of the search direction
$\*d_k \in \real^d$,
which is commonly chosen such that it points in negative gradient
direction ($\innerprod[2]{\*d_k}{\gradient{\*x}{\objfun}(\*x_k)} < 0$).
The step size $\alpha_k$ can then be determined independently of the
algorithm via a \term{line search algorithm.}
One of the most common is the \term{Armijo line search algorithm}
\multicite{Nocedal99Numerical,Ulbrich12Nichtlineare,Valentin14Hierarchische},
which uses a heuristic acceptance criterion
to find $\alpha_k$ with good enough improvement.

\paragraph{Gradient descent}

Gradient descent
\multicite{%
  Ulbrich12Nichtlineare,%
  Valentin14Hierarchische,%
  Toussaint15Introduction%
}
is the simplest gradient-based optimization method,
since it just chooses $\*d_k = -\gradient{\*x}{\objfun}(\*x_k)$.
The method of gradient descent may show a very slow convergence,
if the Hessian $\hessian{\*x}{\objfun}$
of the objective function is ill-conditioned.
One can show that for strictly convex and quadratic functions,
the error $\objfun(\*x_k) - \objfun(\xopt)$
%to the optimal function value
can decrease in each iteration only by the factor of
%$(\frac{\lambda^{\max} - \lambda^{\min}}{\lambda^{\max} + \lambda^{\min}})^2$,
$(\lambda^{\max} - \lambda^{\min})^2/(\lambda^{\max} + \lambda^{\min})^2$,
where $\lambda^{\min}$ and $\lambda^{\max}$ are the minimum and maximum
eigenvalue of $\hessian{\*x}{\objfun}$, respectively
\cite{Ulbrich12Nichtlineare}.
If the condition number
%$\frac{\lambda^{\max}}{\lambda^{\min}}$
$\lambda^{\max}/\lambda^{\min}$
of $\hessian{\*x}{\objfun}$ is large,
then this factor will be very close to one.

\paragraph{NLCG}

A possible remedy for this issue is the method of
non-linear conjugate gradients (NLCG)
\multicite{%
  Nocedal99Numerical,%
  Valentin14Hierarchische,%
  Toussaint15Introduction%
},
which is equivalent to the CG method for solving \spd linear systems
$\mat{A} \*x = \*b$,
if we optimize the strictly convex and quadratic function
$\objfun(\*x) := \frac{1}{2} \tr{\*x} \mat{A} \*x - \tr{\*b} \*x$
\multicite{Reinhardt13Nichtlineare,Valentin14Hierarchische}.
Due to this correspondence,
the NLCG method exactly finds the optimum after only $d$ steps
for strictly convex quadratic functions.
The NLCG quickly converges even for non-convex objective functions,
as three times continuously differentiable functions
with positive definite Hessian are ``similar'' to a
strictly convex quadratic function in a neighborhood of $\xopt$
\cite{Valentin14Hierarchische}.

\paragraph{Newton}

The Newton method
\multicite{%
  Ulbrich12Nichtlineare,%
  Valentin14Hierarchische,%
  Toussaint15Introduction%
}
replaces the objective function with the second-order Taylor approximation
given by
$\objfun(\*x_k + \*d_k)
\!\approx\! \objfun(\*x_k) +
\tr{(\*d_k)} \gradient{\*x}{\objfun}(\*x_k) \,+
\frac{1}{2} \tr{(\*d_k)} \hessian{\*x}{\objfun}(\*x_k) \*d_k$
and determines the search direction such that $\*x_k + \*d_k$ is
the minimum of the approximation, i.e.,
$\*d_k = -(\hessian{\*x}{\objfun}(\*x_k))^{-1} \gradient{\*x}{\objfun}(\*x_k)$.
Despite converging for strictly convex quadratic functions in a single step,
the Hessian must not be ill-conditioned for the Newton method as well,
as we have to solve a linear system with matrix
$\hessian{\*x}{\objfun}(\*x_k)$.
Hence, often a regularization or damping term $\lambda \eye$
for some $\lambda > 0$ is added to the Hessian.

\paragraph{BFGS}

The Newton method has the disadvantage that it needs to evaluate the
Hessian $\hessian{\*x}{\objfun}$,
which may be unavailable or too expensive.
Quasi-Newton methods such as the method of BFGS
(Broyden, Fletcher, Goldfarb, Shanno)
\multicite{%
  Nocedal99Numerical,%
  Ulbrich12Nichtlineare,%
  Toussaint15Introduction%
}
approximate the Hessian by a solution of the secant equation
$\hessian{\*x}{\objfun}(\*x_k) (\*x_k - \*x_{k-1}) \approx
\gradient{\*x}{\objfun}(\*x_k) - \gradient{\*x}{\objfun}(\*x_{k-1})$.
As the solution is not unique for $d > 1$,
Quasi-Newton methods differ in which solution to choose.
The BFGS method performs a simple rank-one update.

\paragraph{Rprop}

Rprop (resilient propagation)
\multicite{Riedmiller93Direct,Toussaint15Introduction}
considers the gradient entries $(\gradient{\*x}{\objfun}(\*x_k))_t$
of each dimension $t = 1, \dotsc, d$ separately
and updates the entries $x_{k,t}$ of $\*x_k$
according to the sign of the respective gradient entry and
individual step size adaption.
Although the algorithm is independent of the exact direction
of $\gradient{\*x}{\objfun}(\*x_k)$,
it was found to often work robustly in machine learning scenarios
\cite{Toussaint15Introduction}.

\paragraph{Levenberg--Marquardt}

The Levenberg--Marquardt method
\multicite{Nocedal99Numerical,Freund07Stoer,Toussaint15Introduction}
can only solve non-linear least-squares problems, i.e.,
the objective function must be of the form
$\objfun(\*x) = \tr{\*\phi(\*x)} \*\phi(\*x) = \sum_{i=1}^m \phi_i(\*x)^2$
for some function $\*\phi\colon \real^d \to \real^m$.
It is an improvement over the Gauss--Newton method
(which is in turn a slight modification of the Newton method)
and can be obtained by replacing the line search in the
Gauss--Newton method with a trust-region approach.



\subsection{Constrained Optimization Methods}
\label{sec:513gradientBasedConstrained}

Algorithms for constrained optimization usually
solve a series unconstrained auxiliary problems with an arbitrary
unconstrained optimization method.
The auxiliary function to be minimized is
the sum of the objective function and non-negative penalty terms,
which penalize if the current point $\*x_k$ is near the boundary
of the feasible domain or even outside.
The magnitude of the penalty terms slowly increase to enforce
the feasibility of the final solution.
Constrained optimization algorithms can roughly be divided
into \term{interior-point or barrier methods,}
where $\*x_k$ always stays in the feasible domain,
and \term{penalty methods,}
where intermediate solutions $\*x_k$ can be infeasible,
in which case the penalty term is applied
(the penalty vanishes inside the domain).

At least for the interior-point methods,
a feasible initial solution is required.
This can be found by solving another auxiliary problem
\cite{Toussaint15Introduction}, for instance
\begin{equation}
  \min_{(\*x, s) \in \real^{d+1}} s
  \quad\text{s.t.}\quad
  s \ge 0,\;\;
  \ineqconfun(\*x) \le s \cdot \*1_{m_{\ineqconfun}},\;\;
  -s \cdot \*1_{m_{\eqconfun}} \le \eqconfun(\*x) \le s \cdot \*1_{m_{\eqconfun}},
\end{equation}
where $\*1_{m_{\ineqconfun}} \in \real^{m_{\ineqconfun}}$ and
$\*1_{m_{\eqconfun}} \in \real^{m_{\eqconfun}}$ are all-ones vectors.
Initial solution for this problem can be explicitly given
(for example, $\*x_0 = \*0$ and
$s_0 = \max(\max(\ineqconfun(\*x_0)), \norm[\infty]{\eqconfun(\*x_0)})$).

\paragraph{Log-barrier}

The log-barrier method
\multicite{Boyd04Convex,Reinhardt13Nichtlineare,Toussaint15Introduction}
is an interior-point method that solves the inequality-constrained problem
$\min \objfun(\*x) \text{ s.t. } \ineqconfun(\*x) \le \*0$ by
adding a logarithmic barrier function term to the objective function
near the boundary, i.e., the method solves
$\min\, [\objfun(\*x) - \mu_k \sum_{i=1}^{m_{\ineqconfun}} \log(-g_i(\*x))]$
for some decreasing $\mu_k \in \posreal$.
Equality constraints can also be solved by incorporating them
into the unconstrained solver (e.g., see \cite{Boyd04Convex}
for an equality-constrained Newton method).

\paragraph{Squared penalty}

The squared penalty method
\multicite{%
  Polak71Computational,%
  Ulbrich12Nichtlineare,%
  Toussaint15Introduction%
}
replaces the constrained problem with the penalized problem
$\min\, [\objfun(\*x) + \mu_k \norm[2]{(\ineqconfun(\*x))_{+}}^2 +
\mu_k \norm[2]{\eqconfun(\*x)}^2]$,
where $\mu_k \in \posreal$ is an increasing penalty parameter and
$(\cdot)_{+} := \max(\cdot, 0)$ denotes the non-negative part.
With increasing $\mu_k$, the constraint violation of the solution of
the penalized problem will decrease, although it may happen
that it never vanishes.

\paragraph{Augmented Lagrangian}

The method of the augmented Lagrangian
\multicite{Reinhardt13Nichtlineare,Toussaint15Introduction}
considers the auxiliary problem
\begin{equation}
  \min_{\*x \in \real^d} \left[
    \objfun(\*x) + \mu_k \sum_{i=1}^{m_{\ineqconfun}} [\lambda_{k,i} > 0]
    ((\ineqconfun[i](\*x))_{+})^2 + \tr{\*\lambda_k} \ineqconfun(\*x) +
    \mu_k \norm[2]{\eqconfun(\*x)}^2 +
    \tr{\*\kappa_k} \eqconfun(\*x)
  \right],
\end{equation}
where $[\lambda_{k,i} > 0] \in \{0, 1\}$ is one if $\lambda_{k,i} > 0$ and
$\*\lambda_k \in \nonnegreal^{m_{\ineqconfun}}$ and
$\*\kappa_k \in \real^{m_{\eqconfun}}$ are estimates of the
Lagrangian multipliers.
They are updated according to the penalty of the previous iteration,
generating a ``virtual gradient'' that drastically decreases
the necessary magnitude of the penalty parameter $\mu_k$
to achieve feasibility of the solution \cite{Toussaint15Introduction}.

\paragraph{Sequential quadratic programming (SQP)}

Sequential quadratic programming (SQP) methods
\multicite{%
  Ulbrich12Nichtlineare,%
  Reinhardt13Nichtlineare,%
  Toussaint15Introduction%
}
are one of the most powerful method classes for constrained optimization.
They are motivated by the Karush--Kuhn--Tucker (KKT) conditions,
which are necessary to hold in any optimal point
(similarly to critical points in unconstrained optimization).
The Newton method can be employed to solve the KKT conditions,
as they can be written as a non-linear system of equations.
The linear system of the resulting \emph{Newton--Lagrange method}
is equivalent to the KKT conditions of a quadratic programming (QP) problem,
for which objective and constraint functions have the
form $\objfun(\*x) = \frac{1}{2} \tr{\*x} \mat{Q} \*x + \tr{\*d} \*x$,
$\ineqconfun(\*x) = \mat{A} \*x - \*b$, and
$\eqconfun(\*x) = \mat{B} \*x - \*c$, respectively.
SQP methods solve one QP problem in each iteration (``sequentially'')
to determine the search direction.
