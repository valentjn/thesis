\section{Overview of Optimization Algorithms}
\label{sec:51algorithms}

\paragraph{Problem setting}

Generally, \term{unconstrained optimization problems} have the form
\begin{equation}
  \label{eq:unconstrainedOptimization}
  \xopt = \argmin \objfun(\*x),\quad
  \*x \in \real^d,
\end{equation}
where $\objfun\colon \real^d \to \real$ is the \term{objective function.}
In contrast, \term{constrained optimization problems} are given by
\begin{equation}
  \label{eq:constrainedOptimization}
  \xopt = \argmin \objfun(\*x),\quad
  \*x \in \real^d\;\;\text{s.t.}\;\;
  \ineqconfun(\*x) \le \*0,
\end{equation}
where
$\ineqconfun\colon \real^d \to \real^{m_{\ineqconfun}}$
($m_{\ineqconfun} \in \natz$)
is the \term{(inequality) constraint function.}
This formulation also contains optimization problems
with equality constraints $\eqconfun(\*x) = \*0$
by setting $\ineqconfun(\*x) := (\eqconfun(\*x), -\eqconfun(\*x))$.
Equality constraints can also be solved by incorporating them
into the unconstrained solver (e.g., see \cite{Boyd04Convex}
for an equality-constrained Newton method).

As sparse grid surrogates $\objfun = \sgintp$ are only defined on the
unit hyper-cube,
the choice of $\*x$ has to be restricted to $\clint{\*0, \*1}$.
In the case of \eqref{eq:unconstrainedOptimization},
this results in a \term{box-constrained optimization problem.}
A simple method for applying unconstrained optimization algorithms
to box-constrained problems is extending $\sgintp$ to $\real^d$ by
$\sgintp(\*x) := +\infty$ for all $\*x \in \real^d \setminus \clint{\*0, \*1}$.
However, more sophisticated
%, algorithm-tailored
approaches are also available \cite{More87Optimization}.

\paragraph{Black-box optimization methods}

Problems of the form \eqref{eq:unconstrainedOptimization} or
\eqref{eq:constrainedOptimization} are \term{black-box optimization problems,}
where we cannot gain any insight into the structure or algebraic
properties of $\objfun$.
Black-box optimization methods perform a series of evaluations
$\objfun(\*x_k)$,
choosing the next evaluation point $\*x_{k+1}$
based on the previous function values $\objfun(\*x_0), \dotsc, \objfun(\*x_k)$.
Gradient-based methods differ from gradient-free approaches
in such a way that they also take values of the gradient
$\gradient{\*x}{\objfun}(\*x_k)$ or of the Hessian
$\hessian{\*x}{\objfun}(\*x_k)$ into account.

A vast array of optimization methods exists in literature.
Some methods are better suited for specific optimization problems
than others.
However, according to the ``no-free-lunch theorem''
and under some assumptions \cite{Wolpert97No},
all methods perform equally well (or equally badly) in the mean of all possible
optimization problems.

\paragraph{Local and global optima}

Most optimization methods depend on an initial point $\*x_0$ and
only find local optima,
where \eqref{eq:unconstrainedOptimization} or
\eqref{eq:constrainedOptimization} only holds for $\*x$
in a neighborhood of $\xopt$.
One can globalize local methods to increase the probability
of finding a global optimum with a Monte-Carlo multi-start approach:
The local method is repeated with different pseudo-random initial points
and the best local optimum is chosen as the result.

In the following,
we give a brief survey of a small selection of optimization methods
(see
\cref{tbl:optimizationMethod},
\cref{fig:optimizationMethodGradientFree}, and
\cref{fig:optimizationMethodGradientBased}),
highlighting the key ingredients for each method.

\begin{table}
  \setnumberoftableheaderrows{1}%
  \begin{tabular}{%
    >{\kern\tabcolsep}=l+l<{\kern5mm}*{5}{+c}<{\kern\tabcolsep}%
  }
    \toprulec
    \headerrow
    Method&                 Type&                    C&    D&  S&    I\\
    \midrulec
    Nelder--Mead&           Simplex heuristic&       \no&  0&  \no&  \yes\\
    Differential evolution& Evolutionary&            \no&  0&  \yes& \yes\\
    CMA-ES&                 Evolutionary&            \no&  0&  \yes& \yes\\
    Simulated annealing&    Temperature heuristic&   \no&  0&  \yes& \no\\
    PSO&                    Swarm heuristic&         \no&  0&  \yes& \no\\
    GP-LCB&                 Bayesian&                \no&  0&  \yes& \no\\
    \midrulec
    Gradient descent&       Descent&                 \no&  1&  \no&  \yes\\
    NLCG&                   Descent&                 \no&  1&  \no&  \yes\\
    Newton&                 Newton&                  \no&  2&  \no&  \yes\\
    BFGS&                   Quasi-Newton&            \no&  1&  \no&  \yes\\
    Rprop&                  Heuristic&               \no&  1&  \no&  \yes\\
    Levenberg--Marquardt&   Least sq., trust-region& \no&  1&  \no&  \yes\\
    \midrulec
    Log-barrier&            Interior-point&          \yes& 0+& --&   \yes\\
    Squared penalty&        Penalty&                 \yes& 0+& --&   \yes\\
    Augmented Lagrangian&   Penalty&                 \yes& 0+& --&   \yes\\
    SQP&                    Quadratic subproblems&   \yes& 2&  --&   \no\\
    \bottomrulec
  \end{tabular}
  \caption[Selection of optimization methods]{%
    Selection of optimization methods.
    The columns show
    if constrained problems are supported (C),
    the order of required derivatives (D),
    if the algorithm is stochastic (S), and
    if the algorithm has been implemented in \sgpp (I).%
  }%
  \label{tbl:optimizationMethod}%
\end{table}



\subsection{Gradient-Free Unconstrained Optimization Methods}
\label{sec:511gradientFree}

\newcommand*{\optImage}[1]{%
  \raisebox{-0.5\height}{\includegraphics{optimizationMethod_#1}}%
}

\begin{figure}
  \optImage{1}\quad\optImage{2}\quad\optImage{3}%
  \\[3mm]%
  \optImage{4}\qquad\optImage{5}\qquad\optImage{6}%
  \caption[Ideas of various gradient-free optimization methods]{%
    Sketch of the ideas of gradient-free optimization methods.%
  }%
  \label{fig:optimizationMethodGradientFree}%
\end{figure}

\paragraph{Nelder--Mead}

The \term{Nelder--Mead method}
\multicite{Nelder65Simplex,Gao12Implementing,Valentin14Hierarchische}
maintains a list of $d + 1$ vertices of a $d$-dimensional simplex,
sorted by ascending function value.
In each iteration,
the method performs one of the operations
\term{reflection,}
\term{expansion,}
\term{outer contraction,}
\term{inner contraction,} and
\term{shrinking}
on the vertices.
Typically, convergence can be detected by
checking the size of the simplex,
as the simplex tends to contract around local minima.
However, there are counterexamples where the method converges to
a non-critical point for an only bivariate objective function
that is strictly convex and twice continuously differentiable
\cite{McKinnon98Convergence}.
%For higher dimensionalities, the method quickly suffers from
%the curse of dimensionality \cite{Gao12Implementing}.

\paragraph{Differential evolution}

The method of \term{differential evolution}
\multicite{%
  Storn97Differential,%
  Zielinski09Optimizing,%
  Valentin14Hierarchische%
}
is an evolutionary meta-heuristic algorithm.
Being similar to genetic algorithms,
the method maintains a \term{population} of $m$ points
that is iteratively updated according to pseudo-random \term{mutations,}
which are weighted sums of the points of the previous generation.
The mutated vector is \term{crossed over} with the original vector
entry by entry.
The resulting \term{offsprings} are only accepted if they lead to
an improvement in terms of objective function value.
%There are various stopping criteria
%that are based on the function value or on the location of the population
%\cite{Zielinski09Optimizing}.

\paragraph{CMA-ES}

\term{CMA-ES (covariance matrix adaption, evolution strategy)}
\cite{Hansen03Reducing}
is an evolutionary algorithm that addresses the issue
that simple evolution strategies do not prefer a search direction
due to the lack of gradients \cite{Toussaint15Introduction}.
The name of the algorithm stems from the fact
that it keeps track of the \term{covariance matrix} of the
Gaussian search distribution.
After $m$ points have been sampled from the current distribution,
the mean of the distribution for the next iteration
is calculated as the weighted mean of the $k$ best samples and
the covariance matrix is adapted accordingly.
An advantage of the method is that if the population is large enough,
local minima are smoothed out \cite{Toussaint15Introduction}.

\paragraph{Simulated annealing}

\term{Simulated annealing}
\multicite{Laarhoven87Simulated,Press07Numerical,Kiranyaz14Multidimensional}
imitates the cooling of a solid by randomly drawing samples
from a proposal distribution and calculating an \term{acceptance probability}
that depends on the function value improvement as well as
on a \term{temperature} $T$.
This temperature is slowly decreased in the course of the algorithm.
Simulated annealing is closely connected to the
Metropolis--Hastings algorithm for drawing random samples of arbitrary
probability distributions.
If run long enough,
%(e.g., if the decrease rate of the temperature is small enough),
simulated annealing is guaranteed to find the global
optimum \cite{Toussaint15Introduction}.

\paragraph{Particle swarm optimization (PSO)}

The method of \term{particle swarm optimization (PSO)}
\multicite{Kennedy95Particle,Zielinski09Optimizing,Kiranyaz14Multidimensional}
can be seen as another evolutionary algorithm
that stems from swarm intelligence.
For each \term{particle} of the population,
not only the \term{position} $\*x_k$ is stored,
but also the current \term{velocity} $\*v_k$,
the best known position in a neighborhood of $\*x_k$
(which may be the whole swarm), and
the best known position of the $k$-th particle.
The next velocity $\*v_{k+1}$ is computed as
a pseudo-randomly weighted sum of $\*v_k$,
the vector from $\*x_k$ to the best neighborhood position, and
the vector from $\*x_k$ to the best own position.

\paragraph{GP-LCB}

\term{GP-LCB (Gaussian process, lower confidence bound)}
\multicite{Srinivas10Gaussian,Toussaint15Introduction} is an example
for a \term{Bayesian optimization} strategy.
The objective function is treated as a stochastic process.
A \term{prior distribution} is updated according to the previous function
evaluations to calculate the \term{posterior distribution.}
The posterior distribution is used to form the \term{acquisition function,}
which in turn determines the point at which the objective
function is evaluated next.
The GP-LCB method is obtained by choosing
\term{Gaussian processes} for the family of stochastic processes and
\term{lower confidence bounds} (which are the sum of the mean
and a multiple of the standard deviation) for the acquisition function.



\subsection{Gradient-Based Unconstrained Optimization Methods}
\label{sec:512gradientBasedUnconstrained}

\begin{figure}
  \optImage{7}\qquad\optImage{8}\qquad\optImage{9}%
  \\[3mm]%
  \optImage{10}\quad\optImage{11}\quad\optImage{12}%
  \caption[Ideas of various gradient-based optimization methods]{%
    Sketch of the ideas of gradient-based or constrained
    optimization methods.%
  }%
  \label{fig:optimizationMethodGradientBased}%
\end{figure}

Most gradient-based optimization algorithms determine in
each iteration $k$ a unit \term{search direction} $\*d_k \in \real^d$
($\norm[2]{\*d_k} = 1$) to update the current iterate $\*x_k$:
\begin{equation}
  \*x_k
  \to \*x_{k+1}
  := \*x_k + \delta_k \*d_k,\qquad
  \delta_k
  := \argmin_{\delta \in \posreal} \objfun(\*x_k + \delta \*d_k),
\end{equation}
where $\delta_k \in \posreal$ is the \term{step size.}
The algorithms essentially differ in the
choice of the search direction $\*d_k$,
which should point in negative gradient
direction ($\innerprod[2]{\*d_k}{\gradient{\*x}{\objfun}(\*x_k)} < 0$).
The step size $\delta_k$ can then be determined independently of the
algorithm via \term{line search,}
for instance, the \term{Armijo line search algorithm}
\multicite{Nocedal99Numerical,Ulbrich12Nichtlineare,Valentin14Hierarchische},
which uses a heuristic acceptance criterion
to find $\delta_k$ with a good enough improvement.

\paragraph{Gradient descent}

\term{Gradient descent}
\multicite{%
  Ulbrich12Nichtlineare,%
  Valentin14Hierarchische,%
  Toussaint15Introduction%
}
%is the simplest gradient-based optimization method,
%since it
just chooses $\*d_k
\!\propto\! -\gradient{\*x}{\!\objfun}(\*x_k)$ (i.e., normalized).
The method suffers from slow convergence,
if the Hessian $\hessian{\*x}{\objfun}$ is ill-conditioned:
One can show that for strictly convex quadratic functions,
the error $\objfun(\*x_k) - \objfun(\xopt)$
%to the optimal function value
can decrease in each iteration only by the factor of
$(\tfrac{\lambda^{\max} - \lambda^{\min}}{\lambda^{\max} + \lambda^{\min}})^2$,
%$(\lambda^{\max} - \lambda^{\min})^2/(\lambda^{\max} + \lambda^{\min})^2$,
where $\lambda^{\min}$ and $\lambda^{\max}$ are the minimum and maximum
eigenvalue of $\hessian{\*x}{\objfun}$, respectively
\cite{Ulbrich12Nichtlineare}.
If the condition number
$\tfrac{\lambda^{\max}}{\lambda^{\min}}$
%$\lambda^{\max}/\lambda^{\min}$
of $\hessian{\*x}{\objfun}$ is large,
then this factor will be very close to one.

\paragraph{NLCG}

A possible remedy for this issue is the method of
\term{non-linear conjugate gradients (NLCG)}
\multicite{%
  Nocedal99Numerical,%
  Valentin14Hierarchische,%
  Toussaint15Introduction%
}.
It is equivalent to the CG method for solving \spd linear systems
$\mat{A} \*x = \*b$,
if we optimize the strictly convex quadratic function
$\objfun(\*x) := \frac{1}{2} \tr{\*x} \mat{A} \*x - \tr{\*b} \*x$
\multicite{Reinhardt13Nichtlineare,Valentin14Hierarchische}, i.e.,
it finds the optimum after only $d$ steps for strictly convex
quadratic functions.
The NLCG method quickly converges even for non-convex objective functions,
as due to the Taylor theorem,
three times continuously differentiable functions
with positive definite Hessian are ``similar'' to a
strictly convex quadratic function in a neighborhood of $\xopt$
\cite{Valentin14Hierarchische}.

\paragraph{Newton}

The \term{Newton method}
\multicite{%
  Ulbrich12Nichtlineare,%
  Valentin14Hierarchische,%
  Toussaint15Introduction%
}
replaces the objective function with the second-order Taylor approximation
given by
$\objfun(\*x_k + \*d_k)
\!\approx\! \objfun(\*x_k) +
\tr{(\gradient{\*x}{\objfun}(\*x_k))} \*d_k \,+
\frac{1}{2} \tr{(\*d_k)} \hessian{\*x}{\objfun}(\*x_k) \*d_k$
and determines the search direction such that $\*x_k + \*d_k$ is
the minimum of the approximation, i.e.,
$\*d_k \propto
-(\hessian{\*x}{\objfun}(\*x_k))^{-1} \gradient{\*x}{\objfun}(\*x_k)$.
Despite converging for strictly convex quadratic functions in a single step,
the Hessian must not be ill-conditioned for the Newton method as well,
as we have to solve a linear system with matrix
$\hessian{\*x}{\objfun}(\*x_k)$.
Hence, often a \term{regularization/damping term} $\lambda \eye$
for some $\lambda > 0$ is added to the Hessian.

\paragraph{BFGS}

The Newton method has the disadvantage that it needs to evaluate the
Hessian $\hessian{\*x}{\objfun}$,
which may be unavailable or too expensive.
\term{Quasi-Newton methods} such as the method of
\term{BFGS (Broyden, Fletcher, Goldfarb, Shanno)}
\multicite{%
  Nocedal99Numerical,%
  Ulbrich12Nichtlineare,%
  Toussaint15Introduction%
}
approximate the Hessian by a solution of the secant equation
$\hessian{\*x}{\objfun}(\*x_k) (\*x_k - \*x_{k-1}) \approx
\gradient{\*x}{\objfun}(\*x_k) - \gradient{\*x}{\objfun}(\*x_{k-1})$.
As the solution is not unique for $d > 1$,
Quasi-Newton methods differ in which solution to choose.
The BFGS method performs a simple rank-one update.

\paragraph{Rprop}

\term{Rprop (resilient propagation)}
\multicite{Riedmiller93Direct,Toussaint15Introduction}
considers the gradient entries $(\gradient{\*x}{\objfun}(\*x_k))_t$
of each dimension $t = 1, \dotsc, d$ separately
and updates the entries $x_{k,t}$ of $\*x_k$
according to the sign of the respective gradient entry,
adapting the step size dimension-wise.
Although the algorithm is independent of the exact direction
of $\gradient{\*x}{\objfun}(\*x_k)$,
it was found to often work robustly in machine learning scenarios
\cite{Toussaint15Introduction}.

\paragraph{Levenberg--Marquardt}

The \term{Levenberg--Marquardt method}
\multicite{Nocedal99Numerical,Freund07Stoer,Toussaint15Introduction}
can only solve \term{non-linear least-squares problems,} i.e.,
the objective function must be of the form
$\objfun(\*x) = \tr{\*\phi(\*x)} \*\phi(\*x) = \sum_{i=1}^m \phi_i(\*x)^2$
for some function $\*\phi\colon \real^d \to \real^m$.
It is an improvement over the \term{Gauss--Newton method}
(which is in turn a slight modification of the Newton method)
and can be obtained by replacing the line search in the
Gauss--Newton method with a \term{trust-region approach.}



\subsection{Constrained Optimization Methods}
\label{sec:513gradientBasedConstrained}

Methods for constrained optimization usually
solve a series of unconstrained \term{auxiliary problems} with an arbitrary
unconstrained optimization method.
The auxiliary function to be minimized is
the sum of the objective function and \term{penalty terms,}
which penalize if the current point $\*x_k$ is near the boundary
of the feasible domain or even outside.
The penalty terms slowly increase to enforce
the feasibility of the final result.
Constrained optimization methods can roughly be divided
into \term{interior-point or barrier methods,}
where $\*x_k$ always stays in the feasible domain,
and \term{penalty methods,}
where intermediate solutions $\*x_k$ may be infeasible,
in which case the penalty term is applied.
%(the penalty vanishes inside the domain).

At least for the interior-point methods,
a feasible initial solution $\*x_0$ is required.
We can find an initial solution by solving another auxiliary problem
\cite{Toussaint15Introduction}, for instance
\begin{equation}
  \min_{(\*x, s) \in \real^{d+1}} s
  \quad\text{s.t.}\quad
  s \ge 0,\;\;
  \ineqconfun(\*x) \le s \cdot \*1_{m_{\ineqconfun}},
\end{equation}
where $\*1_{m_{\ineqconfun}} \in \real^{m_{\ineqconfun}}$
is the all-one vector.
An initial solution for this problem can be explicitly given
(for example, $\*x_0 = \*0$ and
$s_0 = \max(\max(\ineqconfun(\*x_0)), 0)$).

\paragraph{Log-barrier}

The \term{log-barrier method}
\multicite{Boyd04Convex,Reinhardt13Nichtlineare,Toussaint15Introduction}
is an interior-point method that adds
a logarithmic \term{barrier function term} to the objective function
near the boundary.
The method solves
$\min\, [\objfun(\*x) - \mu_k \sum_{i=1}^{m_{\ineqconfun}} \log(-g_i(\*x))]$
for some decreasing $\mu_k \in \posreal$.

\paragraph{Squared penalty}

The \term{squared penalty method}
\multicite{%
  Polak71Computational,%
  Ulbrich12Nichtlineare,%
  Toussaint15Introduction%
}
replaces the constrained problem with the penalized problem
$\min\, [\objfun(\*x) + \mu_k \norm[2]{\nonnegpart{\ineqconfun(\*x)}}^2]$,
where $\mu_k \in \posreal$ is a penalty parameter and
$\nonnegpart{\cdot} := \*\max(\cdot, \*0)$ denotes the non-negative part.
With increasing $\mu_k$, the constraint violation of the solution of
the penalized problem decreases, although it may happen
that it never vanishes.

\paragraph{Augmented Lagrangian}

The method of the \term{augmented Lagrangian}
\multicite{Reinhardt13Nichtlineare,Toussaint15Introduction}
considers the auxiliary problem
\begin{equation}
  \min_{\*x \in \real^d} \bracket*{
    \objfun(\*x) + \mu_k \sum_{i=1}^{m_{\ineqconfun}} [\lambda_{k,i} > 0]
    ((\ineqconfun[i](\*x))_{+})^2 + \tr{\*\lambda_k} \ineqconfun(\*x)
  },
\end{equation}
where $[\lambda_{k,i} > 0] \in \{0, 1\}$ is defined as one,
if and only if $\lambda_{k,i} > 0$, and
$\*\lambda_k \in \nonnegreal^{m_{\ineqconfun}}$ is an estimate of the
\term{Lagrangian multipliers.}
They are updated according to the penalty of the previous iteration,
generating a ``virtual gradient'' that drastically decreases
the necessary magnitude of the penalty parameter $\mu_k$
to achieve feasibility of the solution \cite{Toussaint15Introduction}.

\paragraph{Sequential quadratic programming (SQP)}

\term{Sequential quadratic programming (SQP) methods}
\multicite{%
  Ulbrich12Nichtlineare,%
  Reinhardt13Nichtlineare,%
  Toussaint15Introduction%
}
are one of the most powerful method classes for constrained optimization.
They are motivated by the \term{Karush--Kuhn--Tucker (KKT) conditions,}
which are necessary to hold in any optimal point
(similarly to critical points in unconstrained optimization).
The Newton method can be employed to solve the KKT conditions when
written as a non-linear system of equations.
The linear system of the resulting \emph{Newton--Lagrange method}
is equivalent to the KKT conditions of a
\term{quadratic programming (QP) problem,}
for which objective and constraint functions have the
form $\objfun(\*x) = \frac{1}{2} \tr{\*x} \mat{Q} \*x + \tr{\*d} \*x$ and
$\ineqconfun(\*x) = \mat{A} \*x - \*b$, respectively.
%SQP methods solve one QP problem in each iteration (``sequentially'')
%to determine the search direction.
