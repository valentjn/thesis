\section{Constrained Problems}
\label{sec:a22constrained}

\paragraph{G08}

This problem originates from \cite{Schoenauer93Constrained}.
We changed the domain from $\clint{0, 10}^2$ to increase the
size of feasible region.
In addition, we use different frequencies for the sine terms
as in \cite{Gavana13Global}.
\vspace{-1.6em}

\begin{subequations}
  \begin{gather}
    \centertestfunline{
      \testobjfunscaled{G08}(\xscaled)
      := -\frac{
        \sin^3(2\pi\xscaled[1]) \sin(2\pi\xscaled[2])
      }{
        \xscaled[1]^3 (\xscaled[1] + \xscaled[2])
      },\quad
      \testineqconfunscaled{G08}(\xscaled)
      := \begin{pmatrix}
        \xscaled[1]^2 - \xscaled[2] + 1\\
        1 - \xscaled[1] + (\xscaled[2] - 4)^2
      \end{pmatrix},
    }\\
    \centertestfunline{
      \xscaled \in \clint{0.5, 2.5} \times \clint{3, 6},\quad
      \xoptscaled = (1.227971358337, 4.245373366474),
    }\\
    \centertestfunline{
      \testobjfunscaled{G08}(\xoptscaled) = -0.09582504141804
    }
  \end{gather}
\end{subequations}


\paragraph{G04Squared}

This problem is based on a problem from
\cite{Colville68Comparative} with the objective function
$\testobjfunscaled{G04}(\xscaled)
:= 5.3578547 \xscaled[3]^2 + 0.8356891 \xscaled[1] \xscaled[5] +
37.293239 \xscaled[1] - 40792.141$ and the same constraints
$\testineqconfunscaled{G04}(\xscaled) :=
\testineqconfunscaled{G04Sq}(\xscaled)$.
However, hierarchical cubic not-a-knot B-splines are able to exactly
represent the polynomial $\testobjfunscaled{G04}$ of coordinate degree two
on the whole domain $\clint{\*0, \*1}$,
if the level of the sparse grids is fine enough,
see \thmref{cor:sparseGridRegularNAKPolynomials}.
Therefore, we modified the original G04 problem by squaring the
objective function.
To ensure that this does not change location of the global minimum,
we added a constant before squaring such that the shifted function
is non-negative on $\clint{\*0, \*1}$.
\vspace{-1.6em}

\begin{subequations}
  \begin{gather}
  \centertestfunline{
    \testobjfunscaled{G04Sq}(\xscaled)
    := (5.3578547 \xscaled[3]^2 + 0.8356891 \xscaled[1] \xscaled[5] +
    37.293239 \xscaled[1] - 10120)^2,
  }\\
  \centertestfunline{
    \testineqconfunscaled{G04Sq}(\xscaled)
    := 10^{-3} \scalebox{0.92}{$
      \begin{pmatrix}
        85334.407 + 5.6858 \xscaled[2] \xscaled[5] +
        0.6262 \xscaled[1] \xscaled[4] -
        2.2053 \xscaled[3] \xscaled[5] - 92000\\
        -85334.407 - 5.6858 \xscaled[2] \xscaled[5] -
        0.6262 \xscaled[1] \xscaled[4] +
        2.2053 \xscaled[3] \xscaled[5]\\
        80512.49 + 7.1317 \xscaled[2] \xscaled[5] +
        2.9955 \xscaled[1] \xscaled[2] +
        2.1813 \xscaled[3]^2 - 110000\\
        -80512.49 - 7.1317 \xscaled[2] \xscaled[5] -
        2.9955 \xscaled[1] \xscaled[2] -
        2.1813 \xscaled[3]^2 + 90000\\
        9300.961 + 4.7026 \xscaled[3] \xscaled[5] +
        1.2547 \xscaled[1] \xscaled[3] +
        1.9085 \xscaled[3] \xscaled[4] - 25000\\
        -9300.961 - 4.7026 \xscaled[3] \xscaled[5] -
        1.2547 \xscaled[1] \xscaled[3] -
        1.9085 \xscaled[3] \xscaled[4] + 20000
      \end{pmatrix},
    $}
  }\\
  \centertestfunline{
    \xscaled \in \clint{78, 102} \times \clint{33, 45} \times
    \clint{27, 45}^3,
  }\\
  \centertestfunline{
    \xoptscaled = (78, 33, 29.995256025682, 45, 36.775812905788),
  }\\
  \centertestfunline{
    \testobjfunscaled{G04Sq}(\xoptscaled) = 43.590737882363
  }
  \end{gather}
\end{subequations}
