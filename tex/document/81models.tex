\section{Dynamic Portfolio Lifecycle Models}
\label{sec:81models}

In this section, we give a mathematical framework for
dynamic portfolio choice models,
explain where B-splines on sparse grids come into play,
and mention related literature briefly.



\subsection{Bellman Equation}
\label{sec:811bellmanEquation}

\paragraph{Utility maximization}

Dynamic portfolio lifecycle models aim to maximize the expected
\term{discounted utility} over the lifetime of the individual.
If we neglect stochastic factors, then these models solve
\begin{equation}
  \label{eq:utilityMaximization}
  (\optpolicyfcn_0, \dotsc, \optpolicyfcn_T)
  = \argmax_{\policy_0, \dotsc, \policy_T}
  \sum_{t=0}^T \patience^t \utility(\consume_t(\state_t, \policy_t))
  \quad\text{s.t. specific constraints.}
\end{equation}
Here, $\state_t \in \clint{\*0, \*1} \subset \real^d$ and
$\policy_t \in \real^{m_{\policy}}$
are the state and policy of time $t = 0, \dotsc, T$, respectively.
The constraints ensure that for instance, we do not spend more money
than we actually have.
Starting from a given initial state $\state_0$,
the state $\state_t$ of time $t > 0$ can be computed from
$\state_0$ and $\policy_0, \dotsc, \policy_{t-1}$ with a
\term{state transition function} $(\state_t, \policy_t) \mapsto \state_{t+1}$.
As shown in \cref{fig:dynamicPortfolioChoice},
in each time step, a fraction of the available wealth
is invested in \term{consumption} $\consume_t$,
which can be computed from the state $\state_t$ and the policy $\policy_t$.
We rate the consumption with a \term{utility function}
$\utility(\consume_t)$.
A common choice for $\utility$ is the
\term{constant relative risk aversion (CRRA) utility}
$\utility(\consume_t) := (c_t^{1-\riskav})/(1-\riskav)$
with the \term{risk aversion} $\riskav \in \real \setminus \{1\}$.
Positive and negative values correspond to risk-averse and risk-affine
individuals, respectively.
The factor $\patience \in \pohint{0, 1}$ is the \term{patience}
or \term{time discount factor.}

\begin{SCfigure}
  \includegraphics{dynamicPortfolioChoice_1}%
  \caption[TODO]{%
    Dynamic portfolio lifecycle models.
    The available wealth $\wealth_t$ is invested
    into bonds ($\bond_t$) and two stocks ($\stock_{t,1}, \stock_{t,2}$).
    The rest is consumed ($\consume_t$), resulting in the utility
    $\utility(\consume_t)$.
    In the last time step $T$ \emph{(far right),}
    the whole wealth is consumed, if we do not take inheritance
    into account.%
  }%
  \label{fig:dynamicPortfolioChoice}%
\end{SCfigure}

\paragraph{Limitations of naive utility maximization}

When solving the utility maximization problem \eqref{eq:utilityMaximization},
there are two issues.
First, solving \eqref{eq:utilityMaximization} for all times $t$ at once
implies solving a $(T+1) m$-dimensional optimization problem,
which is usually computationally infeasible.
Second, \eqref{eq:utilityMaximization} does not take stochastic variables
$\stochastic_t$ such as stock return rates into account.
These variables influence the state transition, i.e.,
$(\state_t, \policy_t, \stochastic_t) \mapsto \state_{t+1}$.
Consequently, $\state_t$ cannot be computed from $\state_0$ and
$\policy_0, \dotsc, \policy_{t-1}$ alone,
which complicates the solution of \eqref{eq:utilityMaximization}
even in the expected value.

\paragraph{Bellman principle}

To solve the first issue,
Bellman's principle of optimality \cite{Bellman57Dynamic}
can be applied to problems like
\eqref{eq:utilityMaximization} that are said to have
\term{optimal substructure.}
The principle states that the optimal policy for all times $t = 0, \dotsc, T$
is also optimal with respect to $t = 1, \dotsc, T$, i.e.,
\begin{equation}
  \max_{\policy_0, \dotsc, \policy_T}
  \sum_{t=0}^T \patience^t \utility(\consume_t(\state_t, \policy_t))
  = \max_{\policy_0} \left(
    \utility(\consume_0(\state_0, \policy_0))
    + \patience \max_{\policy_1, \dotsc, \policy_T}
    \sum_{t=1}^T \patience^{t-1} \utility(\consume_t(\state_t, \policy_t))
  \right),
\end{equation}
where we omitted the constraints for brevity.
The inner maximum problem over $\policy_1, \dotsc, \policy_T$
has the same structure as the original problem on the \lhs.
With the \term{value function}
$\valuefcn_t\colon \clint{\*0, \*1} \to \real$,
$\valuefcn_t(\state_t) :=
\max_{\policy_0, \dotsc, \policy_T}
\sum_{t'=t}^T \patience^{t'}
\utility(\consume_{t'}(\state_{t'}, \policy_{t'}))$, this can be rewritten as
\begin{equation}
  \label{eq:simpleBellman}
  \valuefcn_0(\state_0)
  = \max_{\policy_0} \left(
    \utility(\consume_0(\state_0, \policy_0)) +
    \patience \valuefcn_1(\state_1)
  \right)
  \quad\text{s.t. specific constraints,}
\end{equation}
where $\state_1$ is the result of the state transition
from $(\state_0, \policy_0)$.

\paragraph{General Bellman equation}

If we formulate \eqref{eq:simpleBellman} for arbitrary times $t$ and
consider stochastic variables, constraints, and state transition,
then we obtain the following \term{Bellman equation:}
\begin{subequations}
  \label{eq:generalBellman}
  \begin{gather}
    \valuefcn_t(\state_t)
    = \max_{\policy_t} \left(
      \utility(\consume_t(\state_t, \policy_t)) +
      \patience \expectation[t]{
        \valuefcn_{t+1}(\statefcn_t(\state_t, \policy_t, \stochastic_t))
      }
    \right),\quad
    t < T,\\
    \policy_t \in \real^{m_{\policy}}\;\;\text{s.t.}\;\;
    \ineqconfun_t(\state_t, \policy_t) \le \*0,
  \end{gather}
\end{subequations}
where
$\statefcn_t\colon \clint{\*0, \*1} \times \real^{m_{\policy}} \times
\stochdomain_t \to \clint{\*0, \*1}$,
$(\state_t, \policy_t, \stochastic_t) \mapsto \state_{t+1}$,
is the \term{state transition function,}
$\ineqconfun_t\colon \clint{\*0, \*1} \times \real^{m_{\policy}} \to
\real^{m_{\ineqconfun}}$ is the \term{constraint function,}
and the expected value is
\begin{equation}
  \expectation[t]{
    \valuefcn_{t+1}(\statefcn_t(\state_t, \policy_t, \stochastic_t))
  }
  = \int_{\stochdomain_t}
  \valuefcn_{t+1}(\statefcn_t(\state_t, \policy_t, \stochastic_t))
  P_t(\stochastic_t) \diff{}\stochastic_t
\end{equation}
with the probability density function
$P_t\colon \stochdomain_t \to \nonnegreal$ of $\stochastic_t$.
We denote the location of the maximum of \eqref{eq:generalBellman}
as the optimal policy $\optpolicyfcn_t$,
which may be regarded as a function
$\optpolicyfcn_t\colon \clint{\*0, \*1} \to \real^{m_{\policy}}$,
$\state_t \mapsto \optpolicyfcn_t(\state_t)$.



\subsection{Solution with B-Spline Surrogates on Sparse Grids}
\label{sec:812surrogates}

\blindtext{}



\subsection{Related Work}
\label{sec:813relatedWork}

\blindtext{}
